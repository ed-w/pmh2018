\section{2018-07-31 Lecture}

Tangent vector field

Analogy: a tangent vector in the vector space $T_pS$ at the point $p$ of a surface $S \subset \RR^N$ is like a number and a tangent vector field is like a function.

\begin{defn}
  Let $S$ be a regular surface with a chart
  \begin{align*}
    F: V \subset \RR^n &\to U \cap S \\
    \vec u = (u_1,\ldots,u_n) &\mapsto \left( F_1(u),\ldots,F_N(u) \right).
  \end{align*}
  The partial derivative
  \[ \frac{\p F}{\p u_i} = \left( \frac{\p F_1}{u_i}, \ldots, \frac{\p F_N}{\p u_i} \right) \]
  is an $\RR^N$-valued function with variables $u_1,\ldots,u_n$.
  The space of tangent vector fields of $S$ is spanned by linear combinations of the set
  \[ \left\{ \frac{\p F}{\p u_i} \right\}_{i=1}^n. \]
  In this local chart, we have
  \[ \sum_{i=1}^n f_i \frac{\p F}{\p u_i}, \qquad f_i \in \cC^\infty(V). \]
\end{defn}

\begin{rmk}
  Note that $\tfrac{\p F}{\p u_i}$ is the tangent vector of $S$ in the $u_i$-curve direction.
\end{rmk}

\begin{rmk}
  We often write simply
  \[ \frac{\p}{\p u_i} \text{ instead of } \frac{\p F}{\p u_i} \]
  for a notation that is independent of the chart.
\end{rmk}

\begin{rmk}
  In $\RR^N$ it is safe to move vectors (i.e.\@ their basepoints) around.
  In a more general regular surface, the Euclidean idea of parallel no longer makes sense.
  So it is important to keep track of the basepoint of the tangent vector.
\end{rmk}

Differential geometry

Differential calculus (differentiation and integration) on manifolds

We are already familiar with differentiation for functions from $\RR^N \to \RR$.
Differentiation of a function from $S \to \RR$ is a directional derivative (i.e.\@ a tangent vector).
A tangent vector field is a section of the tangent bundle, i.e. a (smooth) map from $S \to TS$ which maps $p \to v \in T_pS$.

How can we define the ``derivative'' of a function from $S \to TS$?
The resultant vector might not be in the tangent bundle.
We can simply project down in to the relevant tangent plane (since we are considering surfaces in $\RR^N$).

\begin{defn}[1.5]
  Let $S$ be a regular surface.
  Let $p \in S$ with co-ordinate chart $F: V \to S \cap U$.
  For any point $q \in S \cap U$ with $F(\vec u)=q$, define two tangent vector fields:
  \begin{align*}
    X(q) &= \sum_{i=1}^n f_i(\vec u) \frac{\p F}{\p u_i} \\
    Y(q) &= \sum_{i=1}^n g_i(\vec u) \frac{\p F}{\p u_i}.
  \end{align*}
  Consider $X(p)$ as the Euclidean vector at $p$ and $Y$ as the $\RR^N$-valued function
  \begin{align*}
    \bar Y: V &\to T_qS \injto T_q\RR^N \isoto \RR^N \\
    q &\mapsto Y(q)
  \end{align*}
  Then we have the derivative of $Y$ in the $X$-direction at $q$:
  \[ \nabla_XY = \proj_{TS} D_X(\bar Y) \]
  where $\proj$ is the Euclidean projection on to the subspace $T_pS \subset T_p\RR^N = \RR^N$.
  Here $D_X(\bar Y)$ is the directional derivative of $\bar Y$ in the direction of $X$ (at a given point $p$).
  This is the \textbf{Levi-Civita covariant derivative}.
\end{defn}

\begin{exer}
  Compute the differentiation of the rotation vector field (of constant angular velocity) of the 2-sphere in its own direction.
\end{exer}

1.2 Fundamental forms

\begin{defn}
  \textbf{A fundamental form}, denoted by $FF$, is a family of symmetric bilinear maps $FF_p: T_pS \times T_pS \to \RR$ parametrised by $p \in S$ such that it is smooth in $p$.
\end{defn}

Let $S \subset \RR^N$.
There are two particularly important fundamental forms:

\begin{defn}[1.7]
The \textbf{first fundamental form}, $I$, is directly induced by the Euclidean inner product in $\RR^N$.
  That is,
  \[ I_p(X,Y) = \ang{X,Y} \]
  for $X,Y \in T_pS \subset T_p\RR^N \cong \RR^N$, and $\ang{\cdot,\cdot}$ is the Euclidean inner product.
\end{defn}

\begin{rmk}
  \lv
  \begin{enum}
    \io
    Think about measuring the speed of motions (i.e.\@ the length of tangent vectors) in $S \subset \RR^N$.

    \io
    Global quantities (motions) could be affected.
    (Infitesimally, $I$ is just the Euclidean inner product).
    As an example, consider $S^1 \subset \RR^2$.
    The diameter of $S^1$ in $\RR^2$ is $2$ but in $S^1$ it is $\pi$.
  \end{enum}
\end{rmk}

Before we look at the second fundamental form:

Now consider $S \subset \RR^N$ with a chart $F: V \to S \cap U$ where $V \subset \RR^n$ is an open subset and $N=n+1$.
This is a hypersurface.
In this setting, we have a normal vector field of $S$ in $\RR^{n+1}$ (up to a sign).

Suppose we have a global unit normal vector field $\nu$ of $S \subset \RR^{n+1}$ which is a smooth $\RR^{n+1}$-valued function on $S$.

\begin{defn}
  Keep the above setting.
  The \textbf{second fundamental form}, $II$, is:
  \[ II_p(X,Y) = \ang{D_X\bar\nu,Y} = I_p(D_X\bar\nu,Y) \]
  where $X,Y \in T_pS$, $\bar\nu(q) = \nu(q) \in T_pS \subset T_p\RR^{n+1} \cong \RR^{n+1}$ and $D_X\bar\nu$ is the directional derivative.
\end{defn}

\begin{exer}
  Prove that $I$ and $II$ are fundamental forms.
\end{exer}
