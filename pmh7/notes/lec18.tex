\section{2018-10-08 Lecture}

Continuing on from last time, we have
\begin{align*}
  \frac{d\abs\gamma}{ds} &= \int_a^b \frac{1}{\sqrt{\left( \frac{\p\gamma}{\p t},\frac{\p\gamma}{\p t} \right)}} \left( \nabla_{\frac{\p\gamma}{\p t}} \frac{\p\gamma}{\p s}, \frac{\p\gamma}{\p t} \right) \, dt \\
  &= \int_a^b \frac{1}{\sqrt{\left( \frac{\p\gamma}{\p t},\frac{\p\gamma}{\p t} \right)}} \frac{\p}{\p t} \left( \frac{\p\gamma}{\p s}, \frac{\p\gamma}{\p t} \right) \, dt - \int_a^b \frac{1}{\sqrt{\left( \frac{\p\gamma}{\p t},\frac{\p\gamma}{\p t} \right)}} \left( \nabla_{\frac{\p\gamma}{\p t}} \frac{\p\gamma}{\p t}, \frac{\p\gamma}{\p s} \right) \, dt
\end{align*}
Now consider $s=0$ and w.l.o.g.\@ choose $t$ such that $\abs{\tfrac{\p\gamma}{\p t}}|_{s=0}=1$ (arc-length parametrisation).
Then
\begin{align*}
  0 &= \frac{d\abs\gamma}{ds}\bigg\vert_{s=0} = \int_a^b \frac{\p}{\p t} \left( \frac{\p\gamma}{\p s}, \frac{\p\gamma}{\p t} \right) \, dt - \int_a^b \left( \nabla_{\frac{\p\gamma}{\p t}} \frac{\p\gamma}{\p t}, \frac{\p\gamma}{\p s} \right) \, dt \\
  &= \left( \frac{\p\gamma}{\p s}, \frac{\p\gamma}{\p t} \right)_{s=0} \bigg\vert^{t=b}_{t=a} - \int_a^b \left( \nabla_{\frac{\p\gamma}{\p t}} \frac{\p\gamma}{\p t}, \frac{\p\gamma}{\p s} \right)_{s=0} \, dt \\
  &= - \int_a^b \left( \nabla_{\frac{\p\gamma}{\p t}} \frac{\p\gamma}{\p t}, \frac{\p\gamma}{\p s} \right)_{s=0} \, dt
\end{align*}
where the last equality holds becasue $\tfrac{\p\gamma}{\p s}=0$ at the endpoints.

This holds for any variation $\gamma(t,s)$ of $\gamma(t)=\gamma(t,0)$.
We claim that this implies
\[ \nabla_{\frac{\p\gamma}{\p t}} \frac{\p\gamma}{\p t} = 0 \]
identically.
Suppose that $\nabla_{\frac{\p\gamma}{\p t}} \frac{\p\gamma}{\p t} \neq 0$ identically.
Then there exists a $t_0 \in (a,b)$ such that $(\nabla_{\frac{\p\gamma}{\p t}} \frac{\p\gamma}{\p t})(p)\neq0$, any by continuity is non-zero everywhere on a neighbourhood $t\in(t_1,t_2)\subset(a,b)$.
Suppose $\gamma\left( (t_1,t_2) \right)$ is in a co-ordinate chart $(U,x_1,\ldots,x_n)$.
Then we can write
\[ \nabla_{\frac{\p\gamma}{\p t}} \frac{\p\gamma}{\p t} = \sum_{i=1}^n f_i \frac{\p}{\p x_i} \]
where WLOG $f_1\neq 0$.
Now construct the following variation $\gamma(t,s)$ of $\gamma(t)$:
\begin{equation*}
  \gamma(t,s)=
  \begin{cases}
    \gamma(t) &\text{outside $U$} \\
    \gamma(t) + \left( s(f_1,\ldots,f_n) \right)\cdot\rho(t) &\text{in $U$}
  \end{cases}
\end{equation*}
where $\rho(t)$ is a cutoff function for $(t_1,t_2)\subset[a,b]$.
Then
\begin{align*}
  \frac{\p\gamma}{\p s}\bigg\vert_{s=0}=
  \begin{cases}
    0 &\text{outside $U$} \\
    \ds \sum_{i=1}^n f_i \frac{\p}{\p x_i} \cdot \rho(t) = \nabla_{\frac{\p\gamma}{\p t}} \frac{\p\gamma}{\p t} \cdot \rho(t) &\text{in $U$}
  \end{cases}
\end{align*}
and so
\begin{align*}
  0 &= \frac{d\abs\gamma}{dt}\bigg\vert_{s=0} = - \int_a^b \left( \nabla_{\frac{\p\gamma}{\p t}} \frac{\p\gamma}{\p t}, \frac{\p\gamma}{\p s} \right)_{s=0} \, dt \\
  &= - \int_a^b \rho(t) \left( \nabla_{\frac{\p\gamma}{\p t}} \frac{\p\gamma}{\p t}, \nabla_{\frac{\p\gamma}{\p t}} \frac{\p\gamma}{\p t} \right)_{s=0} \, dt
  &= - \int_a^b \rho(t) \abs{ \nabla_{\frac{\p\gamma}{\p t}} \frac{\p\gamma}{\p t} }^2_{s=0} \, dt < 0
\end{align*}
a contradiction.

So we have the \textbf{geodesic equation}
\[ \nabla_{\frac{\p\gamma}{\p t}} \frac{\p\gamma}{\p t} = 0. \]

\begin{exer}
  Show that the geodeis equation implies that $|\tfrac{d\gamma}{dt}|$ is a constant along $\gamma$.
\end{exer}

\begin{rmk}
  Without the restriction $|\tfrac{d\gamma}{dt}|=1$, we would get the equation
  \[ \nabla_{\frac{\p\gamma}{\p t}} \frac{\frac{\p\gamma}{\p t}}{\abs{\frac{\p\gamma}{\p t}}} = 0. \]
\end{rmk}

We have also the \textbf{$L^2$-energy functional}
\[ E(\gamma) = \int_a^b \abs{\frac{d\gamma}{dt}}^2 \, dt. \]
This functional depends on the parametrisation.
Its critical points are also given by the geodesic equation:
\[ \frac{dE}{ds} = -2\int_a^b \left( \nabla_{\frac{\p\gamma}{\p t}} \frac{\p\gamma}{\p t}, \frac{\p\gamma}{\p s} \right). \]
Note that we have made no assumption on $|\tfrac{\p\gamma}{\p t}|$.

The relation between $\abs\cdot$ and $E(\cdot)$

By Cauchy-Schwarz we have
\[ \left( \int_a^b \abs{\frac{d\gamma}{dt}} \, dt \right)^2 \leq \int_a^b \abs{\frac{d\gamma}{dt}}^2 \, dt \int_a^b \, dt = (b-a)E(\gamma) \]
with equality when $|\tfrac{d\gamma}{dt}|$ is constant.
So the minimum of $E(\gamma)$ is achieved when $\abs\gamma$ is minimsed and $|\tfrac{d\gamma}{dt}|$ is constant.

Local equation for geodesics

Let $\left( U,\left\{ x^i \right\}_{i=1}^n \right)$ be a co-ordinate chart in $(M,g)$ and let $\gamma=(\gamma^1,\ldots,\gamma^n)$ be the local expression of $\gamma$ in $U$.
Let $\dot\box$ denote a derivative with respect to $t$.
Then $\dot\gamma=\dot\gamma^i\p_i$.
So
\begin{align*}
  \nabla_{\frac{\p\gamma}{\p t}} \frac{\p\gamma}{\p t} &= \nabla_{\dot\gamma^i\p_i} \dot\gamma^j\p_j \\
  &= \dot\gamma^i (\p_i\dot\gamma^j) \p_j + \dot\gamma^i\dot\gamma^j \nabla_{\p_i}\p_j \\
  &= \ddot\gamma^k \p_k + \dot\gamma^i\dot\gamma^j \Gamma_{ij}^k \p_k = 0.
\end{align*}
Hence
\[ \ddot\gamma^k+\dot\gamma^i\dot\gamma^j\Gamma_{ij}^k =0 \]
for $k=1,\ldots,n$.
This is a system of 2nd order ODEs.
ODE theory gives us existence and uniqueness of a smooth solution with initila conditions $\gamma(0)$, $\tfrac{d\gamma}{dt}(0)$.
