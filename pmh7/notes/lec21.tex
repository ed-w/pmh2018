\section{2018-10-16 Lecture}

\begin{rmk}
  When we say that a curve is a geodesic we usually mean that it is constant speed.
\end{rmk}

\begin{rmk}
  The geodesic equation is a system of second order nonlinear ODEs.
  The Jacobi equation is system of second order linear ODEs.
\end{rmk}

By classical ODE theory there exists a unique solution $X(t)$ (a Jacobi field) to the initial value problem (a choice of $X(0)$ and $\dot{X}(0)$). 

\begin{rmk}[4.5]
  Pick $e_1=\dot\gamma$.
  Then the Jacobi equation gives
  \[ \ddot{X}^1 + \Rm(e_i,e_1,e_1,e_1)X^i = 0 \]
  so
  \[ X^1 = X^1(0) + \dot{X}^1(0)t. \]
  This corresponds to a rescaling of the geodesic $\gamma(t)$, that is, $\gamma(t) \rightsquigarrow \gamma(ast)$ for $a\in\RR$.
  Then
  \[ \frac{\p\gamma}{\p s} = at \frac{d\gamma(u)}{du}(ast) = ate_1. \]
  So this is a special Jacobi field $X=X^1e_1$, and $X^1=at$.

  \begin{exer}
    For a geodesic variation $\gamma(s,t)$, we van change it to $\gamma(s,ast)$ for some (properly chosen) $a\in\RR$ such that $\tfrac{\p\gamma}{\p s}|_{s=0}$ has no $e_1$-component.
  \end{exer}

  So we can w.l.o.g\@ assume that the Jacobi field has no $e_1=\dot\gamma$-component.
  Then the resulting variations really are ``variations'' and not reparametrisations of the geodesic curve.
\end{rmk}

\begin{exam}
  Consider a normal co-ordinate chart $\{\p_i\}$ defined by $\exp_p:T_pM\to M$.
  The geodesics are the rays through the origin, so it is clear what possible Jacobi fields $X$ with $X(p)=0$ we can have.
  At $0\in T_pM$ and $p\in M$, the frame $\{\p_i\}$is orthonormal, but not for other $q\in M$ in general.
  The Jacobi field can be used to study those $\p_is$.

  For a geodesic curve $\gamma(t):[0,\infty)\to M$ with $\gamma(0)=p$, we can see that $t\p_2$ is a Jacobi field and so it solves the Jacobi equation.
  We can express this in orthonormal co-ordinates:
  \[ t\p_2 = \sum_{i=2}^n X^ie_i. \]
  Here the co-efficients $X^i$ are defined by the Jacobi equation.
  So the Jacobi equation allows us to link normal co-ordinates with orthonormal co-ordinates along a geodesic.
\end{exam}

4.4 More facts on geodesic

We want to find some global features of geodesics.
The geodesic equation for the length functional of a curve is a local equation.
We have the following fact:
\begin{prop}
  There is always a minimising geodesic curve between any $2$ points in a metrically complete and connected Riemannian manifold.
\end{prop}

\begin{proof}[Outline of proof]
  We can define a distance function
  \[ \dist(p,q) = \inf_{\substack{\gamma(a)=p\\ \gamma(b)=q}}\abs\gamma. \]
  We can show that this is a metric.
  So we have a sequence $(\gamma_k)$ of geodesics with $\abs{\gamma_k}\to\dist(p,q)$.
  Then we have $\gamma_k\to\gamma$ weakly for some $\gamma$ hence it satisfies the geodesic equation $\nabla_{\dot\gamma}\dot\gamma$ weakly.
  We can then show that $\gamma$ is smooth, hence is the desired minimising curve.
\end{proof}

Let $p\in M$ for a geodesically complete and connected Riemannian manifold.
Then the exponential map $\exp_p:T_pM\to M$ is surjective (by the above fact).
However, $\exp_p$ is only diffeomorphic in a neigbourhood of $p$.

When $\exp_p$ is a diffeomorphism, the geodesic is minimising.
(We will not prove this fact.)
Now the orthogonality to the radial direction is preserved by $\exp_p$ because of our discussion on Jacobi fields.
More specifically, let $X=\sum_{i=2}^nX^ie_i$ be a Jacobi field.
Then the geodesic gives the shortest distance in this region (see MATH3968 notes).
So for the distance minimising problem, we only need to study where $\exp_p$ fails to be diffeomorphic.
There are two cases:
\begin{enum}
  \io $d(\exp_p)_v$ is not invertible.
  Then $d(\exp_p)_v(w)=0$ for some $v,w\in T_pM$.
  Now consider the corresponding Jacobi field.
  The Jacobi field becomes $0$ at $\exp_p(v)$.
  All such vectors $w$ form a subspace of $T_v(T_pM)=T_pM$.
  The dimension of this subspace is called the \textbf{index} of this \textbf{conjugate point}.

  \io $\exp_p(v)=\exp_p(w)$ for two different vectors $v,w\in T_pM$.
  For example, the $2$-sphere.
\end{enum}

The loss of minimality of geodesic distance can only happen in codimension 1 set (the \textbf{cut locus}).
