\section{2018-08-20 Lecture}

Last time we defined vector bundles and the tangent bundle.
What is the manifold structure on $TM$? we can define this using gluing.

Let $M$ be a manifold.
Locally we have $M = \bigcup_{i \in I} U_i$ where the $U_i$ are co-ordinate charts.
Note that the union is not disjoint.
Let $F_i: V_i \to U_i$ where $V_i \subset \RR^n$ be the associated maps.
For each $U_i$, take $TU_i$ to be $TV_i = V_i \times \RR^n$.

What happens at the intersections?
Suppose $U = U_1 \cap U_2 \neq \emptyset$ and consider the restrictions of the co-ordinate maps $F_1: V_1 \to U$ and $F_2: V_2 \to U$.
Then we can view $TU$ both as $V_1 \times \RR^n$ and as $V_2 \times \RR^n$.
For $TU \longleftrightarrow V_1 \times \RR^n$ we have
\[ \sum_{i=1}^n a_i \frac{\p}{\p u_i}\Bigg\vert_{(u_1,\ldots,u_n)} \longleftrightarrow (u_1,\ldots,u_n,a_1,\ldots,a_n) \]
and likewise for $TU \longleftrightarrow V_2 \times \RR^n$ we have
\[ \sum_{i=1}^n b_i \frac{\p}{\p v_i}\Bigg\vert_{(v_1,\ldots,v_n)} \longleftrightarrow (v_1,\ldots,v_n,b_1,\ldots,b_n). \]
How can we make these two compatible?
We want
\[ \sum_{i=1}^n a_i \frac{\p}{\p u_i} = \sum_{j=1}^n b_j \frac{\p}{\p v_j}. \]
We will make a change of variables.
Using the chain rule, we have
\[ \sum_{i=1}^n a_i \frac{\p}{\p u_i} = \sum_{i=1}^n \sum_{j=1}^n a_i \frac{\p v_i}{\p u_j} \frac{\p}{\p v_j} \]
so we want
\[ b_j = \sum_{i=1}^n a_i \frac{\p v_j}{\p u_i}. \]
Now define
\begin{align*}
  V \times \RR^n &\to V_2 \times \RR^n \\
  (\vec u, \vec a) &\mapsto (\vec v, \vec b)
\end{align*}
and
\[ \vec b = \left( \frac{\p v_i}{\p u_j} \right)_{i,j=1}^n \vec a. \]
We need to check the following properties:
\begin{enum}
  \io The map $V \times \RR^n \to V \times \RR^n$ is the identity.
  \io The inverse map is given by
  \[ \vec a = \left( \frac{\p u_i}{\p v_j} \right)_{i,j=1}^n \vec b. \]
  \io The order of gluing does not matter (the chain rule):
  \[ \vec c = \left( \frac{\p v_i}{\p w_j} \right)_{i,j=1}^n \vec b \text{ and } \vec b = \left( \frac{\p u_i}{\p v_j} \right)_{i,j=1}^n \vec a \text{ implies } \vec c = \left( \frac{\p u_i}{\p w_j} \right)_{i,j=1}^n \vec a. \]
\end{enum}
Then this gluing defines an equivalence relation so we can define
\[ TM = \coprod_{i \in I} V_i \times \RR^n \Big/ \sim. \]
We need to check the following properties to show that this is a smooth manifold:
\begin{enum}
  \io It is locally Euclidean ($V \times \RR^n \subset \RR^{2n}$).
  \io The transition function
  \[ (\vec u, \vec a) \mapsto \left( \vec v, \left( \frac{\p v}{\p u} \right) \vec a \right) \]
  is smooth (with smooth inverse) on intersections of charts.
\end{enum}
So $TM$ is a smooth vector bundle.

\begin{rmk}[2.12]
  To sum up, we have three views of the tangent space:
  \begin{enum}
    \io Differentiation of scalar functions (velocity)
    \io Equivalence classes of local trivialisations
    \io Whitney's view on manifolds (embedding in Euclidean space)
  \end{enum}
\end{rmk}

\begin{defn}[2.13]
  Let $\Phi: M_1 \to M_2$ be a smooth map of manifolds.
  The \textbf{differential} of $\Phi$ at $p \in M_1$ is the linear map
  \[ (d\Phi)_p: T_pM_1 \to T_pM_2 \]
  defined as follows:
  For any $X \in T_pM_1$ take a motion $\alpha(t)$ with $\alpha(0)=p$ and $\alpha'(0)=X$.
  Then
  \[ (d\Phi)_p(X) = \frac{d\Phi(\alpha(t))}{dt}\Bigg\vert_{t=0}. \]
\end{defn}

We need to show that this definition is well-defined.
In particular, it needs to be independent of the choice of $\alpha$ and is a linear map.

Let $X = \sum_i a_i \tfrac{\p}{\p u_i} \in T_pM_1$ and let $F_1: V_1 \to U_1 \subset M_1$ and $F_2: V_2 \to U_2 \subset M_2$ be two co-ordinate maps with co-ordinates $\vec u=(u_1,\ldots,u_n)$ and $\vec v=(v_1,\ldots,v_n)$ respectively.
Moreover, let $F_1(\vec 0)=p$ and $\vec v = (v_1(\vec u),\ldots,v_n(\vec u))$.
Choose $\alpha(t)$ to be $F_1(a_1t,\ldots,a_nt) \in U_1$.
Then $\tfrac{d\alpha}{dt}=F_1(a_1,\ldots,a_n) \in V_1$.
So
\[ \Phi(\alpha(t)) = \Phi(a_1t,\ldots,a_nt) = (v_1(a_1t,\ldots,a_nt),\ldots,v_n(a_1t,\ldots,a_nt)) \]
which gives
\[ \frac{d\Phi(\alpha(t))}{dt} = \left( \sum_{j=1}^n \frac{\p v_1}{\p u_j}a_j, \ldots, \sum_{j=1}^n \frac{\p v_n}{\p u_j}a_j \right). \]
Note that $a_j$ is the $j$th component of $\tfrac{d\alpha}{dt}|_{t=0}$.
Therefore
\[ d\Phi(X) = \sum_{i=1}^n \left( \sum_{j=1}^n \frac{\p v_i}{\p u_j} a_j \right) \frac{\p}{\p v_j}. \]
At $p$, $(\tfrac{\p v_i}{\p u_j})$ is a fixed matrix and $X$ is related to $d\Phi(X)$ only by the $a_j$s.
Therefore only the vector $\alpha'(0)$ matters and all other information of $\alpha$ is irrelevant.
