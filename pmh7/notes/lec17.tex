\section{2018-10-02 Lecture}

3.2 Curvatures

Ricci curvature and scalar curvature

\begin{defn}[Scalar curvature]
  Define
  \begin{align*}
    \wt\Ric: T_pM &\to T_pM \\
    X \mapsto \wt\Ric(X)
  \end{align*}
  where $\wt\Ric(X)$ is defined by
  \[ \Ric(X,Y) = g(\wt\Ric(X),Y). \]
  Then the \textbf{scalar curvature is}
  \[ R = \tr\wt\Ric = \sum_{i=1}^n \left( \wt\Ric(e_i),e_i \right) = \sum_{i=1}^n \Ric(e_i,e_i) \]
  for an orthonormal basis $\{e_i\}_{i=1}^n$ of $T_pM$.
\end{defn}

\begin{rmk}
  In light of the symmetries of $\Rm$, the ``only'' way (up to a sign) to take traces of $\Rm$ are $\Ric$ and $R$.
  To see this, note that taking the trace just means substituting $e_i$ for any two places in $\Rm(X,Y,Z,W)$ and summing over $i$.
  The only non-trivial way to do so is to pick one in each set of $\{X,Y\}$ and $\{Z,W\}$, and up to a sign there is only one such way to do so.
  This gives $\Ric(Y,Z)$ from which point there is only one way to take the trace.
\end{rmk}

\begin{exam}
  Calculate $\Ric$ and $R$ for $S^2 \subset \RR^3$.
\end{exam}

This concludes section 3.
So we have all the tools we need now and we will apply them.

4. Variational approach (and applications)

Motivation: shortest distance

For $(M,g)$ and for all $p,q\in M$, let $\gamma:[a,b]\to M$ be such that $\{\gamma(a),\gamma(b)\}=\{p,q\}$.
Then the arc-length is
\[ \abs\gamma = \int_a^b \abs{\frac{d\gamma}{dt}}\, dt \]
where
\[ \abs{\frac{d\gamma}{dt}} = \sqrt{g\left(\frac{d\gamma}{dt},\frac{d\gamma}{dt}\right)}. \]

\begin{prop}
  $\abs\gamma$ depends only on the trajectory of the curve $\gamma([a,b])$.
\end{prop}

\begin{proof}
  \lv
  \begin{enum}
    \io
    Let $\mu$ be a orientation-preserving reparametrisation of $t$ where $\mu(a)=c$ and $\mu(b)=d$.
    Then
    \[ \abs{\gamma(\mu)} = \int_c^d \abs{\frac{d\gamma}{d\mu}} \, d\mu = \int_a^b \abs{\frac{d\gamma}{dt}\cdot\frac{dt}{d\mu}}\frac{d\mu}{dt}\, dt = \abs{\gamma(t)} \]
    since $\tfrac{dt}{d\mu}$ is always positive.

    \io
    Now let $\mu$ be an orientation-reversing reparametrisation of $t$ where $\mu(d)=a$ and $\mu(c)=b$ where $c<d$.
    Then
    \[ \abs{\gamma(\mu)} = \int_c^d \abs{\frac{d\gamma}{d\mu}} \, d\mu = - \int_b^a \abs{\frac{d\gamma}{dt}}\left(-\frac{dt}{d\mu}\right)\frac{d\mu}{dt}\ ,dt = \abs{\gamma(t)} \]
    since $\tfrac{dt}{d\mu}$ is always positive.
  \end{enum}
\end{proof}

\begin{rmk}
  Orientation does not matter for scalar quantities such as area. volume etc. but it does matter for vector quantities.
\end{rmk}

We want to find the shortest distance between two points.
From a ``differential'' point of view, it is a critical point of the length functional.

More precisely:
Let $\gamma(t)$ for $t\in[a,b]$ be the shortest distance curve.
A \textbf{smooth variation} (or smooth deformation) of $\gamma$ is a smooth family of curves $\gamma(t,s)$ for $(t,s) \in [a,b] \times (-\eps,\eps)$ such that $\gamma(t,0)=\gamma(t)$ for all $t$, $\gamma(a,s)=\gamma(a)$ for all $s$, and $\gamma(b,s)=\gamma(b)$ for all $s$.
Then
\[ \frac{d\abs{\gamma(s)}}{ds}\bigg\vert_{s=0}=0. \]

\begin{rmk}
  We do not know that a critical point is necessarily a minimum.
\end{rmk}

We now calculate.
Note for a fixed $t$, $\tfrac{\p}{\p s}$ is the covariant derivative along the curve $s \mapsto \gamma(t,s)$, hence it is the same as $\nabla_{\frac{\p\gamma}{\p s}}$.

Let $\gamma=\gamma(t,s)$.
\begin{align*}
  \frac{d\abs\gamma}{ds} &= \frac{d}{ds} \int_a^b \sqrt{\left( \frac{\p\gamma}{\p t},\frac{\p\gamma}{\p t} \right)} \, dt \\
  &= \int_a^b \frac{1}{2\sqrt{\left( \frac{\p\gamma}{\p t},\frac{\p\gamma}{\p t} \right)}} \frac{\p}{\p s} \left( \frac{\p\gamma}{\p t},\frac{\p\gamma}{\p t} \right) \, dt \\,
  &= \int_a^b \frac{1}{\sqrt{\left( \frac{\p\gamma}{\p t},\frac{\p\gamma}{\p t} \right)}} \left( \nabla_{\frac{\p\gamma}{\p s}} \frac{\p\gamma}{\p t},\frac{\p\gamma}{\p t} \right) \, dt
\end{align*}
where we have used the following identity:
\[ \nabla_{\frac{\p\gamma}{\p s}}\frac{\p\gamma}{\p t} - \nabla_{\frac{\p\gamma}{\p t}}\frac{\p\gamma}{\p s} = \left[ \frac{\p\gamma}{\p s}, \frac{\p\gamma}{\p t} \right] = 0. \]
Now $\tfrac{\p\gamma}{\p s}$ and $\tfrac{\p\gamma}{\p s}$ in general cannot be extended to a co-ordinate chart (for example the endpoints) so we cannot immediately apply to torsion-freeness.
We could also define an induced connection, but it is perhaps more transparent to compute directly.

\begin{enum}
  \io
  Let
  \[ \frac{\p\gamma}{\p s} = \sum_{i=1}^n \frac{\p\gamma_i}{\p s}\p_i \text{ and } \frac{\p\gamma}{\p t} = \sum_{i=1}^n \frac{\p\gamma_i}{\p t}\p_i \] 
  in a co-ordinate chart $\{u_i\}_{i=1}^n$, where we have
  \[ \gamma(t,s) = \left( \gamma_1(t,s),\ldots,\gamma_n(t,s) \right). \]
  Then compute the bracket in these co-ordinates.

  \io
  For a local function $F$ over $M$, we have
  \[ \left[ \frac{\p\gamma}{\p s}, \frac{\p\gamma}{\p t} \right]F = \frac{\p^2 F(\gamma(t,s))}{\p s\p t} - \frac{\p^2F(\gamma(t,s))}{\p t\p s} = 0. \]
\end{enum}
