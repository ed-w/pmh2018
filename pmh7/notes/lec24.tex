\section{2018-10-29 Lecture}

Let $\tfrac{\p t}{\p t}=\Delta_x u$ over a bounded set $U \subset \RR^n$ and $t\in[0,T)$ for $T<\infty$.

Here is a simple argument.
Consider the maximum of $u$ over $\ol U\times(0,T]$.
If it is in $\inte U\times(0,T)$, then $\Delta u\leq 0$ at that point.
Also for $t\neq0$ we have
\[ \frac{\p u}{\p t} = \lim_{h\to0+} \frac{u(x,t)-u(x,t-h)}{h} \geq 0. \]
This is almost a contradiction.
(A contradiction would imply that a maximum of $u$ must occur in $U \times \{0\} \cup \p U \times [0,T]$.)
How to make this a true contradiction?

Consider $u=\eps t$.
At the maximum of $u-\eps t$, if it is in $\inte U\times(0,T]$ then $\Delta(u-\eps t)=0$ and $\Delta u\leq0$.
Also $\tfrac{\p(u-\eps t)}{\p t} = \tfrac{\p u}{\p t}-\eps$, so $\tfrac{\p u}{\p t}\geq\eps$, so $0\geq\Delta u$.
This contradicts $\tfrac{\p t}{\p t}=\Delta_x u$.
Hence the maximum of $u-\eps t$ over $\ol U\times[0,T]$ occurs in $U \times \{0\} \cup \p U \times [0,T]$.
Then take $\eps\to0^+$.

In summary, for the heat equation $\tfrac{\p t}{\p t}=\Delta_x u$, the maximum principle argument justifies that the maximum of the solution $u$ is controlled (from above) by its initial and boundary values.

\begin{exer}
  Come up with a similar argument and result using the minimum principle.
\end{exer}

5.3 Ricci flow and mean curvature flow

Ricci flow

For a Riemannian manifold $(M,g)$, consider the evolution of $g=g(t)$ by the Ricci curvature:
\begin{equation*}
  \begin{cases}
    \ds \frac{\p g}{\p t} = -2\Ric(t), \\
    g(0)=g_0.
  \end{cases}
\end{equation*}

Explanation:

The metric $g$ and hence $\Ric$ depend on $t$.

Parabolicity

Let $p\in M$ and take normal co-ordinates $\{x_1,\ldots,x_n\}$ and $\p_i=e_i$ at $p$ at some time $t$.
We have the following classical theorem due to Riemann.

\begin{thm}
  With the above co-ordinates, we have the following Taylor expansion of $g$ at $p$:
  \[ g_{ij} = \delta_{ij} + \frac13 \sum_{kl} \Rm(e_i,e_k,e_j,e_k)x_kx_l + O \left(\abs x^3 \right). \]
\end{thm}

How does this result explain parabolicity?

We want $\tfrac{\p g}{\p t}=-2\Ric(g)$ to give
\[ \frac{\p g_{ij}}{\p t} = \Delta g_{ij}. \]
Then we need
\[ \Delta g_{ij} = \sum_{k} \p_k\p_kg_{ij} = \frac23 \sum_k \Rm(e_i,e_k,e_j,e_k) = -\frac23\Ric(e_i,e_j). \]
Therefore
\[ \frac{\p g_{ij}}{\p t} = -2\Ric_{ij} = 3\Delta g_{ij}. \]
However this is only at $p$ and at the time at which we take the normal co-ordinates with respect to $g(t)$.
This is not the equation to consider to rigorously study this flow because it is only valid at one point in spacetime (at least by our own simple equation).
In real life, Ricci flow is \emph{weakly parabolic} and we still have uniqueness and short-time existence of solutions at least when $M$ is compact.

\begin{exer}
  Calculate the Ricci flow starting from the round sphere.
  Hint: assume that it stays as a round sphere (of evolving radius) and apply the uniqueness principle.
\end{exer}

Mean curvature flow

For a hypersurface (a manifold of codimension $1$) $\Sigma^m$ in $\RR^{n+1}$, set $F_0:\Sigma^n\to\RR^{n+1}$ to be the immersion.
Now consider the evolution of the immersion $F: \Sigma^n\to\RR^{n+1}$:
\[ \frac{\p F}{\p t} = -H\nu \]
where $\nu$ is the normal vector field and $H$ is the mean curvature for the (evolving) hypersurface.
Note that the orientation does not affect $H\nu$ since $\nu\mapsto-\nu$ forces $H\mapsto-H$.

Intuition: we can take $F_0$ as the identity map of the initial surface.
Then under mean curvature flow, every point gets moved to a different position in $\RR^{n+1}$.

Parabolicity of MCF.

Let $p\in F(\Sigma)$ and let $\{x_i\}$ be normal co-ordinates at $p$ with $\p_i=e_i=\p_iFo$.
(Recall that $\Sigma$ is a surface.)
Then computing pointwise, we have
\begin{align*}
  \Delta F &= \sum_{i=1}^n e_ie_iF \\
  &= \sum_{i=1}^n \nabla_{e_i}e_i \\
  &= \sum_{i=1}^n \ol\nabla_{e_i}e_i && \text{the Euclidean connection}  \\
  &= \sum_{i=1}^n \left( \nabla_{e_i}e_i,\nu \right)\nu && \text{since the $e_i$ vanish in the tangential direction} \\
  &= -H\nu
\end{align*}
But both $\Delta G$ and $H\nu$ are globally defined quantities, hence this equation is true globally.
So $\tfrac{\p F}{\p t}=\Delta F$.
