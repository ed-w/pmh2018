\section{2018-08-13 Lecture}

\begin{rmk}[2.3]
  These two definitions are ``abstract'' because one starts with a topological space.
  However, there exists a classical result by Whitney relating it to the earlier definition of regular surfaces in $\RR^N$.
  Roughly speaking, any manifold can be realised as a surface in $\RR^N$ for some $N$, that is, as a submanifold of $\RR^N$ for some $N$.
\end{rmk}

\begin{exer}
  Prove the last claim of the remark (submanifold).
\end{exer}

\begin{exer}[2.4]
  $\RR^n$ is a manifold with the change of variables in multivariate calculus as the transition functions.
  
  Warning: using polar co-ordinates the open set has to exclude the origin.
\end{exer}

\begin{exam}
  Examples of submanifolds:
  \begin{itm}
    \io $\RR^n \subseteq \RR^m$ for $n \leq m$.
    \io The graph of a smooth $\RR^{n-m}$ valued function over $\RR^m$ is a submanifold of $\RR^n$.

    Let $(x,y) \in \RR^n \times \RR^{m-n} = \RR^n$.
    The graph of $f$ is the set of points $(x,f(x))$.
    Consider $(x,y-f(x))$ as the new co-ordinates for $\RR^n$.
    We need to check that the Jacobian is invertible.
    \begin{equation*}
      \frac{\p(x,y-f(x))}{\p(x,y)}=
      \begin{bmatrix}
	\frac{\p x}{\p x} & \frac{\p x}{\p y} \\
	\frac{\p(y-f(x))}{\p x} & \frac{\p(y-f(x))}{\p y}
      \end{bmatrix}
      =
      \begin{bmatrix}
	I_m & 0 \\
	\frac{\p f}{\p x} & I_{n-m}
      \end{bmatrix}
    \end{equation*}
    Clearly the transition function has inverse $(x,y) \mapsto (x,y+f(x))$.
    So it is indeed a co-ordinate chart.
    Then the graph $y=f(x)$ corresponds to the subset of points of the form $(x,0)$ for the new co-ordinate chart.
  \end{itm}
\end{exam}

\begin{exer}[2.7]
  Prove that a submanifold as defined in definition 2.2 is equivalent to being locally a level surface, that is, the zero locus set of $\{f_i\}_{i=1}^k$ where the Jacobian of the $f_i$s with respect to \emph{local} co-ordinates of the ambient manifold is of full rank.
\end{exer}

\begin{proof}
  Let $S$ be a submanifold inside a manifold $M$.
  To show that a property holds locally we need to check it on some co-ordinate chart about every $p \in S$.
  Let $F: S \cap U \to \RR^m \cap V \subset \RR^n$ be the chart.
  The direction $\implies$ is trivial since $S$ is defined locally by the vanishing of the last $n-m$ co-ordinates.
  So if our co-ordinates are $(x,y) \in \RR^m \times \RR^{n-m} = \RR^n$, we can take $\{f_i\}=y$.
  To check that it is full rank, we compute
  \begin{equation*}
    \frac{\p(f_i)}{\p(x,y)}=
    \begin{bmatrix}
      0 & I_{m-n}
    \end{bmatrix}_{(n-m) \times n}
  \end{equation*}
  Now we prove the $\impliedby$ direction.
  Let
  \[ U \cap S = \{ f_1 = \cdots = f_{n-m} = 0 \}. \]
  We have a local chart $V \subset \RR^n$ with local co-ordinates $(x,y)$.
  Then
  \begin{equation*}
    \frac{\p(f)}{\p(x,y)}=
    \begin{bmatrix}
      \frac{\p f}{\p x} & \frac{\p f}{\p y}
    \end{bmatrix}_{(n-m) \times n}
  \end{equation*}
  is of full rank, so we can assume that $\p f/\p y$ is an invertible $(n-m) \times (n-m)$ matrix (by rearranging co-ordinates).
  This is how we get $y$.
  Then we can use the co-ordinate map $(x,y) \mapsto (x,f(x,y))$.
  By a similar calculation to the previous example this map has invertible Jacobian, so we can apply the inverse function theorem to show that there is some (possible smaller) open set $U'$ such that $(x,y) \mapsto (x,f)$ is a change of co-ordinates.
  Then $f=0$ defines the submanifold in the co-ordinates $(x,f)$.
\end{proof}

\begin{rmk}
  A graph is defined by $y=f(x)$ and a level surface by $y-f(x)=0$.
\end{rmk}

\begin{defn}
  Two manifolds $M_1$ and $M_2$ are \textbf{diffeomorphic} if there are smooth maps $f: M_1 \to M_2$ and $g: M_2 \to M_1$ such that $f \circ g = \id_{M_2}$ and $g \circ f = \id_{M_1}$.
  In this case $g=f\inv$ and $f=g\inv$ and they are \textbf{diffeomorphisms}.
\end{defn}

\begin{rmk}
  If $F_1: U_1 \to V_1$ and $F_2: U_2 \to V_2$ are co-ordinate charts for $M_1$ and $M_2$ respectively, then $f$ is defined to be smooth when $F_2 \circ f \circ F_1\inv$ is smooth.
  This is independent on the choice of charts.
  (Here we define the co-ordinate maps from a subset of the manifold to a subset of Euclidean space).
\end{rmk}

\begin{rmk}
  If $M_1$ and $M_2$ are diffeomorphic then they are the same smooth manifold.
  The diffeomorphism $f: M_1 \to M_2$ translates all of the charts for $M_2$ to $M_1$ and they are all compatible.
\end{rmk}

\begin{rmk}[Gluing]
  We can define our manifold structure as a disjoint union of open sets modulo some equivalence relation (gluing the intersections) subject to some comptability condition.
  We can add in as many compatible charts as possible (the maximal atlas).
  So two compatible atlases give the same manifold structure.
  Diffeomorphisms between manifolds are basically a change of variables on the same manifold.
\end{rmk}
