\section{2018-07-30 Lecture}

\begin{itm}
  \io Consultation, backup lecture or tutorial Friday 11--12
  \io 2 assignments, 20\% each and the final exam is worth 60\%
\end{itm}

Course plan:
\begin{itm}
  \io Basic differential geometry
  \io Introduction to Riemannian geometry
  \io Applications to Ricci flow and mean curvature flow at the end
\end{itm}

In MATH3968 we studied curves and surfaces wich are a classical version of what we are going to study.
In the first week or two we will briefly recap curves and surfaces.

Section 1. Recap of Euclidean surfaces (surfaces in $\RR^3$).

Subsection 1.1

\begin{defn}[1.1]
  A \textbf{general regular surface} is a surface $S \cong \RR^N$ such that for each $p \in S$ there is an open neighbourhood $U \subset \RR^N$ such that there is a map $F: V \to S \cap U$ with domain $V \subset \RR^n$ which is homeomorphic and smooth when considered as $F: V \to \RR^N$ and of rank $n$ at each point in $V$.
  That is, the rank of the Jacobian (or differential) $(dF)(q)$ is equal to $n$ for any $q \in V$.
\end{defn}

\begin{rmk}
  Why do we need the condition on the rank?
  Consider the map
  \begin{align*}
    f: [0,\infty) \&\to [0,\infty) \\
    x &\mapsto x^2.
  \end{align*}
  This is a homeomorphism.
  However we have $(df)(x)=2x$ which is not rank $1$ at $0$.
  The inverse map $y \mapsto \sqrt{y}$ has undefined differential at $y=0$.
  So how we view the ray $[0,\infty)$ will be affected by the choice of the chart $f$.

  Here is another example:
  \begin{align*}
    \CC &\to \CC \\
    z &\mapsto z^2
  \end{align*}
\end{rmk}

\begin{rmk}
  We can consider $F$ as a map from $\RR^n$ to $\RR^N$:
  \[ F = \left( F_1(x_1,\ldots,x_n),\ldots,F_N(x_1,\ldots,x_n) \right). \]
  Then the differential is
  \[ dF = \left( \frac{\p F_i}{\p x_j} \right)_{\substack{1 \leq i \leq N \\ 1 \leq j \leq n}}. \]
  We call $F$ a \textbf{co-ordinate chart}.
\end{rmk}

\begin{exam}[1.3]
  Consider the standard unit sphere $S^n \subset \RR^{n+1}$:
  \[ S^n = \left\{ (x_1,\ldots,x_{n+1}) \in \RR^{n+1} \mid x_1^2 + \cdots + x_{n+1}^2 = 1 \right\}. \]
  How can we cover it with charts?
\end{exam}

\begin{rmk}
  What makes a good definition?
  \begin{enum}
    \io The existence of examples
    \io Robustness
  \end{enum}
  In the case of definition 1.1: if $S$ is a regular surface then how rigid is the choice of chart?
  Recall that for each $p \in S$ we choose an open neighbourhood $U$ and a map $F: V \to U \cap S$ which is homeomorphic, smooth and of full rank.
  We don't want our definition to be too restrictive.

  It turns out that the choice of charts is fine as long as the map $F: V \to U \cap S$ is homeomorphic, smooth and of full rank.
\end{rmk}

\begin{defn}[informal]
  Let $S \subset \RR^N$ be a regular surface.
  A \textbf{tangent vector} at $p \in S$ is a vector in $\RR^N$ based at $p$ which is ``tangent'' to $S$.

  What does tangent mean?
  It is a velocity vector at $p$ of a motion (i.e.\@ a parametrised curve) on $S$ through $p$.
  A curve on $S$ can be written as $F(\alpha(t))$ where $F: V \to U \cap S$ is a co-ordinate chart and $\alpha: (-\eps,\eps) \to V$ is a curve in $V \subset \RR^n$ (an open subset).
  Then the tangent vector is:
  \[ \frac{d}{dt} \left( F \circ \alpha \right) (t) \Big|_{t=0} \]

  The set of all tangent vectors of $S$ at $p$ forms a vector space $T_pS$ which is known as the \textbf{tangent space}.
  (Think about this statement.
  Hint: use charts.)

  There is a ``natural'' ``smooth'' structure for the set of all tangent vectors over $S$ (for all points in $S$) called the \textbf{tangent bundle}.
  It can be locally understood using charts.
  \begin{align*}
    F: V &\to S \cap U \\
    \bo u = (u_1,\ldots,u_n) &\mapsto \left( F_1(\bo u),\ldots,F_N(\bo u) \right)
  \end{align*}
  with $(0,\ldots,0) \mapsto p$.
  Consider the co-ordinate curves ($u_i$-curves) for $i=1,\ldots,n$:
  \[ F\left( 
      ( \underset{1}{0}, \ldots, \underset{i-1}{0}, \underset{i}{t}, \underset{i+1}{0}, \ldots, \underset{n}{0} )
    \right)
  \]
  for $t \in (-\eps,\eps)$.
  
  Then $\frac{\p F}{\p u_i}$
\end{defn}

\begin{rmk}
  We can assume that $F(0)=p$ for our chart.
  This is an equivalent definition (because we can translate and compose with the translation).
\end{rmk}
