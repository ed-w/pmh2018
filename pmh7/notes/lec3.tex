\section{2018-08-06 Lecture}

Check that the two fundamental forms are smooth and symmetric.

It is clear that $I$ is symmetric since $I_p(X,Y) = \ang{X,y} = \ang{Y,X} = I_p(Y,X)$.
To check smoothness, take a co-ordinate chart containing $p$:
\begin{align*}
  F: V &\to U \cap S \\
\vec u = (u_1,\ldots,u_n) &\mapsto \left( F_1(u),\ldots,F_N(u) \right)
\end{align*}
for all $p \in U \cap S$.
By abuse of notation, we will also say $p \in V$.
Then
\[ I_p \left( \frac{\p F}{\p u_i}, \frac{\p F}{\p u_j} \right) = \ang{ \frac{\p F}{\p u_i}, \frac{\p F}{\p u_j} } = \sum_{k=1}^N \frac{\p F_k}{\p u_i} \frac{\p F_k}{\p u_j}. \]
Here $S$ is regular, so $F$  is a smooth map $V \to U \cap S \injto \RR^N$, so $I_p(\p_{u_i},\p_{u_j})$ is smooth as a function over $v$ for all $i$ and $j$.

To check that $II$ is symmetric, we compute.
Extend $X, Y \in T_pS$ to a tangent vector field near $p$.
For example, if
\[ X = \sum_{i=1}^n a_i \frac{\p}{\p u_i} \bigg\vert_p, \]
we can can have
\[ X = \sum_{i=1}^n X_i \frac{\p}{\p u_i} \]
over $V$ (or $S \cap U$), where the $X_i$ are smooth functions with $X_i(p) = a_i$.
Likewise if
\[ Y = \sum_{i=1}^n b_i \frac{\p}{\p u_i} \bigg\vert_p, \]
let
\[ Y = \sum_{i=1}^n Y_i \frac{\p}{\p u_i} \]
over $V$ where $Y_i$ is smooth with $Y_i(p)=b_i$ for all $i$.

Now by the product rule, we have
\[ II_p(X,Y) = \ang{D_X \bar\nu, Y} = D_X \ang{\bar\nu, \bar Y} - \ang{\bar\nu, D_X\bar Y} \]
and
\[ II_p(Y,X) = \ang{D_Y \bar\nu, X} = D_Y \ang{\bar\nu, \bar X} - \ang{\bar\nu, D_Y\bar X} \]
So we need to take derivatives of $X$ and $Y$, hence we need a smooth extension near $p$.
Note that the inner product of a tangent vector and a normal vector is zero.
So we want to show
\[ \ang{\bar\nu, D_X Y} = \ang{\bar\nu, D_Y X} \]
or equivalently 
\[ D_XY - D_YX \in T_pS. \]
We now calculate.
\begin{align*}
  (D_XY)(F) &= \left( D_{ \sum_{i=1}^n X_i \frac{\p}{\p u_i} } \sum_{j=1}^n \frac{\p}{\p u_j} \right)(F) \\
  &= \sum_{i=1}^n \sum_{j=1}^n X_i \frac{\p}{\p u_i} Y_j \frac{\p F}{\p u_j} \\
  &= \sum_{i,j} X_i \frac{\p Y_j}{\p u_i} + \sum_{i,j} X_i Y_j \frac{\p^2 F}{\p u_i \p u_j}
\end{align*}
Since $F$ is smooth, we can swap the order of partial derivatives.
Hence $D_XY-D_YX$ is a linear combination of the $\p/\p u_i$, hence is in $T+pS$.

\begin{exer}
  Show that $II_p$ is smooth.
\end{exer}

\begin{defn}
  \lv
  \begin{enum}
    \io
    A \textbf{principal curvature} is an eigenvalue of $II_p$ when considered as a symmetric matrix with respect to a basis of $T_pS$ which is orthonormal with respect to $I_p$.
    \[ \left( II_p(e_i,e_j) \right)_{ij} \qquad \{e_i \in T_pS\}_{i=1}^n \qquad I_p(e_i,e_j) = \delta_{ij} \]
    This is independent of the choice of basis.
    Alternatively, we have the \textbf{shape operator}
    \[ A_p: X \in T_pS \mapsto D_X \bar\nu \in T_pS. \]

    \begin{exer}
      Show that $D_X \bar\nu \in T_pS$.
    \end{exer}
    
    This is also called the second fundamental form:
    \[ II_p(X,Y) = \ang{A_p(X),Y} \]
    and is symmetric (as in $II$).
    $A_p$ is also called the principal curvature.
    A \textbf{principal direction} is an eigenvector of $A_p: T_pS \to T_pS$.

    \io
    The \textbf{mean curvature} is $\tr A_p$ (i.e.\@ the sum of the principal curvatures).

    \io
    The \textbf{Gauss curvature} is $\det A_p$ (i.e.\@ the product of the principal curvatures).
  \end{enum}
\end{defn}

Let us generalise $II$ to a general surface $S \subset \RR^N$ where we do not require $N=n+1$.
Define
\begin{align*}
  II_p: T_pS \times T_pS \times N_pS &\to \RR \\
  (X,Y,\nu) &\mapsto \ang{D_X\bar\nu,Y}(p)
\end{align*}
We need to show that different extensions of  $\nu$ to a normal vector field near $p$ give the same outcome.

\begin{exer}
  Show this.
  There is a hint in the notes.
  Show that
  \[ II_p(X,Y,\nu) = -\ang{\bar\nu, D_XY} \]
  using the same idea as in proving that $II_p$ is symmetric.
  This shows that only the value of $\nu$ at $p$ matters.
\end{exer}

So 
\[ II_p(X,Y) = -\proj_{NS} D_XY(p) \in N_p(S) \]
and
\[ D_XY = \nabla_XY - II(X,Y). \]

1.3 Classical attractions

