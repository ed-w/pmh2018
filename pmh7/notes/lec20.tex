\section{2018-10-15 Lecture}

Note that
\[ P\left( \exp_p(v) \right) = d(\exp_p)_v(v) = \frac{d\exp_p(tv)}{dt}\bigg\vert_{t=1} = \frac{d\gamma(v,t)}{dt}\bigg\vert_{t=1} = \frac{d\gamma(v/\abs{v},\abs{v}t)}{dt}\bigg\vert_{t=1} = \frac{d\gamma(v/\abs{v},s)}{dt}\bigg\vert_{s=\abs{v}}\cdot\abs{v} = \abs{v} \]
since the length of $v/\abs{v}$ is $1$ for $v,v/\abs{v}\in T_pM$.
So $s$ is the arc-length parameter.
Hence
\[ \abs{P(\exp_p(v))}_{\exp_p(V)} = \abs{v}_{T_pM} \]
where the length is computed using $g$ at the points given in the subscripts.
This can also be seen by considering
\[ \abs{P(x^1,\ldots,x_n)}=\abs{\sum_{i=1}^nx^i\p_i}=\sqrt{(x^1)^2+\cdots+(x^n)^2}. \]
Note however that this is \textbf{NOT} because $\{\p_i\}$ is orthornormal!

Define $r=P/\abs{P}$ in a chart around $p$ (but excluding the point $p$ itself).
Clearly $\abs{r}=1$.
From the above computation, we have
\[ r\left( \exp_p(v) \right) = \frac{P(\exp_pv)}{\abs{P(\exp_pv)}} = \frac{\abs{v}}{\abs{P(\exp_pv)}} = \frac{d\gamma(v/\abs{v}),s)}{ds}\bigg\vert_{s=\abs{v}}. \]
So in $T_pM$, $r$ is just the unit radial direction and so, after translation to $M$ through $\exp_p$, is the velocity field of all the unit speed geodesics starting from $p$.
Hence $\nabla_rr=0$ everywhere except at $p$ (this is the geodesic equation).

Now we justify the simplification $\nabla_XY=0$ at $p$.
We will show that at $p$, we have $\nabla_{\p_i}\p_j=0$ for all $i,j=1,\ldots,n$.
Consider a normal co-ordinate chart $(x^1,\ldots,x^n)$.
Then $\alpha(t)=(t,0,\ldots,0)$ is a geodesic starting at $p$ in the $\p_1$-direction.
Then $\nabla_{\p_1}\p_1=0$ for $t\neq0$.
By continuity, we must also have $\nabla_{\p_1}\p_1=\nabla_rr=0$ at $t=0$, that is, at $p$.
Likewise, we have $\nabla_{\p_i}\p_i=0$ at $p$ for all $i=1,\ldots,n$.
Now consider the curve $(t,t,0,\ldots,0)$ for $t>0$.
Then $r=(\p_1+\p_2)/\sqrt{2}$.
So
\[ \nabla_{\p_1+\p_2}(\p_1+\p_2)=2\nabla_rr=0 \]
for $t>0$, hence by continuity for $t=0$ as well.
Then computing at $p$, we have
\[ 0 = \nabla_{\p_1+\p_2}(\p_1+\p_2) = \nabla_{\p_1}p_2 + \nabla_{\p_2}\p_1. \]
By torsion-freeness, we have
\[ \nabla_{\p_1}\p_2-\nabla_{\p_2}\p_1=[\p_1,\p_2]=0 \]
since the Lie bracket is zero.
Hence $\nabla_{\p_2}\p_1=\nabla_{\p_1}\p_2=0$ at $p$, and similarly for the other index pairs.
So $\nabla_{\p_i}\p_j=0$ at $p$ for all $i,j=1,\ldots,n$.

For a fixed vector $X\in T_pM$, we can extend $X=X^i\p_i$ locally with constant coefficients: $X(p)=X^i(p)\p_i$ where $X^i(p)=X^i$ for all $p$ in the chart.
Then
\[ \nabla_XY= X^iY^i\nabla_{\p_i}\p_j=0. \]

4.3 Jacobi field

When we study the variations of curves we get geodesics.
When we study the variations of geodesics we get Jacobi fields.
How does the ``distance'' vary when we perturb the geodesic at the points of interest?
Let $\gamma(s,t)$ be a family of geodesics parametrised by $s$.
Then
\[ \nabla_{\frac{d\gamma}{dt}}\frac{d\gamma}{dt} = 0 \]
by the geodesic equation, and since this is zero for all $s$, we also have
\[ \nabla_{\frac{d\gamma}{ds}} \nabla_{\frac{d\gamma}{dt}}\frac{d\gamma}{dt} = 0. \]

Let us make this more explicit.
Using
\[ \left[ \frac{\p\gamma}{\p s}, \frac{\p\gamma}{\p t} \right] = 0,\]
we have
\begin{align*}
  0 &= \nabla_{\frac{\p\gamma}{\p s}} \nabla_{\frac{\p\gamma}{\p t}}\frac{\p \gamma}{\p t} \\
  &= \Rm\left( \frac{\p \gamma}{\p s}, \frac{\p \gamma}{\p t} \right) \frac{\p \gamma}{\p t} + \nabla_{\frac{\p\gamma}{\p t}} \nabla_{\frac{\p\gamma}{\p s}} \frac{\p \gamma}{\p t} \\
  &= \Rm\left( \frac{\p \gamma}{\p s}, \frac{\p \gamma}{\p t} \right) \frac{\p \gamma}{\p t} + \nabla_{\frac{\p\gamma}{\p t}} \nabla_{\frac{\p\gamma}{\p t}} \frac{\p \gamma}{\p s}.
\end{align*}
Now set $X=\tfrac{\p\gamma}{\p s}$ and restrict to $s=0$.
This gives
\[ 0 = \Rm\left( X, \frac{\p \gamma}{\p t} \right) \frac{\p \gamma}{\p t} + \nabla_{\frac{\p\gamma}{\p t}} \nabla_{\frac{\p\gamma}{\p t}} X.\]
This is known as the \textbf{Jacobi equation}, and a solution $X$ is known as a \textbf{Jacobi field}.
So given a variation of geodesics, we get that the Jacobi field
\[ X = \frac{\p\gamma}{\p s}\bigg\vert|_{s=0} \]
satisfies the Jacobi field.

Now we want to construct the geodesic variation family satisfying the initial values of $X$.
Note that geodesics are much more rigid than any curves; we cannot just perturb them arbitrarily.
So it suffices to perturb just at a point instead of along the whole curve.
We will write down a local expression (in co-ordinates) for the Jacobi equation and then appeal to general ODE theory.

Let $\gamma:[a,b]\to M$ be a geodesic.
Pick an orthonormal basis $\{e_i\}_{i=1}^n$ of the tangent space at $\gamma(a)$ and parallel translate along $\gamma$.
Now set $X=X^ie_i$.
Then substituting this in we get
\[ \ddot{X}^i e_i + \left( \Rm\left( e_i, \frac{d\gamma}{dt} \right)\frac{d\gamma}{dt}, e_j \right)X^ie_j = 0. \]
Then taking co-efficients of each $e_j$ gives
\[ \ddot{X}^j + \left( \Rm\left( e_i, \frac{d\gamma}{dt} \right)\frac{d\gamma}{dt}, e_j \right)X^i = 0. \]
This gives a system of ODEs with uniqueness and existence of solutions coming from the general theory.
