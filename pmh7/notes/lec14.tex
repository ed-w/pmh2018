\section{2018-09-11 Lecture}

We have defined the Levi-Civita connection $\nabla$ for $TM$ and for $T^*M$.
We can extend it to any tensor bundle by $\nabla(a \otimes b) = (\nabla a)\otimes b + a \otimes (\nabla b)$.

Here is another way of viewing a connection $\nabla$.
Originally, we defined
\begin{align*}
  \nabla: \Gamma(M,TM) \times \Gamma(M,TM) &\to \Gamma(M,TM) \\
  (X,Y) &\mapsto \nabla_XY.
\end{align*}
Equivalently, we can have
\begin{align*}
  \nabla: \Gamma(M,TM) &\to \Gamma(M,T^*M \otimes TM) \\
  Y &\mapsto (\nabla Y)(\square) = \nabla_\square Y
\end{align*}
because $\nabla$ is tensorial (pointwise) in the first component.
Here $(\nabla Y)(X) = \ang{\nabla Y,X}$ is the dual action of $T^*M$ on $TM$. % check this
We will write the additional argument as the first component. % check this

3.2 Curvatures

Motivation:
\begin{enum}
  \io
  $\nabla$ is differentiation which is not a pointwise quantity.

  \io
  Classical differential geometry: if $C \subset \RR^3$ is a curve, then $\kappa$ and $\tau$ (based on $\tfrac{d^2\alpha}{ds^2}$) determines $C$.
  
  \io
  In classical differential geometry, the Gauss curvature is defined by the first fundamental form which is generalised to the Riemannian metric.
\end{enum}

This motivates the Riemannian curvature.

\begin{defn}[3.12]
  For a Riemannian manifold $(M,g)$, we have the following composition of $\nabla$ for different bundles:
  \[ \nabla^2: \Gamma(M,TM) \xto{\nabla \text{ for } TM} \Gamma(M,T^*M \otimes TM) \xto{\nabla \text{ for } T^*M \otimes TM} \Gamma(M,T^*M\otimes T^*M\otimes TM) \]
  Namely, for $X,Y,Z\in\Gamma(M,TM)$, we have
  \[ \nabla^2(X,Y,Z) = \nabla_X\nabla_Y Z - \nabla_{\nabla_XY}Z. \]
  Then we define the \textbf{Riemannian curvature} $\Rm$ of $(M,g)$ by
  \[ \Rm(X,Y)Z = \left[ \nabla^2(X,Y) - \nabla^2(Y,X) \right]Z = \nabla^2(X,Y,Z) - \nabla^2(Y,X,Z) \]
  which is a tensor
  \[ \Rm \in \Gamma(M,\overbrace{T^*M}^X\otimes \overbrace{T^*M}^Y\otimes \overbrace{T^*M}^Z\otimes TM). \]
\end{defn}

\begin{rmk}
  What we really mean by $\nabla^2(X,Y,Z)$ is the tensor field $\nabla^2Z=\nabla\nabla Z$ applied to the arguments $(X,Y)$, where the innermost argument is used first.
  Then the above computation for $\nabla^2(X,Y,Z)$ is just the Leibniz rule.\:
  \[ (\nabla^2 Z)(X,Y) = (\nabla_X(\nabla Z))(Y) = \nabla_X( (\nabla_Z)Y) - (\nabla Z)(\nabla_XY) = \nabla_X\nabla_YZ - \nabla_{\nabla_XY}Z. \]
\end{rmk}

Justification of this definition:
\begin{enum}
  \io
  By the Leibniz rule, we have
  \[ \nabla_X(\nabla_YZ) = (\nabla_X\nabla Z)Y + \nabla_{\nabla_XY}Z \]
  so rearranging gives us the right formula for $\nabla^2(X,Y,Z)$.

  \io
  We need to check that $\Rm$ is a tensor, that is it is $\cC^\infty$-linear in all three arguments.
  See the notes for this (but we did do it in class).
  So far we have only used the torsion-free property of $\nabla$.
  (So in principle we can define a curvature tensor for any torsion-free metric).
  Note also that $\nabla^2(X,Y)(Z)$ alone is not a tensor (we need to take the anti-commutator).
\end{enum}

Now we investigate symmetries of the Riemannian curvature tensor $\Rm$.
There is an equivalent way to express this tensor $\Rm(X,Y)(Z)\in\Gamma(M,TM)$:
\[ \Rm(X,Y,Z,W) = g(\Rm(X,Y)Z,W) \]
which is equivalent by nondegeneracy of $g$.
This allows us to write down the symmetries of $\Rm$ in a simpler way.

We have the following symmetry results:
\begin{prop}
  \lv
  \begin{enum}
    \io $\Rm(X,Y,Z,W) = -\Rm(Y,X,Z,W)$.
    \io $\Rm(X,Y,Z,W) = -\Rm(X,Y,W,Z)$ (this is where $\nabla g=0$ is used).
  \end{enum}
\end{prop}

\begin{proof}
  The first is left as an exercise.
  The second is in the notes (but we did it in class).
\end{proof}

\begin{rmk}
  Here is an equivalent definition of $\Rm$:
  \[ \Rm(X,Y)(Z) = \nabla_X \nabla_Y Z - \nabla_Y \nabla_X Z - \nabla_{[X,Y]}Z. \]
\end{rmk}

We also have the First Bianchi Identity:
\begin{prop}
  \[ \Rm(X,Y,Z,W) + \Rm(Y,Z,X,W) + \Rm(Z,X,Y,W) = 0 \]
\end{prop}

