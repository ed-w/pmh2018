\section{2018-03-05 Lecture}

Motivation for functional analysis: the heat equation (section 1.2.1 of Einsiedler/Ward Functional Analysis).
(This lecture is not part of the course proper.)

\begin{defn}
	Let $U$ be open in $\RR^d$.
	Let $u: U \times \RR \to \RR, (x,t) \mapsto u(x,t)$ be the temperature distribution.
	The heat equation is
	\begin{eqn}\label{eq:heat}
		\frac{\p u}{\p t} = c\Delta_x u
	\end{eqn}
	where
	\[\Delta_x = \frac{\p^2 u}{\p x_1^2} + \cdots + \frac{\p^2 u}{\p x_d^2}\]
	is the Laplace operator.
\end{defn}

Why is it the heat equation?

\begin{prop}
	Let $U \subseteq \RR^2$ be open and suppose $f: U \to \RR$ is a $C^2$-function.
	Then
	\begin{eqn}
		\lim_{r \to 0} \frac{1}{r^2 \vol(B_r(x))} \int_{B_r(x)} (f(y)-f(x)) \, dy = c \Delta f(x)
	\end{eqn}
	for any $x \in U$, where $B_r(x)$ is the ball of radius $r$ around $x$, where the integral is with respect to the Lebesgue measure and where $c = \frac 1{2(d+2)}$.
\end{prop}

How do we solve \cref{eq:heat}?

Suppose $u|_{\p V}=b$ is time independent. From physical intuition, the temperature distribution becomes time-independent in the long run.
Then by \cref{eq:heat} the equilibrium solution satisfies
\begin{eqn}\label{eq:dirichlet}
	\begin{cases}
		\Delta u = 0 \\
		u|_{\p u} = b
	\end{cases}
\end{eqn}
The equation $\Delta u=0$ is called the Laplace equation and its solutions are called harmonic functions.
The boundary value problem \cref{eq:dirichlet} is called the Dirichlet problem.

Let us attempt to find solutions by separating variables.
\begin{align*}
	u(x,t) &= F(x)G(t) \\
	F(x)G'(t) &= x(\Delta F(x))G(t) \\
	\frac{G'(t)}{G(t)} &= c \frac{\Delta F(x)}{F(x)}
\end{align*}
By scaling the time we can assume $c=1$.
Then

\[\frac{G'(t)}{G(t)} = \frac{\Delta F(x)}{F(x)}\]

Note that if $G(t) = e^{\lambda t}$ and $\Delta F = \lambda F$ then $F(x)G(t)$ solves the above equation.

To solve this we need the spectral theory of the Laplace operator.

\begin{prop}
	Every sufficiently nice function $f: U \to \RR$ can be decomposed in to a sum $f=\sum_n F_n$ of functions $F_n: \ol U \to \RR$ such that
	\begin{itm}
		\item $\Delta F_n = \lambda_n F_n$ for some $\lambda_n < 0$
		\item $F_n|_{\p U} = 0$
	\end{itm}
	The $F_n$ are eigenfunctions of $\Delta$ and $\lambda_n$ their corresponding eigenvalues.
	Then
	\begin{eqn}
		(x,t) = \sum_n F_n(x) e^{\lambda_n t}
	\end{eqn}
	with boundary values
	\begin{eqn}
		\begin{cases}
			u|_{\p U \times \{t\}}=0 \text{ for all } $t$ \\
			u|_{\p U \times \{0\}}=f
		\end{cases}
	\end{eqn}
\end{prop}

A theme of this subject is to find which operators have eigenfunction decompositions.

We will only talk about continuous linear transformations.
