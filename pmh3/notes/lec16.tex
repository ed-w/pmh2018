\section{2018-05-01 Lecture}

Last time we proved the codimension 1 version of the real Hahn-Banach theorem.
We now prove the general case of the Hahn-Banach theorem over the reals using Zorn's Lemma.

\begin{proof}
  Let $L$ be the set of all linear extensions $f:Z \to \RR$ of $\ph$ with $Y \subseteq Z \subseteq \RR$ satisfyong $f(z) \leq p(z)$ for all $z \in Z$.
  Then define a partial order on $L$ by setting $f_1 \preceq f_2$ if the domain of $f_1$ is a subset of the domain of $f_2$ and the restriction of $f_2$ to the domain of $f_1$ is $f_1$.
  $L$ is non-empty since it contains $\ph$.
  Let $C \subseteq L$ be a chain.
  Now define $f_C$ to be the function defined on the union of the domains of $f_c$ for all $c \in C$ and which restricts to the each of the $f_c$ on their domains.
  It is clear that this function is well defined.
  Then $f_C \in L$, so every chain has an upper bound.
  Then by Zorn's lemma there exists a maximal element $\ol\ph \in L$.

  We claim that the domain of $\ol\ph$ is $X$.
  If it is not, then there exists a $z \in X$ not in the domain of $\ol\ph$.
  Then by the codimension 1 version of the Hahn-Banach theorem there exists a linear extension $\wt\ph$ of $\ol\ph$ to the domian of $\ol\ph$ direct sum with $\RR z$ which is also dominated by $p$.
  Then $\ol\ph \prec \wt\ph$, contradicting the maximality of $\ol\ph$.
\end{proof}

\begin{cor}
  If $(X,\norm{\cdot})$ is a normed space, then for each $a \in X \setminus \{0\}$ there exists a $\ph \in X'$ with $\ph(a)=\norm{a}$ and $\norm{\ph}=1$.
  In particular, if $X \neq 0$ then $X' \neq 0$.
\end{cor}

Dual spaces
\begin{rmk}
  Recall that the (continuous) dual of a normed space $X$ is $X'=\cL(X,K)$.
  Since $K$ is complete, $X'$ is always complete.
\end{rmk}

\begin{defn}
  The \textbf{double dual} or \textbf{bidual} of $X$ is the space $X''=\cL(X',K)$.
  There exists a natural embedding of $X$ in to $X''$.
  For each $x \in X$ define a linear functional $\hat x$ by $\hat x(\ph) = \ph(x)$ for all $\ph \in X'$.
\end{defn}

\begin{prop}
  For each $x \in X$, $\hat x \in X''$.
  Moreover, the map $\Theta: X \to X''$ defined by $x \mapsto \hat x$ is linear, isometric and injective.
\end{prop}

\begin{proof}
  We first prove the first part of the proposition.
  It is easy to see that $\hat x$ is linear.
  Moreover, we have
  \[\abs{\hat x(\ph)} = \abs{\ph(x)} \leq \norm{\ph} \norm{x} \implies \norm{\hat x} \leq \norm{x}\]
  so $\hat x$ is bounded, hence $\hat x \in X''$.

  We now prove the second part of the proposition.
  It is easy to see that $\Theta$ is linear.
  Note that to show that $\Theta$ is an isometry we only need to prove that $\norm{\hat x} \geq \norm{x}$.
  By the corollary to the Hahn-Banach theorem there exists a $\ph_x \in X'$ such that $\ph_x(x) = \norm{x}$ and $\norm{\ph_x}=1$.
  \[\hat{\ph_x} = \ph_x(x) = \norm{x} = \norm{\ph_x}\norm{x} \implies \norm{\hat x} \geq \norm{x}.\]
  That $\Theta$ is injective follows from the isometric property.
\end{proof}

\begin{cor}
  Every normed space has a completion.
\end{cor}

\begin{proof}
  $X$ embeds isometrically via $\Theta$ in to $X''$ which is complete.
  Then
  \[X \cong \im\Theta \subseteq \ol{\im\Theta} \subseteq X''.\]
\end{proof}

\begin{defn}
  A space $X$ is said to be \textbf{reflexive} if $X'' \cong X$ (isometric isomorphism).
\end{defn}

The dual of $\ell^p$.

\begin{thm}
  Let $1<p<\infty$ and let $q = p/(p-1)$ (then $1/p+1/q=1$).
  Then $(\ell^p)' \cong \ell^q$ (isometric isomorphism).
\end{thm}

\begin{cor}
  For $1<p<\infty$, $\ell^p$ is reflexive.
\end{cor}

\begin{lem}
  If $1\leq p<\infty$, then:
  \begin{enum}
    \io
    $\ds x = \sum_{k=1}^\infty x_ke_k$ for all $x = (x_k) \in \ell^p$, where
    \[e_k = (\overset{1}{0},\ldots,\overset{k-1}{0},\overset{k}{1},\overset{k+1}{0},\ldots).\]
  (That is, the partial sums go to zero.)

  \io
  $\ds \ph(x) = \sum_{k=1}^\infty x_k\ph(e_k)$ for all $\ph \in (\ell^p)'$.
  \end{enum}
\end{lem} 

\begin{proof}
  Obvious.
\end{proof}

\begin{rmk}
  If $p=\infty$ then the lemma does not hold.
  For example, take $x=(1)_{n \geq 1}$.
  Then the partial sums $\sum_{k=1}^n x_ke_k$ do not converge to $x$.
\end{rmk}

\begin{proof}[Proof of the theorem]
  If $x \in \ell^p$ and $y \in \ell^q$, then by H\"older's inequality we have
  \[\abs{\sum_{k=1}^\infty x_ky_k} \leq \norm{x}_p\norm{y}_q\]
  In particular, the infinite sum on the left hand side converges.
  Define
  \begin{align*}
    \Theta: \ell^p &\to (\ell^q)' \\
  (\Theta(y))(x) &= \sum_{k=1}^\infty x_ky_k
  \end{align*}
  Denote $\Theta(y)$ by $\ph_y$.
  Then since $\abs{\ph_y(x)} \leq \norm{x}_p \norm{y}_q$ we have $\norm{\ph_y} \leq \norm{y}_q$.
  $\ph_y$ is clearly linear, so the image of $\ph_y$ is in $(\ell^q)'$.

  We now show that $\Theta$ is an isometric isomorphism.
  It is clear that $\Theta$ is linear.
  To see that $\Theta$ is injective, note that
  \[\Theta(y)=\Theta(y') \iff \sum_{k=1}^\infty x_k(y_k-y_k')=0 \text{ for all } x \in \ell^p.\]
  Then choose $x = e_k$, so we have that $y_k=y_k'$ for all $k$.
  We will prove that $\Theta$ is surjective and an isometry in the next lecture.
\end{proof}
