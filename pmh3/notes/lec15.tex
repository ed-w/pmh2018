\section{2018-04-30 Lecture}

\begin{proof}
  Let $A$ be the set of all non-zero linearly independent subsets of $V$ ordered by inclusion.
  For every chain $C$ inside $A$, $u = \bigcup_{c \in C} c$ is an upper bound for $C$.
  Then $u \in A$ since every finite subset of $u$ is contained in $c$ for some $c \in C$.
  So we may apply Zorn's lemma to give the existence of a maximal element $m \in A$.
  Then $\spn m = V$, since if there exists a $w \in V \setminus \spn m$, then $m \prec \{w\} \cup m \in A$, contradicting the maximality of $m$.
\end{proof}

\begin{prop}
  There exists a discontinuous linear functional in any infinite dimensional normed vector space $X$, that is, $X' \subsetneq X^*$.
\end{prop}

\begin{proof}
  Let $X$ be an infinite dimensional normed space.
  Let $B \subseteq X$ be a Hamel basis for $X$.
  Since $B$ is infinite, there exists a sequence $b_1,b_2,\ldots$ in $B$.
  Now define $\phi \in X^*$ on $B$ by
  \begin{equation*}
    \phi(b)=
    \begin{cases}
      n\norm{b_n} & \text{ if } b=b_n \\
      0 & \text{ if } b \in B \setminus\{b_1,b_2,\ldots\}
    \end{cases}
  \end{equation*}
  extending by linearity to all of $X$ (using finite sums of basis vectors).
  Then $\abs{\phi(b_n)} = n\norm{b_n}$ for all $b$, hence $\phi$ is unbounded.
\end{proof}

\begin{rmk}
  Let $e_i$ be the sequence with a $1$ in the $i$th component and zeros in all other components.
  Is $\{e_1,e_2,\ldots\}$ a Hamel basis for $\ell^1$?
  No.
  But it is a Hamel basis for $c_{00}$, the space of finitely supported sequences.
\end{rmk}

\begin{exer}
  Let $X$ be an infinite dimensional Banach space.
  Use Baire's theorem to show that every Hamel basis is uncountable.
\end{exer}

\begin{rmk}
  If $X$ is an infinite dimensional Banach space then no explicit Hamel basis or discontinuous linear functional is known.
\end{rmk}

\begin{defn}
  Let $X$ be a normed space.
  A finite or countably infinite subset $\{x_1,x_2,\ldots\}$ of $X$ is called a \textbf{Schauder basis} if each $x \in X$ can be writte uniquely as
  \[x = \sum_{k=1}^\infty a_kx_k\]
  that is, the partial sums converge to $x$.
\end{defn}

\begin{rmk}
  A Schauder basis is like a Hamel basis but where we allow the sums to be infinite.
\end{rmk}

\begin{exam}
  The set $\{e_1,e_2,\ldots\}$ is a Schauder basis for $\ell^p$ (for $p<\infty$), but not for $\ell^\infty$ since $X$ is not separable.
  If a Banach space $X$ has a Schauder basis then it is separable (consider $\QQ$-linear cominations of the basis vectors).
  The converse is not true (Per Enflo).
\end{exam}

Hahn-Banach Theorem

\begin{thm}[Hahn-Banach theorem]
  Let $X$ be a vector space over $K$ and let $p: X \to \RR$ be a seminorm.
  Let $Y$ be a subspace of $X$.
  If $\ph:Y\to K$ is linear with $\abs{\ph(y)}\leq p(y)$ for all $y \in Y$, then there exists an extension $\wt\ph:X\to K$ of $\ph$ such that:
  \begin{enum}
    \io $\wt\ph$ is linear,
    \io $\wt\ph|_Y=\ph$ and
    \io $|\wt\phi(x)|\leq p(x)$ for all $x \in X$.
  \end{enum}
  We will only do the case $K=\RR$.
  The $\RR$ version looks stronger since the seminorm condition over $\CC$ ($p(\lambda x)=\abs{\lambda}p(x)$ is `quite strong'.
\end{thm}

\begin{proof}
  We first consider the case where $Y$ has codimension $1$, that is, $X = Y \oplus \RR z$ for some $z \in X$.
  Now define
  \[\wt \ph(y+\alpha z) = \ph(y)+\alpha c\]
  where $c = \wt\ph(z)$ must be chosen such that
  \[\ph(y)+\alpha c \leq p(y+\alpha z)\]
  for all $y \in Y$ and $\alpha \in \RR$.
  See Daners' notes for the rest of the proof.
  (I can't be bothered typing it up.)
\end{proof}
