\section{2018-06-04 Lecture}

\begin{proof}[Proof of theorem \ref{22:compact}.2]
  We want to show that if $\lambda \neq 0$ then $\lambda I-T$ is injective if and only if it is surjective.
  We will use the following lemmas:

  \begin{lem}
    Let $X$ be a Banach space, let $T: X \to X$ be compact and let $\lambda \neq 0$.
    If $(\lambda I-T)x_n \to y$ then $([x_n])$ is bounded in $X/\ker(\lambda I-T)$.
  \end{lem}

  \begin{proof}
    Suppose (w.l.o.g.) by way of contradiction that $\norm{[x_n]} \geq n$ for all $n$.
    Let $y_n = [x_n]/\norm{[x_n]} \in X/\ker(\lambda I-T)$.
    Define $[T] : X/\ker(\lambda I-T) \to X/\ker(\lambda I-T)$ by $[T][x]=[Tx]$.

    \begin{exer}
      Show that $[T]$ is compact.
    \end{exer}

    Since $(y_n)$ is bounded, there exists a subsequence $(y_{n_k})$ such that $([T]y_{n_k})$ is converges to some limit point $w$.
    Then
    \[ \lambda y_{n_k} = [\lambda I-T]y_{n_k} + [T]y_{n_k} \to w \]
    since $[\lambda I-T]x_{n_k}$ converges and $\norm{[x_{n_k}]} \to \infty$.
    This shows that $y_{n_k} \to \lambda\inv w$, hence $[\lambda I-T]w=0$ and so $w=0$.
    But $\norm{y_{n_k}}=1$ and $y_{n_k} \to \lambda\inv w$ and $\lambda \neq 0$, a contradiction.
  \end{proof}

  \begin{cor}
    Let $X$ be a Banach space and let $T: X \to X$ be compact.
    If $\lambda \neq 0$ then $\im(\lambda I-T)$ is closed.
  \end{cor}

  \begin{proof}
    If $(\lambda I-T)x_n \to y$, then by the lemma there exists an $M>0$ such that $\inf{x_n-z} \leq M$ for some $z \in \ker(\lambda I-T)$ and for all $n \geq 1$.
    Then there exists a bounded sequence $(x_n-z_n)$ with $z_n \in \ker(\lambda I-T)$.
    Then by compactness there exists a subsequence $(x_{n_k})$ such that $T(x_{n_k}-z_{n_k})$ converges.
    Then
    \[ \lambda (x_{n_k}-z_{n_k}) = (\lambda I-T)(x_{n_k}-z_{n_k}) + T(x_{n_k}-z_{n_k}) = (\lambda I-T)x_{n_k} + T(x_{n_k}-z_{n_k}). \]
  \end{proof}

  If $\lambda \in \sigma(T) \setminus \{0\}$ is not an eigenvalue, then $\lambda I-T$ is injective but not surjective.
  Let $Y_n = \im(\lambda I-T)^n$.
  $Y_n$ is closed for all $n$ by the lemma.
  We will use the following exercise:

  \begin{exer}
    Show that
    \[ X=Y_0 \supset Y_1 \supset Y_2 \supset \cdots \]
    strictly.
    (Use the injectivity of $\lambda I-T$).
  \end{exer}

  By Riesz's lemma there exists a sequence $(y_n)$ with $y_n \in Y_n$ and $\norm{y_n}=1$ for each $n$, and $\norm{y_n-y} \geq 1/2$ for all $y \in Y_{n+1}$.
  We will show that $(Ty_n)$ has no convergent subsequence.

  If $n<m$, then
  \[ \norm{Ty_n-Ty_m} = \norm{(\lambda I-T)y_n - (\lambda I-T)y_m + \lambda y_m - \lambda y_n} \]
  which is in $Y_{m+1}$, so since
  \[ \norm{y-\lambda y_n} = \abs{\lambda} \norm{y/\lambda-y_n} \geq \frac{\abs{\lambda}}{2}, \]
  we have
  \[ \norm{Ty_n-Ty_m} \geq \frac{\abs{\lambda}}{2} > 0. \qedhere \]
\end{proof}

\begin{exer}[Quotient spaces]
  Let $X$ be a normed space and $Y \subseteq X$ a subspace.
  \begin{enum}
    \io
    Show that $X/Y$ is a vector space with $[x_1]+[x_2]=[x_1+x_2]$ and $\lambda[x] = [\lambda x]$.

    \io
    Show that if $Y$ is closed, then $X/Y$ is a normed vector space with 
    \[ \norm{[x]} = \inf_{y \in Y} \norm{x-y} = d(x,Y) \]

    \io
    Show that if $X$ is Banach and $Y$ is closed then $X/Y$ is Banach.
  \end{enum}
\end{exer}
