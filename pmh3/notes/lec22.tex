\section{2018-05-22 Lecture}

\begin{defn}
  Let $T \in \cL(X,Y)$ where $X$ and $Y$ are normed spaces.
  If $\dim \im T <\infty$, then $T$ is said to be of \textbf{finite rank}.
\end{defn}

\begin{exer}
  If $T \in \cL(X,Y)$ is of finite rank then $T$ is compact.
\end{exer}

\begin{proof}
  Let $(x_n)_{n \geq 1}$ be a bounded sequence in $X$.
  Then $(Tx_n)_{n \geq 1}$ is a bounded and contained in $\im T$ which is finite dimensional.
  So it is precompact.
\end{proof}

\begin{thm}[Hilbert-Schmidt integral operators]
  The operator
  \begin{align*}
    T: L^2([a,b]) &\to L^2([a,b]) \\
    f(x) &\mapsto \int_a^b K(x,y) f(y) \, dy
  \end{align*}
  is compact, where $K \in L^2([a,b]\times[a,b])$.
\end{thm}

\begin{proof}
  Let $X=L^2([a,b])$.
  Then $T \in \cL(X)$.
  Now
  \[ \norm{Tf}_2^2 = \int_a^b \abs{ \int_a^b K(x,y) f(y) \, dy } \, dx \leq \int_a^b \left( \int_a^b \abs{K(x,y)}^2 \, dy \norm{f}_2^2 \right) \, dx \leq \norm{K}_2^2 \norm{f}_2^2. \]
  Then $T \in \cL(X)$ with $\norm{T} \leq \norm{K}_2$.
\end{proof}

\begin{exer}
  There exist functions $a_i,b_j \in L^2([a,b])$ with
  \[ K_n(x,y) = \sum_{i=1}^{m_n} \sum_{j=1}^{l_n} a_i(x) b_j(y) \]
  and $K_n \to K$ in $\norm{\cdot}_2$.
  Use step/indicator functions and that the Borel $\sigma$-algebra is generated by open rectangles.
\end{exer}

Define
\begin{align*}
  T_n: L^2([a,b]) &\to L^2([a,b]) \\
  f(x) &\mapsto \int_a^b K_n(x,y) f(y) \, dy
\end{align*}
Then the $T_n$ are of finite rank in $\cL(X)$.
Then $T_n \in \cK(X,X)$ for all $n$, and $\norm{T_n-T} \leq \norm{K_n-K} \to 0$.
Hence $T \in \cK(X,X)$.

Why is $T_n$ of finite rank?
\[ T_nf(x) = \sum_{i=1}^{m_n} \left( \int_a^b \left( \sum_{j=1}^{l_n} b_j(y) \right) f(y) \, dy \right) a_i(x). \]
Then $\im T_n \subseteq \spn\{a_1,\ldots,a_{m_n}\}$.

\begin{rmk}
  Every compact operator is the limit of finite rank (bounded) operators.
\end{rmk}

Spectral theory of compact operators

\begin{thm}\label{22:compact}
  Let $X$ be an infinite dimensional Banach space on $\CC$ and let $T \in \cL(X)$ be a compact operator.
  Then
  \begin{enum}
    \io $0 \in \sigma(T)$.
    \io $\sigma(T) \setminus \{0\} = \sigma_p(T) \setminus \{0\}$.
    \io $\sigma_p(T)$ is either finite (possibly empty) or is a sequence of points converging to $0$.
    \io If $\lambda \in \sigma(T) \setminus \{0\}$ then the eigenspace $\ker(\lambda I-T)$ is finite-dimensional.
  \end{enum}
\end{thm}

\begin{prop}
  Let $X,Y,Z$ be normed spaces.
  \begin{enum}
    \io If $T \in \cK(X,Y)$ and $S \in \cL(Y,Z)$, then $ST \in \cK(X,Z)$.
    \io If $S \in \cL(X,Y)$ and $T \in \cK(Y,Z)$, then $TS \in \cK(X,Z)$.
  \end{enum}
\end{prop}

\begin{proof}
  \lv
  \begin{enum}
    \io
    If $(x_n)$ is a bounded sequence in $X$, then $(Tx_n)$ contains a convergent subsequence $(Tx_{n_k})_k$.
    Since $S$ is continuous, $(STx_{n_k})_k$ also converges.
    
    \io
    If $B$ is a bounded subset of $X$, then $S(B)$ is bounded.
    Therefore $TS(B)$ is precompact.
    \qedhere
  \end{enum}
\end{proof}

\begin{proof}[Proof of theorem \ref{22:compact}.1]
  If $0 \notin \sigma(T)$, then $T$ is invertible, so $T\inv$ is also bounded.
  Then $T \circ T\inv = \id: X \to X$ is compact, a contradiction since $\dim X = \infty$.
\end{proof}

\begin{proof}[Proof of theorem \ref{22:compact}.3]
  We show that for each $r>0$, the number of $\lambda \in \sigma_p(T)$ with $\abs{\lambda}>r$ is finite.
  We will prove this by contradiction.
  Suppose that the above statement is false.
  Then there exist distinct eigenvalues $\lambda_1,\lambda_2,\ldots$ with $\abs{\lambda_n}>r$ for all $n \geq 1$.
  For all $n \geq 1$, choose a non-zero eigenvector $x_n$ corresponding to $\lambda_n$.
  We know from linear algebra that the set $\{x_n\}_{n \geq 1}$ is linearly independent.
  Define $X_n = \spn\{x_1,\ldots,x_n\}$.
  Then $0 \subset X_1 \subset X_2 \subset \cdots$ with each $X_n$ closed (as they are finite-dimensional).

  We will use the following result:
  \begin{lem}
    There exists a sequence $(y_n)$ with $y_n \in X_n$ for all $n$ and $\norm{y_n}=1$ such that $\norm{y_n-y} \geq 1/2$ for all $y \in X_{n-1}$.
  \end{lem}

  \begin{proof}
    This follows from Riesz's Lemma.
  \end{proof}

  We will now show that $(Ty_n)$ has a no convergent subsequence.
  Let $n<m$.
  Then
  \[ \norm{Ty_m-Ty_n} = \norm{\lambda_my_m-(\lambda_my_m-Ty_m+Ty_n)} \]
  \textbf{fill in}
\end{proof}

\begin{proof}[Proof of theorem \ref{22:compact}.4]
  Let $\lambda \in \sigma(T) \setminus \{0\}$.
  Let $B = \{ x \in \ker (\lambda I-T) \mid \norm{x} \leq 1 \}$.
  This is the unit ball in $\ker(\lambda I-T)$.
  Now $x \in \ker (\lambda I-T) \iff Tx = \lambda x \iff \lambda\inv T x = x$.
  Then $B \subseteq \{ \lambda\inv Tx \mid \norm{x} \leq 1 \} = \lambda\inv (T(\ol{B(0,1)})) \subseteq \lambda\inv (\ol{T(\ol{B(0,1)})})$ which is compact.
  So $B$ is compact, hence $\ker (\lambda I-T)$ is finite dimensional.
\end{proof}
