\section{2018-03-26 Lecture}

We will now prove lemma \ref{lem:proj}.

\begin{proof}
	$\implies$: If $m-x \not\perp M$, then there exists an $m' \in M$ such that $\ang{m',m-x}<0$.
	For any $t \in \RR$, set $m_t = m+tm'$.
	Then
	\[\norm{m_t-x}^2 = \ang{m+tm'-x,m+tm'-x} = t^2\norm{m'}^2 + 2t\ang{m',m-x}+\norm{m-x}^2\]
	Now we can choose $t$ small enough such that $\norm{m_t-x}^2 < \norm{m-x}^2$, a contradiction.
	
	$\impliedby$: If $x-m \perp M$, then for all $m' \in M$, we have $\norm{x-m}^2 + \norm{m'}^2 = \norm{x-m'}^2$ since $m'-m \perp x-m$.
	Then $\norm{x-m'}>\norm{x-m}$ for any $m' \neq m$, hence $m = P_M(x)$.
\end{proof}

\begin{defn}
	Let $M \subset \cH$ be non-empty.
	Then define the \textbf{orthogonal complement} of $M$:
	\[M^\perp = \{x \in \cH \mid \ang{x,m}=0 \text{ for all } m \in M\}\]
\end{defn}

\begin{lem}
	$M^\perp \subset \cH$ is a closed subspace.
	Moreover, $(M^\perp)^\perp = \ol{\spn{M}}$
\end{lem}

\begin{cor}
	If $M \subset \cH$ is a closed subspace, then $(M^\perp)^\perp=M$.
\end{cor}

\begin{proof}[Proof of the lemma]
	That $M^\perp \subset \cH$ is closed follows from the continuity of the inner product.
	We will prove the second claim in two parts:
	\begin{enum}
		\io $M \subset (M^\perp)^\perp$ $(\implies \ol{\spn(M)} \subset (M^\perp)^\perp)$
		\par If $m \in M$ then for all $x \in M^\perp$ we have $\ang{x,m}=0 \implies m \in (M^\perp)^\perp$.
		
		\io $(M^\perp)^\perp \subset \ol{\spn(M)} \defeq V$
		\par Now let $x \in (M^\perp)^\perp$.
		Let $v = P_V(x)$.
		If $v = x$ for all $x \in (M^\perp)^\perp$ then we are done.
		Now assume $v \neq x$.
		Since $x-v \perp V$, we have $x-v \in V^\perp$.
		Set $w=x-v$.
		Then
		\[\ang{w,w}= \langle \overbrace{x}^{\in (M^\perp)^\perp},\overbrace{w}^{\in M^\perp} \rangle - \langle \overbrace{v}^{\in \ol{\spn(M)}},\overbrace{w}^{\in V^\perp} \rangle = 0\]
		Hence $w=0$ and so $x=v$.
	\end{enum}
\end{proof}

\begin{cor}
	If $M \subset \cH$ is a closed subspace then $\cH = M \oplus M^\perp$.
\end{cor}

\begin{proof}
	For all $x \in \cH$, $m = P_M(x)$ implies that $x-m \perp M$, that is $x-m=m'$ for some $m' \in M^\perp$.
	Therefore $x = m+m'$.
	To show uniqueness, note that if $x \in M \cap M'$, then $x \perp x$, so $x=0$.
\end{proof}

\begin{defn}
	A set $\{x_i\}_{i \in I}$ is an \textbf{orthogonal system} if for all $x_i,x_j$ for $i,j \in I$ and $i \neq j$ we have $\ang{x_i,x_j}=0$.
\end{defn}

\begin{lem}
	If $\{x_i\}_{i \in I}$ is an orthogonal system, then
	\begin{enum}
		\io Any finite number of non-zero $x_i$s are linearly independent.
		\io If $(x_n)_{n \geq 1}$ is a sequence in $\{x_i\}_{i \in I}$, then
		\[\sum_{n=1}^\infty x_n \text{ converges} \iff \sum_{n=1}^\infty \norm{x_n}^2 \text{ converges}\]
	\end{enum}
\end{lem}

\begin{proof}
	\begin{enum}
		\io If $\ds \sum_{k=1}^N \alpha_k x_k = 0$ then $\ds 0 = \ang{\sum_{k=1}^N \alpha_k x_k, x_{k_0}} = \alpha_{k_0} \norm{x_{k_0}}^2$ for $1 \leq k_0 \leq N$.
		
		\io Let $z_N = \sum_{n=1}^N x_n$.
		Then $\sum_{n=1}^\infty x_n$ converges $\iff$ $(z_N)_{N \geq 1}$ is Cauchy.
		Now for a given $\epsilon$ we have that for large enough $N$,
		\[\norm{z_{N+m}-z_N}^2 = \sum_{n=N+1}^{N+m-1} \norm{x_n}^2 < \epsilon\]
		for all $m$.
		Therefore $(z_N)_{N \geq 1}$ is Cauchy $\iff $ $\sum_{n=1}^N \norm{x_n}^2$ is Cauchy, that is, the sum converges.
		
	\end{enum}
\end{proof}

\begin{defn}
	An \textbf{orthonormal system} is an orthogonal system consisting of unit length vectors.
\end{defn}

\begin{thm}[Bessel inequality]
	Let $M \subset \cH$ be an orthonormal system.
	Then for all $x \in \cH$:
	\[\sum_{m \in M} \abs{\ang{x,m}}^2 \defeq \sum_{\substack{m \in M' \\ M' \subset M \text{ finite}}} \abs{\ang{x,m}}^2 \leq \norm{x}^2\]
	Moreover, the set
	\[\{m \in M \mid \ang{x,m} \neq 0\}\]
	is countable.
\end{thm}

\begin{lem}
	If $M'$ is a finite orthonormal system, then
	\[P_{M'}(x) \defeq P_{\spn(M')}(x) = \sum_{m \in M'} \ang{x,m}m\]
\end{lem}

\begin{proof}
	Clearly $\sum_{m \in M'} \ang{x,m}m \in \spn(M')$.
	It is enough to prove that for all $m' \in M' \subset \spn(M')$ we have $x - \sum_{m \in M'} \ang{x,m}m \perp m'$.
	Then
	\[\ang{x - \sum_{m \in M'} \ang{x,m}m, m'} = \ang{x,m'} - \ang{x,m'}\ang{m',m'} = 0 \qedhere\]
\end{proof}

\begin{proof}[Proof of the Bessel inequality]
	We already know that if $M' \subset M$ is finite, then $\norm{P_{M'}}=1$.
	Therefore
	\[\norm{P_M(x)}^2 = \norm{\sum_{m \in M'} \ang{x,m}m}^2 = \sum_{m \in M'} \abs{\ang{x,m}}^2 \leq \norm{x}^2\]
	This proves the inequality for finite sets.
	Next we will show that
	\[\{m \in M \mid \ang{x,m} \neq 0\}\]
	is countable.
	For all $n \geq 1$, let
	\[M_n \defeq \left\{m \in M \mid \abs{\ang{x,m}} \geq \frac 1n\right\}\]
	Then $M_1 \subseteq M_2 \subseteq \cdots \subseteq M_n \subseteq \cdots$ and $M = \bigcup_{n \geq 1} M_n$.
	Now by the finite Bessel inequality, $\abs{M_n} \leq n^2 \norm{x}^2$ and so $M_n$ is finite, hence the set is countable as required.
	Then taking the supremum over all finite sets gives the infinite case.
\end{proof}

\begin{defn}
	A subset $M \subset \cH$ is called an \textbf{orthonormal basis} if it is a orthonormal system which is \textbf{complete}, that is, $\ol{\spn(M)}=\cH$.
\end{defn}

\begin{exam}
	Let $\cH = \ell^2$.
	Let
	\[e_n = (0,\ldots,0,\overbrace{1}^{n\text{th position}},0,\ldots)\]
	and set $M = \{e_n\}_{n \geq 1}$.
	Clearly $M$ is an orthonormal system.
	To see that $M$ is complete, note that for any $x = (\alpha_i)_{i \geq 1} \in \ell^2$, we have $x = \sum_{n \geq 1} \alpha_n e_n$.
	Therefore $\ol{\spn(M)} = \ell^2$.
\end{exam}