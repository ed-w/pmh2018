\section{2018-03-20 Lecture}

Hilbert Spaces

\begin{prop}
	If $E$ has inner product $\ang{\cdot,\cdot}$, then define a norm by
	\[\norm{u} \defeq \sqrt{\ang{u,u}} \text{ for all } u \in E\]
	Then $\norm{\cdot}$ makes $E$ in to a normed space.
	The triangle ineqaulity follows from the following:
\end{prop}

\begin{prop}[Cauchy-Schwarz inequality]
	For all $u,v \in E$ we have
	\[\abs{\ang{u,v}} \leq \norm{u}\norm{v}\]
	with equality $\iff u \parallel v$.
\end{prop}

\begin{proof}
	Consider the quadratic form
	\[\ang{u+tv,u+tv} \geq 0 \qedhere\]
\end{proof}

\begin{proof}[Proof of triangle inequality]
	\begin{multline*}
		\norm{u+v}^2 = \ang{u+v,u+v} = \norm{u}^2+\norm{v}^2+\ang{u,v}+\ang{v,u} = \norm{u}^2+\norm{v}^2+2\Re\ang{u,v} \\
		\leq \norm{u}^2+\norm{v}^2 + 2\abs{\ang{u,v}} \leq \norm{u}^2+\norm{v}^2+2\norm{u}\norm{v} = \left(\norm{u}+\norm{v}\right)^2 \qedhere
	\end{multline*}
\end{proof}

\begin{defn}
	A space $E$ equipped with an inner product is called a Hilbert space if $(E,\norm{\cdot})$ is a Banach space.
\end{defn}

\begin{rmk}[Invariant subspace problem]
	The following is an open problem:
	
	Let $T: \cH \to \cH$ be a bounded operator.
	Then does there exist a closed subspace $M \subsetneq \cH$ with $T[M] \subset M$?
\end{rmk}

\begin{prop}
	If the parallelogram law
	\[2\norm{a}^2 + 2\norm{b}^2 = \norm{a+b}^2 + \norm{a-b}^2\]
	holds in a normed space $(E,\norm{\cdot})$, then there is an inner product on $E$ such that $\norm{a}^2=\ang{a,a}$.
\end{prop}

\begin{rmk}
	Clearly the converse to the above statement is true (i.e.\@ if the norm in a normed space comes from an inner product then it must satisfy the parallelogram law).
\end{rmk}

\begin{proof}
	Note that
	\[\ang{a+b,a+b}-\ang{a-b,a-b} = 2\ang{a,b}-2\ang{b,a} = 4\Re\ang{a,b}\]
	in any inner product space, so if $K=\RR$, we can define an inner product by
	\[\ang{a,b} = \frac 14 \left(\norm{a+b}^2 - \norm{a-b}^2\right)\]
	for all $a,b \in E$.
	Similarly, if $K=\CC$, we can define
	\[\ang{a,b} = \frac 14 \left(\norm{a+b}^2 - \norm{a-b}^2 - i\norm{x-iy}^2 + i\norm{x+iy}^2\right)\]
	for all $a,b \in E$.
\end{proof}

Orthogonal projection

\begin{thm}
	Let $\cH$ be a Hilbert space and let $M$ be a non-empty subset of $\cH$ containing $x$.
	Define the set
	\[P_M(x) = \left\{ z \in M \mid \norm{z-x} = \dist(x,M) \right\}\]
	If $M \subset \cH$ is closed and convex, then $\abs{P_M(x)}=1$.
\end{thm}

\begin{proof}
	If $x \in M$ then $P_M(x) = \{x\} \implies \abs{P_M(x)=1}$.
	Assume $x \notin M$.
	If $(m_n)$ is a sequence in $M$ such that
	\[\norm{x-m_n} = \dist(x,m_n) \to \dist(x,M) \defeq d\]
	then we will show that $(m_n)$ is Cauchy.
	Clearly $(m_n)$ is bounded, so it converges in $M$ (Bolzano-Weierstrass).
	Hence $P_M(x) \neq \emptyset$.
	
	If $\epsilon>0$, then there exists an $N_\epsilon$ such that for all $f \leq N_\epsilon$, we have $\norm{m_n-x}^2 \leq d^2+\epsilon$.
	We will show that for all $n_1,n_2 \leq N_\epsilon$, we have $\norm{m_{n_1}-m_{n_2}} \leq \epsilon$.
	
	By the parallelogram law,
	\[2\norm{m_{n_2}-x}^2 + 2\norm{m_{n_1}-x}^2 = \norm{m_{n_2}-m_{n_1}}^2 + 4\norm{\frac{m_{n_1}+m_{n_2}}{2}-x}^2\]
	Now
	\[\norm{\frac{m_{n_1}+m_{n_2}}{2}-x} \geq d\]
	since $(m_{n_1}+m_{n_2})/2 \in M$ by convexity.
	Hence
	\begin{eqn}\label{eq:proj-par-ineq}
		\norm{m_{n_2}-m_{n_1}}^2 \leq 2\norm{m_{n_2}-x}^2 + 2\norm{m_{n_1}-x}^2 - 4d^2 \leq 4(d^2+\epsilon)-4d^2 = 4\epsilon
	\end{eqn}
	This completes the proof of existence.
	
	To prove uniqueness, note that if $m_1$ and $m_2$ are both in $P_M(x)$ (i.e.\@ $\norm{m_1-x}^2=\norm{m_2-x}^2=d^2$), then by equation \ref{eq:proj-par-ineq} we have $\norm{m_2-m_1}^2 \leq 4d^2-4d^2=0$, hence $m_1=m_2$.
\end{proof}

\begin{defn}
	Let $M \subset \cH$ be a closed subspace.
	Then $P_M: \cH \to M$ is the \textbf{orthogonal projection map} from $\cH$ on to $M$.
\end{defn}

\begin{thm}\label{thm:prof}
	$P_M$ is linear and bounded with $\norm{P_M}=1$.
\end{thm}

\begin{lem}\label{lem:proj}
	$P_M(x)=m \iff m-x \perp M$ (i.e.\@ $\ang{m-x,m'}=0$ for all $m' \in M$).
\end{lem}

\begin{proof}
	Omitted.
	See the next lecture.
\end{proof}

\begin{proof}[Proof of Theorem \ref{thm:prof}]
	We first prove linearity.
	If $P_M(x_1)=m$ and $P_M(x_2)=m_2$, then by lemma \ref{lem:proj}, $\ang{x_1+x_2-(m_1+m_2),x} \perp M$, hence $P_M(x_1+x_2)=m_1+m_2$.
	
	Now we prove that $\norm{P_M}=1$.
	First note that for all $m \in M$, we have $P_M(m)=m$.
	Hence $\norm{P_M} \geq 1$.
	
	Now let $P_M(x)=m$.
	Then by lemma \ref{lem:proj}, $\ang{m-x,m}=0$.
	Then Pythagoras' theorem (for inner product spaces) gives $\norm{m}^2 + \norm{x-m}^2 = \norm{x}^2$, hence $\norm{m} \leq \norm{x}$.
	Thus $\norm{P_M(x)} \leq \norm{x}$, or $\norm{P_M} \leq 1$.
\end{proof}