\section{Lecture 2018-03-07}

1.4 Composed operations of topological spaces

Let $X$ and $Y$ be topological spaces.

\begin{defn}
	The \textbf{join} of $X$ and $Y$ is defined to be
	\[X*Y \defeq (X \times Y \times I)/\sim\]
	where
	\[(x,y,0) \sim (x,y',0)\]
	for all $x,y,y'$, and
	\[(x,y,1) \sim (x',y,1)\]
	for all $x,x',y$.
	[insert sketch of join here, see Hatcher]
\end{defn}

\begin{defn}
	Assume $A \subset Y$ with a (continuous) map $f:A \to X$.
	Then the space \textbf{$X$ with $Y$ attached along $A$ via $f$} is defined to be
	\[X \sqcup_f Y \defeq (X \amalg Y)/\sim\]
	where $a \sim f(a)$ for all $a \in A$.
	
	As sets, $X \sqcup_f Y$ = $X \amalg (Y \setminus A)$.
	Then we have the following diagram
	\[
	\begin{tikzcd}
		& X \arrow[rd, hook'] \arrow[ld, hook'] &  \\
		X \amalg Y \arrow[rr] &  & X \sqcup_f Y \\
		& Y \arrow[ru, "\phi"] \arrow[lu, hook'] & 
	\end{tikzcd}
	\]
	where $\phi$ is injective $\iff$ $f$ is injective.
\end{defn}

\begin{defn}
	A \textbf{pointed space} is $(X,x_0)$ where $X$ is a topological space and $x_0 \in X$.
	A map $f:(X,x_0) \to (Y,y_0)$ of pointed spaces is a map $f:X \to Y$ of spaces with $f(x_0)=y_0$.
	
	Pointed spaces form a category $\Top_*$.
\end{defn}

\begin{defn}
	The \textbf{wedge sum} of two pointed spaces is defined by
	\[X \vee Y = (X,x_0) \vee (Y,y_0) \defeq (X \amalg Y)/\sim\]
	where $x_0 \sim y_0$.
\end{defn}

\begin{exam}
	$S^1 \vee S^1$ is a figure-8.
\end{exam}

\begin{defn}
	The \textbf{smash product} of two pointed spaces is defined by
	\[X \wedge Y \defeq X \times Y / X \vee Y\]
	where
	\begin{align*}
		X \vee Y &\injto X \times Y \\
		x &\mapsto (x,y_0) \\
		y &\mapsto (x_0,y)
	\end{align*}
\end{defn}

\begin{rmk}
	The smash product will not be important in our course but the wedge sum will be.
\end{rmk}

Homeomorphisms:
$\square^2 \ncong I \ncong \pt$ in $\RR^2$.
(Remove a point.)
We want a coarser notion of equivalence than homeomorphism that gives us bigger equivalence classes.

2 Homotopies
2.1 Homotopy type

\begin{defn}
	Let $X$ and $Y$ be topological spaces.
	\begin{enum}
		\io A \textbf{homotopy} is a family of maps
		\[\{g_z: X \to Y \mid t \in I\}\]
		such that
		\begin{align*}
			G: X \times I &\to Y \\
			(x,t) &\mapsto g_t(x)
		\end{align*}
		is continuous.
		\io Two maps $f_0,f_1: X \to Y$ are \textbf{homotopic} if there exists a homotopy $\{g_t\}$ with $g_0=f_0$ and $g_1=f_1$.
		We denote this by $f_0 \simeq f_1$ or $G: f_0 \Rightarrow f_1$.
		\io A map $f:X \to Y$ is a \textbf{homotopy equivalence} if there exists a map $g:Y \to X$ such that $f \circ G \simeq 1_X$ and $f \circ g \simeq 1_Y$.
		\io $X$ and $Y$ are \textbf{homotopy equivalent} or of the same \textbf{homotopy type} if there exists a homotopy equivalence $f: X \to Y$.
	\end{enum}
\end{defn}

\begin{rmk}
	The notion of homotopy equivalence is a relaxation of the condition for homeomorphism (equalities replaced by homotopy equivalences), i.e.\@ $X \cong Y \implies X \simeq Y$.
\end{rmk}

\begin{defn}
	$X$ is \textbf{contractible} if $X \simeq \pt$, i.e.\@ we have maps $f: X \to \pt$ and $g: \pt \to X$ such that $fg \sim 1_{\pt}$ and $gf \sim 1_X$.
	Now $fg \sim 1_{\pt}$ is automatic.
	Also note that the image of $gf$ is a single point (it is a constant map).
\end{defn}

\begin{exer}
	Show that:
	\begin{itm}
		\io $D^m$ is contractible
		\io $S^1 \simeq D^2 \setminus \{0\}$
		\io $\square^2 \simeq I \simeq \pt$
	\end{itm}
\end{exer}

Is there a homotopy from $S^1$ to $\pt$?
No.
We cannot continuously deform the circle in to a point.
We will give a rigorous proof later with the fundamental group.

We saw that the notion of homotopy is related to the notion of continuous deformations.
We can make this more explicit.

2.2 Retracts and deformation retracts

Let $A \subset X$ with inclusion map $\iota: A \to X$.

\begin{defn}
	A \textbf{rectraction of $X$ on to $A$} is a map $r:X \to X$ with $r[X]=A$ and $r|_A=\iota$.
	Equivalently, it is a map $r:X \to A$ with $r\iota=1_A$.
	If such a retraction exists, then $A$ is a \textbf{retract} of $X$.
\end{defn}

\begin{rmk}
	Hatcher uses a different approach to introduce these notions.
	It might be helpful to read it to get a different point of view.
\end{rmk}

\begin{defn}
	Let $\{f_t:X \to X \mid t \in I\}$ be a homotopy such that
	\begin{itm}
		\io $f_0=1_X$
		\io $f_1:X \to X$ a retraction on to $A$
		\io $f_t|_A=\iota$ for all $t \in I$
	\end{itm}
	Then $\{f_t\}$ is a \textbf{deformation retraction of $X$ on to $A$}, and $A$ is a \textbf{deformation retract} of $X$.
\end{defn}

\begin{rmk}
	Deformation retract is a stronger notion than homotopy equivalence sihce we have a condition on $f_t|_A$.
\end{rmk}

\begin{exer}
	If $A$ is a deformation retract of $X$ then $\iota:A \injto X$ is a homotopy equivalence, so $X \simeq A$.
\end{exer}

\begin{rmk}
	We can use the above exercise to show that $\square^2 \simeq I \simeq \pt$.
\end{rmk}

\begin{rmk}
	$X$ and $Y$ are homotopy equivalent $\iff$ there exists a $Z$ such that $X$ and $Y$ are deformation retracts of $Z$.
	Even though deformation retractions are a special case of homotopy equivalence, we can formulate homotopy equivalences in terms of deformation retractions.
\end{rmk}