\section{2018-03-19 Lecture}

1.3 $\pi_1(S^1)=(\ZZ,+)$
We wil prove the above statement in this lecture.
Since $S^1$ is path-connected, $\pi_1(S^1)$ does not depend on the basepoint.
The intuitive interpretation of this is that any loop in the circle can wind around it any whole number of times in either direction.

\begin{defn}
	A \textbf{covering space} of $X$ is a space $\wt{X}$ and a map $p: \wt X \to X$ such that each $x \in X$ has an open neighbourhood $U$ with $p\inv[U] = \coprod_\alpha V_\alpha$ with each $V_\alpha \subset X$ open and $p: V_\alpha \to U$ is a homeomorphism.
\end{defn}

\begin{exam}
	\begin{itm}
		\io $1_X: X \to X$
		\io Any homeomorphism with codomain $X$
		\io $X \amalg X \to X$ with the inclusion in both components
		\io $X=S^1$, $\wt X = \RR$, $p: \RR \to S^1$, $s \mapsto (\cos 2\pi s, \sin 2\pi s)$
		In a small enough open neighbourhood of any point, we can use the invertibility of sine and cosine to show that the preimage in the real line is a disjoint union of open sets.
		
		Here is a graphical interpretation.
		Define the helix
		\[H = \{(\cos 2\pi s, \sin 2\pi s, s) \mid s \in \RR\} \subset \RR^3\]
		Clearly the definition above specifies a homeomorphism $\RR \cong H$.
		Then let $p$ be the projection on to the plane $z=0$:
		\[p: (\cos 2\pi s, \sin 2\pi s, s) \mapsto (\cos 2\pi s, \sin 2\pi s)\]
	\end{itm}
\end{exam}

\begin{defn}
	A \textbf{lift through $p: \wt X \to X$} of $f: Y \to X$ is a map $\wt f: Y \to \wt X$ with $p \wt f = f$, i.e.\@ the following diagram commutes:
	\[
		\begin{tikzcd}
			Y \ar[rr, "f"] \ar[rd, dashed, "\exists \wt f"] & & X \\
			& \wt X \ar[ur, "p"] &
		\end{tikzcd}
	\]
\end{defn}

\begin{exam}
	Consider the path $\omega_n: I \to S^1$ for $m \in \ZZ$, where
	\[\omega_n(t) = (\cos 2\pi nt, \sin 2\pi nt)\]
	for $t \in I$.
	This corresponds to going around the circle $\abs{n}$ times in the $\sgn{n}$ direction.
	Then a family of lifts through $p: H \to X$ is given by
	\[\wt\omega_n^k(t) = (\cos 2\pi nt, \sin 2\pi nt, nt+k)\]
	where the $k$ 'shifts' the trace of $\wt\omega_n^0$ up or down the helix by $k$ loops.
\end{exam}

\begin{exer}
	Consider a path $f: I \to H$ with $f(0)=(1,0,0)$ and $f(1)=(1,0,n)$ for $n \in \ZZ$.
	Show that $f \simeq \wt\omega_n^0$.
\end{exer}

\begin{prop}{\label{prop:homotopy-lift}}
	Fix a covering space $p: \wt X \to X$.
	Consider a homotopy $g_t: Y \to X$ and a lift $\wt g_0: Y \to \wt X$ of $g_0$.
	Then there exists a unique homotopy $\{\wt g_t: Y \to \wt X\}$ with $p\wt g_t=g_t$ for all $t \in I$.
\end{prop}

\begin{proof}
	See Hatcher for details.
	Let $g_0: Y \to X$ and let $N = g\inv[U]$ where $U \subset X$.
	Now consider
	\[N \xto{g_0|_N} U \xto{\sim} \coprod_\alpha V_\alpha \qedhere\]
\end{proof}

\begin{cor}
	For each path $f: I \to X$ with $f(0)=x_0$ and $\wt x_0 \in p\inv[x_0]$, there exists a unique lift $\wt f: I \to \wt X$ of $f$ with $\wt f(0) = \wt x_0$.
\end{cor}

\begin{proof}
	Take $Y = \pt$ in proposition \ref{prop:homotopy-lift}.
	Then $f: I \to X$ is a homotopy $g_t: Y = \pt \to X$.
	Now apply proposition \ref{prop:homotopy-lift}.
\end{proof}

\begin{cor}
	For each homotopy $\{f_t: I \to X\}$ of paths starting at $x_0$ and $\wt x_0 \in p\inv[x_0]$ there exists a unique homotopy $\{\wt f_t: I \to \wt X\}$ of paths starting at $\wt x_0$ with $f_t = p\wt f_t$.
\end{cor}

\begin{proof}
	Apply proposition \ref{prop:homotopy-lift}.
	Now we just need to check that $\{\wt f_t\}$ is a homotopy of paths.
\end{proof}

\begin{cor}
	For two loops $f$ and $g$ in $X$ at $x_0$ and their unique lifts starting at $\wt x_0$, we have $f \simeq g \iff \wt f \simeq \wt g$.
\end{cor}

\begin{proof}
	The $\implies$ direction follows from the last corollary.
	For the $\impliedby$ direction see the notes.
\end{proof}

\begin{thm}
	Let $x_0 = (1,0) \in S^1$.
	Then there is a group isomorphism
	\begin{align*}
		\Phi: \ZZ &\to \pi_1(S^1,x_0) \\
		n &\mapsto [\omega_n]
	\end{align*}
\end{thm}

\begin{proof}
	Let $\wt x_0 = (1,0,0)$.
	\begin{itm}
		\io $\Phi$ is a group homomorphism, i.e.\@ $[\omega_n][\omega_l]=[\omega_{n+l}]$.
		This is obvious (reparametrisation).
		\io $\Phi$ is surjective.
		Let $f: I \to S^1$ be a loop and $\wt f$ its unique lift starting at $\wt x_0$.
		Then it must end at a point $(0,0,n) \in p\inv[0]$ for some $n \in \ZZ$.
		It can be shown that this forces $\wt f \simeq \wt \omega_n^0$, so $f \simeq \omega_n$, or $\Phi(n)=[f]$.
		\io $\Phi$ is injective.
		Now $\omega_n \sim \omega_l \implies \wt\omega_n^0 \sim \wt\omega_l^0$, but $\wt\omega_n^0 \sim \wt\omega_l^0 \implies m=n$ since they must have the same endpoint. \qedhere
	\end{itm}
\end{proof}