\section{2018-03-28 Lecture}

We cannot expect the epimorphism of lemma \ref{7:vk-lem} to be an isomorphism in general.
One way of thinking about this is that a fixed loop in $X$ which is contained in $A_\alpha \cap A_\beta$ appears in both $\pi_1(A_\alpha,x_0)$ and $\pi_1(A_\beta,x_0)$ and so appears (at least) twice in the free product.
Another way of phrasing this idea is to say that diagram
\[
\begin{tikzcd}
	& \pi_1(A,x_0) \ar[dr, "J_\alpha"] & \\
	\pi_1(A_\alpha \cap A_\beta, x_0) \ar[ur, "j_{\alpha\beta}"] \ar[dr, "j_{\beta\alpha}"] & & \pi_1(X,x_0) \\
	& \pi_1(A_\beta,x_0) \ar[ur, "j_\beta"] &
\end{tikzcd}
\]
commutes, where the maps $j_\alpha$ are induced from the corresponding inclusions of spaces.
(Commutativity follows from the functoriality of $\pi_1$.)

Let us make this more explicit.
Let
\[\omega = [f] \in \pi_1(A_\alpha \cap A_\beta,x_0)\]
Then $j_{\alpha\beta}(\omega) \in \pi_1(A_\alpha,x_0)$ and $j_{\beta\alpha}(\omega) \in \pi_1(A_\beta,x_0)$.
Note that if $\omega \neq 1$, then
\[j_{\alpha\beta}(\omega)j_{\beta\alpha}(\omega\inv) \in *_\alpha \pi_1(A_\alpha,x_0)\]
is a non-empty reduced word and so is not equal to 1.
However,
\[\Phi\left(j_{\alpha\beta}(\omega)j_{\beta\alpha}(\omega\inv)\right) = j_\alpha j_{\alpha\beta}(\omega) j_\beta j_{\beta\alpha}(\omega\inv) = j_\alpha j_{\alpha\beta}(1) = 1\]
by commutativity of the above diagram.

We will see that these elements generate the kernel of $\Phi$.

\begin{thm}[Van Kampen's Theorem]
	Let the space $(X,x_0)$ be the union of a collection $\{A_\alpha\}$ of open subsets each containing $x_0$ and such that each $A_\alpha$, $A_\alpha \cap A_\beta$ and $A_\alpha \cap A_\beta \cap A_\gamma$ is path connected.
	Then the kernel of the epimorphism
	\[\Phi: *_\alpha \pi_1(A_\alpha,x_0) \surjto \pi_1(X,x_0)\]
	is the normal subgroup generated by the set
	\[\left\{ j_{\alpha\beta}(\omega)j_{\beta\alpha}(\omega\inv) \mid \omega \in \pi_1(A_\alpha\cap A_\beta,x_0)\right\}\]	
\end{thm}

\begin{rmk}
	If each $A_\alpha \cap A_\beta$ is simply connected then $\Phi$ is an isomorphism.
\end{rmk}

\begin{proof}
	Omitted.
	See Hatcher.
\end{proof}

\begin{prop}
	Let $x_0 \in A \subset X$ and $y_0 \in Y$ with $A$ a deformation retract of $X$.
	Then $A \vee Y$ is a deformation retract of $X \vee Y$.
\end{prop}

\begin{proof}[Proof (sketch)]
	Take a homotopy $\{f_t: X \to X\}$ with $f_0: X \to A$ and $f_1 = \id_X$ and extend it to a homotopy $\{f_t': X \amalg Y \to X \amalg Y\}$.
	Now consider the diagram
	\[
	\begin{tikzcd}
		X \amalg Y \ar[r, "f_t'"] \ar[d] & X \amalg Y \ar[d] \\
		X \vee Y \ar[r, swap, "f_t''"] & X \vee Y
	\end{tikzcd}
	\]
	where the vertical arrows are the quotient maps.
	Then by definition of quotient (see the chapter 0 exercises) $f_t''$ is a homotopy.
	It is clear that $f_t''$ is also a deformation retraction.
\end{proof}

\begin{exam}
	In this example we will compute the fundamental group $\pi_1(S^1 \times S^1)$ of the figure-eight ($\infty$).
	Let $A$ and $B$ respectively be the figure-eight with a single point removed from the left and right loops respectively.
	Then
	\begin{align*}
		A &\cong \Dd \vee S^1 \simeq S^1 \\
		B &\cong S^1 \vee \Dd \simeq S^1 \\
		A \cap B &\cong \Dd^1 \vee \Dd^1 \simeq \pt
	\end{align*}
	So $\pi_1(A \cap B) = 1$, hence
	\[\pi_1(S^1 \vee S^1) \xleftarrow{\sim} \pi_1(A) * \pi_1(B) \cong \ZZ * \ZZ\]
	
	Then we have $S^1 \vee S^1$ is not homotopy equivalent to $S^1$, $S^1 \times S^1$, $\pt$ etc.
\end{exam}

2.4 Cell complexes

\begin{prop}
	Let $Y = X \amalg_\phi D^n$ and $\phi: \p D^n \to X$.
	Assume that $x$ is path connected and $n>2$.
	Then the inclusion $\iota: X \injto Y$ induces an \underline{isomorphism} of groups $\pi_1(X,x_0) \isoto \pi_1(Y,x_0)$.
\end{prop}

\begin{proof}
	Take $y \in Y \setminus X$.
	Let $A = Y \setminus \{y\}$ and $B = Y \setminus X \cong \Dd^n$.
	Then by corollary \ref{7:cell} we have an epimorphism
	\[\Phi: \pi_1(A,y_0) * \pi_1(B,y_0) \surjto \pi_1(Y,y_0)\]
	Now $y_0 \in A \cap B \cong \Dd \setminus \pt \simeq S^{n-1}$ where last homotopy equivalence comes from the deformation retraction of a ball minus the origin to a sphere contained within it.
	We know that $\pi_1(S^{n-1})=0$ for $n>2$, hence $\pi_1(B)=0$ and $\Phi$ is an isomorphism by van Kampen's theorem.
	It follows that $\pi_1(X,x_0) \cong \pi_1(Y,x_0)$.
\end{proof}

\begin{cor}
	For any path connected cell complex $X$ with dimension $n>2$, the inclusion $\iota: X^2 \to X$ of the 2-skeleton induces an isomorphism $\pi_1(X^2,x_0) \cong \pi_1(X,x_0)$ for all $x_0 \in X$.
\end{cor}

\begin{rmk}
	The fundamental group is good at telling things apart in lower dimensions but in higher dimensions it is not so good.
\end{rmk}

\begin{rmk}
	The fundamental group $\pi_1: \Topp \to \Groups$ is the first in a series of homotopy groups: for all $n \geq 2$ we have $\pi_n: \Topp \to \mathbf{Ab}$.
	However there is no nice higher-dimensional analogue of van Kampen's theorem.
\end{rmk}
