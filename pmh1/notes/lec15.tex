\section{2018-05-02 Lecture}

II Homology
1 Singular Homology

\begin{defn}
  Let $X$ be a topological space and $n \in \NN$.
  Then the group of \textbf{singular $n$-chains} $C_n(X)$ is the free abelian group ($\ZZ$-module) generated by the singular $n$-simplices in $X$.
\end{defn}

\begin{defn}
  A \text{singular $n$-simplex} is a continuous map $\sigma: \Delta^n \to X$.
\end{defn}

\begin{defn}
  The \textbf{boundary homomorphism} is defined as follows:
  \begin{align*}
    \p_n: C_n(X) &\to C_{n-1}(X) \\
    \sigma &\mapsto \sum_{i=1}^n (-1)^i \sigma|_{[v_0,\ldots,\hat v_i,\ldots,v_n]}
  \end{align*}
\end{defn}

\begin{rmk}
  The purpose of the $(-1)$ factor is to make sure everything `aligns' the right way to force the boundary of a boundary to be nothing.
  More formally, we have:
\end{rmk}

\begin{prop}
  We have $\p_n\p_{n+1}=0$, or equivalently $\im\p_{n+1} \subseteq \ker\p_n$.
  In general we do not have equality.
\end{prop}

\begin{defn}
  The \textbf{$n$th singular homology group} $H_n(X)$ is defined as
  \[H_n(X) = \ker\p_n/\im\p_{n+1}.\]
  We denote by $[a]$ the corresponding element in $H_n(X)$ for $a \in \ker\p_n$.
  The group $\ker\p_n$ is called the group of \textbf{$n$-cycles} and the group $\im\p_{n+1}$ is called the group of \textbf{$n$-boundaries}.
  They are both subgroups of $C_n(X)$.
\end{defn}

\begin{rmk}
  Define $\p_0: C_0(X) \to 0$ to be the zero map.
  Then the $C_n$ give a chain complex.
\end{rmk}

\begin{rmk}
  The homology groups are functorial: $H_n: \Top \to \Ab$ is a functor.
  For a map $f: X \to Y$, we get a group homomorphism
  \begin{align*}
    f_\sharp: C_n(X) &\to C_n(Y) \\
    \sigma &\mapsto f\sigma
  \end{align*}
  such that $\p_n f_\sharp = f_\sharp \p_n$ for all $n$.
  So the homology functors give us a homomorphism of chain complexes.
  Since $f_\sharp: C_n(X) \to C_n(Y)$ maps $\ker\p_n$ to $\ker\p_n$ and $\im\p_{n+1}$ and $\im\p_{n+1}$ we have (well-defined) group homomorphisms
  \begin{align*}
    f_*: H_n(X) &\to H_n(Y) \\
    [\alpha] \mapsto [f_\sharp(\alpha)]
  \end{align*}
\end{rmk}

\begin{exer}
  Show that $(fg)_* = f_*g_*$ and that $(1_X)_* = \id_{H_n(X)}$.
\end{exer}

\begin{prop}
  \leavevmode
  \begin{enum}
    \io $\ds H_n(X) \cong \bigoplus_{\alpha \in \pi_0(X)} H_n(X_0)$
    \io $\ds H_0(X) \cong \ZZ^{\oplus \#\pi_0(X)}$
  \end{enum}
\end{prop}

\begin{proof}
  Omitted.
  See lecture notes.
  (It was in the lecture but I can't be bothered typing it up.)
\end{proof}

\begin{exam}
  We have $H_n(\pt)=0$ for $n \geq 1$ and $H_0(\pt)=\ZZ$.
  
  For each $n$ we have one map $\sigma_n: \Delta^n \to \pt$, so $C_n(X)=\ZZ$ for all $n$.
  Now
  \[\p_n\sigma_n = \sum_{i=0}^n(-1)^i\sigma_{n-1} = (n \bmod 2)\]
  so the chain complex looks like
  \begin{equation*}
    \begin{tikzcd}
      \cdots \ar[r] & \ZZ \ar[r, "\id"] & \ZZ \ar[r, "0"] & \ZZ \ar[r, "\id"] & \ZZ \ar[r, "0"] & \ZZ \ar[r] & 0
    \end{tikzcd}
  \end{equation*}
  The fact that the homology groups of the point are not all zero can be annoying.
\end{exam}

\begin{defn}
  Define the \textbf{reduced homology groups} $\wt H_n(X)$ by replacing the $\p_0$ with the map $\eps: C_0(X) \to \ZZ$ defined by $\eps(x)=1$ for all singular 0-simplices (points) $x$.
  Then $\wt H_n(X) = H_n(X)$ for $n \geq 1$ and $\wt H_0(X) = \ker\eps/\im\p_1$.
  This corresponds to putting a $\ZZ$ between $C_0(X)$ and $0$ in our chain complex.
\end{defn}

\begin{rmk}
  We have $\wt H_n(X)=0$ for all $n$.
\end{rmk}

\begin{exer}
  Show that $\wt H_0(X) = \ZZ^{\oplus\#\pi_0(X)-1}$.
\end{exer}
