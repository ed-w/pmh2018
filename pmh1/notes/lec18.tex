\section{2018-05-14 Lecture}

\begin{thm}
  For a pair $(X,A)$ we have a long exact sequence
  \begin{equation*}
    \begin{tikzcd}
      \cdots \ar[r] & H_n(A) \ar[r, "i_*"] & H_n(X) \ar[r, "j_*"] & H_n(X,A) \ar[r, "\delta"] & H_{n-1}(A) \ar[r] & \cdots
    \end{tikzcd}
  \end{equation*}
  terminating with
  \begin{equation*}
    \begin{tikzcd}
      \cdots \ar[r] & H_0(X,A) \ar[r] & 0
    \end{tikzcd}
  \end{equation*}
\end{thm}

\begin{rmk}
  Every short exact sequence of chain complexes
  \begin{equation*}
    \begin{tikzcd}
      0 \ar[r] & A_\bullet \ar[r] & B_\bullet \ar[r] & C_\bullet \ar[r] & 0
    \end{tikzcd}
  \end{equation*}
  gives rise to a long exact sequence of homology groups:
  \begin{equation*}
    \begin{tikzcd}
      \cdots \ar[r] & H_n(A_\bullet) \ar[r] & H_n(B_\bullet) \ar[r] & H_n(C_\bullet) \ar[r, "\delta"] & H_{n-1}(A_\bullet) \ar[r] & \cdots
    \end{tikzcd}
  \end{equation*}
  terminating with
  \begin{equation*}
    \begin{tikzcd}
      \cdots \ar[r] & H_0(C_\bullet) \ar[r] & 0
    \end{tikzcd}
  \end{equation*}
\end{rmk}

If we know the homology groups $H_n(A)$ and the relative homology groups $H_n(X,A)$, the homology groups $H_n(X)$ are more or less determined.

We should think of relative homology as a functor from $\Pair$ to $\Comp$.

2.2 Induced morphisms and naturality

\begin{defn}
  Let
  \[ f: (X,A) \to (Y,B) \]
  be a map of pairs.
  That is, $f:X \to Y$ is a map of topological spaces with $f[A] \subset B$.
  This induces group homomorphisms
  \[ f_*^r: H_n(X,A) \to H_n(Y,B). \]
  (This can be proved analogously to the case of normal homology.)
\end{defn}

\begin{defn}
  A homotopy of pairs
  \[ \{ f_t: (X,A) \to (Y,N) \} \]
  is a homotopy $\{ f_t: X \to Y \}$ with $f_t[A] \subset B$ for all $t$.
\end{defn}

\begin{exam}
  \[ \{ f_t: (I, \p I) \to (Y, \{y_0\}) \} \]
  is a homotopy of loops.
\end{exam}

\begin{rmk}
  A homotopy equivalence $f: (X,A) \to (Y,A)$ induces an isomorphism $f_*^r: H_n(X,A) \to H_n(Y,B)$.
\end{rmk}

\begin{rmk}[Naturality]
  Let $f: (X,A) \to (Y,B)$ be a chain map with $f^0 = f|_A: A \to  B$.
  Then the following diagram commutes:
  \begin{equation*}
    \begin{tikzcd}
      \cdots \ar[r] & H_n(A) \ar[d, "f_*^0"] \ar[r, "i_*"] & H_n(X) \ar[d, "f_*"] \ar[r, "j_*"] & H_n(X,A) \ar[d, "f_*^r"] \ar[r, "\delta"] & H_{n-1}(A) \ar[d, "f_*^0"] \ar[r] & \cdots \\
      \cdots \ar[r] & H_n(B) \ar[r, "i_*"] & H_n(Y) \ar[r, "j_*"] & H_n(Y,B) \ar[r, "\delta"] & H_{n-1}(B) \ar[r] & \cdots
    \end{tikzcd}
  \end{equation*}
\end{rmk}

2.3 Barycentric subdivision.

\begin{defn}
  For a collection $\ca U = \{U_\alpha\}$ of subspaces of $X$, the subgroup $C_n^{\ca U}(X)$ of $C_n(X)$ is generated by all the subgroups $C_n(U_\alpha)$ of $C_n(X)$.
  Then the inclusion $i^{\ca U}: C_n^{\ca U}(X) \to C_n(X)$ is a chain map.
\end{defn}

\begin{prop}
  If $X = \bigcup_\alpha \inte U_\alpha$, then $i_*^{\ca U}: H_n^{\ca U} \to H_n(X)$ is an isomorphism.
\end{prop}

\begin{proof}[Proof outline]
  This can be proven using barycentric subdivision.
  
  Let $\sigma: \Delta^1 \to X$ be arbitrary.
  Then by compactness of $\Delta^1$, it is contained in finitely many preimages of the $U_\alpha$ under $\sigma$.
  Assume that it is contained in two preimages.
  Then we can divide the interval $\Delta^1$ in to two smaller intervals and we can replace the 1-simplex with two smaller 1-simplices which are the restriction to the two intervals.
  To do this, we can take a 2-simplex by turning the divided interval in to a triangle with an altitude going through the divided point.
  Then we can define the 2-simplex structure on the triangle by projecting down on to the original interval and then composing with $\sigma$.
  Then this shows that the original 1-simplex is the sum of the two sub-intervals in the quotient by $\im\p_2$.

  For a description of the barycentric subdivision procedure, see the notes.

  We have a group homomorphism
  \begin{align*}
    S: C_n(X) &\to C_n(X) \\
  \sigma &\mapsto \sum_\alpha (-1)^{s_\alpha} \sigma\alpha
  \end{align*}
  where the sum is over all $(n+1)!$ inclusion maps $\alpha: \Delta^n \to \Delta^n$ in the barycentric subdivision.
  where the signs $(-1)^{s_\alpha}$ are chosen to make $S\p=\p S$.
\end{proof}

We define a map $T: C_n(X) \to C_{n+1}(X)$ based on the map $\gamma: \Delta^{n+1} \to \Delta^n$ which generalises the orthogonal projection map $\Delta^2 \to \Delta^1$.
\[ \gamma(t_0,\ldots,t_{n+1}) = t_0b + (t_1,\ldots,t_n) \]
where $b = (1/(n+1),\ldots,1/(n+1))$.
