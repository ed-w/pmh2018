\section{2018-06-06 Lecture}

5.4 Projective space

Let $k$ be a field.
Then $k\PP^n = (k^{n+1} \setminus \{0\})/\sim$ where $v \sim \lambda v$ for all $\lambda \in k^\times$.
This is the set of all lines through the origin.
As sets, we have $k\PP^N = k^n \amalg k^{n-1} \amalg \cdots \amalg k \amalg \emptyset$ where $k^i$ consists of vectors with the first $i-1$ components zero and the $i$th component 1.
For $k=\RR$ or $k=\CC$, $k^{n+1}$ is a toplogical (metric) space so $k\PP^n$ is a topological space with the quotient topology.

\begin{lem}
  $\RR\PP^k$ is a $k$-dimensional cell complex with exactly one $j$-cell for each $0 \leq j \leq k$.
\end{lem}

\begin{proof}
  We have $\RR\PP^k = (\RR^{k+1}\setminus\{0\})/\sim\ = S^k/\sim$ where $v \sim -v$ (identifying antipodal points).
  Let $q_k: S^k \to \RR\PP^k$ be the quotient map.
  Define the hemisphere $hS^k = \{ v \in S^k \mid v_1 \geq 0 \} \subset S^k$.
  Then $\RR\PP^k = hS^k/\sim$ where $\sim$ is the restriction of the relation $\sim$ on $S^k$.
  Now this equivalence relation is only relevant on the boundary $\p hS^k = S^{k-1}$ where it identifies antipodal points.
  Then $\RR\PP^k \cong S^{k-1}/\sim \sqcup_q hS^k$ where $q_k: S^{k-1} \to S^{k-1}/\sim = \RR\PP^k$ is the quotient map.
  Therefore $\RR\PP^k \cong \RR\PP^{k-1} \sqcup_{q_{k-1}} D^k$.
  This gives a cell complex structure for $\RR\PP^k$.
  Since $\RR\PP^0=\pt$, this gives the result.
\end{proof}

\begin{exam}
  $\RR\PP^1 \cong S^1$.
\end{exam}

\begin{rmk}
  $q_1: S^1 \to \RR\PP^1 \cong S^1$ is not a homeomorphism (it is not bijective).
\end{rmk}

An intuitive look at $\pi_1(\RR\PP^2)$

This can be computed using van Kampen's theorem.
It is $\ZZ/2\ZZ$.
So we expect that $H_1(\RR\PP^2)=\ZZ/2\ZZ$.

Cellular homology of $\RR\PP^2$

The cellular chain complex is
\begin{equation*}
  \begin{tikzcd}
    \cdots \ar[r] & 0 \ar[r] & \ZZ \ar[r, "\p_2^{CW}"] & \ZZ \ar[r, "\p_1^{CW}"] & \ZZ \ar[r] & 0
  \end{tikzcd}
\end{equation*}
Now $\p_1^{CW}(e^1) = \ph(1)-\ph(-1) = 0$ since there is only one $0$-cell.
Also $p_2^{CW}(e^2) = (\deg f)e^1$ where $f: S^1 \xto{q_1} \RR\PP^1 \isoto S^1$ and $\deg f = f_*(1)$.
It is clear that $\deg f = 2$.
Then $H_0(\RR\PP^2)=\ZZ$, $H_1(\RR\PP^2)=\ZZ/2\ZZ$ and $H_n(\RR\PP^2)=0$ for $n \geq 2$.

Note that this is the first place where we have encountered torsion in a homology group.
This is because the attaching map has degree 2, so part of the boundary maps on to itself.
Previously we only had attaching maps with degree 0 (collapsing to a point) and degree 1 (preserving the boundary).

Homology of $\RR\PP^3$

The cellular chain complex is
\begin{equation*}
  \begin{tikzcd}
    \cdots \ar[r] & 0 \ar[r] & \ZZ \ar[r, "0"] & \ZZ \ar[r, "2"] & \ZZ \ar[r, "0"] & \ZZ \ar[r] & 0
  \end{tikzcd}
\end{equation*}
We can deduce that $\p_3^{CW}$ is the zero map from the fact that this is a chain complex.
This alternating pattern generalises but is harder to compute.
So we have
\begin{equation*}
  H_n(\RR\PP^3)=
  \begin{cases}
    \ZZ & n=0 \\
    \ZZ/2\ZZ & n=1 \\
    0 & n=2 \\
    \ZZ & n=3 \\
    0 & n \geq 4
  \end{cases}
\end{equation*}

\begin{lem}
  $\CC\PP^k$ is a $2k$-dimensional cell complex with 1 cell of each even dimension $\leq 2k$ and 0 cells of each odd dimension $<2k$.
\end{lem}

This is because the complex plane is two-dimensional over the reals and the disks used in the construction of cell complexes are real.
This allows us to immediately compute the homology groups:

\begin{cor}
  \begin{equation*}
    H_n(\CC\PP^k)=
    \begin{cases}
      \ZZ & \text{ if $n$ is even and $n \leq 2k$} \\
      0 & \text{ if $n$ is odd or $n > 2k$}
    \end{cases}
  \end{equation*}
\end{cor}
