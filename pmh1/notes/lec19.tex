\section{2018-05-16 Lecture}

Last lecture we defined a chain map $S: C_n(X) \to C_n(X)$ based on the barycentric subdivision of $\Delta^n$ in to $(n+1)!$ smaller copies of $\Delta^n$.
We have a collection of group homomorphisms $T: C_n(X) \to C_{n+1}$ based on $\gamma: \Delta^{n+1} \to \Delta^n$ (an example of which is the projection of a triangle on to one of its faces).
We have $\p T+T\p = 1-S$, hence $S_*=1_*=\id_{H_n(X)}$.
More generally, the map $T_m = \sum_{i=0}^{m-1} TS^i$ satisfies $\p T_k+T_k\p = 1-s^k$.

Motivation for the above:
Let $\cU = \{ U_\alpha \subset X\}$ be a collection of subsets of $X$, and define the subgroup $C_n^\cU(X) = \sum_\alpha X_n(U_\alpha)$ of $C_n(X)$.
Then the inclusion map $i^\cU: C_n^\cU(X) \injto C_n(X)$ induces a map $i_*^\cU: H_n^\cU(X) \to H_n(X)$.

\begin{prop}
  If $X = \bigcup_\alpha \inte U_\alpha$, then there is a chain map $\rho:C_n(X) \to C_n^\cU(X)$ such that $\rho \circ i^\cU = \id_{C_n^\cU}$ and group homomorphisms $D: C_n(X) \to C_{n+1}(X)$ such that $\p D+D\p = 1_{C_n(X)} - i^\cU \circ \rho$, and $D|_{C_n^\cU(X)}=0$.
\end{prop}

\begin{cor}
  As a consequence, we have $\rho_* \circ i_*^\cU = \id_{H_n^\cU(X)}$ and $i_*^\cU \circ \rho_* = \id_{H_n(X)}$.
  Therefore $i_*^\cU: H_n^\cU(X) \isoto H_n(X)$ is an isomorphism.
\end{cor}

\begin{proof}
  For each singular $n$-simplex $\sigma: \Delta^n \to X$, set $k(\sigma) = \argmin_n S^{n}(\sigma) \in C_n^\cU(X)$.
  The Lebesgue covering lemma tells us that there exists an $k$ such that $S^k(\sigma) \in C_n^\cU(X)$.
  Define
  \begin{align*}
    D: C_n(X) &\to C_{n+1}(X) \\
    \sigma &\mapsto T_{k(\sigma)}(\sigma)
  \end{align*}
  Clearly $D|_{C_n^\cU}(X)=0$.
  Now define $\rho' = \id_{C_n(X)} - \p D-D\p$.
  We will show that $\rho' i^\cU = i^\cU$ and that $\rho'=i^\cU\rho$ for some $\rho: C_n(X) \to C_n^\cU(X)$.
  The first claim is clear.

  We now prove the second claim.
  Let $\sigma: \Delta^n \to X$ be arbitrary.
  Then
  \begin{align*}
    \rho'(\sigma) &= \sigma -\p T_{k(\sigma)}(\sigma) - D\p\sigma \\
    &= \rho - (\rho - S^{k(\sigma)}\sigma - T_{k(\sigma)}\p\sigma) _ D\p\sigma \\
    &= S^{k(\sigma)}\sigma + (T_{k(\sigma)}-D)\p\sigma
  \end{align*}
  For a face $\gamma$ in $\sigma$, we have $k(\gamma) \leq k(\sigma)$.
  Hence $\rho'(\sigma)$ is a $S^{k(\sigma)}\sigma$ plues a linear combination of terms $(T_{k(\sigma)}-T_{k(\gamma)})(\gamma)$, each of which equals $\sum_{i=k(\gamma)}^{k(\sigma)-1} TS^i\gamma$.
  Each of the terms are in $C_n^\cU(X)$, hence $\im\rho' \subseteq C_n^\cU(X)$ and so $\rho'$ factors through the inclusion $i^\cU$.
\end{proof}

2.4 Excision

\begin{thm}[Excision]
  \lv
  \begin{enum}
    \io
    For a pair $(X,A)$ and a subset $Z \subset X$ with $\ol Z \subset \inte A$, the inclusion $\iota: (X \setminus Z, A \setminus Z) \injto (X,A)$ induces isomorphisms $\iota_*^n: H_n(X \setminus Z, A \setminus Z) \isoto H_n(X,A)$.
    
    \io
    For subsets $A,B \subset X$ with $\inte A \cup \inte B = X$, the inclusion map $\iota: (B,A \cap B) \injto (X,A)$ induces an isomorphism $\iota_*^n: H_n(B, A \cap B) \isoto H_n(X,A)$.
  \end{enum}
\end{thm}

\begin{rmk}
  The two statements are equivalent (use that if $Z = X \setminus B$ then $\ol Z = X \setminus \inte B$).
\end{rmk}

\begin{proof}
  \textbf{fill in}
\end{proof}
