\section{2018-05-28 Lecture}

\begin{exam}[Examples of $\Delta$-complex structures]
  \lv
  \begin{enum}
    \io $\square^2$: Square plus long diagonal: 2 2-simplices, 5 1-simplices, 4 0-simplices.
    \io $S^1 \times S^1$: $\square^2/\sim$ where we identify opposite edges. 2 2-simplices, 3 1-simplices, 1 0-simplex.
  \end{enum}
\end{exam}

For a $\Delta$-complex $X$ we have subgroups $\Delta_n(X) \leq C_n(X)$ generated by the $\{ \sigma_\alpha: \Delta^n \to X \}$ in the definition of its $\Delta$-complex structure.
Then the boundary homomorphism restricts to $\p: \Delta_n(X) \to \Delta_{n-1}(X)$, giving simplicial homology groups $H_n^\Delta(X) = H_n(\Delta_\bullet(X))$.

\begin{exam}[Examples of simplicial homology]
  \lv
  \begin{enum}
    \io
    $\square^2$:
    \begin{equation*}
      \begin{tikzcd}
	\cdots \ar[r] & 0 \ar[r] & \ZZ^2 \ar[r] & \ZZ^5 \ar[r] & \ZZ^4 \ar[r] & 0
      \end{tikzcd}
    \end{equation*}
    so
    \begin{equation*}
      H_n^\Delta(\square^2)=
      \begin{cases}
	\ZZ & n=0 \\
	0 & n>0
      \end{cases}
    \end{equation*}

    \io
    $S^1 \times S^1$:
    \begin{equation*}
      \begin{tikzcd}
	\cdots \ar[r] & 0 \ar[r] & \ZZ^2 \ar[r] & \ZZ^3 \ar[r] & \ZZ \ar[r] & 0
      \end{tikzcd}
    \end{equation*}
    Then $H_0^\Delta=\ZZ$, $H_1^\Delta=\ZZ^2$, $H_2^\Delta=\ZZ$ and $H_n^\Delta=0$ for all $n \geq 3$.
  \end{enum}
\end{exam}

\begin{rmk}
  Since $\square^2 \cong \Delta^2$, we can compute the simplicial homology of $\square^2$ another way.
\end{rmk}

The inclusions $\Delta_n(X) \injto C_n(X)$ induce a chain map $H_n^\Delta(X) \to H_n(X)$.

\begin{thm}\label{22:equiv}
  The inclusion map induces an isomorphism $H_n^\Delta(X) \isoto H_n(X)$.
\end{thm}

\begin{defn}
  For a $\Delta$-complex $X$, a subspace $A \subseteq X$ is a \textbf{subcomplex} if it is the union of some $\sigma_\alpha(\Delta^{n_\alpha})$.
  Then $\Delta_n(X,A) = \Delta_n(X)/\Delta_n(A)$ and $H_n^\Delta(X,A)=H_n(\Delta_\bullet(X,A))$.
\end{defn}

This gives another map $H_n^\Delta(X,A) \to H_n(X,A)$ induced by the relative inclusion map.

5.2 Equivalence of simplicial and singular homology

\begin{proof}[Proof of theorem \ref{22:equiv}]
  We first consider the special case where $X=\Delta^n$ with the obvious $\Delta$-complex structure (inclusion maps) and set $A=\p\Delta^n$ (the union of all the $(n-1)$-simplices).
  Since $(\Delta^n,\p \Delta^n) \cong (S^n,\p S^n)$, we have
  \begin{equation*}
    H_k(\Delta^n,\p\Delta^n)=
    \begin{cases}
      \ZZ & k=n \\
      0 & k \neq n
    \end{cases}
  \end{equation*}
  We will now show that the map $\id_{\Delta^n}: \Delta^n \to \Delta^n$ yields a generator of $H_n(\Delta^n,\p\Delta^n)$ as $[\id_{\Delta^n}+C_n(\p\Delta^n)] \in H_n(\Delta^n,\p\Delta^n)$.

  Since $\p_n^r(\id_{\Delta^n}+C_n(p\Delta^n))=0+C_{n-1}(\p\Delta^n)$, we have that it is in the homology group.
  That it generates is clear for $n=0$.
  We can prove the remaining cases by induction.
  Let $\Lambda$ be the boundary of one face.
  Then the triple $(\Delta^n,\p\Delta^n,\Lambda)$ gives a s.e.s:
  \begin{equation*}
    \begin{tikzcd}
      0 \ar[r] & C_\bullet(\p\Delta^n,\Lambda) \ar[r] & C_\bullet(\Delta^n,\Lambda) \ar[r] & C_\bullet(\Delta^n,\p\Delta^n) \ar[r] & 0
    \end{tikzcd}
  \end{equation*}
  Then we have a sequence of isomorphisms:
  \[ H_n(\Delta^n,\p\Delta^n) \isoto H_{n-1}(\p\Delta^n,\Lambda) \isoto \wt H_{n-1}(\p\Delta^{n-1}/\Lambda) \isoto \wt H_{n-1}(\Delta^{n-1}/\p\Delta^{n-1}) \isoto H_{n-1}(\Delta^{n-1},\p\Delta^{n-1}) \]
  Then following the maps gives the claim.

  I didn't type up the rest of the proof.
\end{proof}
