\section{2018-03-21 Lecture}

1.4 Induced homomorphisms

\begin{rmk}
	$\pi_1: \Topp \to \Groups$ is a functor.
\end{rmk}

\begin{defn}
	For $\phi: (X,x_0) \to (Y,y_0)$, we have the induced group homomorphism
	\begin{align*}
		\phi_*: (X,x_0) &\to \pi_1(Y,y_0) \\
		[f] &\mapsto [\phi f]
	\end{align*}
\end{defn}

\begin{exer}
	Show that $\pi_1$ is a functor, that is:
	\begin{enum}
		\io $\phi_*$ is well defined and is a group homomorphism.
		\io $(1_X)_*$ is the identity morphism in $\pi_1(X,x_0)$.
		\io $(\psi\phi)_*=\psi_*\phi_*$ for $\psi: (Y,y_0) \to (Z,z_0)$.
	\end{enum}
\end{exer}

\begin{thm}\label{thm:iso}
	If $\phi: X \to Y$ is a homotopy equivalence, then $\phi_*: \pi_1(X,x_0) \to \pi_1(Y,y_0)$ is a group isomorphism.
\end{thm}

\begin{cor}
	Applications:
	\begin{itm}
		\io $S^1$ is not contractible (since $\pi_1(S^1) = \ZZ$ and $\pi_1(\pt)=0$)
		\io The torus: $\pi_1(T^2) = \pi_1(S^1 \times S^1) = \pi_1(S^1) \times \pi_1(S^1) = \ZZ \times \ZZ$
	\end{itm}
\end{cor}

\begin{rmk}
	In category theory we also write $\pi_1(\phi)$ for $\phi_*$ to state the functor more explicitly.
\end{rmk}

\begin{prop}
	Let $(X,x_0)$ be a pointed space and $\{f_t:X \to Y\}$ be a homotopy between maps $f_0$ and $f_1$.
	Let $h: I \to X$ be a path with $h(t)=f_t(x_0)$ for all $t \in I$.
	Then the following diagram commutes:
	\[
		\begin{tikzcd}
			& \pi_1(Y,f_1(x_0)) \ar[dd, "\beta_h"] \\
			\pi_1(X,x_0) \ar[ur, "{f_1}_*"] \ar[dr, "{f_0}_*"] & \\
			& \pi_1(Y,f_0(x_0))
		\end{tikzcd}
	\]
	where $\beta_h$ is the change of basepoint map (recall that it is an isomorphism).
\end{prop}

\begin{proof}
	Let $g: I \to X$ be a loop at $x_0$.
	We have
	\[
	\begin{tikzcd}
		 \left[g\right] \ar[r, mapsto, "{f_1}_*"] & \left[f_1g\right] \ar[r, mapsto, "\beta_h"] & \left[h \cdot f_1g \cdot \ol{h}\right]
		\end{tikzcd}
	\]
	and
	\[
		\begin{tikzcd}
			\left[g\right] \ar[r, mapsto, "{f_0}_*"] & \left[f_0g\right]
		\end{tikzcd}
	\]
	diagram here
	
	Now for all $x \in I$, define $h_s: I \to Y$ by
	\[h_s(t)=h(st)=f_{st}(x_0)\]
	Then
	\[\{h_s \cdot (f_1g) \cdot \ol{h_s}: I \to Y\}\]
	is a homotopy.
\end{proof}

\begin{proof}[Proof of theorem \ref{thm:iso}]
	Take $\psi: Y \to X$ with $\psi\phi \simeq 1_X$ and $\phi\psi \simeq 1_Y$.
	Then $\psi_*\phi_* = (\psi\phi)_* = \beta_h$ and $\phi_*\psi_* = \beta_{h'}$ for some paths $h$ and $h'$.
	Now $\beta_h$ and $\beta_{h'}$ are isomorphisms, hence $\psi_*$ and $\phi_*$ are also isomorphisms.
\end{proof}

\begin{prop}
	Let $A$ be a retract of $X$.
	Then the inclusion $\iota: A \injto X$ induces a monomorphism for $\iota_*: \pi_1(A,x_0) \injto \pi_1(X,x_0)$ for all $x_0 \in A$.
\end{prop}

\begin{proof}
	Let $r: X \to A$ be a retract (with $r|_A = 1_A$).
	Then $r\iota = 1_A$, hence $r_*\iota_*=(r\iota)_*=(1_A)_*$.
\end{proof}

\begin{rmk}
	If $A$ is a deformation retract, then $\iota_*$ is a isomorphism.
	This follows from theorem \ref{thm:iso}.
\end{rmk}

\begin{lem}\label{lem:loop-prod}
	Let $(X,x_0)$ be a pointed space and let $\{A_\alpha\}$ be a collection of open subsets (whose union is $X$)\footnote{This additional condition was added in the next lecture.}.
	If $x_0 \in A_\alpha$ and $A_\alpha \cap A_\beta$ is path connected for all $\alpha$ and $\beta$, then each loop in $X$ at $x_0$ is homotopic to a finite product (composition) of loops with each one contained in some $A_\alpha$.
\end{lem}

\begin{proof}
	See Hatcher.
	The technical details are not very important but the lemma is intuitively obvious.
	(The finiteness is due to compactness.)
\end{proof}