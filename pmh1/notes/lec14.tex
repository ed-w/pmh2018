\section{2018-04-30 Lecture}

Homology

Comparison of the fundamental group and homology

\begin{tabular}[]{| c | c |}
  \hline
  Fundamental Group & Homology \\
  \hline
  $X \mapsto \pi_1(X) \in \Groups$ & $X \mapsto H_n(x) \in \Ab$ for all $n \in \NN$ \\
  Computable & Computable \\
  Functorial & Functorial for all $n$ \\
  Homeomorphisms $\leftrightarrow$ Isomorphisms & Homeomorphisms $\leftrightarrow$ Isomorphisms \\
  $S^n \ ? \ S^m$ for $m \neq n$ and $m,n \geq 2$ & $S^m \not\simeq S^n$ for $m \neq n$ and $m,n \geq 2$ \\
  \hline
\end{tabular}

So homology is a stronger topological invariant than the fundamental group.
Note that there are also higher homotopy groups $\pi_n$ but these are hard to compute.

Simplicial Homology

\begin{defn}
  An \textbf{$n$-simplex} is an ordered set $[v_0,\ldots,v_n]$ of vectors in $\RR^n$ such that the vectors $v_1-v_0,\ldots,v_n-v_0$ are linearly independent.
  We will also abuse notation to use the term $n$-simplex to refer to the convex hull of the $n$-simplex:
  \[\left\{ \sum t_nv_i \mid \sum t_i=1 \text{ and } t_i \geq 0 \text{ for all } n \right\} \subset \RR^m\]
\end{defn}

\begin{defn}
  The standard $n$-simplex $\Delta^n \subset \RR^{n+1}$ is the $n$-simplex with vectors
  \[v_i = (\overset{0}{0},\overset{1}{0},\ldots,\overset{i-1}{0},\overset{i}{1},\overset{i+1}{0},\ldots,\overset{n}{0})\]
  Its convex hull is the set
\[ \left\{ (t_0,\ldots,t_n) \in \RR^{n+1} \mid \sum t_i=1 \text{ and } t_i \geq 0 \text{ for all } i \right\} \]
\end{defn}

\begin{exer}
  Show that:
  \begin{enum}
    \io Any triangle in $\RR^2$ is a $2$-simplex.
    \io No quadrilateral in $\RR^2$ is a $3$-simplex.
    \io Every $n$-simplex is homeomorphic to the standard $n$-simplex.
  \end{enum}
\end{exer}

\begin{defn}
  Let $[v_0,\ldots,v_n]$ be an $n$-simplex.
  A \textbf{face} is obtained by omitting any one point from the vector: $[v_0,\ldots,\wh v_i,\ldots,v_n]$ (the $i$-face).
  The (topological) \textbf{boundary} $\p \Delta^n$ of an $n$-simplex $\Delta^n$ is the union of its faces.
  Denote by $\mr\Delta^n$ the interior $\Delta^n \setminus \p\Delta^n$ of $\Delta^n$.
\end{defn}

\begin{rmk}
  The faces of an $n$-simplex are $(n-1)$-simplices.
\end{rmk}

\begin{defn}
  A simplicial complex is a topological space that consist of (finitely many) simplices glued together along edges.
  Representing a space as a simplicial complex is called triangulation.
\end{defn}

\begin{exam}
  The $2$-sphere can be triangulated with a tetrahedron (three dimensions).
  For the cylinder, if we simply cut the square along the main diagonal, we get a `pseudo-triangulation' since two vertices of each triangle get identified when `glued' together.
  It turns out simplicial homology still works for these pseudo-triangulations but to give a proper triangulation we should first split the square in half (along the axis of the cylinder) and then cut diagonals to get four triangles.
  Likewise for the torus we can give a pseudo-triangulation consisting of the square cut along the long diagonal.
\end{exam}

\begin{exam}[Simplicial homology of the torus]
  We have a chain complex
  \begin{equation*}
    \begin{tikzcd}
      0 \ar[r] & \ZZ^{\# \Delta^2} \ar[r, "\p_2"] & \ZZ^{\#\Delta^1} \ar[r, "\p_1"] & \ZZ^{\#\Delta^0} \ar[r] & 0
    \end{tikzcd}
  \end{equation*}
  where the boundary map $\p$ is defined by
  \[\p_n(\Delta_{[v_0,\ldots,v_n]}) = \sum_{i=0}^n (-1)^i \Delta_{[v_0,\ldots,\wh v_i,\ldots,v_n]}\]
  Since $\p^2 = \p_n \circ \p_{n+1} = 0$, we can define the homology groups:
  \[H_n(X) = \ker\p_n/\im\p_{n+1}\]
\end{exam}
