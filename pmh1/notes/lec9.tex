\section{2018-04-09 Lecture}

Covering spaces

Consider the covering space $p_n: S^1 \to S^1$ defined by $z \mapsto z^n$.
This is corresponds to winding the circle over itself $n$ times.
Think of it as a helix with $n$ turns and the two free ends joined (but there are no self-intersections).

We know that a continuous map between spaces induces a group homomorphism on the fundamental groups.
Consider the covering space $p_n: S^1 \to S^1$.
The generator ($1$) of the covering space is just the loop which goes around the covering circle once, that is, it goes around the helix $n$ times.
Then the map $p_n$ projecting it down ot the circle makes it in to a $n$-fold loop in $\pi_1(S^1)$.
Therefore $p_n$ induces the group homomorphism $\ZZ \to n\ZZ$ for $n \neq 1$.

Now consider the infinite helix (the covering space we have already looked at).
It is homeomorphis to $\RR$ and so has trivial fundamental group.
Therefore the covering space map induces the zero homomorphism.

Note that any space is a covering space of itself with the trivial map.
In this case this is just $p_1$.
Then the induced homomorphism is the identity, so its image subgroup is the whole group $\pi_1(S^1)$.

On the other extreme, the infinite helix is simply connected and so is known as a \textbf{universal covering space}.
It contains any connected covering space, and its induced homomorphism is zero, so its image subgroup is the trivial subgroup.

We will also see that the more sheets a covering space has, the smaller its associated subgroup is.

\begin{thm}[Classification of covering spaces/Galois correspondence]
	If $X$ is a `nice enough' space, then there is a bijection between connected coverings of $(X,x_0)$ and subgroups of $\pi_1(X,x_0)$ given by
	\[\begin{tikzcd}[row sep=tiny]
		(\wt{x},\wt{x}_0) \ar[dd, swap, "p"] & \\
		\ar[r, mapsto] & p_*\left(\pi_1(\wt X,\wt x_0)\right) \leq \pi_1(X,x_0) \\
		(X,x_0) &
	\end{tikzcd}\]
\end{thm}

\begin{rmk}
	This is similar to the fundamental theorem of Galois theory.
\end{rmk}

Review of the basic definitions and facts

\begin{defn}
	A \textbf{covering space} of $X$ is a space $\wt X$ and a map $p: \wt X \to X$ such that every $x \in X$ contains an open neighbourhood $U$ such that
	\[p\inv[U] = \biguplus_\alpha V_\alpha\]
	where each $V_\alpha$ is open in $\wt X$ and $p|_{V_\alpha}: V_\alpha \to U$ is a homeomorphism for each $\alpha$.
\end{defn}

\begin{exer}
	Show that:
	\begin{itm}
		\io $p\inv[x]$ is a discrete set in $\wt X$.
		\io For each $x$ and $U$, the number of sheets is the same as $\#p\inv[x]$ (or there is a bijection in the infinite case).
		\io If $X$ is connected then this cardinality is the same for all $x$ in $X$.
	\end{itm}
\end{exer}

\begin{prop}
	For all paths $f \in X$ starting at $x_0$ and for all $\wt x_0 \in p\inv[x_0]$, there exists a unique lift $\wt f$ in $\wt X$ (that is $p\wt f=f$).
	Moreover, $f \simeq g$ as paths in $X \iff \wt f \simeq \wt g$ as paths in $\wt X$.
\end{prop}

\begin{prop}
	$p_*$ is always injective when $p$ is a covering map.
\end{prop}

\begin{proof}
	If $\wt f \not\simeq \wt g$ then $p\wt f \not\simeq p\wt g$.
\end{proof}

\begin{exer}
	Let $f$ be a loop at $x_0$ in $X$.
	$f \in \im p_* \iff$ the unique lift $\wt f$ is a loop.
\end{exer}

\begin{prop}
	Let $p:(\wt X,\wt x_0) \to (X,x_0)$ be a covering space.
	If $\wt X$ and $X$ are path connected, then there is a bijection between the elements of $p\inv(x_0)$ and the right cosets of $p_*(\pi_1(\wt X, \wt x_0))$ given by
	\begin{align*}
		\Phi: p_*(\pi_1(\wt X,\wt x_0))\backslash\pi_1(X,x_0) &\to p\inv[x_0] \\
		p_*(\pi_1(\wt X,\wt x_0))[g] &\mapsto \wt g(1)
	\end{align*}
	where $\wt g(1)$ is the end point of the (unique) lift of $g$ with start point $\wt x_0$.
\end{prop}
