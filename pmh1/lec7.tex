\section{2018-03-26 Lecture}

\begin{cor}
	Let $n \geq 2$, $\phi:S^{n-1} \to X$ and $Y = X \sqcup_\phi D^n$ where $X$ is path connected.
	Then the inclusion $i:X \injto Y$ induces an epimorphism $\iota_*: \pi_1(X,x_0) \surjto \pi_1(Y,x_0)$ for all $x_0 \in X$.
\end{cor}

\begin{proof}
	Let $y$ be a point in $Y \setminus X$.
	Set $A = Y \setminus \{y\}$ and $B = Y \setminus X \cong \Dd^n$.
	Then $Y = A \cap B$.
	Also $A \cap B \cong \Dd^n \setminus \pt$ and so it is path connected (since $n \geq 2$.
	This is the only place where this condition is used).
	
	Now let $y_0 \in A \cap B$.
	Since $B$ is simply connected, applying lemma \ref{lem:loop-prod} shows that any loop in $(Y,y_0)$ is homotopic to a loop in $(A,y_0)$.
	Thus the inclusion $A \injto Y$ induces an epimorphism $\pi_1(A,y_0) \surjto \pi_1(Y,y_0)$.
	Let $h$ be a path in $A$ from $x_0$ to $y_0$.
	Then since the change of basepoint map is an isomorphism, the diagram commutes
	\[
	\begin{tikzcd}
		\pi_1(A,y_0) \ar[r, twoheadrightarrow] \ar[d, "\beta_h"] & \pi_1(Y,y_0) \ar[d, "\beta_h"] \\
		\pi_1(A,x_0) \ar[r, twoheadrightarrow] & \pi_1(Y,x_0)
	\end{tikzcd}
	\]
	Now $Z \defeq D^n \setminus \{y\}$ has deformation retract $\p D^n$.
	Then since deformation retracts are homotopy equivalences, we have that the composition
	\[\pi_1(X,x_0) \isoto \pi_1(A,x_0) \surjto \pi_1(Y,x_0)\]
	is epic.
\end{proof}

\begin{prop}
	$\pi_1(S^n)=0$ for $n>1$.
\end{prop}

\begin{proof}
	Recall that $S^n \cong D^n/\p D^n \cong \pt \sqcup_\phi D^n$ where $\pt$ `contains' the points of $\p D^n$.
	Then the inclusion $\pt \injto S^n$ induces an epimorphism $0=\pi_1(\pt) \surjto \pi_1(S^n)$.
\end{proof}

2 Van Kampen's Theorem

\begin{defn}
	Let $\{G_\alpha\}_{\alpha}$ be a set of groups.
	\begin{itm}
		\io A \textbf{word} is a finite (and possibly empty) product $g_1g_2\cdots g_k$ with $\gamma_i \in G_{\alpha_i}$.
		\io A word is \textbf{reduced} if $g_i \neq 1_{G_{\alpha_i}}$ and $\alpha_i \neq \alpha_{i+1}$.
		\io The \textbf{reduction} of a word is obtained by removing the identity elements and replacing $g_ig_{i+1}$ by its product whenever they are in the same group.
	\end{itm}
	The \textbf{free product} $*_\alpha G_\alpha$ is the set of reduced words.
	The product of $g_1\cdots g_k$ and $h_1 \cdots h_l$ is the reduction of $g_1\cdots g_kh_1 \cdots h_l$.
	The identity element is the \textbf{empty word}.
\end{defn}

\begin{prop}
	The free product is the coproduct in the category $\Groups$.
	Observe that $G_\beta$ is a subgroup of $*_\alpha G_\alpha$ with $\phi_\beta: G_\beta \injto *_\alpha G_\alpha$.
	Then for any group $H$ with morphisms $\psi_\beta: G_\beta \to H$ there exists a unique morphism $\psi$ such that the following diagram commutes:
	\[
	\begin{tikzcd}
		& G_\beta \ar[ld, swap, "\phi_\beta"] \ar[rd, "\psi_\beta"] & \\
		*_\alpha G_\alpha \ar[rr, dashed, "\exists ! \psi"] & & H
	\end{tikzcd}
	\]
\end{prop}

\begin{lem}
	Let $(X,x_0)$ be a pointed space and let $\{A_\alpha\}$ be a collection of open subsets whose union is $X$.
	Let $x_0 \in A_\alpha$ and let $A_\alpha \cap A_\beta$ is path connected for all $\alpha$ and $\beta$.
	Then there is a group epimorphism
	\[*_\alpha \pi_1(A_\alpha,x_0) \surjto \pi_1(X,x_0)\]
	which restricts to
	\[\iota_*: \pi_1(A_\beta,x_0) \to \pi_1(X,x_0)\]
	for each subgroup $\pi_1(A_\beta, x_0)$ with $\iota:A_\alpha \injto X$.
\end{lem}