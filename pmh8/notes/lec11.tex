\section{2018-09-06 Lecture}

Plan for the forthcoming lectures:

\begin{itm}
  \io Existence, uniqueness and regularity of elliptic boundary value problems.
  \io Theory of distributions (this allows us to find the non-smooth points in solutions)
  \io Wavefront sets (this allows us to find the ``direction'' of the non-smoothness)
\end{itm}

Let $\gamma\in\cC^\infty(\bar\Omega,\Sym(n))$ be a symmetric matrix with eigenvalues of $\gamma(x)\geq c>0$ for all $x\in\bar\Omega$ and let $f\in\cC^\infty(\p\Omega)$.
Then we want to solve $\nabla\cdot(\gamma\nabla u)=0$ and $u|_{\p\Omega}=f$.
We want to find solutions in a Hilbert space, so we try $u\in L^2(\Omega)$.
Then what exactly is $u|_{\p\Omega}$?

Restriction to a codimension one set

\begin{defn}[Sobolev spaces on $\Omega$]
  Let $\Omega$ be an open domain in $\RR^n$ and let $k\in\NN_0$.
  For $u\in\cC^k(\bar\Omega)$, define
  \[ \norm{u}_{H^k(\Omega)}^2 = \sum_{\substack{\alpha\in\NN^n\\\abs{\alpha}\leq k}} \int_\Omega \abs{\p_x^\alpha u}^2 = \sum_{\substack{\alpha\in\NN^n\\\abs{\alpha}\leq k}} \norm{\p_x^\alpha u}_{L^2(\Omega)}^2 \]
  whenever it is finite.
  Now define
  \[ H^k(\Omega) = \ol{ \left\{ u\in\cC^k(\bar\Omega) \; \Big\vert \; \norm{u}_{H^k(\Omega)}<\infty \right\}}^{\norm\cdot_{H^k}}. \]
\end{defn}

\begin{exer}
  Show that when $\Omega=\RR^n$ this is equivalent to what was defined in assignment 1.
\end{exer}

\begin{rmk}
  These spaces can also be defined for $k<0$.
\end{rmk}

\begin{thm}[Restriction to $\p\Omega$]\label{11:boundary}
  Let $\Omega\subseteq\RR^n$ be a smooth bounded domain.
  Then there exists a map
  \[ R: H^k(\Omega) \to H^{k-\frac12}(\p\Omega) \]
  for $k\geq 1$ and $k\in\NN$ such that $R$ is a bounded linear transformation with $Ru=u|_{\p\Omega}$ for $u\in\cC^\infty(\bar\Omega)$.
\end{thm}

\begin{rmk}
  \lv
  \begin{itm}
    \io We have not yet defined $H^s$ for $s\in\RR\setminus\NN$.
    \io This shows that with enough regularity we can get some regularity on the boundary.
    \io This is not exactly a restriction (it is one only on smooth functions).
  \end{itm}
\end{rmk}

First consider the case of a flat boundary.
Let $\Omega=\RR_+^n=\{x \mid x_n>0\}$.
Then the boundary $x_n=0$ is an $(n-1)$-dimensional hyperplane ($\RR^{n-1}$).

\begin{prop}[Restriction to a flat plane]\label{11:plane}
  Let $k\geq1$.
  Then there exists a bounded linear transformation
  \[ R: H^k(\RR_+^n) \to H^{k-\frac12}(\RR^{n-1}) \]
  such that if $u\in\cC^\infty(\ol{\RR_+^n})\cap H^k$ then $(Ru)(x',x_n)=u(x',0)$ where $x'$ represents the first $n-1$ co-ordinates.
\end{prop}

Let us simplify the geometry even more.
Here $\RR^{n-1}$ is the hyperplane $x_n=0$ inside the full space $\RR^n$.

\begin{prop}[Restriction to the interior]\label{11:interior}
  Let $s\in\RR$ with $s>1/2$.
  Then there exists a bounded linear transformation
  \[ R': H^s(\RR^n) \to H^{s-\frac12}(\RR^{n-1}) \]
  such that if $u\in\cC^\infty(\RR^n)\cap H^s$ then $(R'u)(x')=u(x',0)$.
\end{prop}

\begin{proof}[Proof of proposition \ref{11:interior}]
  Let $u\in\cC_c^\infty(\RR^n)$ and define $ru\in\cC_c^\infty(\RR^{n-1})$ by $(ru)(x)=u(x',0)$.
  We will ignore unimportant constant factors here.
  Then
  \[ (ru)(x) = \int_\RR \int_{\RR^{n-1}} e^{ix'\cdot\xi'} \wh u(\xi',\xi_n) \, d\xi' \ d\xi_n \]
  so
  \[ (\wh{ru})(\xi') = \int_\RR \wh u(\xi',\xi_n) \ d\xi_n \]
  then
  \begin{align}
    \norm{ru}_{H^{s-\frac12}(\RR^{n-1})}^2 &= \int_{\RR^{n-1}} (1+\abs{\xi'}^2)^{s-\frac12} \abs{(\wh{ru})(\xi')}^2 \ d\xi' \nonumber \\
    &= \int_{\RR^{n-1}} (1+\abs{\xi'}^2)^{s-\frac12} \abs{\int_\RR \wh u(\xi',\xi_n) \ d\xi_n}^2 \ d\xi'. \label{11:star}
  \end{align}
  Now by Cauchy-Schwarz we have\vspace{-0.5cm}
  \[ \abs{\int_\RR \wh u(\xi',\xi_n) \ d\xi_n} \leq \abs{\int_\RR\frac{(1+\xi_n^2)^\frac s2}{(1+\xi_n^2)^\frac s2} \wh u \ d\xi_n} \leq \overbrace{\left( \int_\RR \frac{1}{(1+\xi_n^2)^s} \ d\xi_n \right)^\frac12}^{<\infty \text{ since } s>1/2} \left( \int_\RR (1+\xi_n^2)^s \abs{\wh u}^2 \ d\xi_n \right)^\frac12. \]
  Substituting this back in to equation \ref{11:star} gives
  \begin{align*}
    \norm{ru}_{H^{s-\frac12}(\RR^{n-1})}^2 &\leq C \int (1+\abs{\xi'}^2)^{s-\frac12} \int (1+\xi_n^2)^s \abs{\wh u}^2 \, d\xi_n \, d\xi' \\
    &\leq C \int (1+\abs\xi^2)^s \abs{\wh u}^2 \, d\xi_n \, d\xi' \\
    &= C\norm{u}_{H^s(\RR^n)}^2
  \end{align*}
  for all $u\in\cC_c^\infty(\RR^n)$ since
  \[ (1+\abs{\xi'}^2)^{s-\frac12}(1+\xi_n^2)^s \leq (1+\abs\xi^2)^s. \]
  Then by the bounded linear transformation (BLT) theorem, $r$ extends to a bounded map $R: H^s(\RR^n) \to H^{s-\frac12}(\RR^{n-1})$.
\end{proof}

\begin{prop}[Extension operator]\label{11:extension}
  For all $k\in\NN_0$ there exists an operator
  \[ E: H^k(\RR_+^n) \to H^k(\RR^n) \]
  such that $(Eu)(x',x_n)=u(x',x_n)$ for all $x_n>0$.
\end{prop}

For $L^2$ the extension is obvious (zero).
For $H^1$ we can reflect across.
In general it is not obvious.

\begin{exer}
  Show that $\cC^\infty(\RR_+^n) \cap H^k(\RR_+^n)$ is dense in $H^k(\RR_+^n)$.
\end{exer}

The plan for the proof is the same as the last proof: show that the operator is bounded and the use the BLT theorem.

\begin{proof}[Proof of proposition \ref{11:extension}]
  Define for $u \in \cC^\infty(\ol{\RR_+^n}) \cap H^k(\RR_+^n)$ the operator
  \begin{equation*}
    (Eu)(x',x_n)=
      \begin{cases}
	u(x) & x_n \geq 0 \\
	\sum_{j=1}^{k+1} \beta_j u\left( x',-\frac{x_n}j \right) & x_n < 0
      \end{cases}
  \end{equation*}
  We claim that $\beta_j$ for $j=1,\ldots,k+1$ can be chosen independently of $u$ such that $Eu \in \cC^k(\RR^n)$.
  Then for $\abs\alpha\leq k$, we have
  \[ \norm{\p_x^\alpha Eu}_{L^2(\RR^n)}^2 \leq \norm{\p_x^\alpha Eu}_{L^2(\RR_+^n)}^2 + \norm{\p_x^\alpha Eu}_{L^2(\RR_-^n)}^2 \leq C\norm{\p_x^\alpha u}_{L^2(\RR_+^n)}^2. \]
  Now we prove the claim.
  We have
  \begin{align*}
    \lim_{x_n\to0^+}Eu(x',x_n) = \lim_{x_n\to0^-}Eu(x',x_n) &\iff u(x',0) = \sum_{j=1}^{k+1} \beta_j u(x',0) \\
    &\impliedby 1 = \sum_{j=1}^{k+1} \beta_j && \text{(0th order condition)} \\
    \lim_{x_n\to0^+}\p_{x^n}Eu(x) = \lim_{x_n\to0^-}\p_{x^n}Eu(x) &\impliedby 1 = \sum_{j=1}^{k+1} \frac{-1}{j} \beta_j && \text{(1st order condition)} \\
    &\vdots \\
    \lim_{x_n\to0^+}\p_{x^n}^mEu(x) = \lim_{x_n\to0^-}\p_{x^n}^mEu(x) &\impliedby 1 = \sum_{j=1}^{k+1} \left(\frac{-1}{j}\right)^m \beta_j && \text{($m$th order condition)}
  \end{align*}
  for $m=0,\ldots,k$.
  This is a system of $k+1$ linear equations in $k+1$ variables with matrix $\left( (-j)^{-(i-1)} \right)_{ij}$ for $1\leq i,j \leq k+1$.
  This is matrix is invertible (this is a Vandermonde matrix) hence a solution exists.
\end{proof}

Now with extension and restriction we have proposition \ref{11:plane}.


