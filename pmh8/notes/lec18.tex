\section{2018-10-09 Lecture}

Some remarks about the assignment and wavefront sets

Let $\chi\in\cC_c^\infty(\RR^n)$ and $u\in L^2(\RR^n)$.
Suppose that $(x_0,\xi_0)\in\RR^n\times S^{n-1}$ and $(x_0,\xi_0)\notin\WF(u)$.
Then $(x_0,\xi_0)\notin\WF(u)$ if and only if there exists a $\rho\in\cC_c^\infty(\RR^n)$ and a $\wt\rho\in\cC_c^\infty(S^{n-1})$ such that $\rho(x_0)=\wt\rho(\xi_0)=1$ and
\[ \abs{\wt\rho\left( \frac\xi{\abs\xi} \right) (\rho u)^\wedge(\xi) } \leq C_k(1+\abs\xi^2)^{-k} \]
for all $\xi\in\RR^n$ and $k\in\NN$.

WLOG $\wh\rho(\xi/\abs\xi)=1$ for $\xi/\abs\xi$ near $\xi_0$ on $S^{n-1}$.
Then
\begin{align*}
  (\chi\rho u)^\wedge(\eta) &= \int \wh\chi(\eta-\xi) (\wh{\rho u})(\xi) \, d\xi \\
  &= \underbrace{\int \wh\chi(\eta-\xi) \left( 1-\wh\rho\left( \frac\xi{\abs\xi} \right) \right) (\wh{\rho u})(\xi)}_{I_1(\eta)} + \underbrace{\int \wh\chi(\eta-\xi) \wh\rho\left( \frac\xi{\abs\xi} \right) (\wh{\rho u})(\xi)}_{I_2(\eta)}.
\end{align*}

Now
\begin{equation*}
  I_2(\eta) = \int \wh\chi(\eta-\xi) \wt\rho\left( \frac\xi{\abs\xi} \right) \frac{(1+\abs\xi^2)^k}{(1+\abs\xi^2)^k} (\wh{\rho u})(\xi).
\end{equation*}
Let $\wt\rho'(\xi)\in\cC_c^\infty(S^{n-1})$ such that $\supp\wt\rho'\subset\subset\left\{ \wt\rho\equiv1 \right\}$.
Then
\begin{equation*}
  \wh\rho'\left( \frac{\eta}{\abs\eta} \right) I_2(\eta) = \int \frac{\wh\chi(\eta-\xi)}{(1+\abs\xi^2)^k} \wt\rho\left( \frac\xi{\abs\xi} \right) (1+\abs\xi^2)^k (\wh{\rho u})(\xi).
\end{equation*}
Now for all $N\in\NN$ we have
\begin{equation*}
  \abs{\wt\rho\left( \frac\xi{\abs\xi} \right) (1+\abs\xi^2)^k (\wh{\rho u})(\xi)} \leq C_{N,k}(1+\abs\xi^2)^{-N}.
\end{equation*}
We also have by \emph{Peetre's inequality}
\begin{equation*}
  \frac1{(1+\abs\xi^2)^k} \leq \frac{(1+\abs{\xi-\eta}^2)^k}{(1+\abs\eta^2)^k}.
\end{equation*}
Putting this all together, we have
\begin{align*}
  \abs{ \wh\rho'\left( \frac{\eta}{\abs\eta} \right) I_2(\eta) } &\leq \int \frac{\abs{\left( (1+\Delta_{\eta-\xi})^k\chi \right)^\wedge(\eta-\xi)}}{(1+\abs\eta^2)^k} C_{N,k} (1+\abs\xi^2)^{-N} \, d\xi \\
  &\leq C(1+\abs\eta^2)^{-k}.
\end{align*}
Hence $I_2(\eta)$ decays to all order in direction $\xi_0$.

Also
\begin{equation*}
  \wt\rho'\left( \frac\eta{\abs\eta} \right)I_1(\eta) = \wt\rho'\left( \frac\eta{\abs\eta} \right) \int \wh\chi(\eta-\xi)\left( 1-\wh\rho\left( \frac\xi{\abs\xi} \right) \right) (\wh{\rho u})(\xi)
\end{equation*}
Now $\xi/\abs\xi$ in $\supp(1-\wt\rho)$ is away from the direction of $\eta/\abs\eta$ in $\supp\wt\rho'$, hence $1/\abs{\eta-\xi}^2\neq0$.
Moreover, we have
\[ \xi\cdot\eta \leq (1-c)\abs\xi\abs\eta \text{ for some } c>0, \]
therefore
\[ \abs{\eta-\xi}^{2k} \leq \frac1{(1-c)^k}\left( \abs\eta^2+\abs\xi^2 \right)^k. \]
Therefore
\begin{equation*}
  \wt\rho'\left( \frac\eta{\abs\eta} \right)I_1(\eta) = \wt\rho'\left( \frac\eta{\abs\eta} \right) \int \frac{\left(\Delta^k\chi \right)^\wedge(\eta-\xi)}{\abs{\eta-\xi}^k} \left( 1-\wh\rho\left( \frac\xi{\abs\xi} \right) \right) (\wh{\rho u})(\xi)
\end{equation*}
so
\begin{equation*}
  \abs{\wt\rho'\left( \frac\eta{\abs\eta} \right)I_1(\eta)} \leq \frac{\wt\rho'\left( \frac{\eta}{\abs\eta} \right)}{\abs\eta^{2k}} \int \left(\Delta^k\chi \right)^\wedge(\eta-\xi) \left( 1-\wh\rho\left( \frac\xi{\abs\xi} \right) \right) (\wh{\rho u})(\xi) \, d\xi
\end{equation*}
when $\abs\eta\to\infty$.

Back to distributions

\begin{defn}[Convergence of test functions]
  Let $\{\ph_j\}$ be a sequence in $\cC_c^\infty(X)$.
  Then $\ph_j \to 0$ if there exists a $K \subset\subset X$ such that:
  \begin{enum}
    \io $\supp\ph_j\subseteq K$ for all $i$, and
    \io $\p^\alpha\ph_j \to 0$ uniformly in $X$ for all $\alpha\in\NN^K$.
  \end{enum}
\end{defn}

\begin{rmk}
  This is a topology generated by a sequence $\{p_j\}$ of seminorms where
  \[ p_j(\ph) = \norm{\p^\alpha_j\ph}_{L^\infty(K_j)}. \]
  (Recall that a seminorm is a norm minus the requirement $p(x)=0\implies x=0$.)
  This topology is metrizable and gives the metric
  \[ d_{\cC_c^\infty(X)}(\ph,\psi) = \sum_{j=0}^\infty 2^{-j} \frac{p_j(\ph-\psi)}{1+p_j(\ph-\psi)}. \]
  This metric makes $\cC_c^\infty(X)$ in to a Fr\'echet space.
\end{rmk}

Recall that $u\in\cD'(X)$ if
\[ \abs{\ang{u,\ph}} \leq C \sum_{\abs\alpha\leq N} \norm{\p^\alpha\ph}_{L^\infty(K)} \]
for all $\ph\in\cC_c^\infty(K)$ and all $K \subset\subset X$.

\begin{thm}
  Let $u: \cC_c^\infty(X) \to \CC$ be linear.
  Then $u\in\cD'(X)$ if and only if
  \[ \lim_{j\to\infty}\ang{u,\ph_j}=0 \text{ for all } \ph_j\to0 \text{ in } \cC_c^\infty(X). \]
\end{thm}

\begin{proof}
  Let $\ph_j\to0$ in $\cC_c^\infty(X)$ and suppose that $u\in\cD'(X)$.
  Then $\ang{u,\ph_j}\to0$ by definition of a distribution.
  Conversely, if $\ang{u,\ph_j}\to0$ for all sequences $\ph_j\to0$ and $u\notin\cD'(X)$, then there exists a $k$ such that for all $N$,
  \[ \sup_{\supp\ph\subseteq K} \frac{\abs{\ang{u,\ph}}}{\sum_{\abs\alpha\leq N} \norm{\p^\alpha\ph}_{L^\infty(K)}} = \infty. \]
  So for all $N$ there exists a $\ph_N$ such that
  \[ \abs{\ang{u,\ph}} \geq N \sum_{\abs\alpha\leq N} \norm{\p^\alpha\ph}_{L^\infty(K)}. \]
  Now set
  \[ \psi_N = \frac{\ph_N}{N\sum_{\abs\alpha\leq N} \norm{\p^\alpha\ph_N}_{L^\infty(K)}}. \]
  Then $\psi_N\to0$ in $\cC_c^\infty(X)$, but $\abs{\ang{u,\psi_N}}\leq1$, a contradiction.
  Hence $u\in\cD'(X)$.
\end{proof}

\begin{defn}[Convergence of distributions]
  For a sequence $u_j\in\cD'(X)$, we define $u_j\to0$ if $\ang{u_j,\ph}\to0$ for all $\ph\in\cC_c^\infty(X)$.  
\end{defn}

\begin{exam}
  \lv
  \begin{enum}
    \io Let $X=\RR$ and $u_j=e^{i\pi jx}$.
      Then
      \[ \int e^{i\pi jx}\ph \to 0 \text{ as } j\to\infty \text{ for all } \ph\in\cC_c^\infty(X), \]
      but $u_j\not\to0$ pointwise.

      \io We have $(\delta_\eps\circ f)\to0$ a.e.\@ in $\RR^n$, but $\delta_\eps\circ f\to\delta\circ f\neq0$ in $\cD'(X)$.
  \end{enum}
\end{exam}
