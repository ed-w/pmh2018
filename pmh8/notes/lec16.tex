\section{2018-10-02 Lecture}

Motivation for distribution theory

Consider Poisson's equation:
\[ \Delta u=F. \]
We can give a ``right-inverse function'' $u(x)$ (or \textbf{Green's function}) for $\Delta$:
\[ u(x) = \int G(x-y)F(y) \, dy \]
where
\begin{equation*}
  G(x-y)=
  \begin{cases}
    C_2\log\abs{x-y} & \text{if }n=2 \\
    C_3\frac1{\abs{x-y}^{n-2}} & \text{if }n\geq3
  \end{cases}
\end{equation*}
Then
\[ \Delta_y G(x-y) = \delta(x,y) \]
and so
\[ \int (\Delta_y\ph) G(x-y) \, dy = \int \ph(y) \delta_0(x,y) \, dy = \ph(x). \]
The function $G$ doesn't really exist on its own (it is undefined at the origin).
We only use it to integrate against another function.

Why distributions and wavefront sets?

\begin{exam}
  \lv
  \begin{enum}
    \io
    We have an operation define a priori on ``good'' functions.
    For example, the restriction map on Sobolev spaces: if $S \subseteq \RR^n$ is a codimension $1$ manifold and if $u\in H^s(\RR^n)$ for $s>1/2$, then $u|_S\in H^{s-1/2}(S)$.
    Here is the distribution version: any distribution $u$ can be restricted to any submanifold of $\RR^n$ of any codimension as long as $WF(u)$ is ``compatible'' with $S$.

    Why is this more powerful (less regularity required) than the Sobolev version?

    \begin{enum}
      \io
      Set
      \[ u(x_1,x_2) = \frac{1}{\sqrt{\abs{x_1}}} \in L\loc^1 \setminus L\loc^2. \]
      The singularities are the $x_2$-axis ($x_1=0$).
      Then $u|_{\{x_2=0\}}=1/\sqrt{\abs{x_1}} \in L\loc^1(\RR)$.
      In fact, $u$ can be restricted to a submanifold $S$ as long as $e_2 \notin T_pS$ if $p \in S \cap \{x_1=0\}$.

      \io
      (Trace of an operator.)
      Consider the heat equation
      \[ \p_t u = \Delta u \]
      and $u(0)=u_0$ on $M$.
      Then
      \[ u(t,x) = \int_M K_t(x,y) u_0(y) \, dy \]
      where $K_0(x,y) = \delta_0(x,y)$.
      It turns out that $K_t(x,y) \in \cC^\infty(M \times M)$ if $t>0$.
      Define
      \[ \tr K_t = \int_m K_t(x,x) \, dx. \]
      This trace is useful because it gives us geometric properties:
      \[ \tr K_t \sim t^{-n/2} \vol(M) + o(t^{-n/2}). \]
      It turns out that we can define $\tr(K(\cdot,\cdot))$ as long as $\diag(M \times M)$ is ``compatible'' with $WF(K(\cdot,\cdot))$.
    \end{enum}

    \io
    (Oscillatory integrals.)
    We want to invert $\Delta+1$, that is, solve $(\Delta+1)u=F$.
    Taking the Fourier transform, we get
    \[ \wh u = \frac1{1+\abs\xi^2} \wh f, \]
    so
    \[ u(x) = \int \frac{e^{i\xi\cdot x}}{1+\abs\xi^2} \int e^{-iy \cdot\xi} F(y) \, dy \, d\xi = \int dy F(y) \int d\xi \frac{e^{i\xi\cdot(x-y)}}{1+\abs\xi^2}. \]
    The integral
    \[ \int d\xi \frac{e^{i\xi\cdot(x-y)}}{1+\abs\xi^2} \]
    is not a-priori integrable, but we can understand this in the context of distributions.

    \io
    (Tracking singularities.)
    This was the original motivation for wavefront sets.
  \end{enum}
\end{exam}

We will now give a precise definition.

\begin{defn}[Distributions in $\RR^n$]
  Let $X\subseteq\RR^n$ be open.
  A linear functional $u: \cC_c^\infty(X) \to \CC$ is called a \textbf{distribution} of order $N$ if there exists an $N\in\NN_0$ such that for all compact $K \subset\subset X$, there exists a $C>0$ such that
  \[ \abs{\ang{u,\ph}} \leq C\sum_{\abs\alpha\leq N} \sup_X \abs{\p^\alpha\ph} \]
  for all $\ph\in\cC_c^\infty(X)$ with $\supp\ph\subseteq K$.
  The vector space over $\CC$ of all distributions on $X$ is denoted $\cD'(X)$.
\end{defn}

\begin{exam}
  \lv
  \begin{enum}
    \io
    Let $u\in L\loc^1(\RR^n)\subseteq\cD'(\RR^n)$.
    Then define
    \[ \ang{u,\ph} = \int u\ph \]
    for all $\ph\in\cC_c^\infty(\RR^n)$.
    When we say that $u$ is a distribution in $L\loc^1$, we mean that it can be represented as in the above equation.
    This distribution is of order $0$.

    \io
    Let $\mu$ be a probability measure on $X$.
    Then for each such $\mu$, define
    \[ \ph \mapsto \int \ph \, d\mu \]
    is a distribution of order $0$.
    If $\mu$ is a point mass at $x_0 \in X$, then
    \[ \int \ph \, d\mu = \ph(x_0) = \ang{\delta_{x_0},u}. \]

    \io
    Let $X=\RR$.
    Then for $\ph\in\cC_c^\infty(\RR)$, define
    \begin{align*}
      \ang{\pv\frac1x,\ph} &\defeq \lim_{\eps\to0^+} \left( \int_{-\infty}^{-\eps} \frac{\ph(x)}{x}\, dx + \int_\eps^\infty \frac{\ph(x)}{x} \, dx \right) \\
      &= \lim_{\eps\to0^+} \bigg( \int_{-\infty}^{-\eps} \log(-x)\ph'(x)\, dx - \left[ \ph(x)\log(-x) \right]_{-\infty}^{-\eps} \\
      &\quad - \int_\eps^\infty \log(x)\ph'(x) \, dx + \left[ \ph(x)\log(x) \right]_\eps^\infty \bigg) \\
      &= \int_{-\infty}^0 \log\abs{x}\ph'(x) - \int_0^\infty \ph'(x)\log\abs{x}
    \end{align*}
    
    \io
    Compute
    \[ \ph \mapsto \lim_{\eps\to0} \int_\eps^\infty \frac{\ph(x)-\ph(0)}{x} \, dx. \]
  \end{enum}
\end{exam}
