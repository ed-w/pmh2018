\section{2018-09-20 Lecture}

Higher regularity

We have shown that for $F \in H\inv(\Omega)$, $f \in H^\frac12(\p\Omega)$ and  $\Omega$ a smooth open domain in $\RR^n$, there exists a unique $u \in H^1(\Omega)$ such that
\begin{equation*}
  \begin{cases}
    \nabla\cdot\gamma\nabla u&=F \\
    u|_{\p\Omega}&=f
  \end{cases}
\end{equation*}
and
\[ \norm{u}_{H^1} \leq \norm{f}_{H^\frac12} + \norm{F}_{H\inv}. \]

\begin{thm}\label{15:thm}
  Let $\nabla\cdot\gamma\nabla u = F\in H^{k-1}(\Omega)$, $k\geq0$ and $u\in H_0^1(\Omega)$.
  Then $u\in H^{k+1}(\Omega)$ and
  \[ \norm{u}_{H^{k+1}(\Omega)} \leq C\norm{F}_{H^{k-1}(\Omega)}. \]
\end{thm}

\begin{exer}[Corollary of theorem \ref{15:thm}]
  Let $\nabla\cdot\gamma\nabla u=F \in H^{k-1}(\Omega)$ and $u|_{\p\Omega}=f \in H^{k+\frac12}(\p\Omega)$ for some $k\geq0$.
  Then $u\in H^{k+1}(\Omega)$ and
  \[ \norm{u}_{H^{k+1}(\Omega)} \leq C\left( \norm{F}_{H^{k-1}(\Omega)} + \norm{f}_{H^{k+\frac12}(\p\Omega)} \right). \]
\end{exer}

\begin{proof}[Proof of theorem \ref{15:thm}]
  We have already done the case $k=0$.
  Now suppose that the statement holds for some $k\geq0$, that is;
  \begin{equation}
    \left.
    \begin{aligned}
      \nabla\cdot\gamma\nabla u &= F \in H^{k-1}(\Omega) \\
      u|_{\p\Omega} &= 0
    \end{aligned}
    \right\}\implies u\in H^{k+1}(\Omega)
    \label{<+label+>}
  \end{equation}
  Suppose that $F \in H^k \subset H^{k-1} \implies u \in H^{k+1}(\Omega)$ and equation \ref{15:star} holds:
  \begin{equation}
    \norm{u}_{H^{k+1}(\Omega)} \leq C\norm{\nabla\cdot\gamma\nabla u}_{H^{k-1}(\Omega)}.
    \label{15:star}
  \end{equation}
  We need to show that $u \in H^{k+2}(\Omega)$ and
  \[ \norm{u}_{H^{k+2}} \leq \norm{\nabla\cdot\gamma\nabla u}_{H^k}. \]
  Let $\chi\in\cC_c^\infty(\RR^n)$ where $\supp\chi\subseteq U\subset\RR^n$ and $(U,\Psi)$ is a co-ordinate neighbourhood for $\p\Omega$.
  Then
  \[ \nabla\cdot\gamma\nabla(\chi u) = \chi\underbrace{\nabla\cdot\gamma\nabla}_{=F \in H^k} u + \underbrace{[\nabla\cdot\gamma\nabla,\chi]}_{1\text{st order operator}}\underbrace{u}_{\in H^{k+1}} \in H^k(\Omega) \]
  (check this).
  Set $\wt\Omega=\Psi(\supp\chi\cap U)$ and 
  \[ \wt u(x) = (\chi u)\circ\Psi\inv(x) \in H^{k+1}(\wt\Omega), \]
  so $\wt u|_{\p\wt\Omega}=0$.

  \begin{exer}
    We have
    \[ \nabla\cdot\wt\gamma\nabla\wt u = \wt F \in H^k(\wt\Omega) \]
    where
    \[ \wt\gamma = (D\Psi)^\intercal \gamma (D\Psi) \]
    and
    \[ \Vert\wt F \Vert_{H^k(\wt\Omega)} \leq \norm{F}_{H^k(\Omega)}. \]
  \end{exer}

  For $j=1,\ldots,n-1$,
  \begin{align*}
    \norm{D_{h,j} \wt u}_{H^{k+1}(\wt\Omega)} &\leq \norm{\nabla\cdot\wt\gamma\nabla D_{h,j}\wt u}_{H^{k-1}(\wt\Omega)} && \text{by equation \ref{15:star}} \\
    &\leq \norm{D_{h,j}(\nabla\cdot\wt\gamma\nabla\wt u)}_{H^{k-1}(\wt\Omega)} + \norm{[\nabla\cdot\wt\gamma\nabla,D_{h,j}]\wt u}_{H^{k-1}(\wt\Omega)} \\
    &\leq \norm{\nabla\cdot\wt\gamma\nabla\wt u}_{H^k} + \norm{\wt u}_{H^{k+1}} && \text{exercise} \\
    &\leq C\norm{\nabla\cdot\wt\gamma\nabla\wt u}_{H^k} && \text{by equation \ref{15:star}}
  \end{align*}
  where $C$ is independent of $h$.
  Then taking $h\to0$, we have
  \begin{equation}
    \norm{\p_{x_j}\wt u}_{H^{k+1}(\wt\Omega)} \leq C\norm{\nabla\cdot\wt\gamma\nabla\wt u}_{H^k(\wt\Omega)}
    \label{15:2star}
  \end{equation}
  for $j=1,\ldots,n-1$.

  \begin{rmk}
    For interior regularity we can stop here.
  \end{rmk}

  Now we need
  \begin{align*}
    &\norm{\p_{x_n}\wt u}_{H^{k+1}(\wt\Omega)} \leq \norm{\nabla\cdot\wt\gamma\nabla\wt u}_{H^k(\wt\Omega)} \\
    \iff &\norm{\p_{x_n}\p_{x_j}\wt u}_{H^k(\wt\Omega)} \leq \norm{\nabla\cdot\wt\gamma\nabla\wt u}_{H^k(\wt\Omega)}
  \end{align*}
  for all $j=1,\ldots,n$.
  If $j=1,\ldots,n-1$, then by equation \ref{15:2star} we have
  \[ \norm{\p_{x_j}\p_{x_n}\wt u}_{H^k} \leq \norm{\p_{x_j}\wt u}_{H^{k+1}} \leq \norm{\nabla\cdot\wt\gamma\nabla\wt u}_{H^k}. \]
  So we only need to check
  \[ \norm{\p_{x_n}^2\wt u}_{H^k} \leq \norm{\nabla\cdot\wt\gamma\nabla\wt u}_{H^k}. \]
  Now write
  \begin{align*}
    \nabla\cdot\wt\gamma\nabla\wt u &= \p_n \wt\gamma^{n,n} \p_n \wt u + \sum_{(l,m)\neq(n,n)} \p_l \wt\gamma^{l,m} \p_m \wt u \\
    &= \wt\gamma^{n,n} \p_n^2 \wt u + \sum_{(l,m)\neq(n,n)} \wt\gamma^{l,m} \p_l\p_m \wt u + Xu
  \end{align*}
  where $X$ is some $1$st order differential operator.

  \begin{exer}
    $\wt\gamma$ is positive definite $\implies$ $\wt\gamma^{n,n}(x)>0$.
  \end{exer}

  So
  \begin{align*}
    \norm{\p_n^2\wt u}_{H^k} &\leq \norm{\nabla\cdot\wt\gamma\nabla\wt u}_{H^k} + \sum_{(l,m)\neq(n,n)} \norm{\p_l\p_m\wt u}_{H^k} + \norm{X\wt u}_{H^k} \\
    &\leq C\norm{\nabla\cdot\wt\gamma\nabla\wt u}_{H^k} + \norm{X\wt u}_{H^k} \\
    &\leq C\norm{\nabla\cdot\wt\gamma\nabla\wt u}_{H^k} + \norm{\wt u}_{H^{k+1}} \\
    &\leq \wt C\norm{\nabla\cdot\wt\gamma\nabla\wt u}_{H^k}
  \end{align*}
  by the induction hypothesis (equation \ref{15:star}).
  Then
  \begin{align*}
    \implies &\norm{\p_{x_j}\p_{x_k}\wt u}_{H^k} \leq C\norm{\nabla\cdot\wt\gamma\nabla\wt u}_{H^k} \\
    \implies &\norm{\wt u}_{H^{k+2}} \leq C\norm{\nabla\cdot\wt\gamma\nabla\wt u}_{H^k}
  \end{align*}
  in $\Omega$.

  \begin{exer}
    Show that this implies that
    \[ \norm{\chi u}_{H^{k+2}(\Omega)} \leq C\norm{\nabla\cdot\gamma\nabla u}_{H^k(\Omega)}. \]
  \end{exer}

  Then considering a partition of unity completes the proof.
\end{proof}

