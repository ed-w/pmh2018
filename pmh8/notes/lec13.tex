\section{2018-09-13 Lecture}

\begin{proof}[Proof of theorem \ref{11:boundary} (continued)]
  Now define for $j\neq0$:
  \[ R(\chi_ju) = R_{\RR^{n-1}} \left( (\chi_ju) \circ \Psi_j \inv \right) \circ \psi_j. \]
  Note that $(\chi_0u)|_{\p\Omega}=0$ so we can ignore it for the purposes of defining $R$.
  Then we can define for all $u\in\cC^\infty(\ol\Omega)$.
  \[ Ru = \sum_{j\neq0} R(\chi_ju). \]
  So we have
  \begin{align*}
    \norm{Ru}_{H^{k-\frac12}(\p\Omega)} &= \sum_{j\neq0} \norm{ R_{\RR^{n-1}} \left( (\chi_ju)\circ\Psi_j\inv \right) }_{H^{k-\frac12}(\RR^n)} \\
    &\leq \sum_{j\neq0} \norm{ (\chi_ju)\circ\Psi_j\inv }_{H^k(\RR_+^n)} \\
    &= C\sum_{j\neq0} \norm{\chi_ju}_{H^k(\Omega)} \leq C'\norm{u}_{H^k(\Omega)} \qedhere
  \end{align*}
\end{proof}

\begin{exer}
  Let $k\in\NN$ with $k\geq1$.
  Then there exists a bounded operator
  \[ R^+: H^{k-\frac12}(\p\Omega) \to H^k(\Omega) \]
  such that $R\circ R^+=\id$.
  Furthermore, $R^+$ can be chosen uniquely so that
  \[ R^+\left( H^{k-\frac12}(\p\Omega) \right) = \ker(R)^\perp. \]
\end{exer}

Now that we have defined restriction to the boundary, we can talk about boundary value problems.

\begin{rmk}
  We will write $u|_{\p\Omega}$ for $Ru$ even if $u\notin\cC^\infty(\ol\Omega)$.
\end{rmk}

We want to solve $\nabla\cdot\gamma\nabla u=0$ in $\Omega$ with $u|_{\p\Omega}=f\in H^\frac12(\p\Omega)$.

\begin{defn}
  Let $s\in\RR$ with $s\geq0$.
  Note that every function in $\cC_c^\infty(\Omega)$ can be naturally extended to a function in $\cC_c^\infty(\RR^n)$ by letting it be zero outside $\Omega$.
  Now define
  \[ H_0^s(\Omega) = \ol{\cC_c^\infty(\Omega)}^{H^s(\RR^n)}. \]
  Note that $H_0^k(\Omega) \subsetneq H^k(\Omega)$ when $k\in\NN$ and $k\geq1$.
\end{defn}

If $u\in H_0^k(\Omega)$ and $k\geq1$, then there exists a sequence $u_j\in\cC_c^\infty(\Omega)$ such that
\[ u_j \xto{H^k(\Omega)} u \text{ and } R(u) = \lim_{j\to\infty}R(u_j)=0. \]
Thus $H_0^k(\Omega) \subseteq \ker(R_{H^k(\Omega)})$ for $k\geq1$.

\begin{exer}
  Show that $\ker(R_{H^1(\Omega)})=H_0^1(\Omega)$.
  (Hint: first do this for $\RR_+^n$.)
\end{exer}

General discussion of dual spaces in functional analysis.

Let $(H,(\cdot,\cdot))$ be a Hilbert space and let $X \subseteq H$ be a Banach space.
For example, we have $H_0^1(\Omega) \subseteq L^2(\Omega)$.
Consider the dual space $X^*$ of $X$.
If $h \in H$, define $\iota_{X^*}: H \to X^*$ by
\[ \ang{\iota_{X^*}h,x}_{X^*,X} = (x,h)_H. \]
So $H$ embeds in to $X^*$, giving us $X \subseteq H \subseteq X^*$.

If $x^* \in X^*$, we say that $x^* \in H$ if $x^* = \iota_{X^*} h$ for some $h \in H$, that is, there exists an $h \in H$ such that $\ang{x^*,x}_{X^*,X} = (h,x)_H$ for all $x \in X$.

\begin{defn}
  Let $s\in\RR$ with $s\leq0$.
  Define
  \[ H^s(\Omega) = \left( H_0^{-s}(\Omega) \right)^*. \]
  Then
  \[ H_0^{-s}(\Omega) \subseteq L^2(\Omega) \subseteq H^s(\Omega) \]
  for all $s\leq0$ and for $u\in H^s(\Omega)$ we say that $u\in L^2(\Omega)$ if
  \[ \ang{v,u}_{H_0^{-s},H^s} = \int_\Omega uv \]
  for all $v \in H_0^{-s}(\Omega)$.
\end{defn}

\begin{prop}[Differentiation]
  The map $u \mapsto \p_{x_j}u$ defined for $u\in\cC^\infty(\ol\Omega)$ extends to a bounded linear map from $H^k(\Omega) \to H^{k-1}(\Omega)$ for all $k\in\ZZ$.
\end{prop}

\begin{proof}
  If $k\geq1$ this is obvious.
  Now let $k\leq0$ and $u\in\cC^\infty(\ol\Omega)$.
  Then
  \begin{align*}
    \norm{\p_{x_j}u}_{H^{k-1}(\Omega)} &= \sup\left\{ \ang{\p_{x_j}u,v} \ \big\vert \ v\in H_0^{1-k}(\Omega) \text{ and } \norm{v}_{1-k} \leq 1 \right\} \\
    &= \sup\left\{ \ang{\p_{x_j}u,v} \ \big\vert \ v\in \cC_c^\infty(\Omega) \text{ and } \norm{v}_{1-k} \leq 1 \right\} \\
    &= \sup\left\{ \int_\Omega \p_{x_j}uv \ \Big\vert \ v\in \cC_c^\infty(\Omega) \text{ and } \norm{v}_{1-k} \leq 1 \right\} \\
    &= \sup\left\{ -\int_\Omega u\p_{x_j}v \ \Big\vert \ v\in \cC_c^\infty(\Omega) \text{ and } \norm{v}_{1-k} \leq 1 \right\} \\
    &\leq \sup\left\{ \int_\Omega uv' \ \Big\vert \ v'\in \cC_c^\infty(\Omega) \text{ and } \norm{v'}_{-k} \leq C \right\} \\
    &= \sup\left\{ \ang{u,v'}_{H^k,H^{-k}} \ \big\vert \ v'\in H_0^k(\Omega) \text{ and } \norm{v'}_{-k} \leq C \right\} \\
    &\leq C\norm{u}_{H^k}.
  \end{align*}

  \begin{exer}
    Show that $\cC^\infty(\ol\Omega)$ is dense in $H^{-\abs{k}}(\Omega)$.
  \end{exer}
  Then the BLT theorem concludes the proof.
\end{proof}

\begin{exer}[Integration by parts]
  Let $u\in H_0^1$ and $v\in L^2$.
  Then
  \[ \ang{\p_{x_j}v,u}_{H^{-1},H_0^1} = -\int_\Omega v\p_{x_j}u .\]
\end{exer}
