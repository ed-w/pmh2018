\section{2018-08-07 Lecture}

Recall: we want to prove that $q \mapsto \Lambda_q$ is injective.
We have Green's identity:
\[ \int_\Omega u_2(q_1-q_2)u_2 = \int_{\p\Omega} u_1(\Lambda_{q_1}-\Lambda_{q_2})u_2 \]
where $(\Delta+q_j)u_j)=0$.
We have $\Lambda_{q_1}=\Lambda_{q_2}$.
We want to find a dense subset of functions in order to show that $q_1=q_2$.
We will do this folowing the method outlined at the end of the last lecture.

\begin{prop}[Complex geometric optics]\label{3:cgo}
  Let $\rho\cdot\rho=0$ and $\abs\rho$ sufficiently large.  
  There exists a solution $u \in \cC^\infty(\bar\Omega)$ solving $(\Delta+q)u=0$ and $u=e^{\rho\cdot x}(1+r)$ where $r$ is a error term which satisfies
  \[ \norm{r}_{L^2} \leq \frac{C}{\abs\rho} \text{ as } \abs\rho \to \infty. \]
\end{prop}

\begin{rmk}
  If there was no $q$, we would expect $u=e^{\rho\cdot x}$.
  The $q$ gives rise to the error term $r$.
\end{rmk}

\begin{rmk}[Physical intuition]
  The quantity $\rho$ is like the frequency of a wave.
  When $\abs\rho$ is large, this is like a high-frequency wave which penetrates the medium and affects the wave very little (so as $\abs\rho$ gets larger the solution looks more like an undisturbed wave).
\end{rmk}

\begin{proof}[Proof of theorem \ref{2:thm2} for $n \geq 3$ assuming proposition \ref{3:cgo}]
  Let $\xi \in \RR^n$.
  Choose $\zeta, \eta \in \RR^n$ of unit length such that $\zeta$, $\eta$ and $\xi$ are mutually perpendicular (here we have used $n \geq 3$).
  Define
  \begin{align*}
    \rho_1 &= t\zeta + i\left( \sqrt{t^2-\abs\xi^2} \eta + \xi \right) \\
    \rho_2 &= -t\zeta + i\left( -\sqrt{t^2-\abs\xi^2} \eta + \xi \right)
  \end{align*}
  for large $t$.
  (We will let $t \to \infty$ later on.)

  So $\rho_1 \cdot \rho_1 = 0$, $\rho_2 \cdot \rho_2 = 0$ and $\abs{\rho_1}, \abs{\rho_2} \to \infty$ as $t \to \infty$.
Then there exist solutions to $(\Delta+q_j)u_j=0$ given by $u_j=e^{\rho_j\cdot x}(1+r_j)$ with $r_1,r_2 \to 0$ in $L_2$ as $t \to \infty$.
  Now apply Green's identity:
  \[ 0 = \int_\Omega u_1(q_1-q_2)u_2 = \int_\Omega e^{2i\xi\cdot x}(1+r_1+r_2+r_1r_2)(q_1-q_2). \]
  The terms $r_i$ hide the appearances of $t$ in the above equation.
  By Cauchy-Schwarz, taking the limit as $t \to \infty$ gives
  \[ \int_\Omega e^{2i\xi\cdot x}(q_1-q_2) = \cF(I_\Omega(q_1-q_2)) = 0. \]
  Then by injectivity of $\cF$ we are done.
\end{proof}

So the proof wasn't so hard once we assumed proposition \ref{3:cgo}.
This proposition hides the bulk of the proof.

Weak solutions

We want to solve $(\Delta+q)u=0$ in $\Omega$ where $u \in \cC^\infty(\bar\Omega)$.
It is difficult to construct a smooth solution directly.
We can create a weak solution first and then show that it is smooth.

\begin{defn}
  A \textbf{weak solution} is an $L^2$ function $u$ such that
  \[ \int_\Omega u(\Delta+q)\ph = 0 \text{ for all } \ph \in \cC_c^\infty(\Omega). \]
  Note that $u$ is not required to be smooth.
  If $u$ is smooth, then integrating by parts gives
  \[ \int_\Omega \ph(\Delta+q)u=0 \text{ for all } \ph \in \cC_c^\infty(\Omega) \]
  which implies that $u$ is a solution in the proper sense.
\end{defn}

Here is how the proof of proposition \ref{3:cgo} will proceed.
Let $\wt\Omega$ be an open subset of $\RR^n$ such that $\Omega \subsetneq \wt\Omega$.
We will construct a weak solution $\wt\Omega$ and we will show that it is in fact smooth on $\Omega$.

\begin{prop}[A famous estimate]\label{3:est}
  Let $q \in L^\infty(\wt\Omega)$.
  Then there exists a $C>0$ such that for all $\rho$ with $\abs\rho$ sufficiently large and $\rho\cdot\rho=0$, we have
  \[ \norm{e^{-\rho \cdot x}(\Delta+q)e^{\rho\cdot x} u} \geq C\abs\rho\norm{u} \text{ for all } u \in \cC_c^\infty(\wt\Omega). \]
  In particular, this means that if $(\Delta+q)e^{\rho\cdot x}u=0$ and $u \in \cC_c^\infty(\wt\Omega)$, then $u \equiv 0$.
  So there are no compactly supported smooth solutions.
\end{prop}

This will allow us to define a coercive inner product.

\begin{cor}[Solvability]\label{3:solv}
  There exists a constant $C>0$ such that for all $\abs\rho$ large and $f \in L^2$, there exists an $r \in L^2(\wt\Omega)$ solving 
  \[ e^{-\rho\cdot x}(\Delta+q)e^{\rho\cdot x}r=f \]
  in the weak sense, and
  \[ \norm{r}_{L^2(\wt\Omega)} \leq \frac{C\norm{f}_{L^2(\wt\Omega)}}{\abs\rho}. \]
\end{cor}

This is a deeper statement than Lax-Milgram from functional analysis.

\begin{proof}[Proof of corollary \ref{3:solv}]
  Define
  \[ \cH = \left\{ e^{\bar\rho \cdot x}(\Delta+q)e^{-\bar\rho\cdot x}u \mid u \in \cC_c^\infty(\wt\Omega) \right\} \subseteq L^2(\wt\Omega). \]
  (This is the range of the adjoint operator.)
  Note that this space is neither complete nor dense in $L^2$.
  Define a linear functional $F: \cH \to \CC$ by
  \[ F\left( e^{\bar\rho \cdot x}(\Delta+q)e^{-\bar\rho\cdot x}u \right) = \int_{\wt\Omega} f \bar u. \]
  This is well-defined because of proposition \ref{3:cgo}.
  In fact it is bounded:
  \[ \abs{ F\left( e^{\bar\rho \cdot x}(\Delta+q)e^{-\bar\rho\cdot x}u \right) } = \abs{ \int_{\wt\Omega} f \bar u } \leq \norm{f}\norm{u} \leq \frac{C\norm{f}}{\abs\rho} \norm{e^{\bar\rho \cdot x}(\Delta+q)e^{-\bar\rho\cdot x}u} \]
  By Hahn-Banach, $F$ extends to a bounded linear functional $F$ on $L^2(\wt\Omega)$ with the same bound.
  So
  \[ \abs{F(w)} \leq \frac{C\norm{f}}{\abs\rho}\norm{w} \text{ for all } w \in L^2(\wt\Omega). \]
  By the Riesz representation theorem, there exists an $r \in L^2(\wt\Omega)$ such that
  \[ F(w) = \int_{\wt\Omega} r\bar w \]
  with
  \[ \norm{r} = \norm{F} \leq \frac{C\norm{f}}{\abs\rho}. \]
  Now we need to check that $r$ is a weak solution.
  \[ F \left( e^{\bar\rho \cdot x}(\Delta+q)e^{-\bar\rho\cdot x}u \right) = \int_{\wt\Omega} r \overline{ e^{\bar\rho \cdot x}(\Delta+q)e^{-\bar\rho\cdot x}u } = \int_{\wt\Omega} f\bar u \]
  for all $u \in \cC_c^\infty$.
  Hence
  \[ e^{\rho \cdot x}(\Delta+q)e^{-\rho\cdot x}r = f \]
  in the weak sense.
\end{proof}

\begin{proof}[Proof of proposition \ref{3:cgo}]
  We need weak solutions to $(\Delta+q)u=0$ in $\wt\Omega$.
  Now we have
  \[ (\Delta+q)e^{\rho\cdot x} = e^{\rho\cdot x}q. \]
  Take $u=e^{\rho\cdot x}(1+r)$ as an ansatz.
  We then get an equation of the form in corollary \ref{3:solv}.
  So we can solve for $r \in L^2(\wt\Omega)$ such that
  \[ \norm{r} \leq \frac{C\norm{q}}{\abs{p}}. \]
  So we have brought it down to proving the estimate (proposition \ref{3:est}).
  This is where the hard work is.
\end{proof}

