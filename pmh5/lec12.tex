\section{2018-04-19 Lecture}

\begin{rmk}
	All representations are over $\CC$ and are finite-dimensional.
\end{rmk}

\begin{defn}
	Let $(V,\rho)$ be a representation of a finite group.
	The \textbf{character} $\chi_V$ is the function $\chi_V:G \to \CC$ defined by $\chi_V(g)= \tr\rho(g)$.
\end{defn}

\begin{rmk}
	Recall that for a linear transformation $\phi \in \End(V)$, the trace $\tr\phi$ is defined as the sum of the diagonal entries of the matrix of $\phi$.
	This is independent of a choice of basis, so it is well-defined.
	Then $\phi$ is also the sum of the eigenvalues of $\phi$ with multiplicities.

	Recall also that $\tr(\phi\psi)=\tr(\psi\phi)$.
	But it is not true that $\tr(\phi\psi)=\tr(\phi)\tr(\psi)$.
	So $\chi_V$ is not a group homomorphism, and indeed $\chi_V(g)$ can equal zero.
	So $\chi_V(ghg\inv) = \tr(\rho(g)\rho(h)\rho(g)\inv) = \chi_V(h)$, so $\chi_V$ is constant on each conjugacy class of $G$.
	
	Note that if $V \cong W$, then $\chi_V=\chi_W$.
	To see this, note that if $\theta: V \isoto W$ is a $\CC G$-isomorphism, then $\rho_W(g)=\theta\rho_V(g)\theta\inv$, so their traces are the same.
\end{rmk}

\begin{defn}
	Define the vector space of \textbf{class functions} on $G$ to be the set of functions from $G$ (to $\CC$) which are constant on conjugacy classes:
	\[F_c(G,\CC) \defeq \{f: G \to \CC \mid f(ghg\inv)=f(g) \text{ for all } g,h \in G\}\]
\end{defn}

\begin{exam}\label{12:ex}
	\leavevmode
	\begin{enum}
		\io
		If $V=0$, then $\chi_V(g)=0$ for all $g \in G$.

		\io
		If $\dim V=1$ then $\chi_V(g)$ is the scalar by which $g$ acts on $V$.
		So in this case $\chi_V$ is effectively the same as $\rho$.
		In fact, $\chi_V \in G^\vee$, the vector space of one-dimensional characters.
		
		\io\label{12:s3}
		Recall the simple two-dimensional simple $\CC S_3$-module $P$.
		The character $\chi_P$ takes the value $2$ on the identity, $0$ on the transpositions and $-1$ on the $3$-cycles.
		We can see from this that there is no basis of $P$ which is preserved (as a set) under the action of $S_3$ since the $3$-cycles have negative trace.
	\end{enum}
\end{exam}

\begin{rmk}
	When describing representing matrices it is enough to consider a generating set of $G$.
	However since characters are not (always) homomorphisms it is not enough to specify the values of a character on a generating set.
\end{rmk}

\begin{rmk}
	If $V$ has a $\CC G$-module direct sum decomposition $V_1 \oplus \cdot \oplus V_m$, then we have
	\[\chi_V = \chi_{V_1}+\cdots+\chi_{V_m}.\]
	This is because you can use a basis of $V$ which consists of the disjoint union of a basis of each of the $V_i$s.
	Then the representing matrices are all block-diagonal with non-zero diagonal blocks the representing matrices for the $V_i$s.
\end{rmk}

\begin{exam}[Example \ref{12:ex}.\ref{12:s3} continued]
	We have $\CC^3 = L \oplus P$ where $L$ is the trivial representation and $\CC^3$ is the standard representation of $\CC S_3$.
	Then $\chi_{\CC^3} = \chi_L+\chi_P$.
	This gives us an easier way to compute $\chi_P$.
	It is clear that $\chi_{\CC^3}(g)$ is just the number of fixed points of $g \in S_3$.
	Then we just subtract $\chi_L=1$ to give $\chi_P$.
\end{exam}

\begin{rmk}
	By Maschke's theorem, any $\CC G$-module $V$ has a direct sum decompoisiton
	\[V \cong S_1^{\oplus m_1} \oplus \cdots \oplus S_r^{\oplus m_r}\]
	where $S_1,\ldots,S_r$ is a complete list of simple modules.
	Then
	\[\chi_V = m_1\chi_{S_1} + \cdots + m_r\chi_{S_r}.\]
\end{rmk}

\begin{defn}
	The \textbf{irreducible characters} of $G$ are the characters of the simple $\CC G$-modules $\chi_1 = \chi_{S_1}, \ldots, \chi_r = \chi_{S_r}$.
\end{defn}

\begin{thm}
	The irreducible characters are a basis of $F_c(G,\CC)$.
\end{thm}

\begin{proof}[Proof (outline)]
	By Wedderburn's theorem it suffices to prove that the vector space
	\[\{f \in \Hom_k(\Mat_n(\CC),\CC) \mid f(AB)=f(BA) \text{ for all } A,B \in \Mat_n(\CC)\}\]
	is one-dimensional and spanned by the trace.
	The proof is left as an exercise (compute sufficiently many $AB-BA$).
	Now $F_c(G,\CC)$ can be identified with
	\[\{f \in \Hom_k(\CC G,\CC) \mid f(ab)=f(ba) \text{ for all } a,b \in G\}.\]
	Then using Wedderburn's theorem, such functions are just linear combinations of functions of the form
	\[\CC G \isoto \prod_{i=1}^r \Mat_{n_i}(\CC) \xto{p_i} \Mat_{n_i}(\CC) \xto{\tr} \CC. \qedhere\]
\end{proof}

\begin{cor}
	The number of simple $\CC G$-modules is equal to the number of conjugacy classes of $G$ (which is equal to the dimension of $F_c(G,\CC)$).
\end{cor}

\begin{cor}\label{12:char}
	If $\chi_V=\chi_W$, then $V \cong W$.
\end{cor}

\begin{proof}
	Let $V = \bigoplus_{i=1}^r S_i^{\oplus m_i(V)}$ and $W = \bigoplus_{i=1}^r S_i^{\oplus m_i(W)}$.
	Then $\chi_V = \sum_i m_i(V)\chi_i$ and $\chi_W = \sum_i m_i(w)\chi_i$.
	By linear independence of the irreducible characters, $\chi_V=\chi_W$ implies that $m_i(V)=m_i(W)$ for all $i$.
\end{proof}

\begin{rmk}
	We know that
	\[V \cong W \iff [\rho_W(g)] = M[\rho_V(g)]M\inv \text{ for some } M \in \GL_n(\CC).\]
	Note that the conjugating matrix $M$ must be the same for all $g \in G$ (but we only need to check on a generating set of $G$).
	Now corollary \ref{12:char} tells us that
	\[[\rho_W(g)] \text{ is similar to } [\rho_V(g)] \text{ for all } g \in G \implies \chi_V =\chi_W \iff V \cong W.\]
	Note that here we may choose a different conjugating matrix for each $g \in G$ (but it is not enough to check only on a generating set of $G$).
	
	Here we have used the notation $[\phi]$ to mean the matrix of the linear transformation $\phi$ with respect to some basis.
\end{rmk}
