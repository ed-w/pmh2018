\section{2018-04-11 Lecture}

Representations of finite groups

\begin{defn}
	Let $G$ be a finite group and $k$ be a(n algebraically closed) field.
	A \textbf{representation} of $G$ over $k$ is a $k$-vector space $V$ and a group homomorphism $\rho: G \to \GL(V)$.
	
	Equivalently, it is a $k$-vector space $V$ and a $k$-algebra homomorphism $\sigma: k[G] \to \End(V)$ (a $k[G]$-module).
\end{defn}

\begin{rmk}
	To see that the two definitions are equivalent, note that $\rho$ determines $\sigma$ since by extending from the vector space basis $G$ in $k[G]$.
	It is clear that $\sigma$ determines $\rho$ by restriction, and since $\rho(g)\rho(g\inv)=\id_V$, we have that the image of $\rho$ is in $\GL(V)$.
\end{rmk}

\begin{rmk}
	If you have a presentation for $G$ with generators $g_1,\ldots,g_n$ and some relations, then specifying $\rho$ amounts to specifying elements $\rho(g_1),\ldots,\rho(g_n) \in \GL(V)$ such that the defining relations are satisfied.
\end{rmk}

\begin{exam}
	\leavevmode
	\begin{enum}
		\io
		Consider $C_n = \ang{x \mid x^n=1}$ for $n \geq 1$.
		A representation of $C_n$ is a $k$-module $V$ and an invertible map $X$ with $X^n=\id_V$.
		
		Let $n=2$, $k=\RR$ and $V=\RR^2$.
		Then we could take $X=\id_V$, $X=-\id_V$ or $X$ to be a reflecion or any other linear transformation with eigenvalues $1$ and $-1$ (glide reflections, (where the eigenspaces not perpendicular)).
		The Jordan normal form theorem tells us that this is a complete list of $\RR[C_2]$-module structures on $\RR^2$.
		
		Note that if $k$ has characteristic two, then the matrix
		$\left(\begin{smallmatrix}
		1 & 1 \\ 0 & 1
		\end{smallmatrix}\right)$
		squares to the identity, so there are more $\RR[C_2]$-module structures on $\RR^2$.
		
		\io
		Consider $D_n = \ang{x,y \mid x^n=1,y^2=1,yx=x\inv y}$.
		A representation of $D_n$ is a $k$-module $V$ together with invertible maps $X$ and $Y$ such that $X^n=Y^2=\id_V$ and $YX=X\inv Y$.
		
		Let $n=2$, $k=\RR$ and $V=\RR^2$.
		(Then $D_2 \cong V_4$, the Klein four-group.)
		Then the defining relations are $X^2=Y^2=1$ and $XY=YX$.
		So we the following list of $\RR[D_2]$-module structures on $\RR^2$:
		\begin{enum}
			\io $X=\id$, $Y=\id$
			\io $X=\id$, $Y=-\id$
			\io $X=\id$, $Y$ a glide reflection
			\io $X=-\id$, $Y=\id$
			\io $X=-\id$, $Y=-\id$
			\io\label{item:casef} $X=-\id$, $Y$ a glide reflection
			\io $X$ a glide reflection, $Y=\id$
			\io $X$ a glide reflection, $Y=-\id$
			\io $X$ and $Y$ commuting glide reflections
			\par
			In this case, $X$ and $Y$ preserve each others' eigenspaces, hence they have the same eigenspaces.
			So there are two sub-cases:
			\begin{enum}
				\io\label{item:case2i'}
				The $\pm1$-eigenspace of $X$ is the $\pm1$-eigenspace of $Y$ $\iff X=Y$.
				\par
				A representation of this type is not faithful (injective).
				
				\io\label{item:case2i''}
				The $\pm1$-eigenspace of $X$ is the $\mp1$-eigenspace of $Y$ $\iff X=-Y$.
				\par
				A representation of this type is faithful.
				In this case, we have
				\begin{align*}
					1 &\mapsto \id \\
					x &\mapsto X \\
					y &\mapsto -X \\
					xy &\mapsto -\id
				\end{align*}
				A standard way of thinking about $V_4$ is to let $X$ be reflection about the $x$-axis and $Y$ reflection about the $y$-axis.
				
				Note that the representation in case \ref{item:casef} is given by
				\begin{align*}
					1 &\mapsto \id \\
					x &\mapsto -\id \\
					y &\mapsto Y \\
					xy &\mapsto -Y
				\end{align*}
				since $XY=-Y$ in this case.
				So the two representations are not the same.
			\end{enum}
		\end{enum}
	\end{enum}
\end{exam}

\begin{rmk}
	Translating the definition of module homomorphism to representations, we see that two representations $(V,\rho)$ and $W<\sigma)$ of $G$ over $k$ are isomorphic $\iff$ there exists an invertible linear map $\Theta:V \isoto W$ such that $\theta\rho(g)=\sigma(g)\theta$ for all $g \in G$, or equivalently, $\theta\rho(g)\theta\inv=\sigma(g)$ for all $g \in G$.
\end{rmk}

\begin{rmk}
	The classifications in examples 1 and 2 were up to the equivalence relation induced by isomorphisms.
	
	If the representations in case \ref{item:case2i'} and \ref{item:case2i''} are isomorphic, then there exists a $\theta \in \GL_2(\RR)$ such that $\theta\rho(x)\theta\inv = \sigma(x)$ and some $\phi \in \GL_2(\RR)$ such that $\phi\rho(y)\phi\inv=\sigma(y)$ but no \underline{single} $\theta \in \GL_2(\RR)$ which does both simultaneously.	
\end{rmk}

\begin{rmk}
	We want to classify finite-dimensional representations of $G$ over $k$.
	Choosing bases, this amounts to a question about matrices: we specify a matrix $\rho(g) \in \GL_n(k)$ for each $g \in G$ satisfying the relations of $G$.
	Then an isomorphism of representations is a single matrix $M \in \GL_n(k)$ which satisfies $M\rho(g)M\inv=\sigma(g)$ for all $g \in G$.
	That is, $M$ simultaneously conjugates elements of one representation to the other.
	(This is known as simultaneous conjugacy or similarity.)
	
	These matrix equations work fine for small groups, but for large groups it becomes difficult.
	We will study group representations using the group algebra instead.
\end{rmk}
