\section{2018-03-22 Lecture}

\begin{prop}\label{prop:dsum}
	If $V$ is a (semisimple) $A$-module with a direct sum decomposition of the form $V = V_1 \oplus \cdots \oplus V_m$ where the $V_i$ are pairwise non-isomorphic simple submodules, then every submodule $U$ of $V$ is a direct sum of some subset of the $V_i$.
\end{prop}

\begin{cor}
	Let $A = \CC[x]$.
	Then every simple $A$-module is one-dimensional, so proposition \ref{prop:dsum} applies to modules $(V,X \in \End V)$ where $X$ is diagonalisable with \underline{distinct} eigenvalues $\lambda_1,\ldots,\lambda_m$.
	If $v_1,\ldots,v_m$ are corresponding eigenvectors, then $V = \CC v_1 \oplus \cdots \oplus \CC v_m$.
	Proposition \ref{prop:dsum} tells us that every $X$-invariant subspace $U \subseteq V$ is spanned by a subset of the eigenvectors.
\end{cor}

\begin{exam}
	Let $V = \CC[x,y]$ and let
	\[A = \ang{x \frac{\p}{\p x}, x \frac{\p}{\p y}, y \frac{\p}{\p x}, y \frac{\p}{\p y}} \subset \End(V)\]
	What are the submodules of $V$?
	
	The generators of $A$ preserve the degree of monomials in $x$ and $y$.
	So there is an obvious direct sum decomposition
	\[V = \bigoplus_{d=0}^\infty V_d \quad \mbox{where} \quad V_d = \spn\{x^d,x^{d-1}y,\ldots,xy^{d-1},y^d\}\]
	
	diagram here showing the action of the operators on the basis of each $V_d$
	
	Is each $V_d$ a simple $A$-module or does it have further submodules?
	Let $U$ be a non-zero submodule of $V_d$.
	If we know that $U$ has to contain a monomial $x^ay^{d-a}$ then we can just use $x \tfrac{\p}{\p y}$ and $y \tfrac{\p}{\p x}$ to get all other monomials $x^by^{d-b}$.
	However an arbitrary subspace does not have to contain an element of a pre-determined basis.
	
	Let $B$ be the subalgebra of $A$ generated by $x \tfrac{\p}{\p x}$ (and optionally also $y \tfrac{\p}{\p y}$).
	Then $V_d = \CC x^d \oplus \CC x^{d-1}y \oplus \cdots \oplus \CC y^d$ is a direct sum of pairwise one-dimensional (hence simple) $B$-submodules.
	The operator $x \tfrac{\p}{\p x}$ acts on $\CC x^a y^{d-a}$ as multiplication by $a$.
	Then since $U$ must also be a $B$-submodule, by proposition \ref{prop:dsum} $U$ must contain some monomial $x^a y^{d-a}$.
	So each $V_d$ is a simple $A$-module and applying proposition \ref{prop:dsum} again we get that every $A$-submodule of $V$ is a direct sum of the $V_d$.
	(By dimensionality the $V_d$ are pairwise non-isomorphic.)
	
	Note that the subalgebra $\CC$ of $A$ generated by $x \tfrac{\p}{\p y}$ does not act semisimply on $V_d$: for example
	\[V_1: \left[x \frac{\p}{\p y}\right]_{x,y} =
		\begin{bmatrix}
			0 & 1 \\
			0 & 0
		\end{bmatrix}
	\]
	$V_1$ is not a semisimple $\CC$-module since $\CC x$ is a submodule with no complementary submodule.
\end{exam}

\begin{defn}
	For any simple $A$-module $S$, the \textbf{isotypic component} of a semisimple $A$-module $V = \bigoplus_{i=1}^n V_i$ corresponding to $S$ is
	\[V_{[S]} = \bigoplus_{V_i \cong S}V_i \subset V\]
\end{defn}

\begin{thm}
%	Let $A$ be any $k$-algebra.
%	Let $V$ be a finite-dimensional semisimple $A$-module:
%	\[V = V_1^{\oplus n_1} \oplus V_2^{\oplus n_2} \oplus \cdots \oplus V_m^{\oplus n_m}\]
%	where the $V_i$ are pairwise non-isomorphic simple submodules.
	Any simple $A$-submodule $U$ of $V$ is contained in its isotypic component $V_{[U]}$.
	Then $V_{[S]} = \sum_U U$, where the sum is over all simple submodules $U$ of $V$ with $U \cong S$.
	In particular, $V_{[S]}$ is independent of the chosen decomposition $\bigoplus_{i=1}^n V_i$ of $V$.
	The decomposition $V = \bigoplus_S V_{[S]}$, where the sum is over all isoclasses of simple $A$-modules $S$, is \underline{canonical}, but the decomposition of each isotypic component is \underline{not canonical}.
\end{thm}

\begin{proof}
	Let $\{p_i: V \to V_i\}$ be the projection maps arising from the direct sum decomposition.
	Let $\phi: U \injto V$ be the inclusion of a simple submodule $U$.
	Then $\phi = p_1\phi + \cdots + p_m\phi$ with each $p_i\phi \in \Hom_A(U,V_i)$.
	Then by Schur's lemma part I, we have that $p_i\phi=0$ unless $U \cong V_i$.
	So
	\[\phi = \sum_{i: V_i \cong U} p_i\phi\]
	and therefore
	\[U = \im\phi \subseteq \bigoplus_{i: V_i \cong U} = V_{[U]} \qedhere\]
\end{proof}

\begin{cor}
	Any submodule $U$ of $V$ is of the form $U_1 \oplus \cdots \oplus U_m$ where each $U_i$ is an $A$-submodule of $V_{[S_i]}$.
\end{cor}

\begin{proof}
	As seen in the last lecture, $U$ is also semisimple, so $U$ is a sum of simple submodules, each of which is contained in one of the $V_{[S_i]}$.
\end{proof}

\begin{rmk}
	This proves proposition \ref{prop:dsum}.
\end{rmk}