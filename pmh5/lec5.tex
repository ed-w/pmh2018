\section{2018-03-21 Lecture}

\begin{defn}
	Let $B$ be a subalgebra of $A$.
	(A subalgebra is a subspace closed under multiplication containing $1$.)
	The \textbf{commutant} or \textbf{centraliser} of $B$ in $A$ is
	\[Z_A(B) = \{a \in A \mid ab=ba \text{ for all } b \in B\}\]
	The \textbf{centre} of A is
	\[Z(A) = Z_A(A) = \{a \in A \mid ab=ba \text{ for all } b \in A\}\]
\end{defn}

\begin{prop}\label{prop:centre}
	Some easy properties:
	\begin{itm}
		\io $Z_A(B)$ is a subalgebra of $A$.
		\io $B$ is commutative $\iff B \subseteq Z_A(B)$.
		\io $A$ is commutative $\iff Z(A)=A$.
		\io For any $B$, $B \subseteq Z_A(Z_A(B))$.
	\end{itm}
\end{prop}

\begin{exam}
	If $(V,\rho)$ is a representation of $A$, then $\End_A(V) = Z_{\End(V)}(\rho[A])$.
\end{exam}

\begin{lem}
	The centre of the endomorphism algebra of a vector space is the subspace of scalar multiplications.
	\[Z(\End(V))=k\]
\end{lem}

\begin{proof}
	If $\phi$ commutes with every linear transformation then it preserves the eigenspaces of every linear transformation, that is, it preserves every subspace of $V$.
	Then if $v_1,v_2 \in V$, then $\phi(v_1)=\lambda_1v_1$, $\phi(v_2)=\lambda_2v_2$ and $\phi(v_1+v_2)=\lambda_1v_1+\lambda_2v_2 = \lambda_(v_1+v_2)$ for some $\lambda,\lambda_1,\lambda_2 \in k$.
	So $\lambda_1=\lambda_2$.
\end{proof}

\begin{note}
	Schur's lemma part II says that is $k$ is algebraically closed and $V$ is a finite dimensional simple $A$-module, then $\End_A(V)=k$, i.e.\@ $Z_{\End(V)}(\rho[A])=k$.
	So proposition \ref{prop:centre} tells us that
	\[\rho[A] \subseteq Z_{\End(V)}\left(Z_{\End(V)}(\rho[A])\right) = Z_{\End(V)}(k) = \End(V)\]
	Later we wil see the density theorem which implies in this case that $\rho[A]=\End(V)$.
	
	The contrapositive states that (if $k$ is algebraically closed and $V$ is a finite-dimensional $A$-module) if $\rho[A] \subsetneq \End(V)$, then $V$ has a non-trivial $A$-module.
	So for the given preconditions, finding whether a module is simple is the same as seeing if all linear transformations arise out of the action of $A$ on the module.
\end{note}

\begin{defn}
	An $A$-module $V$ is \textbf{semisimple} if it is a direct sum of simple submodules.
\end{defn}

\begin{rmk}
	Every simple module is semisimple.
\end{rmk}

\begin{prop}
	$V$ is semisimple $\iff$ for any submodule $W$ of $V$ there exists a direct sum complement (of modules) $W'$ (\i.e.\@ $V = W' \oplus W'$).
\end{prop}

\begin{proof}
	We will only prove the proposition in the case where $V$ is finite-dimensional.
	Zorn's lemma si required in the infinite-dimensional case.
	
	$\impliedby$: Take $V$ if $V$ is simple.
	Otherwise let $W$ be a non-trivial submodule.
	We will need the following lemma which we will not prove.
	\begin{lem}
		The existence of complements property is inherited by submodules.
		That is, if $V = Y \oplus Z$ and $X \subset Y$, then $X = Y \oplus (X \cap Z)$.
	\end{lem}	
	By assumption, $V = W \oplus W'$ for some submodule $W'$.
	Then since $\dim W' < \dim V < \infty$, we can iterate until completion.
	
	$\implies:$
	Assume $V = V_1 \oplus \cdots \oplus V_n$, where each $V_i$ is a simple submodule of $V$.
	Let $J \subseteq [n]$ be a maximal set (w.r.t\@ inclusion) such that
	\[W \cap \bigoplus_{j \in J} V_j = \{0\}\]
	Clearly a maximal set exists since $J = \emptyset$ satisfies the above equation of sets.
	We claim that 
	\[W + \bigoplus_{j \in J} V_j = V\]
	Then the above sum is a direct sum and so we have found a direct sum complement.
	
	To prove the above claim, we just need to show that for each $i \notin J$,
	\[V_i \subseteq W + \bigoplus_{j \in J} V_j\]
	Consider
	\[X_i = V_i \cap \left( W + \bigoplus_{j \in J} V_j \right)\]
	Since $V_i$ is simple, either $X_i = \{0\}$ or $X_i = V_i$.
	In the second case we have the desired inclusion of sets.
	In the first case the sum
	\[V_i + \left( \bigoplus_{j \in J} V_j + W \right)\]
	is direct, contradicting the maximality of $J$.
\end{proof}

\begin{cor}[Corollary to Schur's lemma part II]\label{cor:schur}
	If $k$ is algebraically closed and $(V,\rho)$ is a finite-dimensional irreducible representation of $A$, then for every $z \in Z(A)$, $\rho(z)$ is a scalar multiplication.
\end{cor}

\begin{proof}
	$\rho(z) \in Z(\rho[A]) \subseteq Z_{\End(V)}(\rho[A]) = \End_A(V) = k$.
\end{proof}

\begin{exam}
	Let
	\[A = U(\sli_2) = \CC\ang{e,h,f}/\ang{he-eh-2e,hf-fh+2f,ef-fe-h}\]
	the \textbf{universal enveloping algebra} of the Lie algebra $\sli_2$.
	$Z(A)$ is generated by the \textbf{Casimir element}
	\[\Omega = ef+fe+\frac 12h^2\]
	So on any finite-dimensional simple $U(\sli_2)$-module, $\Omega$ acts as scalar multiplication.
\end{exam}

\begin{prop}
	Let $k$ be algebraically closed.
	If $A$ is commutative, every finite-dimensional simple $A$-module $V$ is one-dimensional.
\end{prop}

\begin{proof}
	By corollary \ref{cor:schur}, every $a \in A$ acts on $V$ as a scalar multiplication, so every subspace of $V$ is a submodule.
	$V$ is simple, so $V$ has no non-trivial subspaces, hence $\dim V = 1$.
\end{proof}

\begin{note}
	If $A$ is commutative and $k$ is algebraically closed, then one-dimensional representations of $A$ are one-dimensional vector spaces $V \cong k$ with an algebra homomorphism $\rho: A \to \End(V) \cong k$, that is, an algebra homomorphism from $A$ to $k$.
	If $A$ is Noetherian, then this is equivalent to a maximal ideal of $A$ (the kernel of the homomorphism).
\end{note}

\begin{exam}
	Let $A = k[x]$.
	Then every simple finite-dimensional $k[x]$-module is one-dimensional.
	Another way of seeing this is that if $V$ is a finite-dimensional vector space and $X \in \End(V)$, then either $\dim V=1$ or there is some non-trivial $X$-invariant subspace (e.g.\@ the span of an eigenvector).
	
	So a $k[x]$-module $V$ is semisimple $\iff$ $V$ is a direct sum of one-dimensional submodules, that is $V$ has a basis consisting of eigenvectors of $X$ (i.e.\@ $X$ is diagonalisable).
\end{exam}