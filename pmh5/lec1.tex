\section{2018-03-07 Lecture}

Historically representations of groups was the first part of representation theory to be studied.
Group representations can be thought of concrete realisations of abstract groups as linear transformations (matrices).

\begin{defn}
	Let $G$ be a group and let $k$ be a field.
	A \textbf{representation} of $G$ over $k$ is a pair $(V,\rho)$ where $V$ is a vector space over $k$ and $\rho: G \to \GL(V)$ is a group homomorphism.
	$GL(V)$ is the group of invertible linear transformations of $V$ under composition.
\end{defn}

\begin{rmk}
	For every $g \in G$, $\rho(g):V \to V$ is linear and invertible.
	The hom property says that $\rho(gh)=\rho(g)\circ\rho(h)$, i.e.\@ $\rho(gh)(v) = \rho(g)\left(\rho(h)(v)\right)$ for all $g,h \in G$ and $v \in V$.
	This implies that $\rho(1_G)=\id_V$ and $\rho(g\inv)=\rho(g)\inv$.	
\end{rmk}

\begin{exam}
	How can we define a group?
	Consider $G \cong D_3 \cong S_3$.
	We can:
	\begin{itm}
		\io Let it be the symmetry group $D_3$ of an equilateral triangle
		\io Let it be the group $S_3$ of permutations of $[3]$
		\io Define it as the set $\{g_1,\ldots,g_6\}$ with a group multiplication table
		\io By a presentation and relations:
		\[G=\ang{x,y \mid x^3=1, y^2=1, yx=x^2y}\]
	\end{itm}
	
	Consider the following ``geometric'' representation of $G$ (as $D_3$):
	Let $V=\RR^2$ and $k=\RR$.
	Define $\rho:G \to \GL(\RR^2)$ by
	\begin{itm}
		\io $\rho(1)$ is the identity operator on $\RR^2$
		\io $\rho(x)$ is rotation (counter-clockwise) by $2\pi/3$
		\io $\rho(x^2)$ is rotation by $4\pi/3$
		\io $\rho(y)$ is reflection in the $x$-axis
		\io $\rho(xy)$ is reflection in the $x$-axis followed by rotation y $2\pi/3$
		\io $\rho(x^2y)$ is reflection in the $x$-axis followed by rotation y $4\pi/3$
	\end{itm}
	[insert sketch of D3 operations here]
	To confirm that this is a representation, we need to check that the group homomorphism property holds.
	
	Another representation is given by the $3 \times 3$ permutation matrices.
	
	The point is that there are many representations of the same abstract group $G$ and it is advantageous to study them all together (by classifying them all).
	
	For example, there are vector spaces of solutions to a linear partial differential operator (such as the Schr\"odinger operator) such that if the operator has a symmetry group, then the space of solutions is a representation of that group.
	In this way we can think of group representation theory as linear algebra in the presence of symmetry.
\end{exam}

More generally, we can look at representations of associative algebras over a field $k$.

\begin{defn}
	Let $A$ be an (associative) $k$-algebra.
	Then a representation of $A$ is a pair $(V,\rho)$ where $V$ is a $k$-vector space and $\rho:A \to \End(V)$ is a homomorphism of $k$-algebras.
\end{defn}

\begin{rmk}
	The above definition is a generalisation of a group representations: a representation of $(V,\rho)$ of $kG$ restricts to a representation $(V,\rho|_G)$ of $G$ where $G$ is considered as a subset of $kG$.
	Converseley, a representation of $G$ extends linearly to a representation of $kG$.
\end{rmk}