\section{2018-03-28 Lecture}

\begin{prop}
	If $V = V_1 \oplus \cdots \oplus V_n$ and $W = W_1 \oplus \cdots \oplus W_m$, then any linear map $\phi \in \Hom(V,W)$ has components
    \[\phi_{ij} = p_i \phi a_j \in \Hom(V_j,W_i)\]
    for $j=1,\ldots,m$ and $i=1,\ldots,n$, where $a_j: V_j \injto V$ is the inclusion and $p_i: W \surjto W_i$ is the projection.
    We can write it as a matrix
    \[
    \begin{pmatrix}
    	\phi_{11} & \phi_{12} & \cdots & \phi_{1n} \\
        \phi_{21} & \phi_{22} & \cdots & \phi_{2n} \\
        \vdots & \vdots & \ddots & \vdots \\
        \phi_{m1} & \phi_{m2} & \cdots & \phi_{mn}
    \end{pmatrix}
    \]
    where the entries of the matrix are linear operators.
    Composition of linear maps corresponds to matrix multiplication: let $\psi: W \to X = X_1 \oplus \cdots \oplus X_p$.
    Then
    \[(\psi\circ\phi)_{ik} = \sum_j \psi_{ij} \circ \phi_{jk}\]
    In particular, we can depict elements in $\End(V)$ as $n \times n$ matrices where the $ij$ entry belongs to $\Hom(V_j,V_i)$.
    
	Now suppose $A$ is an algebra, $V$ and $W$ are $A$-modules and each $V_i$ and each $W_j$ is an $A$-submodule.
    Then $a_j$ and $p_i$ are $A$-module homomorphisms.
\end{prop}

\begin{prop}
	$\phi \in \Hom(V,W)$ is an $A$-module homomorphism $\iff$ $\phi_{ij} \in \Hom(V_j,W_i)$ is an $A$-module homomorphism for all $i$ and $j$.
\end{prop}

\begin{proof}
	$\implies$: $\phi_{ij} = p_i \phi a_j$ is the composition of $A$-module homomorphisms.
    
    $\impliedby$: Define $\phi(v_1+\cdots+v_n) = \sum_i \phi_{ij}(v_j)$ where $v_i \in V$.
    It is clear that $\phi \in \Hom(V,W)$.
    Now for $a \in A$,
    \begin{multline*}
    	\phi(a(v_1+\cdots+v_n)) = \phi(av_1+\cdots+av_n) = \sum_{i=1}^n \phi_{ij}(av_j) \\
    	= \sum_{i=1}^n a \phi_{ij}(v_j) = a \sum_{i=1}^n \phi_{ij}(v_j) = a\phi(v_1+\cdots+v_n) \qedhere
    \end{multline*}
\end{proof}

\begin{cor}
	We can depict elements in $\End_A(V)$ as $n \times n$ matrices where the $ij$-th entry belongs to $\Hom_A(V_j,V_i)$ and where composition of maps corresponds to matrix multiplication.
\end{cor}

\begin{thm}[Finite-dimensional density theorem]
	If $V$ is a finite-dimensional semisimple $A$-module with $\rho:A \to \End(V)$ the algebra homomorphism of the action, then $\rho[A] = Z_{\End(V)}\left(\End_A(V)\right)$.
\end{thm}

\begin{proof}
  "$\subseteq$":
  Since $\End_A(V) = Z_{\End(V)}(\rho[A])$, by the double centraliser theorem this inclusion is obvious.

  "$\supseteq$":
  We want to show that if $\psi \in \End(V)$ is such that $\psi\phi = \phi\psi$ for all $\phi\in\End_A(V)$, then there exists an $a \in A$ such that $av = \psi(v)$ for all $v \in V$.
  First we need the following lemma.
    
\begin{lem}
	Let $A$, $V$ and $\psi$ be as above.
    Then for all $v \in V$, there exists an $a \in A$ such that $av = \psi(v)$.
\end{lem}

Note that this is not what we want because in this case $a$ can depend on $v$.

\begin{proof}
	Since $V$ is semisimple, $V = Av \oplus V'$ for some $A$-submodule $V'$.
    Then define $\phi\in\End_A(V)$ by
    \[
    	\phi =
        \begin{pmatrix}
        	0_{Av} & 0 \\
            0 & \id_{V'}
        \end{pmatrix}
    \]
    Since $\psi\phi = \phi\psi$, $\psi$ must preserve every eigenspace of $\phi$. 
    In particlar, we have $\psi(v) \in \psi[Av] \subseteq Av$.
\end{proof}

    We continue with the proof of the density theorem.
    Let $v_1,\ldots,v_d$ be a basis of $V$.
    By linearity, it is enough to show that there exists an $a \in A$ such that $av_i = \psi(v_i)$ for all $i$.
    Consider $V^{\oplus d} = V^d$ which is a finite-dimensional semisimple $A$-module under the action $a(w_1,\ldots,w_d) = (aw_1,\ldots,aw_d)$.
    Now apply the lemma to $V^d$ where we take $\psi$ to be
    \[
    	\Psi=
    	\begin{pmatrix}
        	\psi & 0 & \cdots & 0 \\
            0 & \psi & \cdots & 0 \\
            \vdots & \vdots & \ddots & \vdots \\
            0 & 0 & \cdots & \psi
        \end{pmatrix}
    \]
    and $v = (v_1,\ldots,v_d)$.
    We need to check that $\Psi$ commutes with all elements of $\End_A(V^d)$, that is
    \[
    	\begin{pmatrix}
        	\psi & 0 & \cdots & 0 \\
            0 & \psi & \cdots & 0 \\
            \vdots & \vdots & \ddots & \vdots \\
            0 & 0 & \cdots & \psi
        \end{pmatrix}
        \begin{pmatrix}
            \phi_{11} & \phi_{12} & \cdots & \phi_{1n} \\
            \phi_{21} & \phi_{22} & \cdots & \phi_{2n} \\
            \vdots & \vdots & \ddots & \vdots \\
            \phi_{m1} & \phi_{m2} & \cdots & \phi_{mn}
    	\end{pmatrix}
    	\overset{?}{=}
        \begin{pmatrix}
            \phi_{11} & \phi_{12} & \cdots & \phi_{1n} \\
            \phi_{21} & \phi_{22} & \cdots & \phi_{2n} \\
            \vdots & \vdots & \ddots & \vdots \\
            \phi_{m1} & \phi_{m2} & \cdots & \phi_{mn}
    	\end{pmatrix}
        \begin{pmatrix}
        	\psi & 0 & \cdots & 0 \\
            0 & \psi & \cdots & 0 \\
            \vdots & \vdots & \ddots & \vdots \\
            0 & 0 & \cdots & \psi
        \end{pmatrix}
    \]
    But this is easily verified as $\psi$ commutes with each of the $\phi_{ij}$ individually.
    
    So the lemma implies that for all such $\psi \in Z_{\End(V)}(\End_A(V))$ there exists an $a \in A$ such that 
    \[(av_1,\ldots,av_d) = a(v_1,\ldots,v_d) = \Psi (v_1,\ldots,v_d) = (\psi(v_1),\ldots,\psi(v_d))\]
    that is, $av = \psi(v)$ for all $v \in V$.
\end{proof}

\begin{exam}
	\begin{enum}
		\io
        Suppose $V=S$ is a finite dimensional simple $A$-module.
        Then $\End_A(S)$ is a division algebra $D$ (by Schur's lemma part I), so you can think of $S$ as a $D$-module.
        The density theorem implies that $\rho[A] = Z_{\End(S)}(D) = \End_D(S)$.        
        If $k$ is algebraically closed then $D=k$, hence $\rho[A] = \End(S)$, that is, the module action homomorphism $\rho:A \to \End(V)$ is surjective.
        
        In general, modules over division algebras are a lot like vector spaces---they have bases and dimensions.
        We can write $d = \dim_D(S)$ and then $S = D^{\oplus d}$.
		Then $\End_D(S)$ consists of $d \times d$ matrices with entries in $\End_D(D) \cong D\op$.
        Then the density theorem implies that $\rho[A] = \End_D(S) \cong \Mat_d(D\op)$.
        
        \io
        Let $V = S_1 \oplus \cdots \oplus S_r$ where $S_1,\ldots,S_r$ are pairwise non-isomorphic simple modules.
        Then by Schur's lemma,
        \[\End_A(V)=\End_A(S_1 \oplus \cdots \oplus S_r)=
	        \begin{pmatrix}
		        D_1 & 0 & \cdots & 0 \\
		        0 & D_2 & \cdots & 0 \\
		        \vdots & \vdots & \ddots & \vdots \\
		        0 & 0 & \cdots & D_r
	        \end{pmatrix}
        \]
        where $D_i = \End_A(S_i)$.
        
        Assume $k$ is algebraically closed.
        Then $D_i = k$ for all $i$.
        So
        \[Z_{\End(V)}
	        (\begin{pmatrix}
		        k & 0 & \cdots & 0 \\
		        0 & k & \cdots & 0 \\
		        \vdots & \vdots & \ddots & \vdots \\
		        0 & 0 & \cdots & k
	        \end{pmatrix})
	        =
	        \begin{pmatrix}
		        \End_k(S_1) & 0 & \cdots & 0 \\
		        0 & \End_k(S_2) & \cdots & 0 \\
		        \vdots & \vdots & \ddots & \vdots \\
		        0 & 0 & \cdots & \End_k(S_r)
	        \end{pmatrix}
        \]
        This is the set of all maps in $\End(V)$ which preserve the subspaces $S_i$ for all $i$.
        Then by the density theorem,
        \[\rho[A] \cong \End_k(S_1) \times \cdots \times \End_k(S_r) \cong \Mat_{d_1}(k) \times \cdots \times \Mat_{d_r}(k)\]
        where $d_i = \dim(S_i)$ for all $i$.
	\end{enum}
\end{exam}