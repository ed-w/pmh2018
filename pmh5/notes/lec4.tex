\section{2018-03-15 Lecture}

\begin{prop}
	Let $\theta: V \to W$ be an $A$-module homomorphism.
	Then $\ker\theta$ is a submodule of $V$ and $\im\theta \cong A/\ker\theta$ is a submodule of $W$.
\end{prop}

\begin{proof}
	Note that $\Hom_A(V,W)$ is a subspace of $\Hom_k(V,W)$ since the condition $\theta(av)=a\theta(v)$ is linear.
\end{proof}

\begin{thm}[Schur's Lemma part I]
	If $V$ and $W$ are simple $A$-modules then any $\theta \in \Hom_A(V,W)$ is either zero or invertible (an isomorphism).
\end{thm}

\begin{proof}
	Since $\ker\theta$ is a submodule of a simple module $V$ then either either $\ker\theta = V \implies \theta=0$ or $\ker\theta=0 \implies \theta$ is injective.
	Since $\im\theta$ is a submodule of a simple module $W$ then either either $\im\theta = 0 \implies \theta=0$ or $\im\theta=W \implies \theta$ is surjective.
\end{proof}

\begin{cor}
	If $V$ and $W$ are simple and $V \not\cong W$, then $\Hom_A(V,W)=0$.
\end{cor}

\begin{cor}
	If $V$ is simple, then $\End_A(V)$ is a division algebra (every non-zero element has a two-sided inverse).
\end{cor}

\begin{exam}
	\begin{enum}
		\io Any one-dimensional $A$-module $V$ is simple.
		In this case $\End_A(V) = \End_k(V) = k$.
		\io Let $A=\RR[x]$ and $V=\RR^2$ with $x$ acting as rotation by $2\pi/3$.
		$V$ is simple because lines (one-dimensional subspaces) are preserved by the rotation.
		Then $\End_{\RR[x]}(V) \subset \End(V) \cong \Mat_2(\RR)$ and $\End_{\RR[x]}(V)$ consists of all matrices in $\Mat_2(\RR)$ which commute with the $2\pi/3$ rotation matrix.
		It turns out that $\End_{\RR[x]}(V)$ is the set of $2 \times 2$ skew-symmetric matrices in $\RR$ which is isomorphic to the complex numbers.
		\io Let $A=V=\HH$ with the left regular representation.
		This is simple because for any non-zero $v \in \HH$, $v$ is invertible hence $\HH v = \HH$.
		Now $\End_\HH(\HH) \subset \End_\RR(\HH) \cong \Mat_4(\RR)$, and $\End_\HH(\HH) = \{\phi \in \End_\RR(\HH) \mid \phi(q_1q_2) = q_1\phi(q_2)\}$
		
		We can think of the associative law $q_1(q_2q_3)=(q_1q_2)q_3$ as a composition of multiplications: $L(q_1)R(q_3)q_2=R(q_3)L(q_1)q_2$.
		This suggests we need to look at right multiplications.
		Indeed, $\End_\HH(\HH) = \{R(q) \mid q \in \HH\}$ where $R(q)q' = q'q$.
		Note that $q \mapsto R(q)$ is not an $\HH$-module homomorphism from $\HH$ to $\End_\HH(\HH)$ since $R(q_1q_2)=R(q_2)R(q_1)$, but $q \mapsto R(\bar{q})$ is, since $q \mapsto \bar q$ is an anti-involution on $\HH$ (i.e.\@ $\overline{q_1q_2}=\bar{q_2}\bar{q_1}$).
	\end{enum}
\end{exam}

\begin{thm}[Schur's Lemma part II]
	If $V$ is a finite dimensional simple $A$-module and $k$ is algebraically closed, then $\End_A(V)=k$.
\end{thm}

\begin{proof}[Proof 1]
	If $\theta \in \End_A(V)$, then $\theta$ has an eigenvalue $\lambda \in k$.
	The eigenspace $V_\lambda^\theta = \ker(\theta-\lambda\id)$ is a submodule of $V$ which is non-zero by assumption, so $V_\lambda^\theta=V$ since $V$ is simple.
	So $(\theta-\lambda\id)v=0$ for all $v \in V$, hence $\theta=\lambda\id$.
\end{proof}

\begin{proof}[Proof 2]
	By part I, $\End_A(V)$ is a (finite dimensional) division algebra over $k$.
	It is enough to show that if $k$ is algebraically closed the only finite dimensional division algebra over $k$ is $k$ itself.
	
	Suppose there exists a division algebra $D \neq k$ and let $d \in D \setminus k$.
	Since $D$ is a division algebra, the subalgebra of $D$ generated by $d$ is commutative (since it is spanned by powers of $d$) and is thus a field extension of $k$.
	Thus $D=k$.
\end{proof}

\begin{thm}[Schur's Lemma part III]
	Let $V$ and $W$ be finite dimensional simple $A$-modules and $k$ is algebraically closed, then
	\[
		\dim \Hom_A(V,W)=
		\begin{cases}
			0 & \text{if } V \not\cong W \\
			1 & \text{if } V \cong W
		\end{cases}
	\]
\end{thm}

\begin{proof}
	We have already shown the first case (part I).
	If $V \cong W$, then let $\phi: V \xto{\sim} W$ be an isomorphism.
	For any $\theta \in \Hom_A(V,W)$, we have $\phi\inv\theta \in \End_A(V) = k$ by part II, so $\phi\inv\theta$ is scalar multiplication by some $\lambda \in k$.
	Hence $\theta=\lambda\phi$ and so $\Hom_A(V,W)$ is one dimensional.
\end{proof}