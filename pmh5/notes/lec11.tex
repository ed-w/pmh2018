\section{2018-04-18 Lecture}

\begin{rmk}
	We will now let $k=\CC$ until advised otherwise.
	Maschke's theorem applies for any finite $G$: every $\CC G$-module is semisimple.
\end{rmk}

Let $S_1,\ldots,S_r$ be a complete list of simple $\CC G$-modules.
We will show that $r$ equal to the number of conjugacy classes in $G$.

\begin{rmk}
	Classifying one-dimensional $\CC G$-modules up to isomorphism is the same as classifying group homomorphisms $\psi: G \to \CC^\times$ up to equality.
	This is because if $V$ is one-dimensional, then $\GL(V) \cong \CC^\times$ where we identify the scalar $\lambda$ and the operation scalar multiplication by $\lambda$.
	Since $1 \times 1$ matrices are invariant under conjugation, different group homomorphisms give non-isomorphic $\CC G$-modules.
	
	Define the \textbf{dual group} (or \textbf{character group})
	\[\wh G = G^\vee = \{\text{homomorphisms } \psi: G \to \CC^\times\}.\]
	This is an abelian group under pointwise multiplication.
	For $\psi \in G^\vee$, we can define a one-dimensional $\CC G$-module $\CC_\psi$ in which $g \in G$ acts on $\CC$ as multiplication by $\psi(g)$.
\end{rmk}

\begin{exam}
	Let $G$ be the quaternion group $Q$.
	We begin by finding all one-dimensional $\CC Q$-modules.
	Consider $Q^\vee$.
	If $\psi: Q \to C^\times$ is a homomorphism, then we must have $\psi(-1) = \psi(iji\inv j\inv) = 1$ and $\psi(i)^2=\psi(i^2)=1$ and similarly for $j$ and $k$.
	So there are four elements of $Q^\vee$.
	(Pick $(\psi(i),\psi(j)) \in \{(\pm 1, \pm 1)\}$ and the rest is defined.)
	
	By the sum of squares formula, we must have $8=4 \cdot 1^2 + 2^2$, so there is one more simple $\CC Q$-module which is two-dimensional.
	Let $Q$ act on $\HH$ (the Hamiltonian quaternions) by left-multiplication.
	We can also make $\HH$ a vector space over $\CC$ by right-multiplication.
	Then $\HH$ as a $\CC$-vector space has basis $\{1,j\}$.
	Then the representing matrices for elements of $\QQ$ are just the Pauli matrices:
	\begin{equation*}
		i=
		\begin{bmatrix}
			i & 0 \\ 0 & -i
		\end{bmatrix}
		\quad
		j=
		\begin{bmatrix}
			0 & 1 \\ -1 & 0
		\end{bmatrix}
		\quad
		k=
		\begin{bmatrix}
			0 & i \\ i & 0
		\end{bmatrix}
	\end{equation*}
	Note that the physics convention is to take out a factor of $i$ in each of the above matrices.
\end{exam}

\begin{rmk}
	Note that $C_n^\vee \cong C_n$ since $\psi$ sends a generator $x$ to an $n$th root of unity in $\CC$.
	However this isomorphism is not natural since it depends on the choice of generator.
	Since any finite abelian group is isomorphic to a direct product of cyclic groups, it follows that we have $G^\vee \cong G$ (but not naturally).
	However, there is a natural isomorphism $(G^\vee)^\vee \cong G$ (\textbf{Pontryagin duality}).
	
	For a non-abelian group $G$, there is a largest abelian quotient (the \textbf{abelianisation}): $G^\text{ab} = G/G'$ where $G'=[G,G]$ is the \textbf{derived} (or \textbf{commutator}) subgroup of $G$: the (normal) subgroup generated by all elements of the form $ghg\inv h\in$ for $g,h \in G$.
	Then $G^\vee=(G^\text{ab})^\vee$.
	
	Given a $\CC G$-module $(V,\rho)$ and $\psi: G \to \CC^\times$ a group homomorphism, we can \textbf{twist} $V$ by $\psi$ to give a new $\CC G$-module $(V^\psi,\rho')$, where $\rho'(g)=\psi(g)\rho(g)$.
	Since this twisting only scales the representing transformations, it does not affect invariant subspaces so $V$ is simple $\iff V^\psi$ is simple.
	This allows us to produce new simple $\CC G$-modules using the dual group.
	(However, we may have $V \cong V^\psi$).
\end{rmk}

\begin{exam}
	Consider $G=S_4$.
	For all $n$, the dual group is $S_n^\vee = \{1,\eps\}$ where $\eps$ is the sign homomorphism.
	So there are only two one-dimensional $\CC S_4$-modules.
	
	We also have the representation
	\[H=\left\{
	\begin{bmatrix}
		a_1 \\ a_2 \\ a_3 \\a_4
	\end{bmatrix}
	\Bigg \vert \
	a_1+a_2+a_3+a_4=0
	\right\}\]
	which is a simple (see tutorial 3) $3$-dimensional $\CC S_4$-module.
	By the previous remark, $H^\eps$ is a simple $\CC S_4$-module, but is it isomorphic to $H$?
	
	We will look at the representing matrices.
	Use the basis
	\[\left\{
	\begin{bmatrix}
		1 \\ -1 \\ 0 \\0
	\end{bmatrix},
	\begin{bmatrix}
	0 \\1 \\ -1 \\ 0
	\end{bmatrix},
	\begin{bmatrix}
	0 \\ 0 \\ 1 \\ -1
	\end{bmatrix}
	\right\}\]
	
	Then the representing matrix for $(1\ 2)$ in $H$ is
	\[\begin{bmatrix}
		-1 & 1 & 0 \\
		0 & 1 & 0 \\
		0 & 0 & 1
	\end{bmatrix}\]
	and in $H^\eps$ it is	
	\[\begin{bmatrix}
		1 & -1 & 0 \\
		0 & -1 & 0 \\
		0 & 0 & -1
	\end{bmatrix}\]
	Since the two matrices are not similar (look at eigenvalues and their multiplicities), we have $H^\eps \not\cong H$.
	
	By the sum of squares formula, we have $\abs{G}=24=2 \cdot 1^2 + 3 \cdot 3^2 + 2^2$, so there must be one more simple $\CC S_4$-module which is two-dimensional.
	
	Recall that $\CC S_3$-has a simple $2$-dimensional module $P$.
	There is a surjective group homomorphism $\phi: S_4 \surjto S_3$ where we regards each element of $S_4$ as acting on the following conjugacy class of $S_4$:
	\[\{(1\ 2)(3\ 4), (1\ 3)(2\ 3), (1\ 4)(2\ 3)\}\]
	We can regard $P$ as a $\CC S_4$-module where $\sigma \in S_4$ acts in the same way as $\phi(\sigma) \in S_3$.
	Since $\phi$ is surjective, $P$ is simple as a $\CC S_4$-module.
\end{exam}
