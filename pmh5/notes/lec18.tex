\section{2018-05-16 Lecture}

The last lecture may or may not be cancelled.

For the rest of the course we will look (briefly) at representations of compact groups, Lie groups and Lie algebras (not in the book).

\begin{defn}
  A \textbf{topological group} is a group $G$ which has a topology such that multiplication and inversion are continuous.
  A topological group is \textbf{compact} if it is compact as a topological space.
\end{defn}

\begin{rmk}
  A subset of a vector space over $\RR$ or $\CC$ is compact if and only if it is closed and bounded (the Heine-Borel theorem).
\end{rmk}

\begin{exam}
  \lv
  \begin{enum}
    \io
    Any group is a topological group with the discrete topology.

    \io
    The group $\GL_n(\RR)$ is a topological group.
    (The topology is the subspace topology induced from the standard topology on $\Mat_n(\RR)$.)
    It is the preimage of $\RR\setminus\{0\}$ under the continuous function $\det: \Mat_n(\RR) \to \RR$, hence it is open.
    It is not bounded because scalar multiples of the identity are in $\GL_n(\RR)$.
    Therefore it is not compact.

    \io
    $\SL_2(\RR)$ is not compact if $n \neq 1$, but it is closed.

    \io
    The \textbf{orthogonal group} $O_n=O_n(\RR)$ of real orthogonal matrices is compact.
    The columns of $A \in O_n$ are an orthonormal basis of $\RR^n$, hence $O_n$ is bounded.
    $O_n$ is closed since the defining relation $A^\perp A=I$ is a set of polynomial equations.
    Note that $\abs{\det A}=1$ for $A \in O_n$.
    The \textbf{special orthogonal group} $SO_n=SL_n(\RR) \cap O_n$ is compact.

    \io
    The \textbf{unitary group} $U_n$ of complex unitary matrices is compact.
    It is bounded since the columns of $A \in U_n$ are an orthonormal basis of $\CC^n$.
    It is closed analogously to the case of $O_n$.
    Note that $\abs{\det A}=1$ for $A \in U_n$.
    The \textbf{special unitary group} $SU_n = SL_n(\CC) \cap U_n$ is compact.
  \end{enum}
\end{exam}

\begin{exam}
  For $n=1$ we have $O_1=\{1,-1\}$, $SO_1=1$, $U_1=\{z \in \CC^\times \mid \abs{z}=1\} \cong \RR/2\pi\ZZ$ and $SU_1=1$.
  For $n=2$, we have
  \[O_2=
    \left\{ 
      \begin{bmatrix}
	\cos\theta & -\sin\theta \\ \sin\theta & \cos\theta
      \end{bmatrix}
    \right\}
    \cup
    \left\{ 
      \begin{bmatrix}
	\cos\theta & \sin\theta \\ \sin\theta & -\cos\theta
      \end{bmatrix}
  \right\}\]
  where the first set is the set of rotations of $\RR^2$ about the origin (eigenvalues $\exp(\pm i\theta)$) and the second set is the set of reflections through lines about the origin (eigenvalues $\pm 1$).
  The first set is precisely $SO_2$.
  Moreover, we have $SO_2 \cong U_1$ under the usual isomorphism between rotations and complex numbers of modulus 1:
  \[
    \begin{bmatrix}
      \cos\theta & -\sin\theta \\ \sin\theta & \cos\theta
    \end{bmatrix}
    \longleftrightarrow
    \exp(i\theta)
  \]
\end{exam}

\begin{exam}
  $SU_2$ is the set of all matrices
  \[
    \begin{bmatrix}
      \alpha & \beta \\ \gamma & \delta
    \end{bmatrix}
  \]
  with complex entries satisfying $\alpha\delta-\beta\gamma=1$ and
  \[
    \begin{bmatrix}
      \delta & -\beta \\ -\gamma & \alpha
    \end{bmatrix}
    =
    \begin{bmatrix}
      \ol\alpha & \ol\gamma \\ \ol\beta & \ol\delta
    \end{bmatrix}
  \]
  So $SU_2$ is the set
  \[\left\{ 
      \begin{bmatrix}
	\alpha & \beta \\ -\ol\beta & \ol\alpha
      \end{bmatrix}
      \mid
      \alpha,\beta \in \CC \text{ with } \abs{\alpha}^2+\abs{\beta}^2=1
  \right\}\]
  or equivalently
  \[\left\{ 
      \begin{bmatrix}
	a+bi & -c-di \\ c-di & a-bi
      \end{bmatrix}
      \mid
      a,b,c,d \in \RR \text{ with } a^2+b^2+c^2+d^2=1
  \right\}\]
  So as a topological space, $SU_2 \cong S^3$ (the unit sphere in $\RR^2$).
  In fact there are only three spheres ($S^0$, $S^1$ and $S^3$) which can be given the structure of a topological group.
  Now we have the following isomorphism of $\RR$-algebras:
  \[\left\{ 
      \begin{bmatrix}
	a+bi & -c-di \\ c-di & a-bi
      \end{bmatrix}
      \mid
      a,b,c,d \in \RR
  \right\}
  \cong \HH
  \]
  so $SU_2$ is isomorphic to $U_1(\HH)$, the group of `unit quaternions': $\{q \in \HH \mid q\ol q=1\}$.
\end{exam}

\begin{exam}
  $SO_3$ is the set of all matrices $A \in \Mat_3(\RR)$ such that $A^\perp A=I$ and $\det A=1$.
  If $\lambda$ is an eigenvalue of $A$, then $\lambda\inv$ is an eigenvalue of $A\inv=A^\perp$, hence of $A$.
  Since $A$ is a real matrix, its eigenvalues must occur in conjugate pairs, hence $\lambda\inv=\ol\lambda$.
  Then $\lambda$ is on the unit circle.
  Since $\det A=1$, the last eigenvalue is $1$, so its eigenvalues are $1$, $\exp i\theta$ and $\exp(-i\theta)$ for some angle $\theta$.
  Therefore $A$ is a rotation matrix about some axis through the origin (unless $A$ is the identity).
  Note that this shows that rotations in $\RR^3$ form a group, a fact which is not geometrically obvious.
\end{exam}

Quaternions were invented to study rotations in $\RR^3$.

\begin{thm}[Hamilton]
  There is a $2:1$ surjective group homomorphism
  \[ \eta: U_1(\HH) \to SO_3 \]
  which sends $a+bi+cj+dk$ to the rotation about the axis $b\bo i + c\bo j + d\bo k$ by an angle of $2\tan\inv(\sqrt{b^2+c^2+d^2}/a)$, where we take the angle to be $\pi$ if $a=0$.
  The `inverse' takes a rotation about the axis defined by a unit vector $\bo u$ through the angle $\ph$ to the two vectors $\pm( \cos\ph/2 + \sin\ph/2\bo u)$ interpreted as elements of $\HH$.
\end{thm}

\begin{exam}
  \lv
  \begin{enum}
    \io $\pm 1$ corresponds to the identity.
    \io $\pm i$ corresponds to rotation by $\pi$ about the $x$-axis.
    \io $1/\sqrt{3}(i+j+k)$ corresponds to rotation by $\pi$ about the axis through $\bo i+\bo j+\bo k$.
    \io $1/2(1+i-j-k)$ corresponds to rotation by $2\pi/3$ about the axis through $\bo i-\bo j-\bo k$.
  \end{enum}
\end{exam}

\begin{defn}
  A (continuous) \textbf{representation} of a topological group is a $\CC$-vector space $V$ and a \textbf{continuous} group homomorphism $\rho: G \to \GL(V)$ with the topology on $\GL(V)$ induced from the topology on $\Mat(\CC)$.
\end{defn}

\begin{rmk}
  Every subrepresentation of a continuous representation is continuous.
\end{rmk}

\begin{exam}
  One-dimensional continuous representations of $U_1$ are the continuous group homomorphisms
  \[ \psi: U_1 \to \CC^\times. \]
  (Recall that $\RR/2\pi\ZZ \cong U_1$.)
  Any \underline{continuous} group homomorphism from $(\RR,+)$ to $\CC^\times$ is of the form $x \mapsto \exp xz$ for some $z \in \CC$.
  So $\psi$ must be of the form
  \[ \psi_n: \exp(i\theta) \mapsto \exp(in\theta) \]
  for some $n \in \ZZ$.
  Since $U_1$ is abelian, the irreducible representations are precisely the one-dimensional representations.
  In particular, there are infinitely many irreducible representations.
\end{exam}

\begin{rmk}
  The situation is much more complicated if we allow representations to be discontinuous.
  If we take a linear functional $h$ on $\RR/\QQ$, then $x \mapsto \exp(ih(x))$ is a homomorphism.
  Since $\RR/\QQ$ has an uncountable Hamel basis there are uncountably many such functionals.
\end{rmk}
