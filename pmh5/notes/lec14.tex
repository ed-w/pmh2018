\section{2018-05-02 Lecture}

More tensor product stuff (mostly omitted).

\begin{rmk}
  Since any element in $V \otimes W$ is a linear combination of pure tensors, it is enough to look only at pure tensors.
  If $\ph \in \End V$ and $\psi \in \End W$, then there is a unique map $\ph\otimes\psi \in \End(V \otimes W)$ such that
  \[(\ph\otimes\psi)(v \otimes w) = \ph(v) \otimes \ph(w)\]
  for all $v \in V$ and $w \in W$.
  We can construct this map by first specifying the value of $\ph\otimes\psi$ on a basis $\{v_i \otimes v_j\}$ of $V \otimes W$ where $\{v_i\}$ and $\{w_j\}$ are bases of $V$ and $W$ respectively.
  Then we need to check that the above equation holds for all pure tensors.
  If $v = \sum_i a_iv_i$ and $w = \sum_j b_jv_j$, we have
  \[(\ph\otimes\psi)(v \otimes w) = \sum_i\sum_j a_ib_i \ph(v_i)\otimes\ph(w_j) = \ph\left( \sum_i a_iv_i \right) \otimes \psi\left( \sum_j b_jv_j \right) = \ph(v)\otimes\psi(w).\]
  Moreover, we have the obvious composition rule $(\ph\otimes\psi)\circ(\ph'\otimes\psi')=(\ph\circ\ph')\otimes(\psi\otimes\psi')$.

  An alternative way to say this is to use the universal property of the tensor product.
  Define a bilinear map by $(v,w) \mapsto \ph(v)\otimes\psi(w)$.
  Then we get a unique linear map which we call $\ph\otimes\psi$.

  In fact, if $V$ and $W$ are finite-dimensional, we have an algebra isomorphism
  \[\End V \otimes \End W \cong \End V \otimes W.\]
  This can be checked using bases or using universal properties.

  We can define a map form $\End V \times \End W$ to $\End V \otimes W$ that takes $(\ph,\psi)$ to $\ph\otimes\psi$.
  It is clear from the definition of $\ph\otimes\psi$ that this map is bilinear.
  Hence it factors through $\End V \otimes \End W$ to give a linear map to $\End V \otimes W$.
  These spaces have equal dimensions so we only need to check for one of injectivity and surjectivity.
\end{rmk}

\begin{prop}
  Let $(V,\rho)$ be a $\CC G$-module and $(W,\sigma)$ be a $\CC H$-module.
  Then $(V \otimes W, \tau)$ is a $\CC G \times H$-module where $\tau(g,h) = \rho(g)\otimes\sigma(h)$.
\end{prop}

\begin{cor}
  If $G=H$, then composing with the diagonal embedding $G \to G \times G$ gives a new group homomorphism $\tau': G \to \GL(V \otimes W)$, so $V \otimes W$ is a $\CC G$-module.
  In this case, we have $g \cdot (v \otimes w) = g \cdot v \otimes g \cdot w$ on pure tensors.
\end{cor}

\begin{prop}
  $\chi_{V \otimes W}(g,h) = \chi_V(g)\chi_W(h)$.
\end{prop}

\begin{proof}
  Choose bases $\{v_i\}_{i=1}^n$ and $\{v_j\}_{j=1}^m$.
  Suppose that $g$ and $h$ have matrices $(a_{ii'})$ and $b_{jj'}$ respectively.
  Then
  \[(g,h)\cdot(v_{i'} \otimes w_{j'}) = g \cdot v_{i'} \otimes g \cdot w_{jj'} = \sum_i \sum_j a_{ii'}b_{jj'} v_i \otimes v_j\].
  So the matrix of $(g,h)$ on $V \otimes W$ in the basis $\{v_i \otimes w_j\}$ is
  \[(a_{ii'}b_{jj'})\]
  where the rows are indexed by $(i,j)$ and the columns by $(i',j')$.
  Then taking traces gives
  \[\chi_{V \otimes W}(g,h) = \sum_i \sum_j a_{ii} b_{jj} = \chi_V(g)\chi_W(h). \qedhere\]
\end{proof}

\begin{cor}
  If $G=H$, then $\chi_{V \otimes W}(g) = \chi_V(g)\chi_W(g)$.
\end{cor}

\begin{exam}
  Let $1$, $\eps$ and $P$ be the trivial, sign and unique irreducible two-dimensional representations of $S_3$ respectively.
  Then $\chi_{P \otimes P} = 1+\eps+\chi_P$, so $P \otimes P \cong \CC \oplus \CC_\eps \oplus P$.
\end{exam}

\begin{rmk}
  It is an open problem to find a combinatorial formula for the multiplicities of simple $\CC S_n$-modules in tensor products of two simple $\CC S_n$-modules.
\end{rmk}

\begin{prop}
  For any vector spaces $V$ and $W$ we have an isomorphism
  \[W \otimes V^* \cong \Hom(V,W)\]
  coming from the bilinear map
  \begin{align*}
    W \times V^* &\to \Hom(V,W) \\
    (w,f) &\mapsto (v \mapsto f(v)w)
  \end{align*}
  This can be checked using bases (and dimensions).
\end{prop}

\begin{cor}
  We have that $W \otimes V^*$ is a $\CC G$-module with the action
  \[g \cdot (w \otimes f) = g \cdot w \otimes g \cdot f\]
  where
  \[(g \cdot f)(v) = f(g\inv \cdot v).\]
  Then $\Hom(V,W)$ is a $\CC G$-module with the action
  \[(g \cdot \ph)(v) = g\ph(g\inv v).\]
  Then
  \[\chi_{\Hom(V,W)}(g) = \chi_{W \otimes V^*}(g) = \chi_W(g)\chi_{V^*}(g) = \chi_W(g)\ol{\chi_V(g)}.\]
\end{cor}
