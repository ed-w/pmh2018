\section{2018-05-30 Lecture}

Representations of Lie algebras

We will study representations of $\sli_2(\CC) = \{
  \left[ \begin{smallmatrix}
    a & b \\ c & -a
  \end{smallmatrix} \right]
  \mid a,b,c \in \CC
\}$
with the commutator bracket $[x,y]=xy-yx$.
The standard basis for $\sli_2$ is:
\begin{equation*}
  e=
  \begin{bmatrix}
    0 & 1 \\ 0 & 0
  \end{bmatrix}
  \quad
  f=
  \begin{bmatrix}
    0 & 0 \\ 1 & 0
  \end{bmatrix}
  \quad
  h=
  \begin{bmatrix}
    1 & 0 \\ 0 & -1
  \end{bmatrix}
\end{equation*}

By definition, a representation of $\sli_2$ is a vector space $V$ over $\CC$ with a Lie algebra homomorphism $\pi: \sli_2 \to \gl(V)$, that is, $\pi$ is linear (determined by $\pi(e)$, $\pi(f)$ and $\pi(h)$) and satisfies $\pi([x,y])=[\pi(x),\pi(y)]$ for all $x,y \in \sli_2$ (it is enough to check on $e$, $f$ and $h$).

The Lie bracket relations in $\sli_2$ are:
\[ [e,f]=h \quad [h,e]=2e \quad [h,f]=-2f \]
The above equations contain all of the information in $\sli_2$.

So a representation of $\sli_2$ is a vector space $V$ over $\CC$ with three linear transformations $\pi(e)=E$, $\pi(f)=F$ and $\pi(h)=H$ satisfying
\[ EF-FE=H \qquad HE-EH=2E \qquad HF-FH=-2F \]
This is a same as a representation of the associative algebra
\[ U(\sli_2) = \CC\ang{E,F,H}/(EF-FE-H,HE-EH-2E,HF-FH+2F) \]
where $(\cdot)$ is a two-sided ideal.
This algebra is infinite dimensional (in fact it has basis $\{E^aH^bF^c \mid a,b,c \in \NN\}$ or similarly for any ordering of $E$, $H$ and $F$).

More generally:
\begin{defn}
  If $\kg$ is a Lie algebra then an enveloping algebra of $\kg$ is an associative algebra $A$ equipped with a Lie algebra homomorphism $\ph: \kg \to A$ where $A$ is a Lie algebra with the commutator bracket.
  There is a unique (up to isomorphism) \textbf{universal enveloping algebra} $U(\kg)$ with $\iota: \kg \to U(\kg)$ such that for any enveloping algebra $\ph: \kg \to A$ there is a unique algebra homomorphism $\tau: U(\kg) \to A$ such that $\tau \circ \iota = \ph$.
\end{defn}

\begin{proof}
  Uniqueness follows from the universal property.
  We can prove uniqueness by direct construction.
  Let $\{x_i\}_{i \in I}$ be a basis of $\kg$.
  Then $[x_i,x_j] = \sum_{k \in I} c_{ij}^k x_k$ for some $c_{ij}^k \in \CC$ (the structure constants).
  Define
  \[ U(\kg) = \CC\ang{\{x_i\}_{i \in I} \, \Bigg\vert \, x_ix_j-x_jx_i-\sum_{k \in I} c_{ij}^k x_k} \]
  and $\iota: \kg \to U(\kg)$ takes $x_i$ to $x_i$.
\end{proof}

\begin{exam}
  Let $\kg = \CC\{x,y\} \cong \CC^2$ with $[x,y]=y$.
  Then $U(\kg) = \CC\ang{x,y}/(xy-yx-y)$.
\end{exam}

\begin{rmk}
  As a special case $(A=\End V$) of the universal property, the Lie algebra representations of $\kg$ are precisely the associative algebra representations of $U(\kg)$.
\end{rmk}

\begin{rmk}
  \lv
  \begin{enum}
    \io $\iota: \kg \to U(\kg)$ is always injective.
    \io $\dim U(\kg)=\infty$ unless $\kg=0$.
    (For example, if $\kg = \CC x \cong \CC$ with $[\cdot,\cdot]=0$, then $U(\kg)=\CC[x]$.)
  \end{enum}
\end{rmk}

\begin{exam}[Representations of $\sli_2$]
  \lv
  \begin{enum}
    \io $V=0$ with $E=H=F=0$
    \io $V=\CC$ with $E=H=F=0$ (since they commute)
    \io $V=\CC^2$ with
    $E=
    \left[ 
      \begin{smallmatrix}
	0 & 1 \\ 0 & 0
      \end{smallmatrix}
    \right]
    $,
    $H=
    \left[ 
      \begin{smallmatrix}
	1 & 0 \\ 0 & -1
      \end{smallmatrix}
    \right]
    $ and
    $F=
    \left[ 
      \begin{smallmatrix}
	0 & 0 \\ 1 & 0
      \end{smallmatrix}
    \right]
    $ (the natural representation).
    Note that this is irreducible since the matrices have no common eigenvector.

    \io $V=\CC^2$ with
    $E=
    \left[ 
      \begin{smallmatrix}
	0 & 0 \\ -1 & 0
      \end{smallmatrix}
    \right]
    $,
    $H=
    \left[ 
      \begin{smallmatrix}
	-1 & 0 \\ 0 & 1
      \end{smallmatrix}
    \right]
    $ and
    $F=
    \left[ 
      \begin{smallmatrix}
	0 & -1 \\ 0 & 0
      \end{smallmatrix}
    \right]
    $.
    This is a representation since $[-x^\perp,-y^\perp]=-[x,y]^\perp$.
    This is also irreducible since the matrices have no common eigenvector.
    We will show that this is isomorphic to the natural representation, that is, there is some matrix
    $\left[
    \begin{smallmatrix}
      \alpha & \beta \\ \gamma & \delta
    \end{smallmatrix}
    \right]$
    that simultaneously conjugates $E$, $H$ and $F$.
    (We can take $\alpha=\delta=0$, $\beta=-1$ and $\gamma=1$.)
  \end{enum}
  Recall that irreducible representations of $SU_2$ were constructed as $\cS^m(\CC^2)$.
  If $V$ and $W$ are representations of $\kg$, then $V \otimes W$ is also a representation of $\kg$: if $x \in \kg$, $v \in V$ and $w \in W$ then
  \[ x(v \otimes w) = xv \otimes w + v \otimes xw \]
  on simple tensors.
  Note that this is not the same as with groups.
  This looks like the product rule since it comes from differentiating $g(v \otimes w) = gv \otimes gw$ where $g=\exp(tA)$.
\end{exam}
