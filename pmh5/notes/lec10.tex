\section{2018-04-12 Lecture}

\begin{thm}[Maschke's Theorem]
	If $G$ is a finite group and $k$ is a field such that the $\cha k$ does not divide $G$, then $kG$ is semisimple, that is, every representation of $G$ over $k$ is semisimple.
\end{thm}

\begin{rmk}
	The converse to Maschke's theorem is also true.
\end{rmk}

\begin{proof}
	Let $V$ be a representation of $G$ over $k$ (a $kG$-module).
	Let $W$ be a $kG$-submodule of $V$.
	We want to show that there is a $kG$-submodule $W'$ of $V$ such that $V=W\oplus W'$.
	Equivalently, we want to show that there is a $kG$-module homomorphism $\pi: V \to W$ such that $\pi$ is the identity on $W$.
	Then defining $W'=\ker\pi$ gives the desired direct sum decomposition (it is easy to check $V=W\oplus\ker\pi$ with $v=\pi(v)+(v-\pi(v))$.
	
	Certainly there is some complementary \underline{subspace} $X$ of $V$ such that $V=W\oplus X$.
	So there is certainly a linear map $\tau:V \to W$ such that $\tau$ is the identity of $W$.
	We make $\tau$ in to a $kG$-module homomorphism by `averaging over the group':
	Define
	\begin{align*}
		\pi: V &\to W \\
		v &\mapsto \frac 1{\abs{G}} \sum_{g \in G} g \tau(g\inv v)
	\end{align*}
	where we have used that $\abs{G}$ is invertible in $k$.	
	Now if $w \in W$, then
	\[\pi(w) = \frac 1{\abs{G}} \sum_{g \in G} g\tau(g\inv w) = \frac 1{\abs{G}} \sum_{g \in G} gg\inv w = w \]
	so $\pi$ is the identity on $W$.
	We now check that $\pi(hv)=h\pi(v)$ for all $h \in G$ and $v \in V$.
	We have
	\[\pi(hv) = \frac 1{\abs{G}} \sum_{g \in G} g\tau(g\inv hv) = \frac 1{\abs{G}} \sum_{g' \in G} hg\tau\left((g')\inv v\right) = h\pi(v)\]
	where we have used the change of variables $g=hg'$.
\end{proof}

\begin{rmk}
	If $\cha k \nmid p$, then for any element $x$ in any $kG$-module $X$, then
	\[x' = \frac 1{\abs{G}} \sum_{g \in G} gx\]
	is always a $G$-invariant element of $X$.
	
	The above proof used this principle in the case where $X=\Hom_k(V,W)$ with the action of $G$ defined by $g \cdot\tau = g\tau g\inv$.
\end{rmk}

\begin{proof}
	For all $h \in G$, we have
	\[hx' = \frac 1{\abs{G}} \sum_{g \in G} hgx = \frac 1{\abs{G}} \sum_{g' \in G} g'x = x'\]
	where we have used the substitution $g'=hg$.
\end{proof}

\begin{cor}[Consequences of Maschke's theorem]
	\leavevmode
	\begin{enum}
		\io
		Any square matrix $A$ of finite order $n$ over $\CC$ is diagonalisable.
		Such a matrix defines a representation of $Z_n$ over $\CC$.
		Then Maschke's theorem implies that this representation is a direct sum of simple $\CC Z_n$-modules, each of which is one-dimensional.
		So the vector space has a basis of eigenvectors for $A$.
		
		Another way to see this is to check that no Jordan block of size larger than $1$ has finite order over $\CC$.
		
		\io
		Applying Wedderburn's theorem to $kG$ (under the assumptions of Maschke's theorem) we have
		\[kG \cong \Mat_{n_1}(D_1) \times \cdots \times \Mat_{n_r}(D_r)\]
		where $D_1,\ldots,D_r$ are division algebras over $k$ with dimensions $n_1,\ldots,n_r$ respectively.
		
		More specifically, there are (up to isomorphism) finitely many simple $kG$-modules $S_1,\ldots,S_r$ with corresponding division algebras $D_i=\End_{kG}(S_i)$ (which are all $k$ if $k$ is algebraically closed).
		Moreover, the algebra homomorphism $\rho: kG \to \End(S_1\oplus\cdots\oplus S_r)$ is injective with image $\End_{D_1}(S_1)\times\cdots\times\End_{D_r}(S_r)$.
		If $k$ is algebraically closed then this implies that
		\[\abs{G} = \dim kG = \dim\left(\End(S_1)\times\cdots\times\End(S_r)\right) = \sum_{i=1}^r (\dim S_i)^2\]
		the \textbf{sum of squares formula}.
	\end{enum}
\end{cor}

\begin{exam}
	\leavevmode
	\begin{enum}
		\io
		Let $G=Z_3$ and $k=\CC$.
		Since $Z_3$ is abelian we know that the simple $\CC Z_3$-modules are one-dimensional.
		A one-dimensional $\CC Z_3$-module is $\CC$ with a chosen third root of unit (to represent $x$).
		So there are three simple modules $\CC_1$, $\CC_\omega$ and $\CC_{\omega^2}$ where $\omega=\exp(2\pi/3)$.
		Then the sum of squares formula tels us that $3=1^2+1^2+1^2$.
		
		More generally, if $G$ is abelian then the number of simple (that is, one-dimensional) $\CC G$-modules (up to isomorphism) equals $\abs{G}$.
		We will see later that if $G$ is not abelian then there exists a simple $\CC G$-module of dimension greater than $1$.
		
		\io
		Let $k=\CC$ and $G=\CC S_3$, so $\abs{G}=6$.
		The one-dimensional representations of $\CC S_3$ are just the group homomorphisms $\chi: S_3 \to \CC^\times$.
		There are two:
		\begin{enum}
			\io
			The trivial representation: $\psi(\sigma)=1$ for all $\sigma\in S_n$
			
			\io
			The sign representation: $\eps(\sigma)=\sgn(\sigma)$
		\end{enum}
		Note that the above two homomorphisms are both constant on conjugacy classes of $S_3$ (\textbf{class functions}).
		(In fact more is true: they are \textbf{characters}.)
		
		There is also a two-dimensional $\CC S_3$-module.
		Let us first consider the \textbf{standard representation} $\CC^3$ of $S_3$ where $S_3$ acts by permuting co-ordinates:
		\[\sigma
		\begin{bmatrix}
			a_1 \\ a_2 \\ a_3
		\end{bmatrix}
		=
		\begin{bmatrix}
			a_{\sigma\inv(1)} \\ a_{\sigma\inv(2)} \\ a_{\sigma\inv(3)}
		\end{bmatrix}\]
		Then there is a direct sum decomposition of $\CC S_3$-submodules $\CC^3=H\oplus L$:
		\[L=\CC
		\begin{bmatrix}
			a_1 \\ a_2 \\ a_3
		\end{bmatrix}
		\quad\text{and}\quad
		H=\left\{
		\begin{bmatrix}
			a_1 \\ a_2 \\ a_3
		\end{bmatrix}
		\Bigg \vert \
		a_1+a_2+a_3=0
		\right\}\]
		Note that $L$ is isomorphic to the trivial representation.
		We can define a basis of $H$ as
		\[\left\{
		\begin{bmatrix}
			1 \\ -1 \\ 0
		\end{bmatrix},
		\begin{bmatrix}
			1 \\ 0 \\ -1
		\end{bmatrix}
		\right\}\]
		With respect to that basis, the representing matrices are:
		\[(1\ 2)=
		\begin{bmatrix}
			-1 & -1 \\
			0 & 1
		\end{bmatrix}
		\quad\text{and}\quad
		(1\ 3)=
		\begin{bmatrix}
			1 & 0 \\
			-1 & -1
		\end{bmatrix}
		\]
		These matrices have no eigenvector in common so there is no one-dimensional subspace of $H$ in which $\CC S_3$-submodule.
		So $H$ is a simple $\CC S_3$-module.
		We will see later that this generalises to $S_n$.
		The sum of squares formula ($1^2+1^2=2^2=6$) tells us that these are all the simple $\CC S_3$-modules.
	\end{enum}
\end{exam}

\begin{rmk}	
	Simple $\CC S_n$-modules are parametrised by partitions of $n$ (Young diagrams).
	For $S_3$, we have
	\begin{align*}
		(3) &\longleftrightarrow \text{the trivial representation} \\
		(2,1) &\longleftrightarrow \text{the two-dimensional simple representation} \\
		(1,1,1) &\longleftrightarrow \text{the sign representation}
	\end{align*}
\end{rmk}
