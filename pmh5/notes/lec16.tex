\section{2018-05-09 Lecture}

\begin{exam}
  Two non-isomorphic groups with the same character table:
  \[ Q = \{ \pm 1, \pm i, \pm j, \pm k \} \quad\text{and}\quad D_4 = \ang{x,y \mid x^4=1,y^2=1,yxy\inv=x\inv} \]
  We have
  \begin{center}
    \begin{tabular}[]{c | c c c c c}
      $Q$ & 1 & $-1$ & $\pm i$ & $\pm j$ & $\pm k$ \\ \hline
      1 & 1 & 1 & 1 & 1 & 1 \\
      $\psi_1$ & 1 & 1 & $-1$ & 1 & $-1$ \\
      $\psi_2$ & 1 & 1 & 1 & $-1$ & $-1$ \\
      $\psi_1\psi_2$ & 1 & 1 & $-1$ & $-1$ & 1 \\
      $\pi$ & 2 & -2 & 0 & 0 & 0
    \end{tabular}
    \hspace{1cm}
    \begin{tabular}[]{c | c c c c c}
      $D_4$ & 1 & $x^2$ & $x,x^3$ & $y,x^2y$ & $xy,x^3y$ \\ \hline
      1 & 1 & 1 & 1 & 1 & 1 \\
      $\psi_1$ & 1 & 1 & $-1$ & 1 & $-1$ \\
      $\psi_2$ & 1 & 1 & 1 & $-1$ & $-1$ \\
      $\psi_1\psi_2$ & 1 & 1 & $-1$ & $-1$ & 1 \\
      $\pi$ & 2 & -2 & 0 & 0 & 0
    \end{tabular}
  \end{center}
  where $\psi_1$, $\psi_2$ and $\psi_1\psi_2$ are the three non-trivial one-dimensional characters and $\pi$ is the Pauli matrix representation.
  (Note that by sum of squares we get the entry $\pi(1)$ and then by orthogonality we get the rest of the row.)
  (The two groups have the same abelianisation.)
  But the two groups are non-isomorphic since $Q$ contains 6 elements of order 4 but $D_4$ contains only 2 elements of order 4.
\end{exam}

So the character table does not allow us to reconstruct the group.
The following definition allows us to distinguish some (but not all) non-isomorphic groups with identical character tables.

\begin{defn}
  The \textbf{Frobenius Schur-indicator} $\FS(\chi_i)$ of an irreducible character is
  \[ \FS(\chi_i) = \frac{1}{\abs{G}} \sum_{g \in G} \chi_i(g^2). \]
\end{defn}

\begin{exam}
  For $Q$, $\FS(\psi_1)=1$ and $\FS(\pi)=-1$, but for $D_4$, $\FS(\psi_1)=1$ and $\FS(\pi)=1$.
\end{exam}

\begin{thm}
  We have
  \begin{equation*}
    \FS(\chi_i)=
    \begin{cases}
      1 &\text{if $\chi_i$ is real-valued and $S_i$ is of symmetric type} \\
      0 &\text{if $\chi_i$ is not real-valued} \\
      -1 &\text{if $\chi_i$ is real-valued and $S_i$ is of skew-symmetric type.}
    \end{cases}
  \end{equation*}
\end{thm}

\begin{rmk}
  Recall that $\chi_i$ is the character of $S_i$ $\iff$ $\ol{\chi_i}$ is the character of $S_i^*$ for simple $\CC G$-modules $S_i$.
  Then $\chi_i$ is real valued $\iff S_i \cong_{\CC G} S_i^*$, that is, $S_i$ is \textbf{self-dual}.
\end{rmk}

\begin{exam}
  Consider $Z_3$ with characters $1$, $\psi$ and $\psi^2$.
  Then $\CC_\psi^2 \cong \CC_{\psi^2}$, so $\FS(\psi)=0$.
\end{exam}

Tensor squares
If $V$ is a vector space with basis $v_1,\ldots,v_n$, then $V^{\otimes 2}$ has basis $v_i \otimes v_j$ for $1 \leq i,j \leq n$.
There is a natural endomorphism
\begin{align*}
  P: V \otimes V &\to V \otimes V \\
  v_i \otimes v_j &\mapsto v_j \otimes v_i.
\end{align*}
Since $P^2=\id$, we have the direct sum decomposition
\[ V^{\otimes 2} = \ca S^2 (V) \oplus \wedge^2 (V) \]
where $\ca S^2(V)$ and $\wedge^2(V)$ are the $+1$ and $-1$-eignepsaces of $P$ respectively.

Recall that $V \otimes V$ is a $\CC G$-module with the action $g \cdot (v \otimes v') = (g \cdot v) \otimes (g \cdot v')$.
This action clearly commutes with $P$, so $V^{\otimes 2} = \ca S^2(V) \oplus \wedge^2(V)$ is a $\CC G$-submodule decomposition.

In general, $\Hom(V,W) \cong W \otimes V^*$.
If $V^{**} \cong V$, then $\Hom(V^*,W) \cong W \otimes V$ (i.e.\@ for $V$ finite-dimensional).
In particular, if $W=V$, then $V \otimes V \cong \Hom(V^*,V)$.
If $V$ is a $\CC G$-module, then this is a $\CC G$-module isomorphism.
This is by definition, since we defined the action of $\CC G$ on $\Hom(V,W)$ by applying the isomorphism from the $\CC G$-module $W \otimes V^*$.
So if $S_i$ is a simple $\CC G$-module, then
\[ (S_i \otimes S_i)^G \cong \Hom(S_i^*,S_i)^G = \Hom_{\CC G}(S_i^*,S_i) \]
Hence
\[ \dim(S_i \otimes S_i)^G=
  \begin{cases}
    1 &\text{if $S_i$ is self-dual} \\
    0 &\text{otherwise.}
  \end{cases}
\]
Moreover
\[ (S_i^{\otimes 2})^G = \ca S^2(S_i)^G \oplus \wedge^2(S_i)^G \]
hence we have either
\[ \dim\ca S^2(S_i)^G = 1 \quad\text{and}\quad \dim\wedge^2(S_i)^G = 0 \qquad\text{(\textbf{symmetric type})} \]
or
\[ \dim\ca S^2(S_i)^G = 0 \quad\text{and}\quad \dim\wedge^2(S_i)^G = 1 \qquad\text{(\textbf{skew-symmetric type})} \]
We know that $\chi_{V \otimes V}(g) = \chi_V(g)^2$.
Tomorrow we will compute $\chi_{\ca S^2(V)}$ and $\chi_{\wedge^2(V)}$ in terms of $\chi_V$.

