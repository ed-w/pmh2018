\section{2018-05-17 Lecture}

Any finite group with the discrete topology is a compact group.
Compact groups can be seen as a generalisation of finite groups.
In the representation theory of finite groups, we often take an average over the group: $\frac{1}{\abs{G}}\sum_{g \in G}$.
In compact groups, we use an \textbf{invariant measure} (the \textbf{Haar measure}) instead.
We will use the following fact:

\begin{thm}
  For a compact group $G$ there is a unique $\CC$-linear function
  \begin{align*}
    \cC(G,\CC) &\to \CC \\
    f &\mapsto \int_G f(g) \, d\mu(g)
  \end{align*}
  such that:
  \begin{enum}
    \io If $f$ takes values in $\RR_{\geq 0}$, then $\ds \int_G f(g) \, d\mu(g) \in \RR_{\geq 0}$.
    \io $\ds \int_G 1 \, d\mu(g) = 1$.
    \io For all $h \in G$ we have
    \[ \int_G f(hg) \, d\mu(g) = \int_G f(g) \, d\mu(g) = \int_G f(gh) \, d\mu(g) \]
    (left- and right-invariance).
  \end{enum}
\end{thm}

\begin{exam}
  \lv
  \begin{enum}
    \io If $G$ is finite, then $\ds \int_G f(g) \, d\mu(g) = \frac{1}{\abs{G}} \sum_{g \in G} f(g)$.
    \io If $G=U_1$, then $\ds \int_{U_1} f(\exp(i\theta)) d\mu(\exp(i\theta)) = \frac{1}{2\pi} \int_{-\pi}^\pi f(\exp(i\theta)) \, d\theta$.
    \io If $G=SU_3$, then
    \[ \int_{SU_2} f(g) d\mu(g) = \frac{1}{\vol(S^3)} \int_{S^3} f(w+xi+yj+zk) \, dw \, dx \, dy \, dz \]
    where $(w,x,y,z) \in S^3$.
    When $f$ is a class function, we can choose a conjugacy class representative so that
    \[ \int_{SU_2} f(g) \, d\mu(g) = \frac 1\pi \int_{-\pi}^\pi f \left( 
	\begin{bmatrix}
	  \exp(i\theta) & 0 \\ 0 & \exp(-i\theta)
	\end{bmatrix}
    \right) \sin^2\theta \, d\theta \]
  \end{enum}
\end{exam}

\begin{rmk}
  If $f$ is a continuous map from $G$ to a finite-dimensional vector space $V$, then the left- and right-invariance condition still holds for
  \[ \int_G f(g) \, d\mu(g) \in V. \]
\end{rmk}

\begin{thm}[Maschke's theorem]
  Any finite-dimensional continuous representation of a compact group is semisimple.
\end{thm}

\begin{proof}
  Same as before but replace $\ds \frac{1}{\abs{G}} \sum_{g \in G}$ with $\ds \int_G d\mu(g)$.
\end{proof}

\begin{thm}
  The irreducible characters (continuous maps $\chi_V: G \to \CC$ for irreducible representations $V$) are orthonormal in the space $\cC_c(G,\CC)$ of continuous class functions with respect to the inner product
  \[ \ang{f,f'} = \int_G f(g) \ol{f'(g)} \, d\mu(g).\]
\end{thm}

\begin{proof}
  Similar to the finite case.
  The results about tensor products and so on still hold.
\end{proof}

Are the characters an orthonormal basis? Not quite.

\begin{thm}
  The Hilbert space completion of $\cC_c(G,\CC)$ is the space $L_c^2(G,\CC)$.
  The irreducible characters form a Hilbert basis of $L^2_c(G,\CC)$.
  That is, if $f \in L^2_c(G,\CC)$ with $\ang{f,\chi_V}=0$ for all irreducible representations $V$, then $f=0$ a.e.
\end{thm}

\begin{proof}
  This proof needs to be modified since the finite-dimensional proof used the group algebra.
  The group algebra needs to be replaced with an $L^2$ version.
\end{proof}

\begin{exam}[The circle group $U_1$]
  The irreducible representations of $U_1$ are preciely the one-dimensional representations (it is abelian).
  One-dimensional continuous representations are
  \begin{align*}
    \rho: \exp(i\theta) &\mapsto \exp(in\theta) \\
    \chi_V: \exp(i\theta) &\mapsto \exp(in\theta).
  \end{align*}
  The orthogonality relations give
  \[ \frac{1}{2\pi} \int_{-\pi}^\pi \exp(in\theta) \exp(-im\theta) \, d\theta =
    \begin{cases}
      1 & \text{ if } n=m \\
      0 & \text{ otherwise}.
    \end{cases}
  \]
  Moreover, $L^2_c(U_1,\CC)=L^2(U_1,\CC)$ is the set of $2\pi$-periodic functions on $\RR$.
  Then Fourier expansion on the circle is the same as expressing a function in terms of the Hilbert basis of irreducible characters of $U_1$.
\end{exam}

\begin{exam}[$SU_2$]
  We will first produce some representations (which will be irreducible).
  There is a natural representation on $\CC^2$.
  We will look at the $m$th symmetric tensor power $\cS^m(\CC^2)$ of $\CC^2$.
  If $\CC^2$ has basis $\{v_1=\bo e_1,v_{-1}=\bo e_2\}$, then $(\CC^2)^{\oplus m}$ has basis $\{v_{\eps_1} \otimes \cdots \otimes v_{\eps_m} \mid \eps_i \in \{-1,1\} \}$.
  Then $\cS^m(\CC^2)$ has basis
  \begin{align*}
    v_m &= v_1 \otimes \cdots \otimes v_1 \\
    v_{m-2} &= v_1 \otimes \cdots \otimes v_1 \otimes v_{-1} + \cdots + v_{-1} \otimes v_1 \otimes \cdots \otimes v_1 \\
    v_{m-2i} &= \text{ the sum of all tensor products of $m-i$ $v_1$s and $i$ $v_{-1}s$.}
  \end{align*}
  where $i=0,\ldots,m$.
  So the basis is $\{ v_m ,v_{m-2}, \cdots, v_{-m}\}$.
  We also have $\dim\cS^m(\CC^2)=m+1$.

  Let $\chi_m = \chi_{\cS^m(\CC^2)}: SU_2 \to \CC$.
  We have
  \begin{align*}
    \begin{bmatrix}
      \exp(i\theta) & 0 \\ 0 & \exp(-i\theta)
    \end{bmatrix}
    v_1 &= \exp(i\theta) v_1 \\
    \begin{bmatrix}
      \exp(i\theta) & 0 \\ 0 & \exp(-i\theta)
    \end{bmatrix}
    v_{-1} &= \exp(-i\theta) v_{-1}
  \end{align*}
  Then in the representation $\cS^m(\CC^2)$, we have
  \begin{equation*}
    \begin{bmatrix}
      \exp(i\theta) & 0 \\ 0 & \exp(-i\theta)
    \end{bmatrix}
    v_{m-2j} = \exp(i\theta)^{m-j} \exp(-i\theta)^j v_{m-2j} = \exp(i\theta)^{m-2j} v_{m-2j}
  \end{equation*}
  therefore
  \begin{equation*}
    \chi_m \left(
    \begin{bmatrix}
      \exp(i\theta) & 0 \\ 0 & \exp(-i\theta)
    \end{bmatrix}
  \right) = \sum_{j=0}^m \exp(i(m-2j)\theta) = \frac{\sin(m+1)\theta}{\sin\theta}.
  \end{equation*}
  So
  \begin{align*}
    \ang{\chi_m,\chi_m} &= \int_{SU_2} \chi_m(g) \ol{\chi_m(g)} \, d\mu(g) \\
    &= \frac 1\pi \int_{-\pi}^\pi \abs{
      \chi_m \left( 
      \begin{bmatrix}
	\exp(i\theta) & 0 \\ 0 & \exp(-i\theta)
      \end{bmatrix}
      \right)
    }^2 \sin^2\theta \, d\theta \\
    &= \frac 1\pi \int_{-\pi}^\pi \sin^2(m+1)\theta \, d\theta = 1
  \end{align*}
  for all $m$.
  Hence $\chi_m$ is an irreducible character and so $\cS^m(\CC^2)$ is an irreducible representation for all $m$.
  For $m \neq n$, we have
  \[ \ang{\chi_m,\chi_n} = \frac 1\pi \int_{-\pi}^\pi \sin(m+1)\theta \, \sin(n+1)\theta \, d\theta = 0. \]
  A harder fact is that if $f \in L^2_c(SU_2,\CC)$ is such that
  \[ \ang{\chi_m,f} = \frac 1\pi \int_{-\pi}^\pi f \left( 
    \begin{bmatrix}
      \exp(i\theta) & 0 \\ 0 & \exp(-i\theta)
    \end{bmatrix}
  \right) \sin(m+1)\theta \, \sin\theta \, d\theta = 0, \]
  then $f=0$ a.e., making the $\chi_m$ a complete orthonomal set.
  So we have found all of the irreducible characters.
  Therefore every irreducible continuous representation of $SU_2$ is isomorphic to $\cS^m(\CC)$ for some $m$.
\end{exam}

