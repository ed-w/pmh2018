\section{2018-04-26 Lecture}

\begin{thm}[Schur's orthogonality relations]
  Let $G$ be a finite group with and let $k=\CC$.
  Define the following inner product on $F_c(G,\CC)$:
  \[\ang{f_1,f_2} = \frac{1}{\abs{G}}f_1(g) \ol{f_2(g)}\]
  Then the irreducible characters $\chi_1,\ldots,\chi_r$ of $G$ are an orthonormal basis of $F_c(G,\CC)$ with respect to this inner product, that is,
  \begin{equation*}
    \ang{\chi_i,\chi_j}=
    \begin{cases}
      1 & \text{ if } i=j \\
      0 & \text{ if } i \neq j
    \end{cases}
  \end{equation*}
\end{thm}

\begin{cor}
  If $V$ is any $\CC G$-module, the multiplicity of $S_i$ in $V$ is $\ang{\chi_V,\chi_i}$ where $\chi_i$ is the character of $S_i$.
\end{cor}

\begin{proof}
  This comes from the formula
  \[V \cong S_1^{\oplus m_1} \oplus \cdots \oplus S_r^{\oplus m_r} \iff \chi_V = m_1\chi_1 + \cdots + m_r\chi_r \qedhere\]
\end{proof}

\begin{exam}
  We often write characters and their values in a table.
  Here is the character table for $S_3$:
  \begin{center}
    \begin{tabular}[]{c | c c c}
      & $1$ & $(x\ y)$ & $(x\ y\ z)$ \\ \hline
      $\chi_1$ & 1 & 1 & 1 \\
      $\chi_\eps$ & 1 & $-1$ & 1 \\
      $\chi_P$ & 2 & 0 & $-1$
    \end{tabular}
  \end{center}
  We compute some inner products as examples:
  \begin{align*}
    \ang{\chi_2,\chi_3} &= \frac{1}{6}(1 \times 1 \times 2 + 3 \times (-1) \times 0 + 2 \times 1 \times (-1)) = 0 \\
    \ang{\chi_3,\chi_3} &= \frac{1}{6}(1 \times 2^2 + 3 \times 0^2 + 2 \times(-1)^2) = 1 \\
  \end{align*}
  If $V$ is a $\CC S_3$-module with character
  \begin{center}
    \begin{tabular}[]{c | c c c}\hline
      $\chi_V$ & 17 & 3 & 5
    \end{tabular}
  \end{center}
  then computing inner products gives $\ang{\chi_V,\chi_1}=6$, $\ang{\chi_V,\chi_\eps}=3$ and $\ang{\chi_V,\chi_P}=4$.
  So $V = \CC^{\oplus 6} \oplus \CC_\eps^{\oplus 3} \oplus P^{\oplus 4}$.

  As a further example, the character table of $C_3$ is
  \begin{center}
    \begin{tabular}[]{c | c c c}
      & 1 & $x$ & $x^2$ \\ \hline
      1 & 1 & 1 & 1 \\
      $\psi$ & 1 & $\omega$ & $\omega^2$ \\
      $\psi^2$ & 1 & $\omega^2$ & $\omega$
    \end{tabular}
  \end{center}
  where $\omega = \exp(e\pi i/3)$.
\end{exam}

Dual spaces

\begin{defn}
  Let $V$ be a vector space.
  Then
  \[V^* = \Hom(V,\CC)\]
  is the \textbf{dual space} of $V$.
  If $\dim V <\infty$, then $\dim V \cong \dim V^*$.
  If $V$ has basis $v_1,\ldots,v_n$, then $V^*$ has basis $v_1^*,\ldots,v_n^*$ where $v_i^*(v_j) = \delta_{ij}$.
\end{defn}

\begin{rmk}
  If $V$ is a $\CC G$-module, then $V^*$ is a $\CC G$-module with the action
  \[(g \cdot f)(v) = f(g\inv \cdot v)\]
  for $g \in G$, $f \in V^*$ and $V \in V$.o
  
  Take bases $v_1,\ldots,v_n$ and $v_1^*,\ldots,v_n^*$ of $V$ and $V^*$ respectively.
  Suppose $g \in G$ has representing matrices $M(g)=(a_{ij}(g))_{ij}$ and $M^*(g)=(a^*_{kl}(g))_{kl}$ relative to the aforementioned bases of $V$ and $V^*$.
  Then we have
  \[g \cdot v_j = \sum_i a_{ij}(g) v_i \quad\text{and}\quad g \cdot v_l^* = \sum_k a_{kl}^*(g) v_k^*.\]
  Then
  \[a_{kl}^*(g) = (gv_l^*)(v_k) = v_l^*(g\inv v_k) = a_{lk}(g\inv).\]
  This shows that
  \[M^*(g) = M(g\inv)^\intercal = ((M(g))\inv)^\intercal.\]
  In particular, we have
  \[\chi_{V^*}(g) = \tr M^*(g) = \tr M(g\inv) = \chi_V(g\inv).\]
  We are now ready to give a characterisation of the complex conjugate of a character.
\end{rmk}

\begin{prop}
  We have
  \[\chi_V(g\inv) = \ol{\chi_V(g)}.\]
\end{prop}

\begin{proof}
  Let $M(g)$ be the representing matrix of $g$ and let $\lambda_1,\ldots,\lambda_n$ be the eigenvalues of $M(g)$.
  Then since $G$ is finite, the eigenvalues must be roots of unity.
  Hence the eigenvalues of $g\inv$ are $\ol\lambda_1,\ldots,\ol\lambda_n$.
  This completes the proof.
\end{proof}

\begin{rmk}
  If $\phi: V \to W$ is a map of vector spaces, then the adjoint map $\phi^*$ is a map from $W^*$ to $V^*$ defined by
  \begin{align*}
    \phi^*: W^* &\to V^* \\
    (\phi^*(f))(v) &= f(\phi(v))
  \end{align*}
  Then in this notation, the dual representation $(V^*,\rho^*)$ of $(V,\rho)$ is the representation with action given by
  \[(\rho^*(g))(f) = (g\inv)^*(f).\]
\end{rmk}

Tensor products

Omitted.
See commutative algebra notes.
