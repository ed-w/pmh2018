\section{2018-05-03 Lecture}

\begin{prop}
  Let $G$ be a finite group and define
  \[\psi_1 = \frac{1}{\abs{G}} \sum_{g \in G} g \in \CC G.\]
  Then for any $\CC G$-module $V$, $\psi_1$ acts as the projection on to the subspace
  \[V^G \defeq \{v \in V \mid gv=v \text{ for all } g \in G\}.\]
  The projection is defined using the direct sum decomposition of $V$ in to isotypic components.
\end{prop}

\begin{rmk}
  By Maschke's theorem, we have
  \[V = V_1 \oplus \cdots \oplus V_n\]
  where the $V_i$ are simple submodules.
  Then
  \[V^G = V_1^G \oplus \cdots \oplus V_n^G = \bigoplus_{i: V_i \cong \CC} V_i,\]
  so $V^G$ is the isotypic component of $V$ corresponding to the trivial $\CC G$-module $\CC$.
\end{rmk}

\begin{proof}
  We just need to check that $\psi_1[V] \subseteq V^G$ (we have already done this).
  Then $\psi_1[V_i] \subseteq V_i^G = 0$ if $V_i \not\cong \CC$.
  If $V_i \cong \CC$, then $\psi_1$ acts on $V_i$ as the identity.
\end{proof}

\begin{cor}
  We have
  \[\dim V^G = \tr\psi_1 = \frac{1}{\abs{G}}\sum_{g \in G}\chi_V(g) = \ang{\chi_V,1}\]
  where we interpret $\psi_1$ as an operator on $V$.
  This is a special case of the general fact that the multiplicity of $S_i$ in $V$ is given by $\ang{\chi_i,\chi_V}$ which follows from Schur's orthogonality relations.
\end{cor}

\begin{proof}[Proof of Schur's orthogonality relations]
  Let $V$ and $W$ be $\CC G$-modules.
  Apply this corollary to the $\CC G$-module $\Hom(V,W)$.
  Then
  \[\dim \Hom(V,W)^G = \frac{1}{\abs{G}} \sum_{g \in G} \chi_{\Hom(V,W)}(g) = \frac{1}{\abs{G}} \sum_{g \in G} \chi_W(g) \ol{\chi_V(g)} = \ang{\chi_V,\chi_W}.\]
  By definition, $\Hom(V,W)^G = \Hom_{\CC G}(V,W)$ since $\phi(v) = g\phi(g\inv v)$ if and only if $\phi$ is a $\CC G$-module homomorphism.
  Therefore
  \[\dim \Hom_{\CC G}(V,W) = \ang{\chi_W,\chi_V}.\]
  Then
  \[\dim \Hom_{\CC G}(S_i,S_j) = \delta_{ij}\]
  by Schur's lemma for $\CC$.
\end{proof}

\begin{rmk}
  We can switch the order of $V$ and $W$ around on one side to obtain
  \[\ang{\chi_V,\chi_W} = \dim \Hom_{\CC G}(V,W)\]
  because the inner products $\ang{\chi_V,\chi_W}$ and $\ang{\chi_W,\chi_V}$ are additive in $V$ and $W$ and they agree (i.e.\@ are real) when $V$ and $W$ are simple (then Maschke's theorem allows us to conclude).
\end{rmk}

We summarise what we know about character tables in the following proposition:
\begin{prop}
  Let $G$ be a finite group with conjugacy classes $C_1,\ldots,C_r$ and irreducible characters $\chi_1,\ldots,\chi_r$.
  (Note that there is the same number of each.)
  Then character table is the matrix
  \[\left(\chi_i(C_j)\right)_{i,j=1}^r\]
  where $\chi_i(C_j)$ means $\chi_i(g)$ for some $g \in C_j$.
  We follow the following conventions when writing character tables:
  \begin{enum}
    \io We label the rows by the characters $\chi_i$.
    \io We label the columns by a representative of each conjugacy class $C_j$ plus its size $\abs{C_j}$.
    \io The first row is the trivial character.
    \io The first column is the identity element (hence contains the dimensions of the corresponding irreducible modules).
    The sum of the squares of the first column is the size of the group.
  \end{enum}
  We have the following properties of the character table:
  \begin{enum}
    \io
    Schur's orthogonality relations tell us that the rows are `weighted orthogonal', that is for rows $i$ and $i'$ we have
    \[\sum_{j=1}^r \abs{C_j} \chi_i{C_j} \ol{\chi_{i'}(C_j)} = \abs{G} \delta_{ii'}.\]

    \io
    We also get a second column orthogonality relation, that is for columns $j$ and $j'$ we have
    \[\sum_{i=1}^r \chi_i(C_j) \ol{\chi_i(C_{j'})} = \frac{\abs{G}}{\abs{C_j}}\delta_{jj'}.\]
    This can be derived by scaling the character table appropriately to get an orthogonal matrix.

    \io
    The complex conjugate of a row is also a row (possibly the same row).
    \[\ol\chi_i = \chi_{S_i^*}\]

    \io
    The complex conjugate of a column is also a column (possibly the same column).
    \[\ol{\chi_i(g)} = \chi_i(g\inv)\]

    \io
    (Twisting) The pointwise product of a row starting with $1$ and any row is also a row (possibly the same row).
    \[\psi \in G^\vee \qquad \psi\chi_{S_i}=\chi_{S_i^\vee}\]

    \io
    (Tensor products) The pointwise product of any two rows is a non-negative integer linear combination of rows.

    \io
    (If all else fails) By definition, $\chi_i(C_j)$ is by definition the trace of a $\chi_i(1) \times \chi_i(1)$ matrix of order dividing $\abs{g}$ for $g \in C_j$.
    (This is independent of the choice of representative.)
    So $\chi_i(C_j)$ is a sum of $\chi_i(1)$ $\abs{g}$th roots of 1 for $g \in C_j$.
  \end{enum}
\end{prop}

\begin{rmk}
  The last item tells us that $\chi_i(C_j)$ is always an algebraic integer(a root of a monic polynomial with integer coefficients) since algebraic integers form a ring.
\end{rmk}

\begin{exam}
  We now construct the character table of $S_4$.
  \begin{center}
    \begin{tabular}[]{r | c c c c c}
      & 1 & $(1\ 2)$ & $(1\ 2)(3\ 4)$ & $(1\ 2\ 3)$ & $(1\ 2\ 3\ 4)$ \\
      & 1 & 6 & 3 & 8 & 6 \\ \hline
      1 & 1 & 1 & 1 & 1 & 1 \\
      $\eps$ & 1 & $-1$ & 1 & 1 & $-1$ \\
      $\rho$ & 3 & 1 & $-1$ & 0 & $-1$ \\
      $\eps\rho$ & 3 & $-1$ & $-1$ & 0 & $-1$ \\
      $\pi$ & 2 & 0 & 2 & $-2$ & 0
    \end{tabular}
  \end{center}
  where $\rho = \chi_H$ and $H$ is the three-dimensional `hyperplane' representation and $\eps\rho = \chi_{H^\eps}$ where $H^\eps \cong H \otimes \eps$.
  We can use both orthogonality relations and the twisting/tensor product rules to construct the character table (there are many ways of doing so).
\end{exam}

