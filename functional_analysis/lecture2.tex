\section{Lecture 2018-03-06}

Normed spaces and Banach spaces

\begin{defn}
	Let $E$ be a vector space over a $K$, where $K$ denotes either $\RR$ or $\CC$.
	A map $E \to \RR_+$ is a \textbf{norm} (denoted by $x \mapsto \norm{x}$) if:
	\begin{enum}
		\io $\norm{x} \geq 0$ for all $x \in E$, and $\norm{x}=0 \iff x=0$
		\io $\norm{\alpha x} = \abs{\alpha} \norm{x}$ for all $\alpha \in K$ and $x \in E$
		\io $\norm{x+y} \leq \norm{x} + \norm{y}$ for all $x,y \in E$
	\end{enum}
	Then $(E,\norm{\cdot})$ is a \textbf{normed space}.
\end{defn}

\begin{prop}
	Let $(E,\norm{\cdot})$ be a normed space.
	Then $d(x,y) \defeq \norm{x-y}$ is a \textbf{metric} on $E$ if:
	\begin{enum}
		\io $d(x,y) \geq 0$ and $d(x,y)=0 \iff x=y$
		\io $d(x,y)=d(y,x)$
		\io $d(x,z) \leq d(x,y) + d(y,z)$
	\end{enum}
\end{prop}

\begin{prop}
	Recall that if $f:X \to Y$ where $X$ and $Y$ are metric spaces, then the following are equivalent:
	\begin{enum}
		\io $f$ is continuous at $x$
		\io $\forall \epsilon>0 \: \exists \delta>0$ such that $d_y(f(x),f(x'))<\epsilon$ whenever $d_X(x,x')<\delta$
		\io $\forall (x_n) \subset X$ such that $x_n \to x$ (i.e.\@ $d(x_n,x) \to 0$) we have $f(x_n) \to f(x)$
	\end{enum}
\end{prop}

\begin{thm}
	Let $(E,\norm{\cdot})$ be a normed space.
	Then the maps
	\begin{enum}
		\io $x \mapsto \norm{x}, E \to \RR_+$
		\io $(x,y) \mapsto x+y, E \times E \to E$
		\io $(\alpha,x) \mapsto \alpha x, K \times E \to E$
	\end{enum}
	are continuous everywhere.
\end{thm}

\begin{exer}
	Prove the above theorem.
\end{exer}

\begin{rmk}
	If $X$ and $Y$ are metric spaces, then $Z = X \times Y$ is also a metric space with metric $d_Z((x,y),(x',y'))=d_X(x,x')+d_Y(y,y')$.
\end{rmk}

\begin{defn}
	A normed space is a \textbf{Banach space} if it is complete, i.e\@ every Cauchy sequence is convergent.
\end{defn}

\begin{exam}
	The following are examples of normed spaces: $\RR$, $\CC$, $\RR^2$, $\CC^3$, $L^2(X,\mu)$.
	Note that $\QQ$ is not a Banach space.
\end{exam}

\begin{exer}
	Show that there is no norm on $\RR^\infty$.
\end{exer}

\begin{thm}
	A normed space $(E,\norm{\cdot})$ is complete if and only if any absolutely converging series converges, i.e.\@
	\[\sum_{n=1}^\infty \norm{x_n}\]
	is finite if and only if
	\[\sum_{n=1}^\infty x_n\]
	exists.
\end{thm}

\begin{proof}
	$\implies$:
	Since $x_n \subset E$ with $\sum\norm{x_n}<\infty$, then let $a \defeq \sum_{n=1}^N x_n$.
	Then $(a_n)$ is Cauchy and it converges.
	
	$\impliedby$:
	Let $(x_n) \subset E$ be a Cauchy sequence.
	We claim that there exists a subsequence $(x_{n_k}) \subset (x_n)$ such that $\norm{x_{n_{k+1}}-x_{n_k}}\leq \frac{1}{2^k}$.
	To see this, let $n_1=N(\tfrac 12), n_2=N\tfrac 14, \ldots n_k=N(\tfrac1{2^k})$ and so on.
	Then
	\[\sum_{k=1}^\infty \norm{x_{n+{k+1}}-x_{n_k}} <\infty\]
	and so
	\[\sum_{k=1}^\infty x_{n+{k+1}}-x_{n_k}\]
	converges.
	Therefore for all $K \geq 1$,
	\[\sum_{k=1}^K (x_{n_{k+1}}-x_{n_k})=x_{n_K+1}-x_{n_1}\]
	Then the below exercise completes the proof.
\end{proof}

\begin{exer}
	Every Cauchy sequence in a metric space with a convergent subsequence converges.
\end{exer}

Main examples of Banach spaces

\begin{defn}
	For any $1 \leq p \leq \infty$ we will introduce $l_p$ and $c_0$.
	Consider
	\begin{eqn}
		K^\infty = \{ (x_1,x_2,\ldots) \mid x_1,x_2,\ldots \in K\}
	\end{eqn}
	Define the $p$\textbf{-norm} on $K^\infty$:
	\begin{eqn}
		\norm{x}_p=
		\begin{cases}
			\left( \sum \abs{x_n}^p \right)^\frac 1p & 1 \leq p \leq \infty \\
			\sup_n \abs{x_n} & p=\infty
		\end{cases}
	\end{eqn}
	Then
	\begin{align}
		l_p &= \{x \in K^\infty \mid \norm{x}_p<\infty\} \\
		c_0 &= \{x \in K^\infty \mid x_n \to 0\}
	\end{align}
\end{defn}

\begin{exer}
	Why cannot $p$ be less than 1?
\end{exer}

\begin{thm}
	$(l_p,\norm{\cdot}_p)$ and $(c_0,\norm{\cdot}_\infty)$ are Banach spaces.
\end{thm}

\begin{lem}[Young's inequality]
	Let $1 \leq p,q \leq \infty$ be \textbf{conjugate}, i.e.\@
	\begin{eqn}
		\frac 1p + \frac 1q = 1
	\end{eqn}
	Then for all $a,b>0$:
	\begin{eqn}
		ab \leq \frac 1p a^p + \frac 1q b^q
	\end{eqn}
\end{lem}

\begin{proof}[Hint for proof.]
	The function $f(x)=e^x$ is \textbf{convex} i.e.\@
	\begin{eqn}
		f(\lambda x + (1-\lambda)y) \leq \lambda f(x) + (1-\lambda) f(y)
	\end{eqn}
	for $0 \leq \lambda \leq 1$ and for all $x,y\in\RR$.
\end{proof}

\begin{exer}
	Complete the above proof.
\end{exer}

\begin{lem}[H\"older's inequality]
	Let $x,y \in K^\infty$.
	Then
	\begin{eqn}
		\norm{xy}_1 \leq \norm{x}_p \norm{y}_q
	\end{eqn}
	where $xy \defeq (x_1y_1,x_2y_2,\ldots)$, and where $p$ and $q$ are conjugate.
\end{lem}

\begin{proof}
	Consider the case where $\norm{x}_p \neq \infty$, $\norm{y}_q \neq \infty$, $x \neq 0$, $y \neq 0$, $p\neq\infty$ or $q\neq\infty$.
	\begin{multline}
		\frac{\norm{xy}_1}{\norm{x}_p\norm{y}_p} = \sum_n \frac{\abs{x_n}}{\norm{x}_p} \sum_n \frac{\abs{y_n}}{\norm{y}_q} \overset{\text{Young}}{\leq} \sum_n \left( \frac 1p \frac{\abs{x_n}^p}{\norm{x}_p^p} + \frac 1q \frac{\abs{y_n}^q}{\norm{y}_q^q} \right) \\
		= \frac 1p \frac{1}{\norm{x}_p^p}  \sum_n \abs{x_n}^p + \frac 1q \frac{1}{\norm{y}_q^q} \sum_n \abs{y_n}^q = \frac 1p + \frac 1q = 1 \qedhere
	\end{multline}
\end{proof}

