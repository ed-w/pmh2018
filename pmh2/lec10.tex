\section{2018-04-10 Lecture}

\begin{rmk}
	There are two ways to define direct and inverse limits:
	\begin{enum}
		\io $(I,\preceq)$ is a directed set.
		\io $(I,\preceq)$ is a poset.
	\end{enum}
	The first case is a special case of the second.
	If we use the second definition, then the direct sum is a special case of the direct limit.
	Proposition 1.14 works for the second definition but the proof of Proposition 1.15 does not (but it is still true).
	It is easy to check that the direct sum preserves exact sequences though.
\end{rmk}

Begin non-assessable content
\begin{defn}[$p$-adic numbers]
	Let $p$ be a prime number.
	Define $R_n = \ZZ/p^n\ZZ$ for $n \in (\PP,\geq)$.
	If $m \leq n$ then $(p^m) \supseteq (p^n)$, so we have the following commutative diagram:
	\[\begin{tikzcd}[row sep=large]
		& \ZZ \ar[rd, twoheadrightarrow] \ar[ld, twoheadrightarrow] & \\
		R_n \ar[rr, "\exists", dashed] & & R_m
	\end{tikzcd}\]
	where the downward arrows are projection maps.
	
	Then
	\[\ZZ_p \defeq \varprojlim R_n\]
	is calles the ring of \textbf{$p$-adic integers}.
	The localisation of this ring is $\QQ_p$ which is the field of \textbf{$p$-adic numbers}.
	
	Let
	\[x = (x_1,x_2,\ldots) \in \ZZ_p \subset \prod_{r=1}^\infty \ZZ/p^r\ZZ.\]
	Then by definition of the inverse limit, we have $x_i \equiv x_j \bmod p$ for $i \geq j$.
\end{defn}

\begin{defn}[A more general direct limit or colimit]
	We can consider a poset $(I,\preceq)$ as a simple graph with vertex set $I$ and arrows $i \to j$ for all $i \preceq j$.
	Transitivity defines an equivalence relation on the set of paths in the graph:
	\[i \to j \to k \sim i \to k.\]
	Now let $\ca G = (\ca V,\ca E,\sim)$ be an arbitrary graph with the transitivity relation $\sim$ as described above.
	We also allow arrows
	\[i \xto{\id} i \quad \text{and} \quad i \xto{0} j.\]
	Now let $\{M_i\}_{i \in \ca V}$ and $\{f_\alpha\}_{\alpha \in \ca E}$
	with $f_\alpha: M_i \to M_j$ where $\alpha = i \to j$.
	Moreover, if $\alpha$ has label $\id$ or $0$ then $f_\alpha = \id$ or $0$ respectively.
	We also have the following identity for the equivalence relation:
	\[\alpha_1 \cdots \alpha_r \sim \beta_1 \cdots \beta_s \implies f_{\alpha_1} \cdots f_{\alpha_r} = f_{\beta_1} \cdots f_{\beta_s}.\]
	Then the \textbf{colimit} $\colim_{\ca G}\{M_i\}$ is the universal $R$-module with maps $\iota_l: M_l \to \colim_{\ca G}\{M_i\}$ for all $l \in \ca V$ such that the following diagram commutes for all edges $\alpha = (i \to j) \in \ca E$:
	\[\begin{tikzcd}[row sep=large,column sep=small]
		& \colim_{\ca G}\{M_i\} & \\
		M_i \ar[ru, "\iota_i"] \ar[rr, "f_\alpha"] & & M_j \ar[lu, "\iota_j"]
	\end{tikzcd}\]
\end{defn}

\begin{exam}
	\leavevmode
	\begin{enum}
		\io
		If $(I,\preceq)$ is a directed set and $\ca G$ the corresponding graph with relations, then $\colim_{\ca G}=\varinjlim$.
		
		\io
		If $\ca E$ is empty then $\colim_{\ca G} = \bigoplus_{\ca V}$.
		
		\io
		If we take
		\[\left( \ca G = \underset{1}{\circ} \underset{0}{\overset{\alpha}{\rightrightarrows}} \underset{2}{\circ} \right) \longmapsto \left( M_1 \xto{f} M_2 \right)\]
		then $\colim_{\ca G}\{M_i\} = \coker f$.
		If we instead take
		\[\ca G = \underset{1}{\circ} \underset{\beta}{\overset{\alpha}{\rightrightarrows}} \underset{2}{\circ}\]
		then $\colim_{\ca G}$ is the coequaliser of $\alpha$ and $\beta$.
		
		\io
		If we take
		\[\ca G=
		\begin{tikzcd}
			1 \ar[r, "\alpha"] \ar[d, "\beta"] & 2 \\
			3 &
		\end{tikzcd}\]
		Then $\colim_{\ca G}$ is the pushout of fibred coproduct.
	\end{enum}
\end{exam}

End non assessable content

Finitely generated modules

\begin{defn}[1.17]
	An $R$-module $M$ is called \textbf{free} if $M \cong \bigoplus_{i \in I} M_i$ such that $M_i \cong R$ in $\RMod$.
\end{defn}

\begin{rmk}
	\leavevmode
	\begin{enum}
		\io We write $\ds R^n = \bigoplus_{i=1}^n R$.
		\io It is important that $M_i \cong R$ as an $R$-module.
	\end{enum}
\end{rmk}

\begin{prop}[1.18]
	$M$ is finitely generated as an $R$-module $\iff$ $M$ is isomorphic to a quotient of $R^n$ for some $n$.
\end{prop}

\begin{proof}
	$\implies$: Let $x_1,\ldots,x_n$ be a set of generators of $M$.
	Then define a surjective map by
	\begin{align*}
		\phi: R^n &\to M \\
		(a_i)_{i=1}^n &\mapsto \sum_{i=1}^n a_ix_i.
	\end{align*}
	Then $M \cong R^n/\ker\phi$.
	
	$\impliedby$: Let $\phi: R^n \to R^n/U \isoto M$.
	Let $\{e_i\}_{i=1}^n$ be the standard generators of $R^n$.
	Then $\{\phi(e_i)\}_{i=1}^n$ generates $M$.
\end{proof}

\begin{rmk}
	Submodules of finitely generated modules need not be finitely generated.
	Take $R=k[x_1,c-2,\ldots]$.
	Clearly $R$ is finitely generated as an $R$-module.
	Then consider $I = (x_1,x_2,\ldots)$.
	Then $I$ is not finitely generated as an $R$-module.
	
	Submodules of free modules need not be free.
	Take $R = k[x]/(x^n)$ and $I=(x)$.
	Then $R$ is clearly free as an $R$-module but $(x)$ is not free as an $R$-module.
\end{rmk}
