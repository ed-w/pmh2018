\section{2018-03-20 Lecture}

Operations on modules

\begin{defn}
	Let $\{M_i\}_{i \in I}$ be a family of submodules of $M$.
	Then
	\begin{enum}
		\io $\displaystyle \sum_{i \in I} M_i = \left\{\sum x_i \mid x_i \in M_i \text{ where finitely many are non-zero}\right\}$ is a submodule
		\io $\displaystyle \bigcap{i \in I} M_i$ is a submodule
	\end{enum}
\end{defn}

\begin{prop}[1.5]
	The third and second isomorphism theorems respectively.
	\begin{enum}
		\io Let $M'' \subset M' \subset M$ be $R$-modules.
		Then
		\[(M/M'')/(M'/M'') \cong M'/M''\]
		\io Let $M_1$ and $M_2$ be submodules of $M$.
		Then
		\[(M_2+M_1)/M_1 \cong M_2/(M_1 \cap M_2)\]
	\end{enum}
\end{prop}

\begin{proof}
	\begin{enum}
		\io Consider the kernel of the map
		\begin{align*}
			M/M'' &\surjto M/M' \\
			v + M'' &\mapsto v + M'
		\end{align*}
		and apply the first isomorphism theorem.
		\io Consider the sequence of maps
		\[M_2 \injto M_1+M_2 \surjto (M_1+M_2)/M_1\]
		Note that the composition is surjective.
		Then apply the first isomorphism theorem.
	\end{enum}
\end{proof}

\begin{defn}[1.6]
	Let $M$ be an $R$-module and let $I<R$.
	\begin{enum}
		\io $\displaystyle IM = \left\{\sum_i a_im_i \mid a_i \in I, x_i \in M \text{ with finitely many non-zero} \right\}$
		\io Let $M'$ and $M''$ be submodules of $M$.
		Then
		\[(M':M'') \defeq \{x \in R \mid xM'' \subseteq M'\}\]
		is an ideal of $R$.
		\io The \textbf{annihilator} of $M$ is defined as
		\[\ann_R(M) = \ann(M) = (0:R) = \{x \in R \mid xM = 0\}\]
		Note that for $I<R$ with $I \subset \ann M$, $M$ is an $R/I$-module.
		\io $M$ is said to be \textbf{faithful} if $\ann_R(M) = 0$.
		Note that $M$ is always a faithful $R/\ann_R(M)$-module.
		\io We write $\ang{m_1,\ldots,m_k}$ with each $m_i \in M$ for the submodule of $M$ generated by $\{m_i\}$.
		It consists of all finite $R$-linear combinations of the $m_i$.
		\io $M$ is said to be \textbf{finitely generated} if $M=\ang{m_1,\ldots,m_k}$ for some finite set $\{m_i\} \subset M$.
		(Note that there may be relations between the generators.)
	\end{enum}
\end{defn}

\begin{defn}[1.7]\label{def:prod}
	Let $\{M_l\}_{l \in I}$ be a family of $R$-modules.
	\begin{enum}
		\io A \textbf{direct sum} of the $M_l$ is an $R$-module
		\[\bigoplus_{l \in I} M_l\]
		together with $R$-module maps
		\[i_k: M_k \to \bigoplus_{l \in I} M_l\]
		for each $k \in I$ such that for each $R$-module $N$ with maps $\{f_i: M_i \to N\}$ there exists a unique map $\phi$ such that the following diagram commutes:
		\[
			\begin{tikzcd}
				\bigoplus_{l \in I} M_l \ar[rr, dashed, "\exists ! \phi"] & & N \\
				& M_k \ar[ul, "i_k"] \ar[ur, swap, "f_k"]
			\end{tikzcd}
		\]
		for all $k \in I$.
		\io A \textbf{direct product} of the $M_l$ is an $R$-module
		\[\prod_{l \in I} M_l\]
		together with $R$-module maps
		\[p_k: \prod M_l \to M_k\]
		for each $k \in I$ such that for each $R$-module $N$ with maps $\{g_k: N \to M_k\}$ there exists a unique map $\psi$ such that the following diagram commutes:
		\[
		\begin{tikzcd}
		\prod_{l \in I} M_l \ar[dr, swap, "p_k"] & & N \ar[ll, dashed, swap, "\exists ! \psi"] \ar[dl, "g_k"] \\
		& M_k & \\
		\end{tikzcd}
		\]
		for all $k \in I$.
	\end{enum}
\end{defn}

\begin{rmk}
	The above definitions (known as a \textbf{coproduct} and \textbf{product} respectively) is valid in any category, but it does not exist in every category.
	When they exist they are unique up to a unique isomorphism.
\end{rmk}

\begin{rmk}
	There may be many isomorphisms between two (isomorphic) (co)products, but there is only one isomorphism which makes the relevant diagram commute.
\end{rmk}

\begin{prop}
	The product and coproduct exist in $\RMod$ and are unique up to a unique isomorphism.
\end{prop}

\begin{proof}
	We will only prove the proposition for coproducts.
	
	We begin by proving uniqueness.
	(This part of the proof is valid in any category.)
	Let $M$ and $M'$ be two direct sums.
	Then the following diagram commutes:
	\[
		\begin{tikzcd}
			& M \arrow[d, "\phi"] \arrow[dd, "\phi'\circ\phi", bend left] \\
			M_k \arrow[ru, "i_k"] \arrow[r, "i_k'"] \arrow[rd, "i_k"] & M' \arrow[d, "\phi'"] \\
			& M
		\end{tikzcd}
	\]
	But since $M$ is a direct sum, there is a unique map from $M$ to $M$ making the diagram commute.
	Clearly the identity is such a map, hence $\phi'\circ\phi$ is the identity on $M$.
	Analogously, $\phi\circ\phi'$ is the identity on $M'$, so $\phi$ is an isomorphism with inverse $\phi'$.
	
	We now prove existence (in the category $\RMod$).
	Set
	\[M_\oplus = \{(m_l)_{l \in I} \mid m_l \in M \text{ with finitely many non-zero}\}\]
	We make $M$ in to an $R$-module in the obvious way.
	Define $i_k: M_k \to M_\oplus$ to be the co-ordinate injections
	\[i_k(m) = (m_l)_{l \in I} \text{ where } m_l =
		\begin{cases}
			m & \text{ if } l = k \\
			0 & \text{ if } l \neq k
		\end{cases}
	\]
	Now let $N$ be as in definition \ref{def:prod}.
	Define the map $\phi$ by
	\begin{align*}
		\phi: M_\oplus &\to N \\
		(m_l)_{l \in I} &\mapsto \sum_{l \in I}f_l(m_l)
	\end{align*}
	Clearly the diagram commutes.
	
	We will note that the direct product of $R$-modules is defined by
	\[M_{\prod} = \{(m_l)_{l \in I} \mid m_l \in M_l\}\]
	which is the direct sum without the finiteness condition.
\end{proof}

\begin{rmk}
	We have
	\begin{enum}
		\io $\ds \bigoplus_{l \in I} M_l$ is a submodule of $\ds \prod_{l \in I} M_l$.
		\io $\ds \bigoplus_{l \in I} M_l = \ds \prod_{l \in I} M_l$ if $I$ is finite.
	\end{enum}
\end{rmk}