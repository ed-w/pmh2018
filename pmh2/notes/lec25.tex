\section{2018-06-04 Lecture}

\begin{proof}[Proof of lemma 4.13]
  By assumption there exists a non-zero $f \in k[X_1,\ldots,X_d,T]$ such that $f(x_1,\ldots,x_d,x_n)=0$.
  Then $f=a_0T^r+a_1T^{r-1}+\cdots+a_r$ where $a_i \in k[x_1,\ldots,x_d]$ for all $i$, $a_0 \neq 0$ and $r>0$.

  If $a_0 \in k$, then $x_n$ is integral over the subring generated by $x_1,\ldots,x_d$.
  Then $x_1,\ldots,x_n$ are integral over the subring generated by $x_1,\ldots,x_{n-1}$, so by 4.4 $A$ is a finitely generated module over the aforementioned subring.
  (This is the case $m=0$.)

  Now suppose that $a_0 \notin k$.
  If $g(X_1,\ldots,X_d,T)=f(X_1+T^m,\ldots,X^d+T^{m^d},T)$, then $g(x_1-x_n^m,\ldots,x_d-x_n^{m^d},x_n)=0$.
  If $m$ is sufficiently large, then $g=c_0T^s+c_1T^{s-1}+\cdots+c_s$ for $c_0 \in k$ (check this).
  Then $x_n$ is integral over $x_1-x_n^m,\ldots,x_d-x_n^{m^d}$.
\end{proof}

Algebraic geometry III

Assume that $k$ is algebraically closed and $A=k[x_1,\ldots,x_n]/I$ is a finitely generated $k$-algebra.
By Hilbert's basis theorem, $A$ is Noetherian.
By the weak Nullstellensatz and the Zariski lemma, $\mSpec A = \cV(I) = \{ \bo a \in k^n \mid f(\bo a)=0 \text{ for all } f \in I \}$ and $\cV(I) \subseteq \AF^n = \mSpec k[x_1,\ldots,x_n]$ where $\AF^n$ is affine $n$-space over $k$.
By the Nullstellensatz there is a one-to-one correspondence between radical ideals in $A$ and closed subsets of $\mSpec A$ and further between closed subsets of $\mSpec A$ and closed subsets of $\AF^n$ contained in $\cV(I)$.
Then for $f \in A$, we have functions
\begin{align*}
  f: \mSpec A &\to k \\
  \km &\mapsto \ol f \in A/\km = k \\
  f: \cV(I) &\to k \\
  \bo a &\mapsto f(\bo a)
\end{align*}
which are the same under the identification.

\begin{lem}[AG.12]
  \lv
  \begin{enum}
    \io $\mSpec A$ is Noetherian (i.e.\@ $\mSpec A$ satisfies the a.c.c\@ for open subseta)
    \io $\AF^n = \cV(I) \amalg (D_{f_1} \cup \cdots \cup F_{f_r})$ where $I=(f_1,\ldots,f_r)$.
  \end{enum}
\end{lem}

Let $\pi: k[x_1,\ldots,x_n] \to A$ be the projection.
Then if $J \leq A$ is a radical ideal, then $J^c = \pi\inv[J]$ is a radical ideal (since the preimage of a prime ideal is a prime ideal).
Since Noetherian rings have finitely many minimal prime ideals, we can write $J^c = \kp_1 \cap \cdots \cap \kp_r$ for some minimal prime ideals $\kp_1,\ldots,\kp_r$.
Then $\cV(J)$ corresponds to $\cV(J^c) = \cV(\kp_1) \cup \cdots \cup \cV(\kp_r)$ (the maximal irreducible components).

\begin{exam}
  $k=\CC$, $A=k[x,y]$, $J=(xy) = (x) \cap (y)$.
\end{exam}

\begin{defn}[AG.13]
  Let $A=k[x_1,\ldots,x_n]/I$ and $B=k[x_1,\ldots,x_m]/J$.
  A morphism of affine algebraic varieties $f: \cV(J) \to cV(I)$ is the restrictoin of a polynomila map $\ol f: \AF^m \to \AF^n$.
  For $X \subseteq \AF^n$ an affine algebraic variety, we define the co-ordinate ring $k[X]$ of $X$ as the maps of all affine algebraic varieties from $A$ to $\AF^1 = k$.
\end{defn}

\begin{prop}[AG.14]
  Let $X \subseteq \AF^m$ be an affine algebraic variety.
  Then $k[X] = k[x_1,\ldots,x_n]/\cI(X)$.
\end{prop}

We know that if $X = \cV(I)$ then $\cI(X) = \rad I$.
\
\begin{rmk}
  $A = k[x_1,\ldots,x_n]I$ is reduced $\iff I = \rad I \iff A = k[\cV(I)]$.
\end{rmk}
