\section{2018-03-27 Lecture}

\begin{prop}[1.10 continued]
	A sequence of $R$-modules
	\[
	\begin{tikzcd}
	0 \ar[r] & M' \ar[r, "f"] & M \ar[r, "g"] & M''
	\end{tikzcd}
	\]
	is exact $\iff$ for every $R$-module $N$, the sequence
	\[
	\begin{tikzcd}
	0 \ar[r] & \Hom_R(N,M') \ar[r, "f_*"] & \Hom(N,M) \ar[r, "g_*"] & \Hom(N,M'')
	\end{tikzcd}
	\]
	is exact (the covariant $\Hom$ functor $\Hom(N,-)$ is left exact).
	The functor acts on morphisms by
	\begin{align*}
	f_*: \Hom(N,M') &\to \Hom(N,M) \\
	\phi &\mapsto f\circ\phi
	\end{align*}
\end{prop}

\begin{proof}
	We will only provide a proof for the contravariant functor. \par
	$\implies$:
	\begin{enum}
		\io
		$g^*$ is injective.\par
		Let $\phi \in \Hom_R(M'',N)$ be such that $g^*(\phi) = \phi g = 0$.
		Since $g$ is surjective it is right-invertible, hence $\phi=0$.
		
		\io
		$\im g^* = \ker f^*$.\par
		The inclusion $\subseteq$ is obvious since $\phi^*(gf) = \phi gf = 0$ since $gf=0$.
		Now we prove the opposite inclusion $\supseteq$.
		
		Let $\psi \in \ker f^*$.
		Then $f^*(\psi) = \psi f = 0$, hence $\ker \psi \supseteq \im f = \ker g$.
		Now define
		\begin{align*}
			\phi: M'' &\to N \\
			x &\mapsto \psi \circ g\inv(x)
		\end{align*}
		This is well-defined because if $y_1,y_2 \in g\inv[x]$, then $y_2-y_1 \in \ker g = \im f$, so there exists a $z \in M'$ such that $f(z) = y_2-y_1$.
		Then since $\psi \in \ker f^*$, we have $\psi f(z) = 0$, that is $\psi(y_2-y_1) = 0$ and so $\psi(y_1)=\psi(y_2)$.
		
		Therefore $\psi = \phi g$ and hence $\psi \in \im g^*$.
	\end{enum}
	
	$\impliedby$:
	\begin{enum}
		\io $g$ is surjective.\par
		Set $N = M''/\im g$ and let $\pi$ be the natural projection.
		Since $g^*$ is injective for all $N$, we have that $\pi g = 0$ which implies that $\pi=0$, hence $\im g = M$ and so $g$ is surjective.

		\io $\im f \subseteq \ker g$.\par
		For all $\phi \in \Hom_R(M'',N)$ we have $\phi gf=0$ since $\im g^* \subseteq \ker f^*$.
		Then set $N = M''$ and $\phi = \id_{M''}$.
		Then $gf=0$.
		
		\io
		$\ker g \subseteq \im f$.\par
		Let $N = M/\im f$ and let $\pi$ be the natural projection.
		Then $\pi \in \ker f^* = \im g^*$.
		Hence there exists a $\psi \in \Hom_R(M'',N)$ such that $\psi g = \pi$.
		Therefore $\ker g \subseteq \ker \pi = \im f$. \qedhere
	\end{enum}
\end{proof}

\begin{prop}[1.11, the snake lemma]
	In $\RMod$, suppose that we have the following commutative diagram of solid arrows:
	\[
	\begin{tikzcd}[column sep=large, row sep=large]
		0 \arrow[r, "\exists", dotted] & \ker\phi \arrow[d, hook] \arrow[r, "\exists", dashed] & \ker\phi' \arrow[d, hook] \arrow[r, "\exists", dashed] & \ker\phi'' \arrow[d, hook] \arrow[llddd, "\exists d"', dashed, bend right] &  \\
		0 \arrow[r, dotted] & M \arrow[r, "f"] \arrow[d, "\phi"] & M' \arrow[r, "g"] \arrow[d, "\phi'"] & M'' \arrow[r] \arrow[d, "\phi''"] & 0 \\
		0 \arrow[r] & N \arrow[r, "f'"] \arrow[d, two heads] & N' \arrow[r, "g'"] \arrow[d, two heads] & N'' \arrow[d, two heads, dashed] \arrow[r, dotted] & 0 \\
		& \coker\phi \arrow[r, "\exists", dashed] & \coker\phi' \arrow[r, "\exists", dotted] & \coker\phi'' \arrow[r, "\exists", dotted] & 0
	\end{tikzcd}
	\]
	Then the dashed arrows exist and form a long exact sequence.
	If the two rows are short exact (that is, the dotted arrows exist), then the dotted arrows attached to the long exact sequence exist and extend the long exact sequence.
\end{prop}

Direct and inverse limits (colimits and limits).

\begin{defn}[1.12]
	A poset $(I,\preceq)$ is called \textbf{directed} if for all $i,j \in I$ there exists a $k \in I$ such that $i \preceq k$ and $j \preceq k$.
\end{defn}

\begin{defn}[1.13]
	\leavevmode
	\begin{enum}
		\io Let $(I,\preceq)$ be a directed set.
		A \textbf{directed system} of $R$-modules indexed by $(I,\preceq)$ is a family of $R$-modules $\{M_i\}_{i \in I}$ and maps
		\[\alpha_j^i: M_i \to M_j \quad \text{for all} \quad i \preceq j\]
		such that the following equations hold:
		\begin{enum}
			\io $\alpha_i^i= \id_{M_i}$
			\io $\alpha_k^j \circ \alpha_j^i = \alpha_k^i$ for all $i \preceq j \preceq k$, that is, the following diagram commutes:
			\[
			\begin{tikzcd}
				M_i \arrow[rd, "\alpha_j^i"'] \arrow[rr, "\alpha_k^i"] &  & M_k \\
				& M_j \arrow[ru, "\alpha_k^j"'] & 
			\end{tikzcd}
			\]
		\end{enum}
		\io The \textbf{direct limit} (also known as the \textbf{inductive} or \textbf{injective} limit)
		of the directed system of $R$-modules
		\[\left(\{M_i\}_{i \in I}, \{\alpha_j^i\}_{i \preceq j}\right)\]
		is an $R$-module
		\[\varinjlim_{i \in I} M_i = \varinjlim M_i = \colim M_i\]
		with maps
		\[\iota_l: M_l \to \varinjlim M_i\]
		such that
		\[\iota_j \circ \alpha_j^i = \iota_i \quad \text{for all} \quad i \preceq j\]
		and is universal with respect to the above property.
		That is, for any $R$-module $N$ with maps $\beta_l: M_l \to N$ such that $\beta_l \circ \alpha_l^k=\beta_k$ for all $k \preceq l$, there exists a unique map $\theta$ making the following diagram commute:
		\[
		\begin{tikzcd}[row sep=large, column sep=large]
			\varinjlim M_i \arrow[rr, "\exists !\theta", dashed] & & N \\
			& M_l \arrow[lu, "\iota_l"'] \arrow[ru, "\beta_l"] &  \\
			& M_k \arrow[luu, "\iota_k", bend left] \arrow[ruu, swap, "\beta_k", bend right] \arrow[u, "\alpha_l^k"] & 
		\end{tikzcd}
		\]
	\end{enum}
\end{defn}

\begin{exam}
	Let $(I,\preceq)$ be a direct system with the discrete partial order and let $\{M_i\}_{i \in I}$ be a family of $R$-modules.
	Then the direct limit of $M_i$ is just their coproduct.
\end{exam}
