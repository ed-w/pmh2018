\section{2018-05-15 Lecture}

\begin{prop}[3.13]
  Let $R$ be a Noetherian ring and $\ph: R \to R'$ is a map of rings.
  Then $\im\ph$ is Noetherian.
\end{prop}

\begin{proof}
  $\im\ph = R/\ker\ph$ is Noetherian as an $R$-module, hence is also Noetherian as an $\im\ph$-module.
\end{proof}

\begin{prop}[3.14]
  Let $R'$ be a subring of $R$ with $R'$ Noetherian and $R$ finitely generated as an $R'$-module.
  Then $R$ is Noetherian.
\end{prop}

\begin{proof}
  By prop.\@ 3.6 we know that $R$ is a Noetherian $R'$-module, hence $R$ is Noetherian as an $R$-module (using the maximum condition, since every $R$-submodule of $R$ is also an $R'$-submodule).
\end{proof}

\begin{prop}
  Let $R$ be Noetherian and $S \subseteq R$ a multiplicative set.
  Then $S\inv R$ is Noetherian.
\end{prop}

\begin{proof}
  There is an inclusion-preserving bijection between ideals in $S\inv R$ with all of the contracted ideals in $R$ (since they are extended ideals).
  Then we can apply the maximum condition.
\end{proof}

\begin{cor}[3.16]
  Let $R$ be Noetherian and $\kp<R$ prime.
  Then $R_\kp$ is Noetherian.
\end{cor}

\begin{thm}[3.17, Hilbert's basis theorem]
  If $R$ is Noetherian, then $R[x]$ is Noetherian.
\end{thm}

\begin{proof}
  Let $I<R[x]$.
  Define the \textbf{initial} (or \textbf{inertial}) \textbf{ideal}:
  \[ \ini(I) = \{ r \in R \mid r \text{ is a leading coefficient of some }f \in I \}. \]
  (Check that $\ini(I)$ is an ideal.)
  Since $R$ is Noetherian, we can choose a set of generators for $\ini(I)$ so that
  \[ \ini(I) = (r_1,\ldots,r_n). \]
  Now for each $i$ choose an $f_i \in I$ such that $f_i = r_ix^{d_i}\ +\ $lower degree terms.
  Set $d=\max{d_i}$ and define
  \[ I' = (f_1,\ldots,f_n) \subseteq I. \]
  Let $f = ax^m\ +\ $lower degree terms $\in I$.
  Then $a \in \ini(I)$.
  If $m \geq d$, then write $a = \sum_{i=1}^n b_ir_i$.
  Then
  \[ f - \sum_{i=1}^n b_i x^{m-d_i}f_i \in I \]
  and has degree $<n$.
  So we can assume that $f=g+h$ where $h \in I'$ is a combination of the $f_i$s, and $g \in I$ with $\deg g<d$.
  Let
  \[ M = \langle1,x,\ldots,x^{d-1}\rangle_R \]
  be an $R$-submodule of $R[x]$.
  We have just showed that $I = (I \cap M) + I'$.
  Now $I \cap M$ is a submodule of $M$ which is a finitely generated $R$-module, hence it is Noetherian.
  Set
  \[ I \cap M = \ang{g_1,\ldots,g_t}_R \]
  Then
  \[ I = (I \cap M) + I' = (f_1,\ldots,f_n,g_1,\ldots,g_t) \subseteq I, \]
  hence $I$ is finitely generated.
\end{proof}

\begin{cor}[3.18]
  \lv
  \begin{enum}
    \io $k[x]$ is Noetherian if $k$ is a field.
    \io If $R$ is Noetherian, then $R[x_1,\ldots,x_n]$ is Noetherian.
  \end{enum}
\end{cor}

\begin{defn}[3.20]
  Let $R$ be a commutative ring, $A$ a ring and $f: R \to A$ a map of rings such that $f[R] \subseteq Z(A)$.
  Then $A$ is an $R$-module with the action $r \cdot a = f(a)$. for all $r \in R$ and $a \in A$.
  Then $A$ is called an \textbf{$R$-algebra}.
\end{defn}

\begin{cor}[3.19]
  If $R$ is Noetherian and $A$ is a finitely-generated (commutative) $R$-algebra, then $A$ is Noetherian.
\end{cor}

\begin{proof}
  We have $A \cong R[x_1,\ldots,x_n]/I$ for some $n$ and some ideal $I$, so $A$ is Noetherian.
\end{proof}

\begin{cor}[3.21]
  \lv
  \begin{enum}
    \io All finitely generated rings are Noetherian (since commutative rings are $\ZZ$-algebras).
    \io All finitely generated $k$-algebras are Noetherian (where $k$ is a field).
  \end{enum}
\end{cor}

\begin{prop}[Artin-Tate lemma, 3.22]
  Let $R \subseteq A \subseteq B$ be rings where $R$ is Noetherian, $B$ is finitely generated as an $R$-algebra and finitely generated as an $A$-module.
  Then $A$ is finitely-generated as an $R$-algebra, hence Noetherian.
\end{prop}

\begin{proof}
  Let $x_1,\ldots,x_m$ be the $R$-algebra generators of $B$ and $y_1,\ldots,y_n$ the $R$-module generators of $B$.
  Then for all $i$ we have
  \[ x_i = \sum_{j=1}^n a_{ij}y_j \text{ where } a_{ij} \in A \text{ for all } j, \]
  and for all $i$ and $j$,
  \[ y_iy_j = \sum_{k=1}^n a_{ijk} y_k \text{ where } a_{ijk} \in A \text{ for all } k. \]
  Let $A_0$ be the $R$-algebra generated by the set 
  \[ \{ a_{ij}, a_{ijk} \mid 1 \leq i,j,k \leq n \}. \]
  Note that $R \subseteq A_0 \subseteq A$, and $A_0$ is finitely generated, hence Noetherian.
  Then any $x \in B$ can be written as an $A_0$-linear combination of the $y_i$s.
  Hence $B$ is a finitely generated $A_0$-module, hence it is a Noetherian $A_0$-module.
  Then since $A \subseteq B$ is an $A_0$-submodule of $B$, hence it is a finitely generated $A_0$-module.
  Since $A_0$ is finitely generated as an $R$-algebra, hence $A$ is a finitely generated $R$-algebra which is thus Noetherian.
\end{proof}
