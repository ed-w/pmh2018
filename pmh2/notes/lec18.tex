\section{2018-05-08 Lecture}

(still non-assessable content)

\begin{defn}
  Take a point $x$ in a topological space $X$ and let $U$ and $V$ be two open neighbourhoods of $x$.
A \textbf{germ} is an equivalence class of functions from $X$ to another topological space $Y$ defined by the following relation: if $\ph: U \to Y$ and $\ph': V \to Y$ are two maps, then $(\ph,U) \sim (\ph',V)$ if there exists an open neighbourhood $W \subset U \cap V$ of $X$ such that $\ph|_W = \ph'|_{W'}$.
\end{defn}i

\begin{defn}[AG.11]
  Let $X$ be a topological space, $\ca F$ a sheaf on $X$ and $x \in X$.
  The \textbf{stalk} of $\ca F$ at $x$ is
  \[ \ca F_x = \varinjlim_{x \in U} \ca F(U). \]
\end{defn}

\begin{exam}
  Let $X = \Spec R$, $\ca F = \wt M$, $x = \kp$ and $S = R \setminus \kp$  where $\kp$ is a prime ideal.
  Then
  \[ \wt M_\kp = \varinjlim_{\kp \in D_f} \wt M(D_f) = \varinjlim_{f \in S} M_f = M_\kp. \]
  So stalks are just localisations at prime ideals.
  Similarly, we have
  \[ (\ca O_{\Spec R})_\kp = R_\kp. \]
\end{exam}

We want to see $R$ as functions on $\Spec R$.
For $\mSpec R \subseteq \Spec R$, we have
\[ r \in R \qquad r: m \mapsto r + \km \in R/\km. \]

begin assessable content

\S 3 Noetherian and Artinian rings (and modules)

\begin{lem}[3.1]\label{18:cond}
  Let $(\Sigma, \preceq)$ be a poset.
  The following are equivalent:
  \begin{enum}
    \io\label{18:acc} Every increasing sequence $x_1,x_2,\ldots$ in $\Sigma$ eventually stabilises.
    \io\label{18:max} Every non-empty subset of $\Sigma$ has a maximal element.
  \end{enum}
\end{lem}

\begin{proof}
  Omitted. Trivial.
\end{proof}

\begin{defn}
  Let $M$ be an $R$-module.
  \begin{enum}
    \io
    Let $\Sigma$ be the set of all submodules of $M$ ordered by inclusion.
    Then condition $\ref{18:acc}$ is called the \textbf{ascending chain condition} (ACC) and condition $\ref{18:max}$ is called the \textbf{maximal condition}.
    If $M$ satisfies both (or either) of conditions \ref{18:acc} and \ref{18:max} then $M$ is called \textbf{Noetherian}.

    \io
    Let $\Sigma$ be the set of all submodules of $M$ ordered by reverse inclusion.
    Then condition $\ref{18:acc}$ is called the \textbf{descending chain condition} (ACC) and condition $\ref{18:max}$ is called the \textbf{minimal condition}.
    If $M$ satisfies both (or either) of conditions \ref{18:acc} and \ref{18:max} then $M$ is called \textbf{Artinian}.
  \end{enum}
\end{defn}

\begin{exam}
  \leavevmode
  \begin{enum}
    \io
    Let $R=\ZZ$.
    Then any finite $\ZZ$-module satisfies the ACC and DCC.

    \io
    Let $R=\ZZ$ and $M=\ZZ$.
    We have $n\ZZ \subseteq m\ZZ \iff m \mid n$.
    Then $M$ is Noetherian but not Artinian.

    \io
    Let
    \[ G_p = \{x \in \QQ/\ZZ \mid \ord(x) = p^r \text{ for } r \in \NN\}. \]
    We have submodules
    \[ G_p^r = \{x \in \QQ/\ZZ \mid (\ord(x) \mid p^r) \}. \]
    Then $G_p^0 \subset G_p^1 \subset G_p^2 \subset \cdots$, so $G_p$ is not Noetherian.

    \begin{exer}
      Show that $G_p$ is Artinian by showing that the $G_p^r$ are all of the submodules of $G_p$ (then the minimal condition is satisfied).
    \end{exer}

    \io
    Let
    \[ H_p = \left\{ \frac{m}{p^r} \in \QQ \mid m,r \in \ZZ, r \geq 0 \right\}. \]

    \begin{exer}
      Show that there is an exact sequence
      \begin{equation*}
	\begin{tikzcd}
	  0 \ar[r] & \ZZ \ar[r] & H_p \ar[r] & G_p \ar[r] & 0
	\end{tikzcd}
      \end{equation*}
    \end{exer}
    Then by a theorem we will prove later, $H_p$ is neither Artinian nor Noetherian.
  \end{enum}

  \io
  Consider the polynomial ring in infinitely many variables: $R = k[x_i \mid i \in \NN]$ and let $M=R$.
  Since $(x_0) \subset (x_0,x_1) \subset (x_0,x_1,x_2) \subset \cdots$, $M$ is not Noetherian.
  Since $(x_0) \supset (x_0x_1) \supset (x_0x_1x_2) \supset \cdots$, $M$ is not Artinian.
\end{exam}

\begin{prop}[3.2]
  Let $M$ be an $R$-module.
  Then $M$ is Noetherian $\iff$ every submodule of $M$ is finitely generated.
\end{prop}

\begin{proof}
  $\implies$:
  Let $N$ be a non-zero submodule of $M$.
  Let $\Sigma'$ be the set of all finitely generated submodules of $N$.
  Clearly $\Sigma' \neq 0$ so it has a maximal element $N_0$.
  If $N_0 \subsetneq N$, there exists an $x \in N \setminus N_0$.
  Hence $N_0 + Rx \subseteq N$.
  But $N_0 + Rx$ is finitely generated, hence it is in $\Sigma'$, contradicting the maximality of $N_0$.
  Therefore $N=N_0$, so $N$ is finitely generated.

  $\impliedby$:
  Let $M_1 \subseteq M_2 \subseteq \cdots$ be an ascending chain in $M$.
  Then
  \[ N = \bigcup_{i=1}^\infty M_i \]
  is a submodule, so we have
  \[ N = \ang{x_1,\ldots,x_n}_R = \sum_{i=1}^n Rx. \]
  Now there exists some $k \geq 1$ such that $x_1,\ldots,x_n \in M_k$.
  Hence $N=M_k$, so the chain stabilises and $M$ is Noetherian.
\end{proof}
