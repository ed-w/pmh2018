\section{2018-04-09 Lecture}

\begin{prop}[1.14]
	The direct limit in $\RMod$ exists and is unique.
\end{prop}

\begin{rmk}
	As a universal property it is unique up to a unique isomorphism compatible with the directed system.
\end{rmk}

\begin{proof}
	We prove existence only.
	Uniqueness is proved in the same way as for universal properties.
	
	Let
	\[\varinjlim M_i = \left( \bigoplus_{i \in I} M_i \right)/M'\]
	where $M'$ is the submodule generated by
	\[i_j(m_j) - i_l\circ\alpha_l^j(m_j) \quad \text{for all} \quad m_j \in M_j \text{ and } j \preceq l\]
	and where
	\[\iota_l:
	\begin{tikzcd}
		M_l \ar[r, hookrightarrow, "i_l"] & \bigoplus_{i \in I} M_i \ar[r, twoheadrightarrow] & \varinjlim M_i
	\end{tikzcd}\]
	Now by definition, we have $\iota_j(m_j)=\iota_l\circ\alpha_l^j(m_j)$ for all $m_i \in M_j$ and $j \preceq l$.
	
	To show universality, let us assume that $N$ is an $R$-module with maps $\beta_j: M_j \to N$ such that $\beta_j \circ \alpha_j^k=\beta_k$ for all $k \preceq j$.
	Then by the universal property of the direct sum, we have a map
	\[\theta: \bigoplus_{i \in I} M_i \to N\]
	such that
	\[\theta \circ i_l(m_l) = \beta_l(m_l)\]
	for all $l \in I$.
	
	Now
	\[\theta\left((\iota_j(m_j)-\iota_l\circ\alpha_l^j(m_j)\right) = \beta_j(m_j)-\beta_l\circ\alpha_l^j(m_j) = \beta_j(m_j)-\beta_j(m_j) = 0\]
	So $\theta[M']=0$ and therefore it descends to $\theta: \varinjlim M_i \to N$.
	To show that it is unique, suppose $\phi: \bigoplus_i M_i \to N$ satisfies $\phi \circ \iota_l(m_l) = \beta_l(m_l)$.
	For any $m \in \bigoplus_i M_i$ we have
	\[m = \sum_{j \in J} \iota_j (m_j)\]
	where $J \subseteq I$ is finite.
	Then
	\[\phi(m) = \sum_{j \in J} \phi\circ\iota_j(m_j) = \sum_{j \in J} \beta_j(m_j) = \sum_{j \in J} \theta\circ\iota_j(m_j) = \theta(m) \qedhere\]
\end{proof}

\begin{prop}[1.15]
	The direct limit of a direct system of exact sequences is exact.
\end{prop}

\begin{proof}
	It suffices to prove the proposition for three-term sequences of the form
	\[\left(\left\{
	\begin{tikzcd}
		M_i \ar[r, "f_i"] & N_i \ar[r, "g_i"] & P_i
	\end{tikzcd}
	\right\}_{i \in I},
	\left\{\left(\alpha^i_j: M_i \to M_j,\beta^i_j: N_i \to N_j,\gamma^i_j: P_i \to P_j\right)\right\}_{i \preceq j}\right).\]	
%	\[\begin{tikzcd}
%	(M_i,\alpha_j^i)_{i \in I} \ar[r, "(f_i)_{i \in I}"] & (N_i,\beta_j^i)_{i \in I} \ar[r, "(g_i)_{i \in I}"] & (P_i,\gamma_j^i)_{i \in I}
%	\end{tikzcd}\]	
	We require that the maps in a directed system of exact sequences make the following diagram commute for all $i \preceq j$:
	\[\begin{tikzcd}
		M_i \ar[r, "f_i"] \ar[d, "\alpha^i_j"] & N_i \ar[r, "g_i"] \ar[d, "\beta^i_j"] & P_i \ar[d, "\gamma^i_j"] \\
		M_j \ar[r, "f_j"] & N_j \ar[r, "g_j"] & P_j
	\end{tikzcd}\]
	(The rows in the above diagram are exact.)
	
	Now since $(\iota_l^N \circ f_l) \circ \alpha_l^k = \iota_l^N \circ \beta_l^k \circ f_k = \iota_k^N \circ f_k$, $\varinjlim N_i$ satisfies the property defining $\varinjlim M_i$ (but is not universal).
	A similar property holds for $\varinjlim P_i$.
	Therefore we have the following commutative diagram
	\[\begin{tikzcd}[row sep=huge, column sep=large]
	\varinjlim M_i \arrow[rr, "\exists !\theta", dashed] & & \varinjlim N_i \arrow[rr, "\exists !\phi", dashed] & & \varinjlim P_i & \\
	& M_l \arrow[lu, "\iota_l^M"'] \arrow[ru, "\iota_l^N \circ f_l"] \arrow[rr, "f_l"] & & N_l \arrow[lu, "\iota_l^N"'] \arrow[ru, "\iota_l^P \circ g_l"] \arrow[rr, "g_l"] & & P_l \arrow[ul, "\iota_l^P"'] \\
	& M_k \arrow[luu, "\iota_k^M"] \arrow[ruu, swap, "\iota_k^N\circ f_k"] \arrow[u, "\alpha_l^k"] \arrow[rr, "f_k"] & & N_k \arrow[luu, "\iota_k^N"] \arrow[ruu, swap, "\iota_k^P\circ g_k"] \arrow[u, "\beta_l^k"] \arrow[rr, "g_k"] & & P_k \arrow[u, "\gamma_l^k"] \arrow[uul, "\iota_l^P"]
	\end{tikzcd}\]
	
	We will first show that $\im\theta \subseteq \ker\phi$.
	Now
	\[\phi\circ\theta\circ\iota_l^M = \iota_l^P\circ g_l\circ f_l = 0\]
	since $g_lf_l=0$ for all $l \in I$.
	Since every element of $\varinjlim M_i$ is the sum of images of elements in the $M_i$, we conclude that $\phi\circ\theta=0$.
	
	Now we show that $\im\theta \supseteq \ker\phi$.
	Let $x \in \ker\phi$ be a coset representative in $\bigoplus_i N_i$.
	Then we may write
	\[x=\sum_{l\in J} \iota_l^N(x_l)\]
	for some finite set $J \subseteq I$.
	Then
	\[0=\phi(x) = \sum_{l \in J} \phi\circ\iota_l^N (x_l) = \sum_{l \in J} \iota_l^P\circ g_l(x_l)\]
	Since $x$ has finitely many nonzero entries, there exists a $j \succeq l$ for all $l \in J$ such that $0 = \gamma_j^l\circ g_l(x_l) = g_j\circ\beta_j^l(x_l)$ for all $l$ (since the sum is direct).
	Therefore $\beta_j^l(x_l) \in \ker g_j = \im f_j$ for all $l \in J$.
	Then there exists a $y_l \in M_j$ such that $f_j(y_l) = \beta_j^l(x_l)$.
	Then
	\[x = \sum_{l\in J} \iota_l^N(x_l) = \sum_{l \in J} \iota_j^N\circ\beta_j^l(x_l) = \sum_{l \in J} \iota_j^N \circ f_j(y_l) = \sum_{l \in J} \theta\circ\iota_j^M(x_l)\]
	and so $x \in \im\theta$.
\end{proof}

\begin{rmk}
	We have:
	\begin{enum}
		\io The direct sum of exact sequences is exact.
		\io If $U_i$ is a submodule of $M_i$, that is, we have the following exact sequence:
		\[\begin{tikzcd}
			0 \ar[r] & U_i \ar[r] & M_i \ar[r] & M_i/U_i \ar[r] & 0
		\end{tikzcd}\]
		then $\varinjlim U_i$ is a submodule of $\varinjlim M_i$ and $\varinjlim M_i/U_i$ is a quotient of $\varinjlim M_i$.
		Moreover, we have
		\[\varinjlim M_i/U_i \cong \varinjlim M_i / \varinjlim U_i\]
	\end{enum}
\end{rmk}

\begin{defn}[1.16]
	\begin{enum}
		\io Let $(I,\preceq)$ be a directed set.
		An \textbf{inverse system} of $R$-modules indexed by $(I,\preceq)$ is a family of $R$-modules $\{M_i\}_{i \in I}$ and maps
		\[\beta_j^i: M_j \to M_i \quad \text{for all} \quad i \preceq j\]
		such that the following equations hold:
		\begin{enum}
			\io $\beta_i^i= \id_{M_i}$
			\io $\beta_j^i \circ \beta_k^j = \beta_k^i$ for all $i \preceq j \preceq k$, that is, the following diagram commutes:
			\[
			\begin{tikzcd}
			M_k \arrow[rd, "\beta_k^j"'] \arrow[rr, "\alpha_k^i"] &  & M_i \\
			& M_j \arrow[ru, "\alpha_j^i"'] & 
			\end{tikzcd}
			\]
		\end{enum}
		\io The \textbf{inverse limit} (also known as the \textbf{projective} limit)
		of the directed system of $R$-modules
		\[\left(\{M_i\}_{i \in I}, \{\beta_j^i\}_{i \preceq j}\right)\]
		is an $R$-module
		\[\varprojlim_{i \in I} M_i = \varprojlim M_i = \lim M_i\]
		with maps
		\[\pi_l: \varprojlim M_i \to M_l\]
		such that
		\[\beta_j^i \circ \pi_j = \pi_i \quad \text{for all} \quad i \preceq j\]
		and is universal with respect to the above property.
		That is, for any $R$-module $N$ with maps $\alpha_j: N \to M_j$ such that $\beta_j^i \circ \alpha_j=\alpha_i$ for all $i \preceq j$, there exists a unique map $\theta$ making the following diagram commute:
		\[
		\begin{tikzcd}[row sep=large, column sep=large]
		\varprojlim M_i \arrow[rd, "\pi_j"] \arrow[rdd, "\pi_i", bend right] & & N \arrow[ll, "\exists !\theta", dashed, swap] \arrow[ld, "\alpha_j" '] \arrow[ldd, "\alpha_i" ', bend left]\\
		& M_j \arrow[d, "\beta_j^i"] &  \\
		& M_i & 
		\end{tikzcd}
		\]
	\end{enum}
\end{defn}

\begin{rmk}
	\leavevmode
	\begin{enum}
		\io We can construct $\varprojlim M_i \subset \prod M_i$.
		\io An inverse limit of exact sequences is in general not exact.
		\io Direct limits are a special case of colimits and inverse limits are a special case of limits.
	\end{enum}
\end{rmk}
