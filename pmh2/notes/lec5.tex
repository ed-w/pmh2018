\section{2018-03-19 Lecture}

\begin{exam}
	Let $R=\CC[x]$.
	Then $\mSpec R = \{(x-c) \mid c \in \CC\ \leftrightarrow \CC$ and $\Spec R = \CC \sqcup \{0\}$.
	The closure $\ol{(0)}=\Spec R$.
	For an $f \in \CC[x]$, associate a function $\phi_f: \mSpec R \to \CC$ by $(x-c) \mapsto f(c)$ (the evaluation).
	
	The Zariski topology: $I<\CC[x] \implies I = (f)$ for some $f \neq 0$.
	Then $\cV{f}$ is the set of linear factors dividing $f$, or equivalently the set of zeros of $f$.
	Note that $\abs{\cV(f)}<\infty$, so the closed subsets of $\Spec R$ are $\Spec R$ and finite subsets in $\mSpec R$.
	Thus the Zariski topology is not Hausdorff (two distinct points may not have two disjoint open neighbourhoods).
\end{exam}

begin assessable stuff

\S 1: Modules

\begin{defn}[1.1]
	A \textbf{(left) $R$-module} $M$ is an abelian group $(M,+)$ with a binary operation:
	\begin{align*}
		\mu_M = M: R \times M &\to M \\
		(x,m) \mapsto xm
	\end{align*}
	such that
	\begin{enum}
		\io $x(m+n)=xm+xn$
		\io $(x+y)m=xm+ym$
		\io $(xy)m=x(ym)$
		\io $1_Rm=m$
		for all $m,n \in M$ and $x,y \in R$.
	\end{enum}
\end{defn}

\begin{exam}
  	\leavevmode
	\begin{enum}
		\io If $R=k$ is a vector space, then $R$-modules are $k$-vector spaces.
		\io If $R=k[X]$ then an $R$-module $M$ is a $k$-vector space with a linear map $\wt{X} \in \End M$.
		\io If $R=k[X,Y]/(f(X,Y))$, then an $R$-module $M$ is a $k$-vector space and a pair of commuting $k$-linear maps $\wt{X},\wt{Y} \in \End M$ such that $f(\wt{X},\wt{Y})=0$.
		\io Any ideal $I$ of $R$ is an $R$-module.
	\end{enum}
\end{exam}

\begin{defn}[1.2]
	Let $M$ and $N$ be two $R$-modules.
	An \textbf{$R$-module homomorphism} is a map $f:M \to N$ such that
	\begin{enum}
		\io $f$ is a morphism of abelian groups, and
		\io $xf(m)=f(xm)$ for all $x \in R$ and $m \in M$.
	\end{enum}
\end{defn}

\begin{rmk}
	The $R$-modules and their homomorphisms form a category $\RMod$.
	Injections and surjections are the monomorphisms and epimorphisms respectively (unlike with rings where nor all epimorphisms are surjective).
\end{rmk}

\begin{prop}[1.3]
  	\leavevmode
	\begin{enum}
		\io The set $\Hom_R(M,N)$ is an abelian group with addition.
		\io The set $\Hom_R(M,N)$ is an $R$-module with pointwise scalar multiplication.
		\io There is a natural isomorphism $\Hom_R(R,M) \xto{\sim} M$ by $f \mapsto f(1)$.
	\end{enum}
\end{prop}

\begin{defn}[1.4]
  	\leavevmode
	\begin{enum}
		\io A submodule $M'$ of $M$ is a subgroup $M'<M$ such that $\mu_M$ restricts to $\mu: R \times M' \to M'$.
		\io The quotient of $M$ by $M'$ by a submodule $M$ is the quotient group $M/M'$ with $\mu_{M/M'}$ the push down of $\mu_M$:
		\[
			\begin{tikzcd}
				R \times M \arrow[r, "\mu_M"] \arrow[d] & M \arrow[d] \\
				R \times M/M' \arrow[r, "\mu_{M/M'}"'] & M/M'
			\end{tikzcd}
		\]
	\end{enum}
\end{defn}

\begin{exam}
  	\leavevmode
	\begin{enum}
		\io If $I<R$ is an ideal then $R/I$ is an $R$-module.
		\io If $f \in \Hom_R(M,N)$, then $\ker f$ is a submodule of $M$ and $\im f$ is a submodule of $N$.
		\io The first isomorphism theorem for groups implies the first isomorphism theorem for modules:
		\[\ol{f}: M/\ker f \xto{\sim} \im f\]
		is an $R$-module map for any $f \in \Hom_R(M,N)$.
		We just need to check that the inverse of an $R$-module map is also an $R$-module map.
		\io The cokernel of $f \in \Hom_R(M,N)$ is defined as:
		\[\coker f = N/\im f\]
	\end{enum}
\end{exam}

\begin{rmk}
	$\ker f$ and $\coker f$ are universal objects:
	\[
		\begin{tikzcd}
			M \arrow[rr, "f"] & & N \\
			& \ker f \arrow[ul, hook, "\iota"] \arrow[ur, "0"] & \\
			& L \arrow[uul, bend left, "g"] \arrow[uur, bend right, "0"] \arrow[u, dashed, "\exists !"] &
		\end{tikzcd}
	\]
	\[
		\begin{tikzcd}
			M \arrow[rr, "f"] \arrow[dr, "0"] \arrow[ddr, bend right, "0"] & & N \arrow[dl, twoheadrightarrow, "\pi"] \arrow[ddl, bend left, "g"] \\
			& \coker f \arrow[d, dashed, "\exists !"] & \\
			& L &
		\end{tikzcd}
	\]
\end{rmk}
