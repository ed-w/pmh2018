\section{2018-05-01 Lecture}

\begin{lem}[2.14]
  \leavevmode
  \begin{enum}
    \io The map $S\inv(\square) = (\square)^e$ commutes with finite sums, products and intersections of ideals.
    \io $\rad S\inv I = S\inv(\rad I)$.\\
    (Note that $\supseteq$ is already true for any extension of an ideal.)
  \end{enum}
\end{lem}

\begin{rmk}
  In general, $S\inv(I:J) \neq (S\inv I: S\inv J)$.
\end{rmk}

\begin{cor}[2.15]
  $\nilrad S\inv R = S\inv (\nilrad R)$.
\end{cor}

\begin{cor}[2.16]
  Let $\kp$ be prime in $R$.
  Then there is a bijection of sets
  \[\Spec R_\kp \longleftrightarrow \left\{ \fr q \in \Spec R \mid \fr q \subseteq \kp \right\}\]
\end{cor}

\begin{proof}
  By Proposition 2.13 we have a bijection
  \[\Spec R_\kp \longleftrightarrow \left\{ \fr q \in \Spec R \mid \fr q \cap (R \setminus \kp) = \emptyset \right\}\]
  and $\fr q \cap (R \setminus \kp) = \emptyset \iff \fr q \subseteq \kp$.
\end{proof}

\begin{rmk}
  Let $\fr q \subseteq \kp$ be prime ideals.
  Then there is a ring isomorphism
  \[R_\kp/\fr q^e \cong (R/\fr q)_{\kp^e}\]
  where on the left hand side the extension is with respect to the localisation and on the right hand side the extension is with respect to the quotient.
  The spectrum of both of these rings is in bijection with the set
  \[ \left\{ \fr r \in \Spec R \mid \fr q \subseteq \fr r \subseteq \kp \right\}.\]
\end{rmk}

Local properties

\begin{defn}
  A property $P$ of a ring (or an $R$-module) $A$ is called \textbf{local} if $A$ has property $P$ if and only if $A_\kp$ has property $P$ for all $\kp \in \Spec R$.
  A property is sometimes called \textbf{very local} if the above holds with $\Spec$ replaced by $\mSpec$.
\end{defn}

\begin{prop}
  $M=0$ is a local property.
\end{prop}

\begin{proof}
  It is clear that if $M=0$ then all localisations of $M$ are zero.
  Now assume $M_\kp=0$ for all $\kp \in \Spec R$.
  In particular, $M_\km=0$ for all $\km \in \mSpec R$.
  Now let $x \in M \setminus \{0\}$ and $I = \ann_R(x)$.
  Since $I \subsetneq R$ we must have $I \subseteq \km_x$ for some $\km_x \in \mSpec R$.
  Then $x/1=0$ in $M_{\km_x}$, so there exists a $u \in R \setminus \km_x$ such that $ux=0$.
  Then $u \in \ann_R(x)$, a contradiction.
\end{proof}

\begin{cor}[2.18]
  The map
  \begin{align*}
    M &\to \prod_{\km \in \mSpec R} M_\km \\
    m &\mapsto \left( \frac{m}{1} \in M_\km \right)_{\km \in \mSpec R}
  \end{align*}
  is injective.
\end{cor}

\begin{proof}
  This is proved analogously to Proposition 2.17.
\end{proof}

\begin{prop}[2.19]
  Let $\ph: M \to N$ be an $R$-module map.
  Then $\ph$ being (in/sur/bi)-jective is a local property.
  (This is defined in terms of kernels and cokernels.)
\end{prop}

\begin{proof}
  We have an exact sequence
  \begin{equation*}
    \begin{tikzcd}
      0 \ar[r] & \ker\ph \ar[r] & M \ar[r, "\ph"] & N \ar[r] & \coker\ph \ar[r] & 0
    \end{tikzcd}
  \end{equation*}
  from which applying the localisation map gives an exact sequence
  \begin{equation*}
    \begin{tikzcd}
      0 \ar[r] & (\ker\ph)_\kp \ar[r] & M_\kp \ar[r, "\ph_\kp"] & N_\kp \ar[r] & (\coker\ph)_\kp \ar[r] & 0
    \end{tikzcd}
  \end{equation*}
  This gives the following isomorphisms:
  \[(\ker\ph)_\kp \cong \ker(\ph_\kp) \quad\text{and}\quad (\coker\ph)_\kp \cong \coker(\ph_\kp)\]
  Then since $M=0$ is a local property, applying this to the modules $\ker\ph$ and $\coker\ph$ gives that $\ph$ is injective if and only if $\ph_\kp$ is injective for all $\kp \in \Spec R$, and similarly for surjectivity.
\end{proof}

\begin{cor}
  $R$ being reduced (containing no nilpotent elements) is a local property.
\end{cor}

\begin{proof}
  By considering $\nilrad R$ as an $R$-module, we have that $\nilrad R =0$ if and only if $(\nilrad R)_\kp=0$ for all $\kp \in \Spec R$.
  Then since $(\nilrad R)_\kp = \nilrad(R_\kp)$ this completes the proof.
\end{proof}

\begin{prop}[2.21]
  Flatness of a module is a local property.
\end{prop}

\begin{proof}
  Suppose that $M$ is flat.
  Then $M_\kp = R_\kp \otimes_R M$ is flat as an $R_\kp$-module (since localisation is exact).

  Now assume that $M_\kp$ is flat for all $\kp \in \Spec R$.
  Then if
  \begin{equation*}
    \begin{tikzcd}
      0 \ar[r] & N \ar[r] & N'
    \end{tikzcd}
  \end{equation*}
  is exact, then
  \begin{equation*}
    \begin{tikzcd}
      0 \ar[r] & N_\kp \ar[r] & N_\kp'
    \end{tikzcd}
  \end{equation*}
  is exact for all $\kp \in \Spec R$.
  By assumption, for each $\kp \in \Spec R$ we get an exact sequence
  \begin{equation*}
    \begin{tikzcd}
      0 \ar[r] & N_\kp \otimes_{R_\kp} M_\kp \ar[r] & N_\kp' \otimes_{R_\kp} M_\kp
    \end{tikzcd}
  \end{equation*}
  Then we have the following commutative diagram with top row exact
  \begin{equation*}
    \begin{tikzcd}
      0 \ar[r] & N_\kp \otimes_{R_\kp} M_\kp \ar[r] \ar[d, "\cong"] & N_\kp' \otimes_{R_\kp} M_\kp \ar[d, "\cong"] \\
      0 \ar[r] & (N \otimes_R M)_\kp \ar[r] & (N' \otimes_R M)_\kp
    \end{tikzcd}
  \end{equation*}
  since localisation commutes with tensor products.
  Then the bottom row is exact, so since injectivity is a local property, the sequence
  \begin{equation*}
    \begin{tikzcd}
      0 \ar[r] & N \otimes_{R} M \ar[r] & N' \otimes_{R} M
    \end{tikzcd}
  \end{equation*}
  is exact.
\end{proof}

Begin non-assessable content

Outlook: Algebraic Geometry II

We have seen that for a ring $R$ we get a topological space $\Spec R$ where the closed sets in $\Spec R$ correspond to the radical ideals in $R$.

\begin{defn}[AG.4]
  A topological space homeomorphic to $\Spec R$ for some $R$ is called an \textbf{affine algebraic variety}.
\end{defn}

Let $I<R$.
Then
\[\cV(I) = \{ \kp \in \Spec R \mid I \subseteq \kp \} \overset{1:1}{\longleftrightarrow} \Spec R/I\]
is an affine algebraic variety.

Here is a special case of the above definition.
Let $I = (f)$.
Then $\Spec R = \cV(I) \amalg D_f$ for some open set $D_f$.
Now
\[D_f = \{\kp \in \Spec R \mid f \notin \kp \} = \{ \kp \mid f^n \notin \kp \text{ for all } n \geq 0\} \leftrightarrow \Spec R_f.\]
This proves the following corollary.

\begin{cor}
  $D_f$ is an affine algebraic variety.
\end{cor}

Now we look at the general case where $I$ is any ideal of $R$.
We have
\[ \Spec R = \cV(I) \amalg \bigcup_{f \in I} D_f.\]
\begin{enum}
  \io
  Why is there no intersection?
  If $|fr q \in D_f$ then $f \notin \fr q$ so $I \subsetneq \fr q$ and therefore $q \notin \cV(I)$.

  \io
  Why are they complements?
  If $\fr q \supsetneq I$ then there exists an $f \in I$ such that $f \notin \fr q$, that is, $\fr q \in D_f$.
\end{enum}

This shows that any open set is the union of some $D_f$s.
Then the set $\{D_f\}_{f\in R}$ is a basis of the open topology.
