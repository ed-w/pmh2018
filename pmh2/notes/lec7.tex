\section{2018-03-26 Lecture}

\begin{rmk}
	What do we know about the category $\RMod$?
	\begin{itm}
		\io $\Hom_R(M,N)$ is an abelian group and composition is bi-additive. (This makes $\RMod$ in to a \textbf{pre-additive category}.)
		\io It has finite products and coproducts.
		\io It has a zero object (initial an terminal). (This and the above make $\RMod$ in to a \textbf{additive category}.)
		\io Every morphism has a kernel and a cokernel. (This and the above make $\RMod$ in to  \textbf{pre-abelian category}.)
		\io Given an exact sequence
		\[
		\begin{tikzcd}
			\ker f \ar[r, hookrightarrow, "i"] & M \ar[r, "f"] & N \ar[r, twoheadrightarrow, "p"] & \coker f
		\end{tikzcd}
		\]
		we have
		\[\coker i = \ker p\]
		(This is the first isomorphism theorem.)
		(This and the above make $\RMod$ in to an \textbf{abelian category}.)
	\end{itm}
\end{rmk}

\begin{exam}
	\leavevmode
	\begin{enum}
		\io The product in $\mathbf{Sets}$ is the Cartesian product.
		The product in $\mathbf{Top}$ is the Cartesian product with the product topology.
		The product in $\mathbf{Groups}$ is the Cartesian product with component-wise multiplication.
		The product in $\mathbf{Rings}$ is the Cartesian product with component-wise multiplication and addition.
		
		\io The coproduct in $\mathbf{Sets}$ is the disjoint union.
		The coproduct in $\mathbf{Top}$ is the disjoint union with the obvious topology.
		The coproduct in $\mathbf{Top_*}$ is the wedge sum.
		The coproduct in $\mathbf{Groups}$ is the free product.
		The coproduct in $\mathbf{Ab}=\ZZ\mbox{-}\mathbf{Mod}$ is the direct sum.
		
		\io Assume that $I$ is an infinite set.
		Then
		\[\Hom_R \left(\bigoplus_{i \in I} M_i, N\right) \cong \prod_{i \in I} \Hom_R \left(M_i, N\right)\]
	\end{enum}
\end{exam}

Some homological algebra

\begin{defn}[1.9]
	A sequence of $R$-modules and morphisms
	\[
	\begin{tikzcd}
		\cdots \ar[r] & M_{i+1} \ar[r, "f_i"] & M_i \ar[r, "f_{i-1}"] & M_{i-1} \ar[r] & \cdots
	\end{tikzcd}
	\]
	is called \textbf{exact} at position $i$ (or $M_i$) if
	\[\im f_i = \ker f_{i+1}\]
	A sequence is \textbf{exact} if it is exact at every position.
\end{defn}

\begin{exam}
	\leavevmode
	\begin{enum}
		\io The sequence
		\[\
		\begin{tikzcd}
			0 \ar[r] & M' \ar[r, "f"] & M
		\end{tikzcd}
		\]
		is exact $\iff$ $f$ is monic.
		
		\io The sequence
		\[\
		\begin{tikzcd}
			M \ar[r, "f"] & M' \ar[r] & 0
		\end{tikzcd}
		\]
		is exact $\iff$ $f$ is epic.
		
		\io The \textbf{short exact sequence}
		\[\
		\begin{tikzcd}
			0 \ar[r] & M \ar[r, "f"] & M' \ar[r, "g"] & M'' \ar[r] & 0
		\end{tikzcd}
		\]
		(is exact) $\iff$ $f$ is monic, $g$ is epic and $\ker g = \im f$, that is
		\[M'' \cong M'/\ker g \cong M'/\im f \cong \coker f\]
		(We also have $M \cong \ker f$.)
		
		\io Consider the long exact sequence
		\[
		\begin{tikzcd}
			\cdots \ar[r, "x"] & k[x]/(x^2) \ar[r, "x"] & k[x]/(x^2) \ar[r, "x"] & k[x]/(x^2) \ar[r, "x"] & \cdots
		\end{tikzcd}
		\]
		where the maps are multiplication by $x$.
		However the sequence
		\[
		\begin{tikzcd}
			0 \ar[r] & k \cong k[x]/(x) \ar[r, hookrightarrow, "x"] & k[x]/(x^2) \ar[r, "x^2"] & k[x]/(x^3) \ar[r, "x^2"] & k[x]/(x^3)
		\end{tikzcd}
		\]
		is not exact at $k[x]/(x^3)$ since $\im x^2 \subsetneq \ker x^2$.
	\end{enum}
\end{exam}

\begin{prop}[1.10]
	A sequence of $R$-modules
	\[
	\begin{tikzcd}
		M' \ar[r, "f"] & M \ar[r, "g"] & M'' \ar[r] & 0
	\end{tikzcd}
	\]
	is exact $\iff$ for every $R$-module $N$, the sequence
	\[
	\begin{tikzcd}
	0 \ar[r] & \Hom_R(M'',N) \ar[r, "g^*"] & \Hom(M,N) \ar[r, "f^*"] & \Hom(M',N)
	\end{tikzcd}
	\]
	is exact (the contravariant $\Hom$ functor $\Hom(-,N)$ is left exact).
	The functor acts on morphisms by
	\begin{align*}
		f^*: \Hom(M,N) &\to \Hom(M',N) \\
		\phi &\mapsto \phi \circ f
	\end{align*}
\end{prop}
