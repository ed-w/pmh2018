\section{2018-03-06 Lecture}

Operations on ideals

\begin{defn}
	Let $\{I_j \mid j \in J\}$ be a family of ideals in $R$.
	Define
	\begin{itm}
		\io $\sum_j I_j = \left\{ \sum_{j \in J} x_j \text{ finite} \mid x_J \in I_j \right\}$ is the smallest ideal containing all the $I_j$.
		\io $\bigcap_j I_j$ is the largest ideal contained in all the $I_j$.
		\io If $J = [n]$, then
		\[\prod_j I_j = \left\{ \sum_{k,\text{ finite}} x_k^1 x_k^2 \cdots x_k^n \mid x_k^j \in I_j\right\}\]
		\io A special case of the above: if $I_1=I_2=\cdots=I_n=I$, then $\prod_j I_j = I^n$.
		(By convention, $I^0=R$.)
	\end{itm}
\end{defn}

\begin{exam}
	Let $R=\ZZ$, $I_1 = s\ZZ$ and $I_2 = t\ZZ$, where $s,t \in \NN$.
	Then
	\begin{itm}
		\io $I_1 I_2 = st\ZZ$
		\io $I_1 \cap I_2 = \lcm(s,t)\ZZ$
		\io $I_1+I_2 = \gcd(s,t)\ZZ$
	\end{itm}
	If $s$ and $t$ are prime, then $I_1 \cap I_2 = st\ZZ$ and $I_1+I_2=\ZZ$.
\end{exam}

\begin{prop}
	Let $I$, $I'$ and $I''$ be ideals in $R$.
	Then
	\begin{itm}
		\io $I(I'+I'') = II'+II''$
		\io $I \cap (I'+I'') = I \cap I' + I \cap I''$ if $I',I'' \subset I$
		\io $(I+I')(I \cap I') \subset II'$
		\io $II' \subset I \cap I'$
	\end{itm}
\end{prop}

\begin{defn}
	If ideals $I$ and $I'$ of $R$ satisfy $I+I'=R$, then $I$ and $I'$ are said to be \textbf{coprime}.
\end{defn}

\begin{prop}
	If $I$ and $I'$ are coprime then $I \cap I = II'$.
\end{prop}

\begin{prop}[Lattice isomorphism theorem]
	There is an inclusion-preserving bijection for any ideal $I<R$:
	\begin{align*}
		\{\text{ideals in } R/I\} &\overset{\sim}{\longleftrightarrow} \{\text{ideals } J \subset R \text{ with } I \subset J\} \\
		\pi_I(y) &\longleftrightarrow y
	\end{align*}
\end{prop}

\begin{proof}
	Exercise.
\end{proof}

\begin{note}
	If $f_1,\ldots,f_r \in R$, we write $(f_1,\ldots,f_r)$ for the ideal
	$Rf_1+\cdots+Rf_r$.
\end{note}

\begin{defn}
	Here are some special types of ideals.
	\begin{enum}
		\io An ideal $I$ generated by one element ($I=(x)$ for some $x \in R$) is called a \textbf{principal ideal}.
		\io An ideal $I \lneq R$ is a \textbf{maximal ideal} if the only ideal $J<R$ with $I \subsetneq J$ is $R$ itself.
		\io A subset $S \subset R$ is \textbf{multiplicative} if $1_R \in S$ and $f,g \in S$ implies $fg \in S$.
		\io An ideal $I$ is a \textbf{prime ideal} if $R \setminus I$ is a multiplicative set.
		A more traditional definition: an ideal is prime if $ab \in I$ implies $a \in I$ or $b \in I$ (and $I \neq R$).
		\io A ring is an \textbf{integral domain} if $(0)=\{0\}$ is a prime ideal.
		A more traditional definition: $ab=0$ implies $a=0$ or $b=0$ (no zero divisors), and $R \neq 0$.
	\end{enum}
\end{defn}

\begin{prop}
	\leavevmode
	\begin{enum}
		\io $I<R$ is prime $\iff$ $R/I$ is an integral domain
		\io $I<R$ is maximal $\iff$ $R/I$ is a field
	\end{enum}
\end{prop}

\begin{proof}
	\leavevmode
	\begin{enum}
		\io Apply the quotient map to the definition of zero divisors.
		\io A ring $R$ is a field if and only if its only ideals are $0$ and $R$. \qedhere
	\end{enum}
\end{proof}

\begin{cor}
	$I$ is maximal $\implies$ $I$ is prime.
\end{cor}

\begin{exer}
	\leavevmode
	\begin{enum}
		\io If $R \neq 0$, then $R$ contains a maximal ideal (use Zorn's lemma).
		\io If $I\lneq R$ then there exists a maximal ideal $\km$ containing $I$.
		\io If $x \in R$ is not invertible, then there exists a maximal ideal $\km$ containing $x$.
	\end{enum}
\end{exer}

\begin{exam}
	\leavevmode
	\begin{enum}
		\io All ideals of $\ZZ$ are of the form $(n)$ for $n \in \NN$.
		\begin{enum}
			\io $(n)$ is prime $\iff$ $n$ is prime or $n=0$.
			\io $(n)$ is maximal $\iff$ $n$ is prime.
		\end{enum}
		\io All ideals of $k[x]$ are of the form $(f)$ where $f \in k[x]$.
		\begin{enum}
			\io $(f)$ is prime $\iff$ $f$ is irreducible or 0.
			\io $(f)$ is maximal $\iff$ is irreducible.
		\end{enum}
	\end{enum}
\end{exam}
