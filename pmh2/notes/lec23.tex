\section{2018-05-28 Lecture}

\begin{lem}\label{23:l1}
  $(\rad I)^N \subseteq I$ for some $N>0$ if $R$ is Noetherian.
\end{lem}

\begin{proof}
  $I$ is finitely generated and $x^N \in I$ for any generator $x$.
\end{proof}

\begin{lem}\label{23:l2}
  Let $R$ be a ring with a finite collection of maximal ideals $\{\km_i\}_{i=1}^n$ such that $\prod_{i=1}^n \km_i=0$.
  Then $R$ is Noetherian if and only if it is Artinian.
\end{lem}

\begin{proof}
  Exercise.
  Quotient by the maximal ideals to reduce to vector spaces.
\end{proof}

\begin{lem}\label{23:l3}
  If $R$ is a Noetherian ring, then $R$ has finitely many minimal prime ideals.
\end{lem}

\begin{proof}
  Exercise.
  Show that every radical ideal is a finite intersection of prime ideals.
\end{proof}

\begin{proof}[Proof of theorem 3.36]
  $\implies$: We already know that if $R$ is Artinian, then $\dim R=0$.
  Let $\mSpec R=\{\km_1,\ldots,\km_n\}$.
  Then
  \[ \prod_{i=1}^n \km_i \subseteq \cap_{i=1}^n \km_i = \ca J(R) = \nilrad R. \]
  Raising everything to the power $N$ gives $\prod_{i=1}^n \km_i^N = 0$ for some $N$, since $(\nilrad R)^N=0$ for some $N$ (by 3.34).
  This implies that $R$ is Noetherian (by lemma \ref{23:l2}).

  $\impliedby$:
  $R$ has finitely minimal prime ideals (by lemma \ref{23:l3}).
  Since $\dim R=0$, they are all maximal.
  Then $\nilrad R = \bigcap_{i=1}^n \km_i$, so $\prod_{i=1}^n \km_i^N=0$ for some $N$, as before (by lemma \ref{23:l1}).
  This implies that $R$ is Artinian (by lemma \ref{23:l2}).
\end{proof}

\begin{rmk}
  If $R$ is Artinian and local with maximal ideal $\km$, then $\Spec R = \{\km\}$ and $\nilrad R = \km$.
  Hence every element of $R$ is either a unit or nilpotent.
\end{rmk}

\begin{thm}[3.37]
  Let $R$ be an Artinian ring.
  Then $R = \prod_{i=1}^n R_i$ where each $R_i$ is an Artinian local ring.  \end{thm}

begin non-assessable content

\S4 Integrality and Normality

\begin{defn}[4.1]
  Let $S \subseteq R$ be a subring.
  An element $x \in R$ is \textbf{integral over $S$} if
  \[ x^n + s_1x^{n-1} + \cdots + s_{n-1}x + s_n = 0 \]
  for some $s_1,\ldots,s_n \in S$.
  If $A$ is an $R$-algebra, we say that $x \in A$ is integral over $R$ if it is integral over the image of $R$.
  If every element of a ring $R$ is integral over $S \subseteq R$, then we say that $R$ is integral over $S$.
\end{defn}

\begin{prop}[4.2]
  Let $S \subseteq R$ be a subring.
  An element $x \in R$ is integral over $S$ if and only if there exists a faithful $(S'=\ang{S,x})$-module $M \subseteq R$ which is finitely generated as an $S$-module.
\end{prop}

\begin{proof}
  $\implies$:
  Suppose that $x^n+s_1x^{n-1}+\cdots+s_n=0$.
  Set $M=\sum_{i=0}^nSx^i \subseteq R$.
  It is clearly finitely generated as an $S$-module and it is also an $S'$-module since $xM \subseteq M$.
  It is clear that $M$ is faithful since $1 \in M$.

  $\impliedby$:
  Let $M=\ang{e_1,\ldots,e_n}_S$ be the submodule with $xM \subseteq M$.
  Then $xe_i = \sum s_{ij} e_j$ for some $s_{ij} \in S$, so $\sum s_{ij}'e_i=0$ for some $s_{ij}' \in S'$.
  Let $C=(s_{ij})_{ij}$ be the coefficient matrix.
  Then by Cramer's rule, $(\det C)e_i=0$ for all $i$.
  Since $M$ is faithful as an $S'$-module, $\det C=0$.
  This gives the desired monic polynomial in $x$ with coefficients in $S$.
\end{proof}

\begin{rmk}[Cramer's rule]
  If $C \in \Mat_n(R)$ such that $C \bo x = 0$ where $\bo x = [x_i]^\perp$, then $(\det X)x_i=0$ for all $i$.
  This is proved in much the same way as with vector spaces.
\end{rmk}

\begin{prop}[4.3]
  An $R$-algebra $A$ is finitely generated as an $R$-module if and only if it is finitely-generated as an $R$-algebra by elements integral over $R$.
\end{prop}

\begin{proof}
  Suppose $A$ is generated by $x_1,\ldots,x_n$ as an $R$-algebra with $x_i^{n_i}+r_{i_1}x_i^{n_i-1}+\cdots+r_{i_{n_i}}=0$.
  Then $A$ is generated by $\{ x_1^{a_1} \cdots x_na^n \mid 0 \leq a_i < n_i \}$ as an $R$-module.
\end{proof}
