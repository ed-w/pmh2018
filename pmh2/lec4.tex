\section{2018-03-13 Lecture}

\begin{prop}[0.17]
	The lattice isomorphism theorem restricts to prime ideals:
	There is an inclusion-preserving bijection for any ideal $I<R$:
	\begin{align*}
	\{\text{ideals in } R/I\} &\overset{\sim}{\longleftrightarrow} \{\text{ideals } J \subset R \text{ with } I \subset J\} \\
	\pi_I(y) &\longleftrightarrow y
	\end{align*}
\end{prop}

\begin{proof}
	Exercise.
\end{proof}

\begin{rmk}
	\begin{enum}
		\io If $f:R \to R'$ and $p<R'$ is prime, then $f\inv[p]$ is prime.
		(This is not true for maximal ideals e.g.\@ $\ZZ \injto \QQ$.)
		\io If $f:R \to R'$ is surjective and $p<R$ is prime, then $f[p]$ is prime.
	\end{enum}
\end{rmk}

\begin{defn}[0.18]
	The Jacobson radical of $R$ is
	\[J(R) = \bigcap_{\km \in \mSpec R} \km\]
\end{defn}

\begin{note}
	Note that $\nilrad R \subset J(R)$.
\end{note}

\begin{prop}[0.19]
	We have that
	\[J(R) = \{x \in R \mid 1_R - xy \in R^\times \text{ for all } y \in R\}\]
\end{prop}

\begin{proof}
	$\subseteq:$ Assume $x \in J(R)$ and there exists a $y \in R$ such that $1-xy \notin R^\times$.
	By the existence of maximal ideals, $1-xy \in \km$ for some $\km \in \mSpec R$.
	Now since $x \in J(R) \in \km$, we have $xy \in \km$.
	Thus $1 \in \km$, a contradiction.
	
	$\supseteq$: Assume $1-xy \in R^\times$ for all $y \in R$ but there exists an $\km \in \mSpec R$ such that $x \notin \km$.
	Then since $\km$ is maximal, $\km + (x) = R$, so $1=u+ax$ for some $u \in m$ and $a \in R$.
	Then $1-ax \in \km$.
	But this means that $1-ax \notin R^\times$, a contradiction.
	(This is because $(u)=R$ for any $u \in R^\times$.)
\end{proof}

\begin{defn}[0.20]
	A \textbf{local ring} is a ring $R$ with exactly one maximal ideal $\km$.
	The field $R/\km$ is the \textbf{residue field} of $R$ (and $\km$).
\end{defn}

Outlook: Algebraic Geometry I (not assessable)

Define the \textbf{vanishing set} of an ideal by
\[\cV(I) = \{\kp \in \Spec R \mid I \subset \kp\}\]
We have
\begin{align*}
	R \text{ a ring} &\longrightarrow \Spec R \\
	I<R &\longrightarrow \cV(I)
\end{align*}
We want to put a topology on $\Spec R$.

\begin{prop}[AG.1]
	\begin{enum}
		\io $I \subset I' \implies \cV(I) \supset \cV(I')$
		\io $\cV(0) = \Spec R$ and $\cV(R) = \emptyset$
		\io $\cV(I) \cup \cV(J) = \cV(I \cap J) = \cV(IJ)$
		\io $\bigcap_{i \in I} \cV(I_i) = \cV(\sum_{i \in I} I_i)$
	\end{enum}
\end{prop}

\begin{proof}
	1,2 and 4 are omitted.
	We prove 3.
	fill in
\end{proof}

\begin{thm}[AG.2]
	$\Spec R$ is a topological space with the closed sets being $\{\cV(I)\}_{I<R}$.
	This topology is known as the \textbf{Zariski topology}.
\end{thm}

We can generalise this.
If $S$ is a subset of $R$, then define $\cV(S) = \{\kp \in \Spec R \mid S \subset \kp\}$.
If $Y \subset \Spec R$, then define the \textbf{ideal} of $Y$ to be $\cI(Y) = \cap_{\kp \in Y} \kp$.

\begin{rmk}
	\begin{enum}
		\io $\cV(S) = \cap_{x \in S} \cV((x))$ is closed.
		\io $\cI(Y)$ is a radical ideal.
		\io $\cV(\cI(Y)) = \overline{Y}$
		\io If $Y$ is closed, then $\cV(\cI(Y))=Y$.
		\io $\cV(S) = \cV((S))$.
	\end{enum}
\end{rmk}

\begin{prop}[AG.3]
	\begin{enum}
		\io There is a one-to-one order-reversing correspondence between radical ideals of $R$ subsets of $\Spec R$.
		\io The closed points of closed subsets $Y$ of $\Spec R$ are the maximal ideals in $Y$.
	\end{enum}
\end{prop}

\begin{proof}
	\begin{enum}
		\io Note that $\cI(\cV(J)) = \rad J$.
		\io Trivial. \qedhere
	\end{enum}
\end{proof}

\begin{rmk}
	There are non-closed points in $\Spec R$.
	It can happen that $x \in \Spec R$ and $\overline{x} = \Spec R$.
\end{rmk}

One non-rigorous way of thinking about it is that $\Spec R$ is the space and $R$ are the functions on the space.

Let $r \in R$.
We want to assign to $r$ a function $\phi_r: \mSpec R \to ?$ mapping $\km$ to $\overline{r} \in R/\km$.
We need to find a codomain $?$ such that we have a well-defined function for each $\km$ (e.g.\@ we want to embed the $R/\km$ in one big field).