\section{2018-04-17 Lecture}

The tensor product

\begin{defn}[1.24]
	Let $R$ be a commutative ring.
	\begin{enum}
		\io
		A map $f: M \times M' \to N$ is called \textbf{$R$-bilinear} if
		\begin{align*}
			f_M: M &\to \Hom_R(M',N) \\
			m &\mapsto f(m,-)
		\end{align*}
		and
		\begin{align*}
			f_{M'}: M' &\to \Hom_R(M,N) \\
			m' &\mapsto f(-,m')
		\end{align*}
		are linear (and well-defined).
		
		A more standard definition is that the following properties are satisfied:
		\begin{enum}
			\io $f(x+y,z)=f(x,z)+f(y,z)$
			\io $f(x,y+z)=f(x,y)+f(x,z)$
			\io $f(rx,y)=rf(x,y)=f(x,ry)$
		\end{enum}
		
		\io
		The \textbf{tensor product} $M \otimes_R N$ is an $R$-module with a bilinear map $p: M \times N \to M \otimes_R N$ such that for any $R$-module $P$ and any bilinear map $f:M \times N \to P$ there exists a unique map $\phi$ such that the following diagram commutes:
		\[\begin{tikzcd}
			M \times N \ar[r, "p"] \ar[dr, "f", swap] & M \ar[d, "\phi"] \otimes_R N \\
			& N
		\end{tikzcd}\]
	\end{enum}
\end{defn}

\begin{rmk}
	This definition only works for commutative rings.
\end{rmk}

\begin{prop}[1.25]
	The tensor product of two $R$-modules exists and is unique.
\end{prop}

\begin{proof}
	The uniqueness follows from the universal property.
	Let $T$ be the free $R$-module $R^{M \times N}$ generated by the set $M \times N$ modulo the relations
	\begin{gather*}
		(m+m',n)-(m,n)-(m',n) \\
		(m,n+n')-(m,n)-(m,n') \\
		(rm,n) - r(m,n) \\
		(m,rn) - r(m,n)
	\end{gather*}
	for all $r \in R$, $m,m' \in M$ and $n,n' \in N$.
	Let $p$ be the inclusion $M \times N \injto R^{M \times N}$ followed by the projection $R^{M \times N} \surjto T$.
\end{proof}

\begin{rmk}
	\leavevmode
	\begin{enum}
		\io
		Denote the image of $(m,n)$ under $p$ by $m \otimes_R n$.
		This is not unique.
		
		\io
		We will usually omit the subscript $R$ where there is no ambiguity.
	\end{enum}
\end{rmk}

\begin{exam}
	Let $R=\ZZ$, $M=\ZZ$, $M'=2\ZZ$ and $N=\ZZ/2\ZZ$.
	Then $M \otimes_R M \isoto M$ with multiplication, $M \otimes N \isoto N$ with multiplication and $M' \otimes N \isoto N$ by taking the second component.
\end{exam}

\begin{exer}
	Show that $\ZZ/2\ZZ \otimes_\ZZ \ZZ/3\ZZ = 0$.
\end{exer}

\begin{rmk}
	If $M$ is generated by $m$ elements and $N$ is generated by $n$ elements, it is possible for $M \otimes_R N$ to be generated by less than $mn$ elements.
\end{rmk}

2. Localisation and fractions

This is a generalisation of the construction of a ring of fractions.

\begin{defn}[2.1]
	Let $S \subseteq R$ be a multiplicative set (recall that it must contain 1).
	Then $S\inv R$ is a ring with a map $\iota_S: R \to S\inv R$ such that $i_S[S] \subseteq (S\inv R)^\times$ and it is universal with this property: that is, if $f: R \to R'$ is a map of rings such that $f[S] \subseteq (R')^\times$ then there exists a unique map $\phi$ such that
	\[\begin{tikzcd}
		R \ar[r, "\iota_s"] \ar[rd, "f", swap] & S\inv R \ar[d, "\phi"] \\
		& R'
	\end{tikzcd}\]
	$S\inv R$ is called the \textbf{localisation of $R$ at $S$}.
\end{defn}

\begin{rmk}
	This definition only works for commutative rings.
\end{rmk}

\begin{prop}[2.22]
	$S\inv R$ exists and is unique.
\end{prop}

\begin{proof}
	Set
	\[S\inv R = R \times S/\sim\]
	where
	\[(x,s)\sim(y,t) \iff \text{ there exists a } u \in S \text{ such that } u(xt-ys)=0.\]
	We write $x/s$ for the equivalence class of $(x,s)$ under $\sim$.
	(We will omit the verification that $\sim$ is an equivalence relation.)
	
	To make $S\inv R$ in to a ring, define
	\[\frac xs + \frac yt = \frac{xt+ys}{st} \quad \text{and} \quad \frac xs \cdot \frac yt = \frac{xy}{st}.\]
	(We will omit the verification that these operations are well-defined.)
\end{proof}