\section{2018-03-12 Lecture}

\begin{defn}
	A principal ideal domain (PID) is an integral domain such that every ideal is principal.
\end{defn}

\begin{prop}[0.9]
	In a PID every non-zero prime ideal is maximal.
\end{prop}

\begin{proof}
	Let $(x) \neq 0$ be prime and $(x) \subsetneq (y)$.
	Then $x=yz$ for some $x \in R$, so $yz \in (x)$.
	Since $y \notin (x)$, we have $z \in (x)$ so $z=tx$ for some $t \in R$.
	Then $x=yz=ytx$ or $x(1-yt)=0$, so $yt=1$ and so $1 \in (y)$, that is $(y)=R$.
\end{proof}

\begin{defn}[0.10]
	The \textbf{prime spectrum} $\Spec R$ of $R$ is the set of all prime ideals of $R$.
	Then \textbf{maximal spectrum} $\mSpec R$ of R is the set of all maximal ideals of $R$.
\end{defn}

\begin{exam}
	\begin{enum}
		\io $\Spec \ZZ = \{0\} \cup \{\text{primes}\}$ and $\mSpec \ZZ = \{\text{primes}\}$
		\io $\Spec k[x] = \{0\} \cup \{\text{irreducible polynomials in }X\}$.
		\io If $k$ is algebraically closed then the only irreducible polynomials are linear.
		Then $\Spec k[x] = \{(0)\} \cup k$ where we identify $\alpha \in k$ and $(x-\alpha) \in k[x]$.
		Once again $\mSpec$ is just $\Spec$ without $\{(0)\}$.
	\end{enum}
\end{exam}

When we look at $\Spec$ and $\mSpec$ in integral domains, the zero ideal plays a special role.
How can we extend this notion to rings with zero divisors?

\begin{defn}[0.11]
	\begin{enum}
		\io An element $x \in R$ is \textbf{nilpotent} if there exists an $n \geq 0$ such that $x^n=0$.
		\io The \textbf{nilradical} of $R$ is the set of all nilpotent elements in $R$.
	\end{enum}
\end{defn}

\begin{prop}[0.12]
	\begin{enum}
		\io $\nilrad R$ is an ideal and $\nilrad(R/\nilrad R)=0$.
		\io $\nilrad R$ is the intersection of all of the prime ideals of $R$.
	\end{enum}
\end{prop}

\begin{proof}
	\begin{enum}
		\io Since $R$ is commutative $R \nilrad R = R$.
		If $x,y \in \nilrad R$ with $x^n=0$ and $y^m=0$, then $(x+y)^m=0$ by the binomial theorem.
		So $\nilrad R$ is closed under addition.
		Clearly $\nilrad R$ is closed under additive inversion.
		Hence $\nilrad R$ is an ideal.
		
		Let $x+\nilrad R \in R/\nilrad R$ be nilpotent.
		Then there exists an $n > 0$ such that $(x+\nilrad R)^n = x^n + \nilrad R = \nilrad R$, hence $x^n \in \nilrad R$ which implies that $x \in \nilrad R$.
		So $\nilrad(R/\nilrad R)=0$.
		
		\io Let $I = \bigcap_{\kp \in \Spec R} \kp$.
		Let $x$ be nilpotent with $x^n=0$.
		Then $x^n \in \kp$ for all prime ideals $\kp$.
		Then applying the definition of prime (Euclid's lemma) iteratively, we get that $x \in \kp$ for all $\kp$.
		Hence $\nilrad R \subset I$.
		
		Now suppose that $x$ is not nilpotent.
		Define
		\[\Sigma = \{J<R \mid x^n \notin J \text{ for all } n>0\}\]
		Now $\Sigma$ is partially ordered by inclusion and is non-empty since $(0) \in \Sigma$.
		Clearly any chain is bounded by its union which is also in $\Sigma$.
		Then by Zorn's Lemma there exists a maximal element $\kp' \in \Sigma$.
		We need to show that $\kp'$ is prime.
		
		Let $y,y' \in \kp'$.
		Then $\kp'+(y)$ and $\kp'+(y')$ are not in $\Sigma$ since $\kp'$ is maximal.
		So $x^m \in \kp'+(y)$ and $x^n \in \kp' + (y')$, so $x^{n+m} \in \kp' + (yy')$ for some $m$ and $n$.
		Then $\kp' + (yy') \notin \Sigma$, so $yy' \notin \kp'$, hence $\kp'$ is prime (contrapositive of Euclid's lemma).
		Then since $x^n \notin \kp'$ for all $n>0$, we have $x \notin \kp'$ and so $x \notin I$.
		\qedhere
	\end{enum}
\end{proof}

\begin{exam}
	$\nilrad \ZZ = 0$, $\nilrad k[x] = 0$, $\nilrad(k[x]/(x^3))=(x+(x^3))$
\end{exam}

How can we generalise the `nice' properties of prime ideals?

\begin{defn}[0.13]
	Let $I<R$.
	The radical of $I$ is
	\[\rad I = \{x \in R \mid x^n \in I \text{ for some } n \geq 0\}\]
\end{defn}

\begin{lem}[0.14]
	\begin{enum}
		\io $\rad (0) = \nilrad R$
		\io If $\pi_I: R \to R/I$ then $\rad I = \pi_I\inv[\nilrad R/I]$.
		\io $\rad I$ is an ideal.
	\end{enum}
\end{lem}

\begin{proof}
	It follows directly from the definitions.
\end{proof}

\begin{defn}[0.15]
	A \textbf{radical ideal} is an ideal $I$ with $I = \rad I$.
\end{defn}

\begin{lem}[0.16]
	For $I<R$, we have
	\[\rad I = \bigcap_{\substack{\kp \in \Spec R \\ I \subset \kp}} \kp\]
\end{lem}

\begin{proof}
	Apply the previous lemma.
\end{proof}